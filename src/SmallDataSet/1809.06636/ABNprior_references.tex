%%%%%%%%%%%%%%%%%%%%%%%% referenc.tex %%%%%%%%%%%%%%%%%%%%%%%%%%%%%%
% sample references
% %
% Use this file as a template for your own input.
%
%%%%%%%%%%%%%%%%%%%%%%%% Springer-Verlag %%%%%%%%%%%%%%%%%%%%%%%%%%
%
% BibTeX users please use
% \bibliographystyle{}
% \bibliography{}
%
%COMMENTED BECAUSE ONLY INSTRUCTIONS:
%\biblstarthook{
%References may be \textit{cited} in the text either by number (preferred) or by author/year.\footnote{Make sure that all references from the list are cited in the text. Those not cited should be moved to a separate \textit{Further Reading} section or chapter.} The reference list should ideally be \textit{sorted} in alphabetical order -- even if reference numbers are used for the their citation in the text. If there are several works by the same author, the following order should be used:
%\begin{enumerate}
%\item all works by the author alone, ordered chronologically by year of publication
%\item all works by the author with a coauthor, ordered alphabetically by coauthor
%\item all works by the author with several coauthors, ordered chronologically by year of publication.
%\end{enumerate}
%The \textit{styling} of references\footnote{Always use the standard abbreviation of a journal's name according to the ISSN \textit{List of Title Word Abbreviations}, see \url{http://www.issn.org/en/node/344}} should be as follows:
%~\cite{science-contrib, science-online, science-mono, science-journal, science-DOI} .
%}

\begin{thebibliography}{99.}%
% and use \bibitem to create references.
%
% Use the following syntax and markup for your references if
% the subject of your book is from the field
% "Mathematics, Physics, Statistics, Computer Science"
%
% Contribution

% Reference to a link with new prior:


\bibitem{Jansen2003}
Jansen, R., Yu, H.Y., Greenbaum, D., Kluger, Y., Krogan, N.J., Chung, S.B.,
  Emili, A., Snyder, M., Greenblatt, J.F., Gerstein, M.: A Bayesian networks
  approach for predicting protein-protein interactions from genomic data.
\newblock Science \textbf{302}(5644), 449--453 (2003)

\bibitem{Dojer2006}
Dojer, N., Gambin, A., Mizera, A., Wilczynski, B., Tiuryn, J.: Applying dynamic
  Bayesian networks to perturbed gene expression data.
\newblock BMC Bioinformatics \textbf{7}, 249 (2006)

\bibitem{Poon2007}
Poon, A.F.Y., Lewis, F.I., Pond, S.L.K., Frost, S.D.W.: Evolutionary
  interactions between n-linked glycosylation sites in the HIV-1 envelope.
\newblock PLOS Computational Biology \textbf{3}(1), e11 (2007)

\bibitem{Djebbari2008}
Djebbari, A., Quackenbush, J.: Seeded Bayesian networks: Constructing genetic
  networks from microarray data.
\newblock BMC Systems Biology \textbf{2}, 57 (2008)


\bibitem{Hodges2010}
Hodges, A.P., Dai, D.J., Xiang, Z.S., Woolf, P., Xi, C.W., He, Y.Q.: Bayesian
  network expansion identifies new ROS and biofilm regulators.
\newblock PLOS One \textbf{5}:3, e9513 (2010)

% 4) Reference for applied work on pigs from Fraser
\bibitem{Lewis2012} 
Sanchez-Vazquez, M.J., Nielen, M., Edwards, S.A., Gunn, G.J. and Lewis F.I.: 
Identifying associations between pig pathologies using a 
multidimensional machine learning methodology. BMC  Veterinary Research \textbf{8}:151, 1--11 (2012)


% 3) Reference for applied work on euthanasia with Sonja
\bibitem{Hartnack2016} 
Hartnack, S., Springer, S., Pittavino, M. and Grimm, H.: Attitudes of Austrian veterinarians towards euthanasia in small animal practice: impacts of age and gender on views on euthanasia. BMC  Veterinary Research \textbf{12}, 1--14 (2016)

% 2) Reference for applied work on Lepto with Anou
\bibitem{Pittavino2017a} 
Pittavino, M., Dreyfus, A., Heuer, C., Benschop, J., Wilson, P., Torgerson, P., Furrer, R.: Comparison between Generalized Linear Modelling and Additive Bayesian Network. Identification of Factors associated with the Incidence of Antibodies against Leptospira interrogans sv Pomona in Meat Workers in New Zealand. Acta Tropica \textbf{173}, 191--199 (2017)


% 4) Journal article
\bibitem{Rijmen2008} Rijmen, F.: Bayesian networks with a logistic regression model for the conditional probabilities. International Journal of Approximate Reasoning \textbf{48}:2, 659--666 (2008)

% Reference predictor removal
\bibitem{zorn2005solution}
Zorn, Ch.: A solution to separation in binary response models. Political Analysis \textbf{13}:2, 157--170 (2005)

%reference prior (Jeffroeys prior)
\bibitem{firth1993bias}
 Firth, D.: Bias reduction of maximum likelihood estimates. Biometrika \textbf{80}:1, 27--38 (1993)

% reference Gelman 
\bibitem{gelman}
 Gelman, A., Jakulin, A., Pittau, M. G., and Su, Y. S.: A weakly informative default prior distribution for logistic and other regression models. The Annals of Applied Statistics \textbf{2}:4, 1360--1383 (2008)

% Reference for Lindley's paradox
\bibitem{Lindley1957}
Lindley, D.V.: A statistical paradox. Biometrika \textbf{44}:1--2, 187--192 (1957)

\bibitem{Krat:Furr}
Kratzer, G. and Furrer, R.:
    Information-theoretic scoring rules to learn additive Bayesian network applied to epidemiology.
    arXiv:1808.01126 (2018)

% reference Koivisto
\bibitem{koivisto}
Koivisto, M., and Sood, K.: Exact Bayesian structure discovery in Bayesian networks. Journal of Machine Learning Research 5(May), 549--573 (2004)



% 6) abn R package reference
\bibitem{Kratzer}
Kratzer, G., Pittavino, M., Lewis, F., Furrer, R.: abn: an R package for modelling multivariate data using additive Bayesian networks. R package version 1.2. https://CRAN.R-project.org/package=abn (2018)


% 5) R reference
\bibitem{R} 
R Development Core Team. R: a language and environment for statistical computing.
R Foundation for Statistical Computing, Vienna, Austria. http://www.R-project.org (2017)

% Reference for Score Equivalence
\bibitem{Heckerman1995}
Heckerman, D., Geiger, D. and Chickering, D. M.: Learning Bayesian networks: the combination of knowledge and statistical data. Machine Learning \textbf{20}:3, 197--243 (1995).

% 1) PhD thesis as Monograph Contribution 
\bibitem{Pittavino2016} 
Pittavino, M.: Additive Bayesian Networks for Multivariate Data: Parameter Learning,
Model Fitting and Applications in Veterinary Epidemiology. PhD Thesis. University of Zurich (2016)



% Reference to new prior used:
\bibitem{Diaconis1979}
Diaconis, P. and Ylvisaker, D.: Conjugate priors for exponential families. The Annals of Statistics \textbf{7}:2, 269--281 (1979)


\bibitem{Chen2003}
Chen, M. and Ibrahim, J.: Conjugate priors for generalized linear models. Statistica Sinica \textbf{13}:389, 391, 461--476 (2003)

% Reference for the change of variables proposition:
\bibitem{Gut1995}
Guti\'errez-Pe\~na, E., and Smith, A. F. M.: Conjugate parameterizations for natural exponential families. Journal of the American Statistical Association \textbf{432}:90, 1347--1356 (1995)




% % 5) Reference for applied work on Lepto data with Anou
% \bibitem{Pittavino2017b} 
% Pittavino, M., Dreyfus, A., Heuer, C., Benschop, J., Wilson, P., Torgerson, P., Furrer, R.: Data on Leptospira interrogans sv Pomona in Meat Workers in New Zealand. Data in Brief \textbf{13}, 587--596, (2017) 


%% Journal article by DOI
%\bibitem{Friedman1999} Friedman, N., Goldszmidt, M., Wyner, A.: Data analysis with Bayesian networks: A Bootstrap approach, Proc. 15th Conf. on Uncert. in Artif. Intell. (UAI'99), San Francisco: Morgan Kaufmann, 196-205 (1999)
%
%% 7) abn R package reference
%\bibitem{Koivisto2004}
%Koivisto, M., Sood, K. Exact Bayesian structure discovery in Bayesian networks.
%Jour. of Mach. Lear. Res. \textbf{5}, 549-573 (2004)
%
%% 8) Journal article by Proc.
%\bibitem{Plummer2003}
%Plummer, M.: JAGS: a program for analysis of Bayesian graphical models using Gibbs
%701 sampling. Proc. 3rd Int. Work. Dist. Stat. Comp. (DSC 2003), Vienna, Austria, 1-10 (2003)

%

%
\end{thebibliography}
