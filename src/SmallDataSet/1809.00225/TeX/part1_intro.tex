%%%%%%%%%%%%%%%%%%%%%%%%%%%%%%%%%%%%%%%%%%%%%%
\section{Introduction}
\label{sec:intro}
%%%%%%%%%%%%%%%%%%%%%%%%%%%%%%%%%%%%%%%%%%%%%%

The \gls{lhc}~\cite{Evans:2008zzb} at CERN is scheduled to undergo a decisive upgrade~\cite{Apollinari:2015bam}, 
which will extend its physics potential well beyond the initial design goal~\cite{Jakobs:2011zz,Schmidt:2016jra}. 
The first phase (Phase-I) of the upgrade will take place during the long technical shutdown of the \gls{lhc}, in 
2019-2020, to improve the injector and the collider collimation system. In the second phase (Phase-II), scheduled 
for 2024-2026, the \gls{lhc} will be further enhanced with new technologies, including 11-12\,\si{T} triplet magnets 
and compact crab cavities with ultra-precise phase control. The final High Luminosity LHC (HL-LHC) is expected to 
begin operation in 2026, delivering proton-proton ($pp$) collisions with a maximum instantaneous luminosity of 
\SI{7.5e34}{cm^{-2}.s^{-1}} (7.5 times higher than the initial design). In terms of total integrated luminosity, 
a goal of 3000-4000 $\si{fb^{-1}}$ is defined.

In order to perform in the high intensity radiation environment of the HL-LHC, where up to 200 inelastic $pp$ collisions 
(pile-up events) per \SI{25}{ns} bunch crossing are expected, the ATLAS detector~\cite{PERF-2007-01} is
scheduled for upgrade of its sub-detector systems as well as the Trigger and Data AcQuisition (TDAQ) 
strategies~\cite{CERN-LHCC-2015-020}. In this context, the Phase-II upgrade program foresees the replacement of 
the readout electronics of the \gls{TileCal}~\cite{TCAL-2010-01,Collaboration:2285583}, the central hadronic 
calorimeter of ATLAS, with new architectures that will be able to deliver reliable measurements during the 
HL-LHC operation. Among the different designs proposed and evaluated for the \gls{fe} readout electronics, 
the Laboratoire de Physique de Clermont-Ferrand (LPC) presented the solution of an ASIC that embedds both
the analog processing and digitisation of the detector signal, the \gls{fatalic}. The following sections 
intend to provide a detailed description of the \gls{fatalic} motivation and design, and present simulation 
and experimental measurements to demonstrate its potential as a \gls{fe} readout system.

%%%%%%%%%%%%%%%%%%%%%%%%%%%%%%%%%%%%%%%%%%%%%%
\section{The ATLAS Tile Calorimeter}
\label{sec:TileCal}
%%%%%%%%%%%%%%%%%%%%%%%%%%%%%%%%%%%%%%%%%%%%%%

\gls{TileCal} is a sampling calorimeter, constructed of steel plates as absorber and scintillating tiles as active 
medium, and it is important for the measurement of jet- and missing-energy, jet substructure, electron isolation and 
triggering (including muon information). The position of \gls{TileCal} in the ATLAS calorimeter complex can be seen 
in Fig.\,\ref{fig:tile_a}. It is divided into four cylinders, two of which form the central Long-Barrel (LB) while 
the other two constitute Extended-Barrel (EB) partitions, covering the pseudorapidity\footnote{
%%%%-----------------------------------------------------------------------
ATLAS uses a right-handed coordinate system, centered at the nominal interaction point. 
The $x$-axis points towards the center of the LHC ring, the $y$-axis points upwards and 
the $z$-axis points along the beampipe. Cylindrical coordinates ($r$,$\phi$) are used in 
the transverse plane, $\phi$ being the azimuthal angle around the $z$-axis. The pseudorapidity 
is defined in terms of the polar angle $\theta$ as $\eta = -\ln(\tan\frac{\theta}{2})$.}
%%%%-----------------------------------------------------------------------
range $|\eta|<1.7$. Each Tile cylinder is made of $64$ wedge modules (Fig.\,\ref{fig:tile_b}) in the azimuthal
coordinate. The \glspl{pmt} with the associated \gls{fe} readout electronics and high voltage distribution cards 
are inserted at the outer radius of each module, hosted by a train of two \SI{1.4}{m} long ``drawers'' which form
a ``super-drawer'' (one super-drawer can host up to 48 \glspl{pmt}).

\begin{figure}[t]
  \begin{center}
    \subfloat[]{\includegraphics[width=0.64\textwidth]{figures/tile_cal.pdf}\label{fig:tile_a}}
    \subfloat[]{\includegraphics[width=0.36\textwidth]{figures/tile_module.pdf}\label{fig:tile_b}}\\
    \subfloat[]{\includegraphics[width=1.00\textwidth]{figures/tile_towers.pdf}\label{fig:tile_c}}
    \caption{(a) The calorimeter system of ATLAS. (b) Schematic of a wedge module. (c) Tile cell mapping.
    \label{fig:general-context:tilcal-views}}
  \end{center}
\end{figure}

The scintillation light is collected from the two opposide sides of the tiles by \gls{wls} fibres, which are bundled,
to define readout cells, and coupled to a pair of \glspl{pmt}. The current pulse induced at the \gls{pmt} anodes, with 
a typical width of \SI{10}{\ns}, is therefore proportional to the energy deposited in the respective cell and is fed to 
the \gls{fe} electronics to be amplified, shaped and digitised at the \SI{40}{MHz} \gls{lhc} clock rate. This arrangement 
of the readout employs a total of 9852 \glspl{pmt} and segments \gls{TileCal} into three radial layers (Fig.\,\ref{fig:tile_c}), 
classified as "A", "BC" and "D", with depth 1.5, 4.1, and 1.8 interaction lengths, respectively,
at $\eta=0$. Each A-BC-D cell grouping in the same $\eta$-direction forms a projective tower currently used for the fast 
trigger. Additional scintillators (E cells) are installed in the region between the LB and the EB, mainly to measure energy 
escaping the electromagnetic calorimeter, while at higher $|\eta|$, Minimum Bias Trigger Scintillators (MBTS) are used to 
monitor minimum-bias event rates. The E and MBTS scintillators have been used to monitor the luminosity during van der Meer 
scans\footnote{
%------------------------------------------------------------
Van der Meer scans enable the measurement of the absolute luminosity of a particle collider by 
sweeping the beams transverselly across each other. Being responsible for luminosity measurements 
in ATLAS, \gls{TileCal} must be able to measure very low Minimum Bias event rates.}
%------------------------------------------------------------
~\cite{Balagura:2011yw} and are also useful in electron identification.

To ensure stable measurements, \gls{TileCal} incorporates three calibration systems. A \gls{cis}~\cite{Tang:2013vya} 
is used to monitor the response of the \gls{fe} electronics to known injected charge, but also to derive the conversion 
factors between the electronic responses and the input charge. Next, the \gls{cs} system~\cite{Starchenko:2002ju,Shalanda:2003rq} 
uses a radioactive $^{137}$Cs $\gamma$-source, hydraulicaly circulated through a system of tubes that traverses every row 
of scintillating tiles. The illumination of the tiles with \SI{0.662}{MeV} $\gamma$ produces a uniform, low current signal 
at the \gls{pmt} anodes that allows inter-calibration of the detection chains (scintillators, \gls{wls} fibers, \glspl{pmt}, 
\gls{fe} electronics). After initial adjustment of the \gls{pmt} gains, the \gls{cs} system is used to measure the small
variations among the read-out responses, which are used to derive the necessary calibration coefficients with respect to
a unique reference value. Lastly, the laser system~\cite{system:2016tae} injects light pulses to the \glspl{pmt} for 
frequent monitoring of the gains between \gls{cs} scans.

%---------------------------------
\section{Upgrade of the Tile readout}
\label{sec:PhysicsSpecs}
%---------------------------------

In the current Tile readout scheme, the 256 super-drawers are interfaced to back-end Read-Out Drivers (RODs) through
256 optical links (plus another 256 links for redundancy) with a bandwidth of \SI{800}{Mbps} per link. Each super-drawer
stores the digitised samples from the read-out Tile cells in pipeline memories, while analog trigger signals, corresponding 
to each Tile tower, are transmitted to the off-detector trigger pre-processor. These trigger signals are formed by summation 
of the \gls{pmt} outputs, in dedicated analog boards, with rough adjustment in time but without calibration to account for gain
variations. Upon trigger acceptance, the data of each super-drawer are forwarded to the RODs.

For the HL-LHC, the TDAQ strategy\,\cite{,Collaboration:2285583} foresees a fully digital calorimeter trigger with 
higher granularity and precision. Each super-drawer is replaced by four independent mini-drawers, with half the length 
of the current drawers to allow easier access to the electronics and improve the reliability of the cooling circuits. 
At the \SI{40}{MHz} rate, each mini-drawer transmits the entire set of digitised data to an off-detector PreProcessor 
(PPr)\,\cite{Carrio:2014hqa} over two \SI{9.6}{Gbps} optical links (the full readout scheme employs 2048 optical links 
plus another 2048 links for redundancy). The PPr stores the data in pipeline memories and, in parallel, transmits calibrated 
trigger primitives (energy measured in grouped or individual Tile cells) to the trigger system. Upon receipt of a trigger 
acceptance signal, the reconstructed energy, the time and a quality factor for each Tile cell are forwarded to the data 
network for event aggregation and storage.

In the new Tile readout system, the present reliable but outdated Tile \gls{fe} electronics will be replaced by a new
architecture that will be able to endure the harsh radiation conditions of the HL-LHC and handle the expected dynamic
range of the signal. The lowest expected signal from physics events is defined by the minimum ionisation Landau peak 
of muons traversing an A cell (smallest Tile cell) at normal incidence, with a most probable value of
\SI{350}{MeV}~\cite{Adragna:2009zz}. Considering the typical electromagnetic (EM) scale constant of \SI{1.05}{pC/GeV} 
and the $e/\mu$ response ratio of $0.91$, the respective charge delivered by a single \gls{pmt} is approximately 
\SI{200}{fC}. On the other hand, the largest expected signal is that of energetic jets, depositing up to \SI{1.3}{TeV} 
in EM scale (\SI{1.5}{TeV} in hadronic scale) in a single Tile cell. To cover this range, the maximum charge requirement 
for one readout channel is set to \SI{800}{pC}. There is however interest in extending the range up to \SI{1.2}{nC} to 
be able to measure higher energy jets, expected in rare and possibly new physics events. In addition to the processing 
of physics signals, each \gls{fe} electronic card must include a separate channel for large time-constant integration 
of low amplitude currents. This channel is needed for calibration scans using the \gls{cs} system ($\SI{50}{nA}-\SI{100}{nA}$) 
but also for the monitoring of minimum-bias event rates and the instantaneous luminosity ($\SI{20}{pA}-\SI{10}{\mu A}$).

