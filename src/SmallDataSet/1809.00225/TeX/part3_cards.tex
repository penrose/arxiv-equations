%%%%%%%%%%%%%%%%%%%%%%%%%%%%%%%%%%%%%%%%%%%%%%%%%%%%%%%%%%%%%%%%%%%%%%
\section{Associated cards}
\label{sec:cards}
%%%%%%%%%%%%%%%%%%%%%%%%%%%%%%%%%%%%%%%%%%%%%%%%%%%%%%%%%%%%%%%%%%%%%%

Fig.\,\ref{fig:cards:overview} displays a fully equipped mini-drawer, hosting 12 \glspl{pmt}, 12 ``All-in-One'' cards, each of 
which supports one \gls{fatalic} unit and the dedicated \gls{cis}, and the Mainboard, which controls the All-in-One cards and 
transmits the data to the Daughterboard. Both the All-in-One cards and the Mainboard have been designed by LPC for the purposes 
of \gls{fatalic}. The Daughterboard, also shown in Fig.\,\ref{fig:cards:overview}, is the on-detector interface to the back-end 
electronics. It is divided into two independent sides, each of which establishes a \SI{9.6}{Gb/s} uplink to deliver the data of 
six readout channels to the off-detector PPr, while a \SI{4.8}{Gb/s} downlink is used to transmit the \gls{lhc} clock as well as 
control and configuration commands.

\begin{figure}[t]
  \begin{center}
    \includegraphics[width=1.0\textwidth]{figures/cards/full-readout_romain}
    \caption{Schematic view of the readout chain. Left: the All-In-One card supporting FATALIC. Right: a 
    fully equipped mini-drawer housing 12 \glspl{pmt} along with the associated readout electronic cards.
    \label{fig:cards:overview}}
  \end{center}
\end{figure}

%---------------------------------
\subsection{All-in-one card}
%---------------------------------

The All-in-One card supports \gls{fatalic} and the \gls{cis}. It is based on a 6-layer Printed Circuit Board (PCB)
with dimensions $\SI{7.0}{cm}\times\SI{4.7}{cm}$. On one side, a 7-pin connector attaches the card to the 
high-voltage divider, at the basis of the \gls{pmt}, while on the opposite side a 40-conductor ribbon cable establishes 
communication with the Mainboard. Finally, nine on-board potentiometers adjust the pedestal for each channel and the low
voltages supplying FATALIC and other blocks.

A schematic diagram of the \gls{cis} is given in Fig.\,\ref{fig:cards:cis}. The charge injection is driven by a 12-bit 
\gls{dac} with maximum output \SI{4.095}{Volts}. The \gls{dac} charges one of the three available
capacitors, \SI{5.6}{pF}, \SI{39}{pF} or \SI{330}{pF}, for the scanning of the high-, medium- or low-gain dynamic 
range, respectively. The connected capacitance is defined through analog switches, controlled by two timing signals
from the Mainboard. The same timing signals control the charge/discharge (through appropriately adjusted resistances) 
cycles in order to reproduce the \gls{pmt} pulse shape. Finally, the \gls{dac} is also connected to a Howland DC current source
to allow calibration of the slow channel.

\begin{figure}[t]
  \begin{center}
    \includegraphics[width=0.9\textwidth]{figures/cards/cis}
    \caption{Schematic diagram of the \gls{cis}.\label{fig:cards:cis}}
  \end{center}
\end{figure}

%---------------------------------
\subsection{Mainboard}
%---------------------------------

The twelve All-in-One cards of one mini-drawer are controlled by the $\SI{28}{cm}\times\SI{10}{cm}$ Mainboard, which 
serialises the data for transmission to the Daughterboard and distributes clocks and commands from the Daughterboard 
to the \gls{fe} electronics. It also distributes the \SI{10}{V} low-voltage supply, by means of point-of-load regulators, 
to power both the All-in-One cards and the Daughterboard. The connection to the Daughterboard is established with a 
400-pin FPGA Mezzanine Connector (FMC).

The Mainboard is divided into two sections, each of which receives a separate \SI{10}{V} supply and carries two Altera 
Cyclone4 FPGAs. Each FPGA controls three All-in-One cards and communicates with the Daughterboard through a Serial Peripheral 
Interface (SPI) port. The data of each of the two output fast channels are transmitted to the Daughterboard in Low-Voltage 
Differential Signaling (LVDS) with a ``2-Lane Output, 16-bit Serialisation'' at \SI{320}{Mbps}. The necessary synchronisation 
clocks (\SI{160}{MHz} and \SI{40}{MHz}) are also generated by the FPGA, in common for the three cards. On the other hand, 
slow channel data are digitally summed in the FPGA over a time interval of \SI{10}{ms} and delivered to the Daughterboard 
through a standard Inter-Integrated Circuit (I$^2$C) bus.

%\begin{figure}[t]
%  \begin{center}
%    \includegraphics[width=0.8\textwidth]{figures/cards/serialization.jpg}
%    \caption{Serialisation scheme.\label{fig:mbserialisation}}
%  \end{center}
%\end{figure}

\clearpage

