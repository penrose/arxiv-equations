%%%%%%%%%%%%%%%%%%%%%%%%%%%%%%%%%%%%%%%%%%%%%%%%%%%%%%%%%%%%%%%%%%%%%%
\section{Performance with particle beams}
\label{sec:TestBeamResults}
%%%%%%%%%%%%%%%%%%%%%%%%%%%%%%%%%%%%%%%%%%%%%%%%%%%%%%%%%%%%%%%%%%%%%%

The 24 prototype FATALIC were tested in the reconstruction of real energy deposits of hadrons, electrons and muons,
provided by the H8 secondary particle beam of CERN. Two mini-drawers, equipped with 12 \gls{fatalic} \gls{fe} electronics 
each, were inserted into a demonstrator Tile module, providing full readout of Tile cells A1-5, BC1-5, D1 and partial readout
of Tile cells A6 and D0. First, the detection chains by running \gls{cs} scans, using the slow channel. The fast channels 
performance was then probed with particle beams of different energies and compositions.

%Fig.\,\ref{fig:tb:setup} demonstrates the mini-drawer assembly, before insertion, as well as a three 
%of the TileCal modules at the beam hall.

%\begin{figure}[!b]
%  \centering
%  \subfloat[][]{\includegraphics[width=0.48\textwidth]{figures/testbeam/tb_minidrawer.jpg}\label{fig:tb:minidrawer}}
 % \hspace{4pt}
%  \subfloat[][]{\includegraphics[width=0.48\textwidth]{figures/testbeam/tb_module.jpg}\label{fig:tb:module}}
%  \caption{(a) A TileCal mini-drawer, equipped with 12 FATALIC FEs, before insertion into the TileCal module. (b) Three of the TileCal demonstrator modules in the beam hall. }
%  \label{fig:tb:setup}
%\end{figure}

%~~~~~~~~~~~~~~~~~~~~~~~~~~~~~~~~~~~~~~~~~~~~~~~~~
\subsection{Inter-calibration of readout channels}
\label{subsec:Calibration}
%~~~~~~~~~~~~~~~~~~~~~~~~~~~~~~~~~~~~~~~~~~~~~~~~~

As the \gls{cs} source traverses a Tile cell, the response of the respective \glspl{pmt} exhibits a characteristic
plateau (Fig.\,\ref{fig:pl_a}), which reflects the sequential excitation of the tiles, with local maxima (minima), 
generated when the source traverses scintillator tiles (steel plates).
%
\begin{figure}[!tb]
  \centering
  \subfloat[][]{\includegraphics[width=0.48\textwidth]{figures/testbeam/integrator_plateau}\label{fig:pl_a}}
  \subfloat[][]{\includegraphics[width=0.48\textwidth]{figures/testbeam/integrator_calib}\label{fig:pl_b}}
  \caption{(a) Characteristic plateau, showing the response of one \gls{pmt} to the \gls{cs} source signal.
  (b) Plateau variations before and after adjustment of the high voltage of each \gls{pmt}.}
  \label{fig:pl}
\end{figure}
%
Since the energy deposited by the source is uniform, variations of the measured responses can be used to inter-calibrate 
the detection chains. This is performed by equalising each plateau $p_{i}$ to the overall mean $\langle p \rangle$. Since 
the dependence of the \gls{pmt} gain on the applied high voltage is $\sim$$V^{\beta}$, where $\beta\simeq 7$ for the 
particular \glspl{pmt}, the high voltage is corrected to $V^{'}_{i} = V_{i}\cdot(\langle p \rangle/p_{i})^{-7}$. 
Fig.\,\ref{fig:pl_b} demonstrates the plateaus before and after equalisation. As shown, residual variations are 
successfully reduced to the noise level and can be used to derive correction factors for each channel's response.

%~~~~~~~~~~~~~~~~~~~~~~~~~~~~~~~~~~~~~~~~~~~~~~~~~
\subsection{Measurement of particle deposits}
\label{subsec:TBparticles}
%~~~~~~~~~~~~~~~~~~~~~~~~~~~~~~~~~~~~~~~~~~~~~~~~~

The following paragraphs present the results of data analysis, using electron, muon and hadron beams, targeting the center of
each A-cell at 20$^{\circ}$ incidence. The deposited energy is reconstructed using Optimal Filtering, as described in 
Section\,\ref{sec:EnergyReco}, and is expressed in units of input charge by applying the fC/ADC conversion factors, derived using 
the \gls{cis}. Unless specified otherwise, the deposited energy is estimated from the sum of the measurements in the targeted 
A and BC cell. Adjacent cells are also taken into account for containment. The average energy, deposited by a specific beam 
constituent, is obtained from the mean of a gaussian line-shape interpolated around the respective characteristic peak.

%~~~~~~~~~~~~~~~~~~~~~~~~~~~~~~~~~~~~~~~~~~~~~~~~~
\subsubsection*{Electrons}

The reconstruction of \SI{20}{GeV}, \SI{50}{GeV} and \SI{100}{GeV} electron signal is tested with the beam targeting Tile cells 
A2-A5. Since the electromagnetic shower is contained within a short distance in the Tile module, the electron energy is obtained
from the respective distribution in each targeted A-cell. The results are summarised in Table\,\ref{tab:elecs} and displayed 
graphically in Fig.\,\ref{fig:elec_reso} as a function of the beam energy. Using these measurements (from 11\,Tile cells), the 
EM scale constant is estimated $1.04\pm 0.1$\si{pC/GeV}, which is consistent with the nominal value of $1.05\pm 0.1$\si{pC/GeV}, 
derived in precise test-beam studies\,\cite{Adragna:2009zz}, based on more than 200 Tile Cells, using electrons of different
energies with $20^{\circ}$ incidence. Finally, an example showing the total reconstructed energy distribution (A and BC cells 
combined) is presented in Fig.\,\ref{fig:ereco_a}, while Fig.\,\ref{fig:ereco_b} displays the two-dimensional distribution in 
the A/BC cell plane, where the characteristic deposits of the different beam constituents (electrons, muons and pions) can be distinguished.

\begin{figure}[tb]
  \centering
  \subfloat[][]{\includegraphics[width=0.48\textwidth]{figures/testbeam/elec_linearity}}
  \subfloat[][]{\includegraphics[width=0.48\textwidth]{figures/testbeam/elec_reso}}
  \caption{Relative mean (a) and standard deviation (b) of the reconstructed electron energy in A-cells as a function of 
  the beam energy. Error bars represent statistical uncertainties from the gaussian fit to the energy distribution in each cell.}
  \label{fig:elec_reso}
\end{figure}

\begin{table}[h]
  \begin{center}{\small
  \begin{tabular}{ r c c c c c c c c }
  {\bfseries Beam} & \multicolumn{2}{c}{\bfseries Cell A2} & \multicolumn{2}{c}{\bfseries Cell A3} & \multicolumn{2}{c}{\bfseries Cell A4} & \multicolumn{2}{c}{\bfseries Cell A5}\\
  \toprule
  GeV & $Q_\text{reco}$\,[pC] & $\sigma$\,[pC] & $Q_\text{reco}$\,[pC] & $\sigma$\,[pC] & $Q_\text{reco}$\,[pC] & $\sigma$\,[pC] & $Q_\text{reco}$\,[pC] & $\sigma$\,[pC]\\
   20   & 23.2 & 2.3  &  20.1 & 2.4  &  21.6 & 2.5  &  19.7 & 2.2\\
   50   & 54.3 & 3.5  &  50.1 & 3.5  &  52.5 & 3.7  &  -    & -  \\ 
  100   & 98.2 & 5.6  &  98.7 & 5.5  & 103.1 & 4.8  &  95.1 & 5.1\\ 
  \bottomrule
  \end{tabular}}
  \end{center}
  \caption{\label{tab:elecs}Reconstructed electron energy in the targeted A-cells. }
\end{table}

\begin{figure}[t]
  \centering
  \subfloat[][]{\includegraphics[width=0.48\textwidth]{figures/testbeam/T4_E100}\label{fig:ereco_a}}
  \subfloat[][]{\includegraphics[width=0.508\textwidth]{figures/testbeam/elec2D_black}\label{fig:ereco_b}}\\
  \subfloat[][]{\includegraphics[width=0.48\textwidth]{figures/testbeam/T4_M165}\label{fig:ereco_c}}
  \subfloat[][]{\includegraphics[width=0.48\textwidth]{figures/testbeam/T3_H30}\label{fig:ereco_d}}
  \caption{(a) Reconstructed energy distribution with 100\,GeV electron beam targeting cell A4. (b) Two-dimensional distribution
  in the A/BC-cell plane, demonstrating the characteristic deposits of the different beam consituents. (c) 165 GeV muon beam 
  targeting cell A4. (d) 30 GeV\,hadron beam targeting cell A3 (adjacent towers are also included for containment).\label{fig:ereco}}
\end{figure}

%~~~~~~~~~~~~~~~~~~~~~~~~~~~~~~~~~~~~~~~~~~~~~~~~~
\subsubsection*{Muons}

The reconstruction of muon signal is tested with \SI{165}{GeV} muon beams targeting cells A2-A5. The reconstructed energy distribution
for a characteristic case is shown in Fig.\,\ref{fig:ereco_c}. The most probable values (mpv), estimated for each targeted Tile tower 
(A- and BC-cells), are listed in Table\,\ref{tab:muons}. The results are also expressed in terms of energy loss per unit distance (\dedx), 
using the track-length in each Tile cell (\SI{31.925}{cm} for A-cells and \SI{89.391}{cm} for BC-cells, for 20$^\circ$ incidence). The 
overal energy loss is $14.2\pm 1.9$ \si{fC/cm}, which corresponds to $15.0\pm 2.0$ \si{MeV/cm} considering the EM calibration 
constant of \SI{1.04}{MeV/pC} estimated above and the $e/\mu$ response ratio of 0.91. The result is consistent with the estimate of
\SI{15.2}{MeV/cm}, reported in previous test-beam studies using \SI{180}{GeV} muons at projective angles.

\begin{table}[h]
\begin{center}{\small
\begin{tabular}{ r c c c c  }
{\bfseries Cell Type} & {\bfseries Tower 2(+3)} & {\bfseries Tower 3(+4)} & {\bfseries Tower 4(+5)} & {\bfseries Tower 5}\\
\toprule
          & $Q_\text{reco}$\,[pC] & $Q_\text{reco}$\,[pC] & $Q_\text{reco}$\,[pC] & $Q_\text{reco}$\,[pC]\\
A         &     0.49 &  0.41 & 0.47 & 0.41\\
BC        &     1.66 &  1.17 & 1.11 & 1.22\\ 
A+BC      &     2.21 &  1.60 & 1.60 & 1.67\\
\midrule
          &   $dQ/dx$\,[fC/cm] & $dQ/dx$\,[fC/cm] & $dQ/dx$\,[fC/cm] & $dQ/dx$\,[fC/cm]\\
A         &   15.3 & 12.8 & 14.7 & 12.8\\
BC        &   18.6 & 13.1 & 12.4 & 13.6\\ 
A+BC      &   18.2 & 13.2 & 13.2 & 13.8\\ 
\bottomrule
\end{tabular}}
\end{center}
\caption{\label{tab:muons}Reconstructed energy and energy loss per unit distance of 165\,GeV muons targeting cells A2-A5.}
\end{table}

\begin{table}[h]
  \begin{center}{\small
  \begin{tabular}{ r c c c }
  {\bfseries Beam} & {\bfseries Tower 2+3+4} & {\bfseries Tower 3} & {\bfseries Tower 2+4}\\
  \toprule
  GeV   & $Q_\text{reco}$\,[pC] & $Q_\text{reco}$\,[pC] & $Q_\text{reco}$\,[pC]\\
   30   &     24.4 &   20.1 &  2.8\\
  180   &    150.9 &  121.4 & 27.9\\ 
  \bottomrule
  \end{tabular}}
  \end{center}
  \caption{\label{tab:hadrons}Reconstructed energy of 30\,GeV and 180\,GeV pions in the targeted tower\,3 and adjacent towers\,2,4.}
\end{table}

%~~~~~~~~~~~~~~~~~~~~~~~~~~~~~~~~~~~~~~~~~~~~~~~~~
\subsubsection*{Hadrons}

Hadron signal is studied with beams of \SI{30}{GeV} and \SI{180}{GeV} pions targeting Tile cell A3. To account for energy leakage
towards adjacent Tile cells of the same module, deposits measured in Tile cells A2, BC2, A4, BC4 are also added to the measurement. 
The results are summarised in Table\,\ref{tab:hadrons}, while the reconstructed energy distribution for the case of \SI{30}{GeV} pions is
shown in Fig.\,\ref{fig:ereco_d}. As expected, the measured energy is lower than the beam energy, since a single Tile module 
cannot provide full coverage, in solid angle, of the hadronic shower. The energy leakage towards neighbouring Tile cells of the 
same module is found approximately 11\% and 18\% in the cases of \SI{30}{GeV} and \SI{180}{GeV}, respectively.

