Belief propagation is a heuristic procedure for computing marginal probabilities of graphical models.
We choose to use a slightly different belief propagation implementation compared to~\cite{duclos2010fast}, as ours seems to be more natural for the bit-flip noise model.
We divide the lattice into $2\times 2$ unit cells.
Let $G$ be a bipartite graph, where one part corresponds to unit cells, and another part to coarse-grained edges.
Two vertices in $G$ is connected when the coarse-grained edge is adjacent to the unit cell.
This later decides how the messages flow in the graph.
However, we assign two variables $\{x(e_i),x(e_j)\}$ to each vertex corresponding a coarse-grained edge which $e_i$ and $e_j$ form.
In this section, the symbol $e$ or $e_i$ will denote original edges of the lattice (i.e. not coarse-grained).
We define $E$ to be the set of all edges $e$, $E_{cg} \subset E$ to be the set of $e$ which are components of coarse-grained edges (i.e. red edges in \autoref{fig:lattice}), and $\bar{E}_{cg} = E \setminus E_{cg}$.
Given a syndrome $S$, the unnormalized probability of an error configuration $\vec{x}\equiv \{x(e)\}_{e\in E}$ can be written as
\begin{equation}
p(\vec{x}) = g(S, \vec{x})\prod_{e\in E} p_e(x(e)) ,
\end{equation}
where $g(S, \vec{x}) = 1$ if $\vec{x}$ has syndrome $S$, and otherwise $g(S, \vec{x}) = 0$.
It is obvious $g(S, \vec{x})$ can be factorized to local terms.
Thus, the marginal distribution for $e\in E_{cg}$ can then be factorized according to $G$ as
\begin{equation}
\sum_{\{x(e), e\in \bar{E}_{cg}\}}p(\vec{x}) =  \prod_c f_c(\{x(e)\}_c) ,
\end{equation}
where the product is taken over all unit cells $c$, and $\{x(e)\}_c$ are the set of $x(e)$ such that $e\in E_{cg}$ is adjacent to $c$.
We can then apply the standard belief propagation to the graph $G$.
Without further explanation, we choose to use the following rule.
A unit cell $c_k$ sends to each of its adjacent cell $c_n$ a message containing 4 real numbers
\begin{equation}
\{m_{c_k, c_n}(x(e_i), x(e_j))\} \quad \text{for} \  x(e_i),x(e_j) = 0,1 ,
\end{equation}
where $e_i$ and $e_j$ form the coarse-grained edge between $c_k$ and $c_n$.
When we already fix an error configuration $\vec{x}$, we may use the simplified notations
\begin{equation}
p_e \equiv p_e(x(e)), \qquad m_{c_k, c_n} \equiv m_{c_k, c_n}(x(e_i), x(e_j)).
\end{equation}
To compute an out-going message from cell $c$, we take messages from the other three directions, and consider them to be the probability of error configuration on respective edges.
We then sum over all error configurations in the cell which give the correct syndrome of the 4 plaquette stabilizer checks.
\begin{figure}
	\includegraphics[width=0.5\linewidth]{bp_cell}
	\caption{An illustration of message passing for a unit cell.}
	\label{fig:bp_cell}
\end{figure}
More concretely, we define $\vec{x}_c$ to be $\vec{x}$ restricted to edges in $c$ (i.e. all edges in \autoref{fig:bp_cell}), and $g'(S, \vec{x}_c)$ checks whether $\vec{x}_c$ is compatible with $S$ similar to $g$.
We assume we want to send the messages from $c$ to $c_n$, while the incoming messages are from cells $I = \{c_k\}$.
Then we have
\begin{equation}
m_{c, c_n} = \sum_{\vec{x}_c} g'(S, \vec{x}_c)\prod_{c_k\in I} m_{c_k, c} \prod p_{e_i}.
\end{equation}
For the last term $\prod p_{e_i}$, the product is taken over the blue edges in \autoref{fig:bp_cell}, assuming $c_n$ is on the right of $c$.

In the end of the message passing, we can compute the marginal probability by
\begin{equation}
P(x(e_i),x(e_j)) = m_{c_k, c_n}m_{c_n,c_k} / \left(p_{e_i}p_{e_j}\right) ,
\end{equation}
where $e_i$ and $e_j$ are the edges between $c_n$ and $c_k$.
From the joint distribution $P(x(e_i),x(e_j))$ we can compute the distribution $P\left(x(e_i)+x(e_j)\mod 2\right)$.
It is not hard to see that the above message passing rules will lead to the correct marginal probability when the underlying graph is a tree.
(Note this is not the case for the square lattices we are considering)

The key differences between our implementation and the one in~\cite{duclos2013fault} is
\begin{itemize}
	\item Ours utilizes all four stabilizer checks in each unit cell while in~\cite{duclos2013fault} only three are used.
	\item Each message contains 4 real numbers in our implementation while only 1 in~\cite{duclos2013fault}.
\end{itemize}
