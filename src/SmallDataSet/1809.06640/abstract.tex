We still do not have the perfect decoders for topological codes that can satisfy all needs of different experimental setups.
Recently, a few neural network based decoders have been studied, with the motivation that they can adapt to a wide range of noise models, and can easily run on dedicated chips without a full-fledged computer.
The later feature might lead to fast speed and the ability to operate in low temperature.
However, a question which has not been addressed in previous works is whether neural network decoders can handle 2D topological codes with large distances.
In this work, we provide a positive answer for the toric code~\cite{Kitaev2003Faulttolerantanyon}.
The structure of our neural network decoder is inspired by the renormalization group decoder~\cite{duclos2010fast,duclos2013fault}.
With a fairly strict policy on training time, when the bit-flip error rate is lower than $9\%$, the neural network decoder performs better when code distance increases.
With a less strict policy, we find it is not hard for the neural decoder to achieve a performance close to the minimum-weight perfect matching algorithm.
The numerical simulation is done up to code distance $d=64$.
Last but not least, we describe and analyze a few failed approaches.
They guide us to the final design of our neural decoder, but also serve as a caution when we gauge the versatility of stock deep neural networks.