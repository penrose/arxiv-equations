\documentclass[11pt]{article}
\usepackage[utf8]{inputenc}
\usepackage{almost_sure}


\setlength\parindent{0pt}
\allowdisplaybreaks[4]


\newcommand{\serious}[1]{\large{\textbf{\textcolor{red}{#1}}}}
\newcommand{\change}[1]{\textcolor{orange}{#1}}


\newcommand{\Addresses}{{% additional braces for segregating \footnotesize
  \bigskip\bigskip \small

  Bjoern Bringmann, \textsc{University of California, Los Angeles, Department of Mathematics, 520 Portola Plaza, Los Angeles, CA 90095}\\\nopagebreak
  Email address: \texttt{bringmann@math.ucla.edu}

}}
\title{Almost sure local well-posedness for a derivative nonlinear wave equation}
\author{Bjoern Bringmann}
\begin{document}
\pagenumbering{arabic}

\maketitle

\let\thefootnote\relax\footnotetext{\emph{MSC2010}: 35L05, 35L15, 35L71.}
\let\thefootnote\relax\footnotetext{\emph{Keywords}: nonlinear wave equations, probabilistic well-posedness, quadratic nonlinearity, paraproducts}
\begin{abstract} \noindent
We study the derivative nonlinear wave equation \( - \partial_{tt} u + \Delta u = |\nabla u|^2 \) on \( \mathbb{R}^{1+3} \). The deterministic theory is determined by the Lorentz-critical regularity \( s_L = 2 \), and both local well-posedness above \( s_L \) as well as ill-posedness below \( s_L \) are known. In this paper, we show the local existence of solutions for randomized initial data at the super-critical regularities \( s\geq 1.984\). In comparison to the previous literature in random dispersive equations, the main difficulty is the absence of a (probabilistic) nonlinear smoothing effect. To overcome this, we introduce an adaptive and iterative decomposition of approximate solutions into rough and smooth components. In addition, our argument relies on refined Strichartz estimates, a paraproduct decomposition, and the truncation method of de Bouard and Debussche. 
\end{abstract}


\section{Introduction}
We consider the Cauchy problem for the nonlinear wave equation 
\begin{equation}\label{intro:eq_nlw}
\begin{cases}
-\partial_{tt} u + \Delta u = |\nabla u|^2  \qquad\qquad \text{for} ~ (t,x)\in \mathbb{R}^{1+d} \\
u|_{t=0}=f_0 , ~ \partial_t u|_{t=0}=f_1 
\end{cases}~,
\end{equation}
with initial data \( (f_0,f_1)\in H_x^{s}(\mathbb{R}^d)\times H_x^{s-1}(\mathbb{R}^d)\) and dimension \( d \geq 2 \). The choice of the nonlinearity \( |\nabla u|^2 \) is mainly for simplicity, and the methods of this paper also apply to a general quadratic derivative nonlinearity. In particular, using the sign change \( u \mapsto -u \), one can convert \( |\nabla u|^2 \) into \( -|\nabla u|^2 \).\\
 The deterministic theory of \eqref{intro:eq_nlw} is by now well-understood. 
Due to the scaling symmetry of the equation, one expects local well-posedness only for \( s \geq d/2\). Using Lorentz-transformations (cf. \cite{SS98,Tao06}) one obtains a second obstruction to local well-posedness, and the Lorentz-critical regularity is given by \( (d+5)/4 \). The local well-posedness of \eqref{intro:eq_nlw} for 
\begin{equation*}
s> s_d:=\max \left( \frac{d}{2}, \frac{d+5}{4} \right)
\end{equation*}
was proven by Ponce-Sideris \cite{PS93}, Zhou \cite{Zhou03}, and Tataru \cite{Tataru99}. In contrast, the ill-posedness for \( s \leq s_d \) was proven by Lindblad \cite{Lindblad93, Lindblad96} for certain derivative nonlinear wave equations. In particular, a minor modification of the example on \cite[p. 511]{Lindblad93} applies to \eqref{intro:eq_nlw} in dimension \( d=3 \).\\
The purpose of this paper is to understand whether the ill-posedness in low regularity spaces is witnessed by generic or only exceptional sets of initial data. This leads us to consider the Cauchy problem \eqref{intro:eq_nlw} for random initial data \( (f_0^\omega, f_1^\omega) \in H^s(\mathbb{R}^d)\times H^{s-1}(\mathbb{R}^d) \). In recent years, the study of random dispersive PDEs has seen an enormous growth of interest. We refer the reader to the survey paper \cite{BOP18} for a detailed summary, and mention the related works
\cite{BOP15, BOP17, Bourgain94,Bourgain96,Bringmann18,BT08I,BT08II,DLM17,DLM18,KMV17,LM13,LM16,Pocovnicu17}. In this paper, we construct the random initial data using the Wiener randomization \cite{BOP15,LM13}. For this, let \( f \in L^2(\mathbb{R}^d) \) be arbitrary but fixed. Let \( \varphi\colon \mathbb{R}^d \rightarrow [0,1] \) be a smooth and compactly supported function such that the translates \( \{ \varphi(\cdot -k): k \in \mathbb{Z}^d \} \) form a partition of unity. Then, the Wiener decomposition of \( f \) is given in frequency space by
\begin{equation}\label{intro:eq_decomp}
\widehat{f}(\xi) = \sum_{k\in \mathbb{Z}^d} \varphi(\xi-k) \widehat{f}(\xi)~. 
\end{equation}
The Wiener randomization is now defined by randomizing the coefficients in \eqref{intro:eq_decomp}.  Let \( (\Omega, \mathscr{F}, \mathbb{P} ) \) be a probability space and let 
\( \{ g_k(\omega) \colon k \in \mathbb{Z}^d \} \) be a family of independent standard complex Gaussians. Then, we define
\begin{equation}\label{intro:eq_randomization}
\widehat{f^\omega}(\xi) = \sum_{k\in \mathbb{Z}^d} g_k(\omega) \varphi(\xi-k) \widehat{f}(\xi)~. 
\end{equation}
Thus, \( f^\omega \) is a random linear combination of functions that are frequency localized on cubes of scale \( \sim 1 \). The Gaussians may also be replaced by any family of independent uniformly sub-Gaussian random variables. Furthermore, if \( \varphi(\xi) = \varphi(-\xi) \) and \( f \) is real-valued, one can condition on the event that \( g_k = \overline{g_{-k}} \) for all \( k \in \mathbb{Z}^d \) to obtain real-valued functions \( f^\omega \).  \\

The first probabilistic result on wave equations with a quadratic derivative nonlinearity was recently obtained in \cite{CCMNS18}. The authors proved the following theorem. 

\begin{thm}[{\cite[Theorem 2.1 and Remark 2.3]{CCMNS18}}]\label{thm:staffilani}
Let \( (f_0 , f_1) \in H^1(\rthree)\times L^2(\rthree) \) and let \( (f_0^\omega,f_1^\omega) \) be as in \eqref{intro:eq_randomization}. Let \( F^\omega(t) = W(t)(f_0^\omega,f_1^\omega) \) be the solution to the linear wave equation with initial data \( (f_0^\omega,f_1^\omega) \). Furthermore, let \( u^{(j)} \) be the \( j\)-th Picard iterate, which is given by
\begin{align*}
u^{(0)}(t) &:= F^\omega(t)~, \\
u^{(j)} (t) &:= F^\omega(t) + \duh |\nabla u^{(j-1)}|^2\ds \qquad \forall j\geq 1~. 
\end{align*}
For any sufficiently small \( T>0 \), we have for almost every \( \omega \in \Omega \) that
\begin{equation*}
(u^{(j)}, \partial_t u^{(j)} ) \in \big( C_t^0 \dot{H}^1_x \times C_t^0 L^2_x\big)([0,T]\times \rthree) \qquad \forall j\geq 1~. 
\end{equation*}
\end{thm}
\begin{rem}
In fact, the theorem in \cite{CCMNS18} is slightly more general, and holds for any dimension \( d=2,3,4 \) and any quadratic derivative nonlinearity. Furthermore, the randomization in \cite{CCMNS18} uses random signs instead of Gaussians.
\end{rem}
The randomness in Theorem \ref{thm:staffilani} is essential. For deterministic data, the statement of the theorem may even fail for the first iterate \( u^{(1)} \), see \cite{FK00,Zhou97}. The bounds in \cite{CCMNS18} on the size of \( u^{(j)} \), however,  are not uniform in \( j\geq 0\), and are not sufficient to conclude the existence of a solution.  In fact, this is mentioned as an open problem on \cite[p.3]{CCMNS18}. \\
 The main theorem of this paper solves this problem (in three dimensions) for certain super-critical regularities \( s < 2 = s_3  \).

\begin{thm}[Main theorem]\label{main_thm}
Assume that \( (f_0,f_1) \in H_x^s(\rthree)\times H_x^{s-1}(\rthree)\), where \( s \geq 1.984 \). In addition, let \( 0 < T_0 \ll 1 \) and \( \sigma=1.1 \). Then, there exists a random function \( u \) and random times \( 0 < T(\omega) \leq T_0 \) such that 
\begin{equation}\label{intro:eq_u_space}
\begin{aligned}
u&\in  \big( L_\omega^2 C_t^0 H_x^s \medcap L_\omega^2 L_t^2 W_{x}^{\sigma,\infty} \big)(\Omega \times [0,T_0] \times \rthree)~,\\
\partial_t u &\in\big( L_\omega^2 C_t^0 H_x^{s-1} \big)(\Omega \times [0,T_0] \times \rthree)~,
\end{aligned}
\end{equation}
and such that for almost every \( \omega \in \Omega \) it holds that 
\begin{equation}\label{intro:eq_u_duhamel}
u(t)= W(t) (f_0^\omega, f_1^\omega) + \duh |\nabla u(s)|^2 \ds \qquad \forall t \in [0,T(\omega)]~. 
\end{equation}
\end{thm}
\begin{rem}
A minor modification of the arguments should lead to a similar result in dimension \( d=2 \). We expect the restriction \( s \geq 1.7281 \), which lies below the critical regularity \( s=1.75 \). In contrast, the extension to high-dimensions \( d \geq 4 \) may be more difficult, and likely involves \( X^{s,b} \)-type spaces \cite{Bourgain93}. The techniques of this paper may also apply to nonlinear wave equations with null-forms \cite{KM93}, but we have not pursued this direction yet. 
\end{rem}
In the following we sketch the main ideas behind the proof of Theorem \ref{main_thm}. We first describe why a common combination of the Da Prato-Debussche trick (cf. \cite{PD02}) and nonlinear smoothing estimates cannot be applied to \eqref{intro:eq_nlw}. As above, let \( F^\omega(t) \) be the solution to the linear wave equation with initial data \( (f^\omega_0,f^\omega_1) \). Then, we decompose the solution as \( u(t)= F^\omega(t)+w(t) \), and obtain the equation
\begin{equation}\label{intro:eq_nlw_nl}
\begin{cases}
-\partial_{tt} w + \Delta w =  |\nabla w|^2 + 2 \nabla w \cdot \nabla F + |\nabla F|^2 \qquad\qquad \text{for} ~ (t,x)\in \mathbb{R}^{1+d} \\
w|_{t=0}=0, ~ \partial_t w|_{t=0}=0
\end{cases}~.
\end{equation}
Previous works \cite{BOP15,BOP17,DLM17,OP16,Pocovnicu17} then use a contraction mapping argument at a sub-critical regularity \( \nu > s_d \) to show that a solution \( w(t) \) of  \eqref{intro:eq_nlw_nl} exists almost surely. In addition to probabilistic Strichartz estimates for \( F^\omega(t) \), this requires a (probabilistic) nonlinear smoothing estimates for \( w(t) \). For example, in the case of the nonlinear Schrödinger equation, this can be proven using either bilinear Strichartz estimates \cite{BOP15,BOP17,Brereton16} or local smoothing estimates \cite{DLM18}. However, the equation \eqref{intro:eq_nlw_nl} does not exhibit nonlinear smoothing. To see this, we examine the low-high interaction term \( \nabla P_1 F(t) \cdot \nabla P_{\gg 1} F(t) \). Heuristically, we have for any \( \nu >s_d>s \) that 
\begin{equation*}
|\nabla|^{\nu} \duh \nabla P_1 F \cdot \nabla P_{\gg 1} F^\omega \ds \simeq \int_0^t \sin((t-s)|\nabla|)~ \nabla P_1 F \cdot |\nabla|^{\nu-1} \nabla P_{\gg 1} F \ds 
\end{equation*}
Thus, the linear evolution \( F_n^\omega(t) \) is attacked by more than \( s \) derivatives. Since the Duhamel integral does not increase the spatial regularity, and the bilinear Strichartz estimates for the wave equation do not gain derivatives \cite{FK00}, we cannot show a nonlinear smoothing estimate for this term. In fact, by choosing the initial data to be frequency localized on two cubes of scale \( \sim 1\), one at distance \( \sim 1 \) and one at distance \( \sim N \gg 1 \) from the origin, we see that this term may have the same regularity as the initial data. \\
In the above heuristic, we have seen that the low-high interactions form the main obstacle towards the well-posedness of \eqref{intro:eq_nlw_nl} at a regularity \( \nu> s_d \). In other dispersive equations, such as the Benjamin-Ono equation \cite{Tao04}, the low-high interactions can be removed by using a gauge transformation. However, \eqref{intro:eq_nlw_nl} does not appear to have such structure. Instead, we remove the low-high interactions by viewing them as part of the linear evolution for the high-frequency data. To make this precise, we first need to introduce an iterative method. For \( n\geq 0 \) and \( N= 2^n \), we set 

\begin{equation*}
 Q_1 f^\omega(x):= g_0(\omega) P_0 f(x)\quad \text{and} \quad Q_{N} f^\omega(x) := \sum_{N/2\leq \| k \|_2 < N} g_k(\omega) P_k f (x), ~\text{where}~  N \geq 2~.
 \end{equation*}
 We remark that the family of random functions \( \{ Q_N f \}_{N\geq 1} \) is jointly independent, which is essential for the argument. Furthermore, we define
 \begin{equation*}
 Q_{\leq N}f^\omega(x):= \sum_{M\leq N} Q_{M}f^\omega(x)~.
 \end{equation*}
 Since the frequency-truncated initial data is smooth, there exists a solution \( u_n \) of 

 \begin{equation}\label{intro:eq_un}
\begin{cases}
 -\partial_{tt} u_n + \Delta u_n = |\nabla u_n|^2\\
 u_n|_{t=0} = Q_{\leq N} f_0^\omega~,~~\partial_t u_n |_{t=0} = Q_{\leq N} f_1^\omega ~.
\end{cases}
\end{equation}
Our goal is to prove the convergence of \( u_n \) in the low regularity space \( C_t^0 H_x^s \), and define the solution \( u \) as the limit of the sequence \( u_n \). First, we define the increment \( v_n \) by writing \( u_n = u_{n-1} + v_n \). To simplify the notation, we use the convention \( u_{-1} = 0\). Then, the equation for \( v_n \) reads 

\begin{equation}\label{intro:eq_vn}
\begin{cases}
 -\partial_{tt} v_n + \Delta v_n = |\nabla v_n|^2 + 2 \nabla u_{n-1} \cdot \nabla v_n   \\
 v_n|_{t=0} = Q_{N} f_0^\omega~,~ ~ \partial_t v_n |_{t=0} = Q_{N} f_1^\omega ~.
\end{cases}
\end{equation}
To control \( v_n \) uniformly in \( n \geq 0 \), it is necessary to decompose it into a rough, linear component and a smooth, nonlinear component. For a fixed parameter \( \gamma \in (0,1) \), we define the adapted linear evolution \( F_n^\omega \) as the solution to 

\begin{equation}\label{intro:eq_Fn}
\begin{cases}
-\partial_{tt} F_{n}^\omega+\Delta F_{n}^\omega =  2  \nabla P_{\leq N^\gamma} u_{n-1} \cdot \nabla F_{n}^\omega\\
F_{n}^\omega|_{t=0} = Q_N f_0^\omega~,~~ \partial_t F_{n}^\omega|_{t=0} = Q_N f_1^\omega~. 
\end{cases}
\end{equation}
As a consequence, the equation for the nonlinear component \( w_n = v_n -F_n^\omega \) is given by 
 \begin{equation}\label{intro:eq_wn}
 \begin{cases}
 -\partial_{tt} w_n + \Delta w_n =|\nabla F_n^\omega + \nabla w_n|^2 + 2 \nabla u_{n-1} \nabla w_n + 2  \nabla P_{>N^\gamma} u_{n-1}  \cdot \nabla F_n^\omega~,\\
 w_n|_{t=0} = 0~, ~~ \partial_t w_n|_{t=0} =0~. 
 \end{cases}
 \end{equation}
 To obtain the lowest regularity \( s \), we will later choose \( \gamma=0.88 \), see \eqref{fin:eq_parameters}. 
 Therefore, the inhomogeneous term \( \nabla P_{> N^\gamma} u_{n-1} \cdot \nabla F_n^\omega \) in \eqref{intro:eq_wn} is essentially a high-high interaction. We can then hope to control \( w_n \) at a higher regularity than \( F_n^\omega \). \\
In the final paragraph of this introduction, we provide a brief outline of the argument. While the equation \eqref{intro:eq_Fn} is linear in \( F_n^\omega \), it is highly nonlinear in the random variables \( \{ g_k \colon \| k \|_2 < N/2 \} \). The resulting difficulties on the probabilistic side of the argument can be solved using the truncation method of de Bouard and Debussche \cite{BD99}. In order to prove probabilistic Strichartz estimates for \( F_n^\omega \), one needs to control the effect of the variable-coefficient term \( \nabla P_{\leq N^\gamma} u_{n-1} \nabla F_n^\omega \) on the frequency support of \( F_n^\omega \). For this, we use a result of Geba and Tataru \cite{TG05} and the re-centered Besov-type spaces from Section \ref{sec:function_spaces}. Finally, we control the nonlinear component \( w_n \). To handle the low-high interaction term \( \nabla P_1 w \cdot \nabla F_n^\omega \), we place \( w_n \) in a function space \( \YN \) that is concentrated at frequencies \( \sim N \).

\paragraph{Acknowledgements}~\\
The author thanks his doctoral advisor Terence Tao for his invaluable guidance and support. The author also thanks Rowan Killip and Monica Visan for many helpful discussions.

\section{Notation and Preliminaries}
In this section, we will provide the necessary notation and preliminaries for the rest of the paper. In Section \ref{sec:function_spaces}, we construct spaces of frequency-localized functions. In Section \ref{sec:strichartz}, we recall the Strichartz estimates for the wave equation. In particular, we describe the refinement of Klainerman and Tataru \cite{KT99}. 

\subsection{Function Spaces}\label{sec:function_spaces}
For any function \( f\in L^1(\mathbb{R}^d) \), we define its Fourier transform \( \widehat{f} \)  by
\begin{equation*}
\widehat{f}(\xi) := \frac{1}{(2\pi)^{\frac{d}{2}}} \int_{\mathbb{R}^d} \exp(-i x \cdot \xi) f(x) \dx~. 
\end{equation*}
Let \( \varphi \colon \mathbb{R}^d \rightarrow \mathbb{R} \) be a smooth, compactly supported function s.t. \( \varphi|_{B(0,1)} \equiv 1 \) and \( \varphi|_{\mathbb{R}^d\backslash B(0,2)} \equiv 0 \). We set \( \psi_1(\xi) = \varphi(\xi) \) and \( \psi_M(\xi) := \varphi(\xi/M)- \varphi(2\xi/M) \), \( M \geq 2 \). For any dyadic \( M \geq 1 \), we define the re-centered Littlewood-Paley operators by
\begin{equation*}
\widehat{P_{M;k}f}(\xi):= \psi_M(\xi-k) \widehat{f}(\xi)
\end{equation*}
The standard Littlewood-Paley projections \( P_M \) are given by \( P_{M;0} \).  We also use the fattened Littlewood-Paley projections \( \widetilde{P}_M \), which are defined using multipliers \( \widetilde{\psi}_M \) with slightly larger support. \\
The following function spaces are partly motivated by the frequency envelopes in \cite{Tao01,Tao04}. 
We first define two weight functions \( c \colon 2^\mathbb{N} \rightarrow \mathbb{R}_+ \). \\
To capture functions that are localized at frequencies \( \sim N \), we set
\begin{equation}\label{prelim:eq_cN}
c_{N,D}(M):= \max \left( \frac{N}{M}, \frac{M}{N} \right)^{D}~.
\end{equation}
In addition, to capture functions localized at frequencies \( \lesssim N \), we set 
\begin{equation}\label{prelim:eq_clN}
c_{\leq N,D}(M):= \max \left( 1,\frac{M}{N} \right)^{D}~.
\end{equation}
Next, let \( u\colon \mathbb{R}^{1+3} \rightarrow \mathbb{R} \) be a function on space-time. We define frequency localized versions of the \( L_t^\infty L_x^2 \)-norm by
\begin{align}
\| u \|_{\XNT}&:= \sum_{M\geq 1} c_{N,D}(M) \| P_M u \|_{L_t^\infty L_x^2([0,T]\times \rthree)}~,\\
\| u \|_{\XlNT}&:= \sum_{M\geq 1} c_{\leq N,D}(M) \| P_M u \|_{L_t^\infty L_x^2([0,T]\times \rthree)}~.
\end{align}
Similarly, we define frequency localized versions of the Strichartz-type \( L_t^2 L_x^\infty \)-norm by 
\begin{align}
\| u \|_{\SNT}&:= \sum_{M\geq 1} c_{N,D}(M) \| P_M u \|_{L_t^2 L_x^\infty([0,T]\times \rthree)}~,\\
\| u \|_{\SlNT}&:= \sum_{M\geq 1} c_{\leq N,D}(M) \| P_M u \|_{L_t^2 L_x^\infty([0,T]\times \rthree)}~.
\end{align}
The function spaces corresponding to the norms above are given by 
\begin{equation}\label{prelim:eq_spaces}
\begin{aligned}
\XNT &:= \{ u \in C_t^0 L_x^2([0,T]\times \rthree)\colon \| u \|_{\XNT} < \infty \}~, \\
\XlNT& := \{ u \in C_t^0 L_x^2 ([0,T]\times \rthree)\colon \| u \|_{\XlNT} < \infty \}~, \\
\SNT &:= \{ u \in L_t^2 L_x^\infty([0,T]\times \rthree)\colon \| u \|_{\SNT} < \infty \}~, \\
\SlNT &:= \{ u \in L_t^2 L_x^\infty([0,T]\times \rthree ) \colon \| u \|_{\SlNT} < \infty \}~. 
\end{aligned}
\end{equation} 
Note that the \( \XNT \) and \( \XlNT \)-spaces only contain functions in \( C_t^0 L_x^2([0,T]\times \rthree) \). We now record some basic properties of these spaces. 

\begin{lem}\label{prelim:lem_basic_properties}
The spaces \( \XNT \), \( \XlNT \), \( \SNT \), and \( \SlNT \) equipped with their corresponding norms are complete. \\
Furthermore, for each \( u \in \XNT \), the mapping
\begin{equation}\label{prelim:eq_continuity}
t \in [0,T] \mapsto \| u \|_{\XN([0,t])} 
\end{equation}
is continuous. An analogous continuity statement also holds for the other function spaces. 
\end{lem}
The relevance of \eqref{prelim:eq_continuity} stems from our use of the truncation method of \cite{BD99}, see Section \ref{sec:it}.
\begin{proof}
The completeness follows from standard arguments in graduate real analysis, and the proof is omitted. \\
It remains to show the continuity statement \eqref{prelim:eq_continuity}. Since \( w\in C_t^0 L_x^2([0,T]\times \rthree)\), each individual summand \( t \mapsto \| P_M u \|_{L_t^\infty L_x^2([0,T]\times \rthree)} \) is continuous. Since \( \| u \|_{\XN([0,t])} \) is a uniform limit of the partial sums in \( M \geq 1 \), the result follows. 
\end{proof}
Equipped with the functions spaces above, we are now ready to define the function space \( \YN \) for the solution \( w_n \) of \eqref{intro:eq_wn}. For given parameters \( \nu > 2 \), \( \sigma = \nu-1 - \), and \( \eta,D > 0 \), we set 
\begin{equation}\label{prelim:eq_YN}
\begin{aligned}
\YN([0,T]):=\{& u \colon [0,T]\times \rthree \rightarrow \mathbb{R} | ~ \bra^\nu u, \bra^{\nu-1} \partial_t u\in ( \XNeta \medcap \XlN)([0,T]), \\
& \text{and} \quad \bra^\sigma u \in (\SNeta \medcap \SlN)([0,T]) \}~. \\
\end{aligned}
\end{equation}
The corresponding norm is defined by
\begin{equation*}
\begin{aligned}
\| u \|_{\YN([0,T])} :&= \| \bra^\nu u \|_{(\XNeta \bigcap \XlN)([0,T])} +  \| \bra^{\nu-1} \partial_t u \|_{(\XNeta \bigcap \XlN)([0,T])}  \\
&+ \| \bra^\sigma u \|_{(\SNeta \bigcap \SlN)([0,T])}~. 
\end{aligned}
\end{equation*}
The main regularity parameter is \( \nu > 2 \), and it describes the number of derivatives of \( w_n \) that are controlled in the \( L_t^\infty L_x^2 \)-type norm. The value of \( \sigma \) is then determined by the deterministic Strichartz estimates. Finally, the parameters \( \eta> 0 \) and \( D >0 \) describe the localization to frequencies \( \sim N \) and \( \lesssim N \), respectively. Due to high-high to low frequency interactions in the quadratic term \( |\nabla w_n|^2 \), we have to choose \( \eta < \nu - 1 \). In contrast, there is essentially no transfer from low to high frequencies over short time intervals, and hence \( D >0 \) can be chosen arbitrarily large. \\
The (nearly) optimal choice of the parameters leads to \( \nu = 2.1001 \), see \eqref{fin:eq_parameters}. This may seem surprising, since this is an absolute amount above the critical regularity \( s_3 = 2 \). The additional regularity is used to control the effect of the variable-coefficient term \( \nabla P_{\leq N^\gamma} u_{n-1} \cdot \nabla F_n^\omega \) on the frequency support of the randomized initial data, see Proposition  \ref{lin:prop_freq_envelope}.


Recall that the atoms in the Wiener randomizaton are localized in frequency space to cubes of scale \( \sim 1 \). To take advantage of this, we introduce the following Besov-type spaces.
Let \( \gamma \in (0,1) \) and \( k \in \mathbb{Z}^3 \) with \( \| k \|_2 \sim N \). We define the weight function
\begin{equation}\label{prelim:eq_ck}
c^{\rho,\gamma}_{k,D}(M) := M^\rho \max \left( 1, \frac{M}{N^\gamma} \right)^{D}~. 
\end{equation}
Using this weight function, we set
\begin{equation*}
\| f \|_{\Bk} := \sum_{M\geq 1} c^{\rho,\gamma}_{k,D}(M) \| P_{M;k} f \|_{L_x^2(\rthree)} \quad \text{and} \quad \Bk := \{ f \in L_x^2(\rthree)\colon \| f \|_{\Bk}  < \infty \}~. 
\end{equation*}

\subsection{Strichartz Estimates}\label{sec:strichartz}

First, we state a local Strichartz estimate in the form needed for this paper. 

\begin{lem}[Strichartz Estimate]\label{prelim:lem_strichartz}
Let \( \nu> 2 \) and let \( \sigma= \nu - 1 - \delta \), where \( \delta >0 \) is small. Let \( 0 < T \leq 1 \) and let \( u \) be a solution of 
\begin{equation}\label{prelim:eq_strichartz}
\begin{cases}
-\partial_{tt} u + \Delta u = F  \qquad\qquad \text{for} ~ (t,x)\in [0,T]\times \rthree \\
u|_{t=0}=f_0 , ~ \partial_t u|_{t=0}=f_1 
\end{cases}~.
\end{equation}
Then, we have that 
\begin{align*}
&\| u \|_{C_t^0 H_x^\nu([0,T]\times \rthree)}+\| \partial_t u \|_{C_t^0 H_x^{\nu-1}([0,T]\times \rthree)} + \| \langle \nabla \rangle^{\sigma} u \|_{L_t^2 L_x^\infty([0,T]\times \rthree)} \\
&\lesssim_{\nu,\sigma} \| f_0 \|_{H_x^\nu(\rthree)} + \| f_1 \|_{H_x^{\nu-1}(\rthree)} + \| \langle \nabla \rangle^{\nu-1} F \|_{L_t^1 L_x^2([0,T]\times \rthree)}~. 
\end{align*}
In particular, \( u \in C_t^0 H_x^\nu([0,T]\times \rthree) \) and \( \partial_t u \in C_t^0 H_x^{\nu-1}([0,T]\times \rthree) \). 
\end{lem}

\begin{proof}
This lemma follows directly from the (global) Strichartz estimates in \cite{KT98}. In order to deal with the inhomogeneous norms, we also use that 
\begin{equation*}
\Big \| \frac{\sin(t|\nabla|)}{|\nabla|} \Big \|_{H_x^{\nu-1}(\rthree) \rightarrow H_x^{\nu}(\rthree)} \leq \max( 1, |t|) = 1~. 
\end{equation*}
The local estimate with an \( \epsilon\)-loss at the endpoint \( (2,\infty) \) follows from Hölder's inequality, Bernstein's estimate, and \cite[Corollary 1.3]{KT98} with \( (q,p)=(2+,\infty-)\).  
\end{proof}

In the following, we recall a refined Strichartz estimate from \cite{KT99}. This estimate has already been used in the context of the Wiener randomization in \cite{DLM17}. 
\begin{lem}[{Refined Strichartz Estimate \cite{KT99}}]\label{prelim:lem_refined_strichartz}
Assume that \( k \in \mathbb{Z}^3 \) with \( \| k \|_2 \sim N \), and let \( 1 \leq M \ll N \). Furthermore, let \( (q,p) \) be a sharp wave-admissible Strichartz pair, i.e., \( 2 \leq q,p < \infty \) and 
\begin{equation*}
\frac{1}{q}+ \frac{1}{r} = \frac{1}{2}~. 
\end{equation*}
Then, it holds for all \( T>0 \) that
\begin{equation*}
\Big \| \duh P_{M;k} F \ds \Big \|_{L_t^q L_x^p([0,T]\times \rthree)} \lesssim \Big( \frac{M}{N} \Big)^{\frac{1}{2}-\frac{1}{p}} N^{-1} N^{\frac{3}{2}-\frac{1}{q}- \frac{3}{p}} \| F \|_{L_t^1 L_x^2([0,T]\times \rthree)}~. 
\end{equation*}
\end{lem}
The refined Strichartz estimate exhibits a gain in \( M/N \). Since the projection onto small balls at a large distance from the origin essentially rules out the Knapp counterexamples, this is to be expected. 

\section{The truncated equations}\label{sec:it} 
Recall from the introduction that \( u_n \), \( F_n^\omega \), and \( w_n \) are supposed to solve \eqref{intro:eq_un}, \eqref{intro:eq_Fn}, and \eqref{intro:eq_wn}. However, we cannot directly work with the weak formulation of these equations. The problem is unrelated to any estimates in the deterministic part argument, and comes only from the moments with respect to \( \omega \in\Omega \). Due to the quadratic term \( |\nabla w |^2 \), it is not possible to apply a contraction mapping argument in \( L_\omega^r L_t^q L_x^p \)-type spaces, since there is no gain of integrability in \( \omega \). To illustrate this further, one can compare the following two model problems. If \( A(t) \) is a nonnegative continuous function satisfying \( A(0)=0 \) and \( A(t) \leq A + t^{\frac{1}{2}} A(t)^2 \) for all \( t\geq 0 \), a bootstrap argument implies that \( A(t) \leq 2A \) for all \( t\leq (4A)^{-2} \). In contrast, if \( A_t \) is a nonnegative stochastic process with continuous paths satisfying \( A_0 = 0 \) and \( \mathbb{E}[A_t]\leq A + t^{\frac{1}{2}} \mathbb{E}[A_t^2] \) for all \( t \geq 0 \), we cannot directly deduce any bound on \( A_t\).



 The same problem often occurs in the theory of stochastic partial differential equations, and can be solved using the truncation method of de Bouard and Debussche \cite{BD99}.
 Let \( \theta \colon \mathbb{R}_{\geq 0} \rightarrow [0,1] \) be a smooth function s.t. \( \theta|_{[0,1]} = 1 \) and \( \theta|_{[2,\infty)} = 0 \). To simplify the notation, we write \( M=2^m \) and \( N=2^n \). We now define the cutoff functions
 \begin{align}
 \theta_{F,w;\leq n-1}(s)
 &:= \theta\left(\sum_{m=0}^{n-1} \Big(\| \langle \nabla \rangle^{\sigma^\prime} F_{m}^\omega \|_{\SlMprime([0,s])} +\| \langle \nabla \rangle^{\sigma} w_m \|_{\SlM([0,s])}+ \| \langle \nabla \rangle^{\nu} w_m \|_{ \XlM([0,s])}\Big) \right)~, \notag \\
\theta_{F;n}(s) &:= \theta\left( \| \langle \nabla \rangle^{\sigma^\prime} F_{n}^\omega \|_{\SNprime([0,s])}\right)~, \notag\\
\theta_{w;n}(s) &:= \theta\left( \| \langle \nabla \rangle^{\sigma} w_n \|_{\SlN([0,s])}+ \| \langle \nabla \rangle^{\nu} w_n \|_{ \XlN([0,s])}\right)~. \label{it:eq_theta_wn}
\end{align}
Let \( F_{\leq n-1}^\omega := \sum_{m=0}^{n-1} F_m^\omega \) and \( w_{\leq n-1} := \sum_{m=0}^{n-1} w_m \). 
For future use, we remark that 
\begin{equation}\label{it:eq_triangle}
\begin{aligned}
\| \bra^{\sigma^\prime} F_{\leq n-1}^\omega \|_{\SlNprime}&= \| \sum_{m=0}^{n-1} \bra^{\sigma^\prime} F_m^{\omega} \|_{\SlNprime} \leq \sum_{m=0}^{n-1} \| \bra^{\sigma^\prime} F_m^\omega \|_{\SlNprime} \leq \sum_{m=0}^{n-1} \| \bra^{\sigma^\prime} F_m^\omega \|_{\SlMprime} ~,\\
\| \bra^\sigma w_{\leq n-1} \|_{\SlN} &= \| \sum_{m=0}^{n-1} \bra^{\sigma} w_m \|_{\SlN} ~ \leq \sum_{m=0}^{n-1} \| \bra^{\sigma} w_m \|_{\SlN} ~  \leq \sum_{m=0}^{n-1} \| \bra^{\sigma} w_m \|_{\SlM}~.
\end{aligned}
\end{equation}
Then, we let \( F_n^\omega \) be a solution of the truncated equation
\begin{equation}
F_n^\omega(t)= W(t) (Q_N f_0^\omega, Q_N f_1^\omega) + 2 \duh \theta_{F,w;\leq n-1}(s) P_{\leq N^\gamma} \nabla u_{n-1}(s) \nabla F_n^\omega(s) \ds~. \label{it:eq_truncated_Fn}
\end{equation}
In Section \ref{sec:lin}, it will be useful to decompose \( F_n^\omega \) into a superposition of the solutions corresponding to each individual individual pair \( (P_k f_0, P_k f_1) \). Thus, we define \( F_{n,k} \) as the solution of 
\begin{equation}
F_{n,k}(t)= W(t) (P_k  f_0, P_k f_1) + 2 \duh \theta_{F,w;\leq n-1}(s) P_{\leq N^\gamma} \nabla u_{n-1}(s) \nabla F_{n,k}(s) \ds~. \label{it:eq_truncated_Fnk}
\end{equation}
Finally, the nonlinear component \( w_{n}(t) \) is defined as the solution of 
\begin{align}
w_n(t) &= \duh  \theta_{F;n}(s) |\nabla F_n^\omega|^2\ds  
		+ 2 \duh \theta_{F;n}(s) \nabla F_n^\omega \nabla w_n \ds   \notag\\
		&+ \duh \theta_{w;n}(s) |\nabla w_n|^2\ds\label{it:eq_truncated_wn}  
		+ 2 \duh \theta_{F,w;\leq n-1}(s) \nabla u_{n-1} \nabla w_n \ds   \\
		&+ 2 \duh \theta_{F,w;\leq n-1}(s) \nabla P_{>N^\gamma} \nabla u_{n-1} \nabla F_n^\omega \ds \qquad~.   \notag
\end{align}
Once we have obtained uniform bounds on \(F_n^\omega \) and \( w_n \), we can then remove the truncation by passing to a small random time interval, see Section \ref{sec:final}.

\section{The adapted linear evolution \(F_n^\omega\)}\label{sec:lin}

In this section, we study the adapted linear evolution \( F_n^\omega \). Our main objective is to understand the frequency localization of the functions \( F_{n,k} \) and \( F_n^\omega \), which we then use to prove probabilistic Strichartz estimates. In order to avoid continually interrupting the main argument, we deal with any issues of (strong) measurability in the appendix.

\begin{lem}[Frequency localization of \( F_n^\omega \)] \label{lin:lem_frequency_Fnomega}
Let \( F_n^\omega \) be a solution of \eqref{it:eq_truncated_Fn}, let \( s \geq 1 \), and let \( D^\prime >0\). Then, we have that
\begin{equation}\label{lin:eq_frequency_Fnomega}
\| \langle \nabla \rangle^s F_n^\omega \|_{\XNprime([0,1])} + \| \langle \nabla \rangle^{s-1} \partial_t F_n^\omega \|_{\XNprime([0,1])} \lesssim \| (Q_N f_0^\omega, Q_N f_1^\omega ) \|_{H_x^s\times H_x^{s-1}} ~. 
\end{equation}
\end{lem}
\mbox{Lemma  \ref{lin:lem_frequency_Fnomega}} is a direct consequence of the work of Geba and Tataru, see \cite[Proposition 3.1]{TG05}. In the proof below, we only describe their result and translate it into our notation. The argument in \cite{TG05} is based on the energy method and microlocal analysis. We remark that their proof could be simplified in our situation, since the principal symbol of \eqref{it:eq_truncated_Fn} has constant coefficients. \\
If we were able to afford an \(\epsilon\)-loss in \( N\), Lemma \ref{lin:lem_frequency_Fnomega} would also follow from Proposition \ref{lin:prop_freq_envelope} and Khintchine's inequality. However, the \( L_t^\infty\)-norm prevents us from decoupling the \( F_{n,k} \) without losing a power of \( N \).

\begin{proof} Let \( q\colon \mathbb{R} \rightarrow [1,\infty) \). We set 
\begin{equation*}
E_q(F(t)) := \frac{1}{2} \| q(|\nabla| ) \partial_t F(t) \|_{L_x^2(\rthree)}^2 + \frac{1}{2} \| q(|\nabla|) \nabla F(t) \|_{L_x^2(\rthree)}^2~. 
\end{equation*}
For any \( \widetilde{D}> 0 \),  \cite{TG05} constructs a multiplier \( q \) satisfying the growth conditions (see \cite[(16)]{TG05}) 
\begin{equation*}
q(|\xi|)  ~~ \begin{cases}
\begin{tabular}{ll}
\( > N^{(1-\gamma)\widetilde{D}} \) & if \( |\xi|\ll N \)\\
\( = 1 \) & if \( \frac{N}{4} \leq |\xi|\leq 4N \)\\
\( > \left( \frac{|\xi|}{N^\gamma} \right)^{\widetilde{D}}\) & if \( |\xi| \gg N \)
\end{tabular}
\end{cases}~. 
\end{equation*}
and the Gronwall-bound (see \cite[(15)]{TG05})
\begin{equation*}
\frac{\mathrm{d}}{\mathrm{d}t} E_q( F_n^\omega(t)) \lesssim \| \theta_{F,w;\leq n-1}(t) \nabla P_{\leq N^\gamma} u(t) \|_{L_x^\infty} E_q(F_n^\omega(t))~. 
\end{equation*}
Since the principal symbol of \eqref{it:eq_truncated_Fn} has constant coefficients, the multiplier \( q \) in \cite{TG05} may be chosen independent of \( t \). It follows that 
\begin{align*}
\sup_{t\in [0,1]} E_q(F_n^\omega(t)) &\lesssim \exp(  C \| \theta_{F,w;\leq n-1}(t) \nabla P_{\leq N^\gamma} u(t) \|_{L_t^1 L_x^\infty([0,1]\times \rthree)}) \| (Q_N f_0^\omega, Q_N f_1^\omega) \|_{\dot{H}^1\times L^2} \\
&\lesssim  \| (Q_N f_0^\omega, Q_N f_1^\omega) \|_{\dot{H}^1\times L^2} ~. 
\end{align*}
Using the fundamental theorem of calculus, we also obtain a bound on  \( \|  q(D) F_n^\omega(t) \|_{L_x^2} \). Putting everything together, we obtain that 
\begin{alignat*}{3}
\| \langle \nabla \rangle^s P_M F_n^\omega \|_{L_t^\infty L_x^2([0,1]\times \rthree)} &\lesssim N^{-(1-\gamma) \widetilde{D} } \| (Q_N f_0^\omega, Q_N f_1^\omega)\|_{H^s\times H^{s-1}} \qquad \qquad &&\text{if}  ~ 1\leq M\ll N~, \\
\| \langle \nabla \rangle^s P_M F_n^\omega \|_{L_t^\infty L_x^2([0,1]\times \rthree)} &\lesssim \| (Q_N f_0^\omega, Q_N f_1^\omega)\|_{H^s\times H^{s-1}} \qquad \qquad &&\text{if}  ~ M \sim N~, \\
\| \langle \nabla \rangle^s P_M F_n^\omega \|_{L_t^\infty L_x^2([0,1]\times \rthree)} &\lesssim \left( \frac{M}{N} \right)^{s-1} \left( \frac{M}{N^\gamma} \right)^{-\widetilde{D}} \| (Q_N f_0^\omega, Q_N f_1^\omega)\|_{H^s\times H^{s-1}} \qquad \qquad &&\text{if}  ~ M \gg  N~, 
\end{alignat*}
The same bounds also hold for \( \bra^{s-1} \partial_t F_n^\omega \). The lemma follows by choosing \( \widetilde{D}>0 \) sufficiently large. 
\end{proof}


\begin{prop}[Frequency profile of the adapted linear evolution] \label{lin:prop_freq_envelope}
Let \( (f_0,f_1) \in H_x^1 \times L_x^2 \). Let \( k \in \mathbb{Z}^3 \) with \( \| k \|_2 \sim N \), let \( \sigma > 1 \), let \( \rho:= \sigma-1-\delta >0 \), and let \( D^{\prime\prime} >0 \) be  arbitrarily large. Assume that \( \phi \colon \mathbb{R}^{1+3} \rightarrow \mathbb{R} \) has frequency support in the ball \( \| \xi \|_2 \lesssim N^\gamma \) and satisfies \( \langle \nabla \rangle^\sigma \phi \in L_t^1 L_x^\infty(\mathbb{R}\times \rthree)\). Furthermore, let \( F_k \) be the solution of 
\begin{equation}\label{lin:eq_Fk}
-\partial_{tt} F_k + \Delta F_k = 2 ~ \nabla \phi \cdot \nabla F_k~, \qquad (F_k,\partial_t F_k )|_{t=0}= ( P_k f_0,P_k f_1) ~. 
\end{equation}
Then, we have for all \( 0< T \leq 1 \) that 
\begin{align*}
&\| \nabla F_k \|_{L_t^\infty \Bkprime ([0,T]\times \rthree)}+ \| \partial_t F_k \|_{L_t^\infty \Bkprime ([0,T]\times \rthree)}+  \| F_k \|_{L_t^\infty \Bkprime ([0,T]\times \rthree)}\\
& \lesssim_{\sigma,\rho,\gamma,D^\prime} \| (P_k f_0 , P_k f_1) \|_{H^1\times L^2} \exp( C_{\sigma,\rho,\gamma,D^\prime} \| \langle \nabla \rangle^{\sigma}  \phi \|_{L_t^1 L_x^\infty([0,T]\times \rthree)} )~. 
\end{align*}
\end{prop}
\begin{proof} 
Let \( c= c^{\rho,\gamma}_{k,D^{\prime\prime}} \) be as in \eqref{prelim:eq_ck} and let \( \nabla_{x,t} \) be the gradient with respect to both variables. Then, we have that
\begin{equation}\label{lin:eq_Fk_switch}
\| \nabla_{x,t} F_k \|_{L_t^\infty \Bkprime} = \| c(M) \nabla_{x,t} P_{M;k} F_k \|_{L_t^\infty \ell_M^1 L_x^2}\leq 
\| c(M) \nabla_{x,t} P_{M;k} F_k \|_{\ell_M^1 L_t^\infty L_x^2}~. 
\end{equation}
Thus, we have to control \( \| \nabla_{x,t} P_{M;k} F_k \|_{L_t^\infty L_x^2} \). 
From Duhamels formula, it follows that
\begin{align*}
&\| P_{M;k} \nabla_{x,t} F_k \|_{L_t^\infty L_x^2([0,T]\times \rthree)} \\
&\lesssim \| P_{M;k} \nabla_{x,t} W(t) (P_kf_0,P_k f_1) \|_{L_t^\infty L_x^2([0,T]\times \rthree)} + \| P_{M;k} \left( \nabla F_k \cdot \nabla \phi \right) \|_{L_t^1 L_x^2([0,T]\times \rthree)} \\
&\lesssim 1_{M\leq 4} \| (P_kf_0,P_k f_1) \|_{\dot{H}^1\times L^2} + \| P_{M;k} \left( \nabla F_k \cdot \nabla \phi \right) \|_{L_t^1 L_x^2([0,T]\times \rthree)} 
\end{align*}
Then, we estimate
\begin{align*}
&\| P_{M;k} \left( \nabla \phi \cdot \nabla F_k \right) \|_{L_t^1 L_x^2([0,T]\times \rthree)} \\
&\lesssim \Big \| \sum_{K\ll M} \sum_{L\sim M} \| \nabla P_L \phi \cdot \nabla P_{K;k} F_k \|_{L_x^2} + \sum_{K\sim M} \| \nabla \phi  \cdot \nabla P_{K;k} F_k \|_{L_x^2} + \sum_{K\sim L \gg M } \| \nabla P_L \phi \cdot \nabla P_{K;k} F_k \|_{L_x^2} \Big\|_{L_t^1([0,T])} \\
&\lesssim 1_{\scriptscriptstyle M \lesssim N^\gamma}  M^{1-\sigma} (\sup_{K\ll M}  c(K)^{-1}) \Big \|   \| \langle \nabla \rangle^{\sigma} \phi \|_{L_x^\infty} \| \nabla F_k \|_{\Bkprime}   \Big \|_{L_t^1([0,T])}
+ \Big \| \|\langle \nabla \rangle^{\sigma} \phi \|_{L_x^\infty} \sum_{K\sim M} \| \nabla P_{K;k} F_k \|_{L_x^2}  \Big \|_{L_t^1([0,T])}\\
&~~+ 1_{\scriptscriptstyle{ M\lesssim N^\gamma}}  \sup_{K\gg M} (K^{1-\sigma} c(K)^{-1})  \Big \| \| \langle \nabla \rangle^\sigma \phi \|_{L_x^\infty} \| \nabla F_k \|_{\Bkprime} \Big \|_{L_t^1([0,T])}
\end{align*}
By multiplying with \( c(M) \), summing in \( M \), and interchanging \( \ell_M^1 \) and \( L_t^1 \) in the second contribution, we obtain that 
\begin{align*}
&\| \nabla F_k \|_{L_t^\infty \Bkprime([0,T])}+ \| \partial_t F_k \|_{L_t^\infty \Bkprime([0,T])} \\
&\lesssim \Big( 1+ \sum_{M\lesssim N^\gamma} M^{1-\sigma} c(M) \sup_{K\lesssim M} c(K)^{-1} + \sum_{M\lesssim N^\gamma} c(M) \sup_{K\gg M} K^{1-\sigma} c(K)^{-1} \Big)   \Big \| \| \langle \nabla \rangle^\sigma \phi \|_{L_x^\infty} \| \nabla F_k \|_{\Bkprime} \Big \|_{L_t^1([0,T])} \\
&\lesssim \Big( 1+ \sum_{M\lesssim N^\gamma} M^{1-\sigma+\rho} + \sum_{M\lesssim N^\gamma} M^{1-\sigma} \Big)   \Big \| \| \langle \nabla \rangle^\sigma \phi \|_{L_x^\infty} \| \nabla F_k \|_{\Bkprime} \Big \|_{L_t^1([0,T])}\\
&\lesssim   \Big \| \| \langle \nabla \rangle^\sigma \phi \|_{L_x^\infty} \| \nabla F_k \|_{\Bkprime} \Big \|_{L_t^1([0,T])}~.
\end{align*}
The proposition then follows from Gronwall's inequality. For the inhomogeneous term, we also use the fundamental theorem of calculus.\\
We remark that the definition of \( c(M) \) for \( M \gtrsim N^\gamma \) does not enter in a significant way. The weight only needs to grow in  \( M \) and satisfy a local constancy condition.
\end{proof}
\begin{cor} Under the same conditions as in Proposition \ref{lin:prop_freq_envelope}, we have that
\begin{equation*}
\begin{aligned}
&\| \nabla F_{n,k} \|_{\ell_k^2 L_t^\infty \Bkprime} + \| \partial_t F_{n,k} \|_{\ell_k^2 L_t^\infty \Bkprime}+ \|  F_{n,k} \|_{\ell_k^2 L_t^\infty \Bkprime}\\
& \lesssim \| (\widetilde{P}_N f_0, \widetilde{P}_N f_1) \|_{H^1\times L^2}  \exp( C_{\sigma,\rho,\gamma,D} \| \langle \nabla \rangle^{\sigma}  \phi \|_{L_t^1 L_x^\infty([0,T]\times \rthree)} )~. 
\end{aligned}
\end{equation*}
\end{cor}
\begin{proof}
This follows directly from Proposition \ref{lin:prop_freq_envelope} and \begin{equation*}
\| (P_k f_0, P_k f_1) \|_{\ell_{\| k\|_2 \sim N}^2 ( \dot{H}_x^1 \times L_x^2)} \lesssim \| (\widetilde{P}_N f_0, \widetilde{P}_N f_1) \|_{\dot{H}^1\times L^2}~. 
\end{equation*}
\end{proof}

\begin{prop}[Probabilistic Strichartz Estimates]\label{lin:prop_probabilistic_strichartz}
Let \( F_n^\omega \) be a solution of \eqref{it:eq_truncated_Fn}, let \( s>1 \), \( \sigma^\prime > \sigma > 1 \), and let \( D^\prime > 0 \).  Let \( \delta > 0 \) be as in Proposition \ref{lin:prop_freq_envelope}. Furthermore, we assume that \begin{equation}\label{lin:eq_consistency_condition_strichartz}
\sigma < \frac{3}{2}~. 
\end{equation}
Then, it holds for all \( 0 < T \leq 1 \) and all \( r \geq 1 \) that
\begin{equation}\label{lin:eq_probabilistic_strichartz}
\begin{aligned}
&\| \langle\nabla\rangle^{\sigma^\prime} F_n^\omega \|_{L_\omega^r \SNprime(\Omega \times [0,T])} \\
&\lesssim \sqrt{r} T^\frac{1}{2} N^{2\delta} \| (P_N f_0, P_N f_1) \|_{H_x^{\sigma^\prime}(\rthree)\times H_x^{\sigma^\prime-1}(\rthree)} \\
&+ \sqrt{r} T^{\frac{1}{2}} N^{\sigma^\prime -s +1- \gamma (\sigma-1) - \frac{1}{2} (1-\gamma) + 2\delta }  \| (\widetilde{P}_N f_0, \widetilde{P}_N f_1) \|_{H_x^s(\rthree)\times H_x^{s-1}(\rthree)}
\end{aligned}
\end{equation}
\end{prop}

\begin{rem}
The power on \( N \) in the estimate above can be motivated by writing
\begin{equation*}
N^{\sigma^\prime -s +1- \gamma (\sigma-1) - \frac{1}{2} (1-\gamma) }= \underbrace{N^{\sigma^\prime-s}}_{\substack{\text{difference of}\\\text{derivatives}}} \cdot \underbrace{N^1}_{\substack{\text{deterministic}\\ \text{Strichartz}}} \cdot \underbrace{{N^{-\gamma(\sigma-1)}}}_{\substack{\text{gain frequency}\\ \text{localization}}} \cdot \underbrace{N^{-\frac{1}{2}(1-\gamma)}}_{\substack{\text{gain refined}\\\text{Strichartz}}}~.
\end{equation*}
The nearly optimal choice of the parameters leads to \( \sigma^\prime = 1.13205 \), see \eqref{fin:eq_parameters}. From \eqref{fin:eq_parameters}, we also have that \( {\sigma= \nu-1-=1.1001-} \), and thus the random evolution \( F_n^\omega \) has a higher number of derivatives bounded in \( L_t^2 L_x^\infty \) than \( w_n \).
\end{rem}
\begin{proof} Let \( (q,p) =(2+,\infty-) \) be a sharp wave-admissible Strichartz pair. During this proof, it is convenient to define
\begin{equation*}
\| u \|_{\SNqpprime([0,T])} := \sum_{M\geq 1} c_{N,D^\prime}(M) ~  \| P_M u \|_{L_t^q L_x^p([0,T]\times \rthree)}~. 
\end{equation*}

We separate the proof in three steps: \\

\textbf{Step 1: Estimate for the individual \( F_{n,k} \).}~ \\
Since \( F_n^\omega \) is a random linear combination of the \( F_{n,k} \), we need to control their Strichartz-type norms. To this end, we prove that if \( F_k \) is a solution of \eqref{lin:eq_Fk} and \( D^\prime > 0 \), then there exists a \(D^{\prime\prime}>0 \) s.t.
\begin{equation}\label{lin:eq_refined_strichartz}
\begin{aligned}
\| \langle \nabla \rangle^{\sigma^\prime} F_k \|_{\SNqpprime([0,T])}
 &\lesssim T^{\frac{1}{q}}  \| (P_k f_0,P_k f_1) \|_{H_x^{\sigma^\prime} \times H_x^{\sigma^\prime-1}} \\
 &+  N^{\sigma^\prime-\gamma(\sigma-1)-\frac{1}{2}(1-\gamma)+\delta }  \| \langle \nabla \rangle^\sigma \phi \|_{L_t^1 L_x^\infty( \sIT)} \| \nabla F_k\|_{L_t^\infty \Bktil ([0,T])}~.
\end{aligned}
\end{equation}
Using Hölder's inequality and Bernstein's estimate, we have that
\begin{equation*}
\| \langle \nabla \rangle^{\sigma^\prime} P_M W(t) ( P_k f_0, P_k f_1) \|_{L_t^q L_x^p}  
\lesssim T^{\frac{1}{q}} \IMN   \| (P_k f_0, P_k f_1) \|_{H_x^{\sigma^\prime}\times H_x^{\sigma^\prime-1}}~. 
\end{equation*}
This leads to the first summand on the right-hand side of \eqref{lin:eq_refined_strichartz}. Next, we control the Duhamel term. 
Using the refined Strichartz estimates (see Lemma \ref{prelim:lem_refined_strichartz}), we have that \\
\begin{align*}
&\| \langle \nabla \rangle^{\sigma^\prime} P_M \int_0^t \frac{\sin((t-s)|\nabla|)}{|\nabla|} \nabla \phi \cdot \nabla F_k \ds \|_{L_t^q L_x^p} \\
&\lesssim \sum_{1\leq L \leq N^\gamma} \sum_{1\leq K \ll N} \| \langle \nabla \rangle^{\sigma^\prime} P_M \int_0^t \frac{\sin((t-s)|\nabla|)}{|\nabla|} \nabla P_L \phi \cdot P_{K;k} \nabla F_k \ds \|_{L_t^q L_x^p} \\
&~~+ \sum_{K\sim N} \| \langle \nabla \rangle^{\sigma^\prime} P_M \int_0^t \frac{\sin((t-s)|\nabla|)}{|\nabla|} \nabla  \phi \cdot P_{K;k} \nabla F_k \ds \|_{L_t^q L_x^p}\\
&~~+ \sum_{K\gg N} \| \langle \nabla \rangle^{\sigma^\prime} P_M \int_0^t \frac{\sin((t-s)|\nabla|)}{|\nabla|} \nabla P_L \phi \cdot P_{K;k} \nabla F_k \ds \|_{L_t^q L_x^p}~. \\
&\lesssim M^{\sigma^\prime} \sum_{1\leq L \leq N^\gamma} \sum_{1\leq K \ll N} \left( \frac{\max(L,K)}{N} \right)^{\frac{1}{2}-} \| P_M ( P_L \nabla \phi \cdot P_{K;k} \nabla F_{k} )\|_{L_t^1L_x^2} \\
&~~+ M^{\sigma^\prime} \sum_{K\sim N} \| P_M( \nabla \phi \cdot \nabla F_k ) \|_{L_t^1L_x^2} \\
&~~ + M^{\sigma^\prime} \sum_{K\gg N} \| P_M(\nabla \phi \cdot \nabla F_k) \|_{L_t^1 L_x^2}\\
&\lesssim  \Bigg[ M^{\sigma^\prime} \IMN \sum_{1\leq L \leq N^\gamma} \sum_{1\leq K \ll N } \left( \frac{\max(L,K)}{N} \right)^{\frac{1}{2}-} L^{1-\sigma} K^{1-\sigma+\delta} \max\left( 1, \frac{K}{N^\gamma} \right)^{-D^{\prime\prime}}  \\
&~~+ \IMNl M^{\sigma^\prime} \sum_{K\sim N} K^{1-\sigma+\delta} \left( \frac{K}{N^\gamma} \right)^{-D^{\prime\prime}} + \IMNg M^{\sigma^\prime} \sum_{\substack{ K \gg N\\ K \sim M}} K^{1-\sigma+\delta} 
\left( \frac{K}{N^\gamma} \right)^{-D^{\prime\prime}} \Bigg]\\
&~~\cdot \| \langle \nabla \rangle^\sigma \phi \|_{L_t^1 L_x^\infty} \| F_k \|_{L_t^\infty \Bkprime} \\
&\lesssim \bigg[ \IMN N^{\sigma^\prime-\frac{1}{2} (1-\gamma) + \gamma (1-\sigma) + \delta + } +
\IMNl N^{-2D^\prime}  + \IMNg M^{\sigma^\prime+1-\sigma+\delta} \left( \frac{M}{N^\gamma} \right)^{-D^{\prime\prime}} \bigg]\\ 
&~~\cdot \| \langle \nabla \rangle^\sigma \phi \|_{L_t^1 L_x^\infty} \|  \nabla F_k \|_{L_t^\infty \Bktil} \\
&\lesssim \bigg[ \IMN N^{\sigma^\prime-\frac{1}{2} (1-\gamma) + \gamma (1-\sigma) + \delta + } +
\IMNl N^{-2D^\prime}  + \IMNg N^{-(1-\gamma)D^{\prime\prime}} \left( \frac{M}{N} \right)^{-D^{\prime\prime}+1-\sigma+\delta} \bigg]\\ 
&~~\cdot \| \langle \nabla \rangle^\sigma \phi \|_{L_t^1 L_x^\infty} \|  \nabla F_k \|_{L_t^\infty \Bktil} \\
&\lesssim  N^{\sigma^\prime-\frac{1}{2} (1-\gamma) + \gamma (1-\sigma) + \delta + } \max \left( \frac{M}{N}, \frac{N}{M} \right)^{-D^\prime-1} \| \langle \nabla \rangle^\sigma \phi \|_{L_t^1 L_x^\infty} \| \nabla F_k \|_{L_t^\infty \Bkprime} ~.
\end{align*}
In the evaluation of the sum over \( 1\leq L \leq N^\gamma, 1\leq K \ll N \), we have used the condition \( \sigma < \frac{3}{2} \). 
After multiplying with \( c_{N,D^\prime}(M) \) and summing over \( M \) this completes the proof of \eqref{lin:eq_refined_strichartz}.   
We now apply \eqref{lin:eq_refined_strichartz} to the functions \( F_{n,k} \). Due to the cutoff, \( \sigma^\prime > \sigma\), and \eqref{it:eq_triangle}, we have that
\begin{equation*}
\| \langle \nabla \rangle^{\sigma} \left( \theta_{F,w;\leq n-1}(t) u_{n-1}(t,x) \right) \|_{L_t^2 L_x^\infty(\mathbb{R}\times \rthree)} \lesssim 1~. 
\end{equation*}
Thus, it follows from \eqref{lin:eq_refined_strichartz} and  Proposition \ref{lin:prop_freq_envelope} that
\begin{equation}\label{lin:eq_refined_strichartz_Fnk}
\begin{aligned}
&\| \langle \nabla \rangle^{\sigma^\prime} F_{n,k} \|_{\SNqpprime} \\
&\lesssim  T^\frac{1}{q} \| (P_k f_0, P_k f_1) \|_{H_x^{\sigma^\prime}\times H_x^{\sigma^\prime-1}} +  T^{\frac{1}{2}}N^{\sigma^\prime-\frac{1}{2} (1-\gamma) + \gamma (1-\sigma) + \delta + } \| \nabla F_{n,k} \|_{L_t^\infty \Bktil([0,T])}  \\
&\lesssim T^\frac{1}{q} \| (P_k f_0, P_k f_1) \|_{H_x^{\sigma^\prime}\times H_x^{\sigma^\prime-1}} 
+  T^{\frac{1}{2}} N^{\sigma^\prime-\frac{1}{2} (1-\gamma) + \gamma (1-\sigma) + \delta + } \| ( P_k f_0 , P_k f_1 ) \|_{H_x^1\times L_x^2}  \\
&\lesssim T^\frac{1}{q} \| (P_k f_0, P_k f_1) \|_{H_x^{\sigma^\prime}\times H_x^{\sigma^\prime-1}} 
+  T^{\frac{1}{2}} N^{\sigma^\prime-s+1-\frac{1}{2} (1-\gamma) + \gamma (1-\sigma) + \delta + } \| ( P_k f_0 , P_k f_1 ) \|_{H_x^s\times H_x^{s-1}} 
\end{aligned}
\end{equation}
\textbf{Step 2: Probabilistic Decoupling in \( \SNqpprime\).} ~\\
In this step, we use \eqref{lin:eq_refined_strichartz_Fnk} to prove the analog of \eqref{lin:eq_probabilistic_strichartz} in \( \SNqpprime \).
We now use a standard combination of Khintchine's inequality and Minkowski's integral inequality to extend the estimate from \( F_{n,k} \) to  \( F_n^\omega \), see e.g. \cite{BOP15,LM16}. 
Since we cannot use Minkowski's integral inequality to switch \( \ell_M^1 \ell_k^2 \) into \( \ell_k^2 \ell_M^1 \), we use \eqref{lin:eq_refined_strichartz_Fnk} with a slightly larger \( D^\prime \). 

In this step, we crucially rely on the independence of the individual \( (Q_N f_0^\omega, Q_N f_1^\omega)_N \). 
Recall that the functions \( F_{n,k} \) are measurable with respect to the sigma-algebra \( \mathscr{F}_{n-1}  = \sigma( g_l \colon \| l \|_2 < N/2 ) \). In the equations below, we let \( k\in \mathbb{Z}^3 \) be in the annulus
\( N/2 \leq \| k \|_2 < N \). 
We have for all \( T>0 \) and all \( r \geq \max(q,p) \) that 
\begin{align}
&\| \langle \nabla \rangle^{\sigma^\prime} F_n^\omega \|_{L_\omega^r \SNqpprime}\notag \\
&= \mathbb{E}\Big[ \mathbb{E} \Big[ \| c_{N,D^\prime}(M) \sum_{k} g_k \langle \nabla \rangle^{\sigma^\prime} P_M F_{n,k} \|_{\ell_M^1 L_t^q  L_x^p}^r\Big| \mathscr{F}_{n-1} \Big] \Big] ^{\frac{1}{r}} \notag\\
&\leq \mathbb{E}\Big[ \Big\| \mathbb{E} \Big[ \big| c_{N,D^\prime}(M) \sum_{k} g_k \langle \nabla \rangle^{\sigma^\prime} P_M F_{n,k} \big|^r \Big| \mathscr{F}_{n-1}\Big]^{\frac{1}{r}} \Big\|_{\ell_M^1 L_t^q  L_x^p}^r \Big] ^{\frac{1}{r}} \notag\\
& \lesssim\sqrt{r}  \mathbb{E} \Big[ \| c_{N,D^\prime}(M)  \langle \nabla \rangle^{\sigma^\prime} P_M F_{n,k} \|_{ \ell_M^1 L_t^q L_x^p\ell_k^2}^r \Big] ^{\frac{1}{r}} \notag\\
& \lesssim\sqrt{r}  \mathbb{E} \Big[ \| c_{N,D^\prime+1}(M) \langle \nabla \rangle^{\sigma^\prime} P_M F_{n,k} \|_{\ell_M^2 L_t^q  L_x^p\ell_k^2}^r \Big] ^{\frac{1}{r}} \label{lin:eq_prob_strichartz_qp} \\
& \leq\sqrt{r}  \mathbb{E} \Big[ \| c_{N,D^\prime+1}(M)  \langle \nabla \rangle^{\sigma^\prime} P_M F_{n,k} \|_{\ell_k^2\ell_M^2 L_t^q  L_x^p}^r \Big] ^{\frac{1}{r}} \notag\\
& \lesssim\sqrt{r}  \mathbb{E} \Big[ \| c_{N,D^\prime+1}(M)  \langle \nabla \rangle^{\sigma^\prime} P_M F_{n,k} \|_{\ell_k^2\ell_M^1 L_t^q  L_x^p}^r \Big] ^{\frac{1}{r}}\notag \\
& \lesssim\sqrt{r}  \mathbb{E} \Big[ \|   \langle \nabla \rangle^{\sigma^\prime} P_M F_{n,k} \|_{\ell_k^2 L_t^q \SNqpplus}^r \Big] ^{\frac{1}{r}} \notag\\
&\lesssim \sqrt{r} T^\frac{1}{q} \|(\widetilde{P}_N f_0, \widetilde{P}_N f_1)\|_{H_x^{\sigma^\prime}(\rthree)\times H_x^{\sigma^\prime-1}(\rthree)} \notag\\
&~~+  \sqrt{r} T^{\frac{1}{2}} N^{\sigma^\prime-s+1-\frac{1}{2} (1-\gamma) + \gamma (1-\sigma) + \delta + } \| (\widetilde{P}_N f_0, \widetilde{P}_N f_1) \|_{H_x^s(\rthree)\times H_x^{s-1}(\rthree)} ~.\notag
\end{align}
The same estimate for smaller \( r\geq1 \) then follows by using Hölder's inequality in \( \omega \). \\

\textbf{Step 3: Moving from \((2,\infty)\) to \( (q,p) \).} ~\\
Using Bernstein's estimate, we have that
\begin{align*}
&\| \langle \nabla \rangle^{\sigma^\prime} F_n^\omega \|_{\SNprime}\\
&\leq \sum_{L\geq 1} c_{N,D^\prime}(L) \| \langle \nabla \rangle^{\sigma^\prime} P_L F_n^\omega \|_{L_t^2 L_x^\infty} \\
&\lesssim T^{\frac{1}{2}-\frac{1}{q}} \sum_{L\geq 1} L^{0+} c_{N,D^\prime}(L)  \| \langle \nabla \rangle^{\sigma^\prime} P_L F_n^\omega \|_{L_t^q L_x^p}\\
&\lesssim T^{\frac{1}{2}-\frac{1}{q}} \sup_{L\geq 1} (L^{0+} \max \left( \frac{N}{L}, \frac{L}{N}\right)^{-1} ) \| \langle \nabla \rangle^{\sigma^\prime} F_n^\omega \|_{\SNqpplus} \\
&\lesssim  T^{\frac{1}{2}-\frac{1}{q}} N^{0+} \| \langle \nabla \rangle^{\sigma^\prime} F_n^\omega \|_{\SNqpplus}~. 
\end{align*}
Then, the proposition follows from \eqref{lin:eq_prob_strichartz_qp}, where \( D^\prime \) is replaced by \( D^\prime+1\).
\end{proof}


\section{The nonlinear evolution \( w_n \)}
Recall that the nonlinear evolution \( w_n \) solves the truncated equation
\label{nl:eq_truncated_wn}
\begin{align}
w_n(t) &= \duh  \theta_{F;n}(s) |\nabla F_n^\omega|^2\ds  \notag \\
		&~~~+ 2 \duh \theta_{F;n}(s) \nabla F_n^\omega \nabla w_n \ds \notag\\
		&~~~+ \duh \theta_{w;n}(s) |\nabla w_n|^2\ds\notag\\
		&~~~+ 2 \duh \theta_{F,w;\leq n-1}(s) \nabla F_{\leq n-1}^\omega \nabla w_n \ds \label{nl:eq_truncated_wn}\\
		&~~~+ 2 \duh \theta_{F,w;\leq n-1}(s) \nabla w_{\leq n-1} \nabla w_n \ds\notag \\
		&~~~+ 2 \duh \theta_{F,w;\leq n-1}(s) \nabla P_{>N^\gamma} \nabla u_{n-1}\nabla F_n^\omega \ds~\notag \\
\end{align}
The main result of this section provides control of the nonlinear component \( w_n \) in \( \YN \). 

\begin{prop}[Control of the nonlinear component \( w_n \)]\label{nl:prop_wn}
Assume that \( \nu> 2 \), \( \sigma=\nu-1- \), \( s> 1 \), \( \sigma^\prime > 1 \), and \( \max( \nu-\sigma^\prime, \sigma-1 ) < \eta < \nu - 1 \). Let \( D\geq D_0(s,\nu,\sigma^\prime,\sigma,\eta) \) and \( D^\prime\geq D_0^\prime(s,\nu,\sigma^\prime,\sigma,\eta,D) \) be sufficiently large. Furthermore, assume that \( 0<T_0 = T_0(s,\nu,\sigma^\prime,\sigma,\eta,D,D^\prime) \) is sufficiently small. \\
Then there exists a unique solution \( w_n \in \YN([0,T_0]) \) of  \eqref{nl:eq_truncated_wn}. Furthermore, we have for all \( 0 \leq T \leq T_0 \) that
\begin{equation}\label{nl:eq_a_priori}
\begin{aligned}
\| w_n \|_{\YN([0,T])}
&\lesssim T^{\frac{1}{2}} \left( N^{\nu-s-\gamma(\sigma^\prime-1)}+ N^{(1-\gamma)(\nu-1)+1-\sigma^\prime} \right) \left( \| \langle \nabla \rangle^s F_n^\omega \|_{\XNprime}+ \| \langle \nabla \rangle^{\sigma^\prime} F_n^\omega \|_{\SNprime}\right)~.
\end{aligned}
\end{equation}
\end{prop}

\subsection{Bilinear Estimates}
In this section we prove the main bilinear estimates for the Duhamel terms in \eqref{nl:eq_truncated_wn}. In order to group similar estimates together, we work with a paraproduct decomposition. We define
\begin{align*}
\Plh(v,w) &:= \sum_{L,K\colon L \ll K} \duh \nabla P_L v \cdot \nabla P_K w ~\ds ~,\\
\Phl(v,w) &:=\sum_{L,K\colon L \gg K } \duh \nabla P_L v \cdot \nabla P_K w ~\ds ~, \\
\Phh(v,w) &:= \sum_{L, K \colon L \sim K} \duh \nabla P_L \cdot \nabla P_K w ~\ds ~.  
\end{align*}
Our motivation for distinguishing between low-high and high-low interactions stems from the terms in \eqref{nl:eq_truncated_wn}. Whereas the first factor is often localized at frequencies \( \lesssim N\), the second factor is always localized at frequencies \( \sim N \).\\
We now summarize the necessary estimates for the proof of Proposition \ref{nl:prop_wn}. The functions \( F,G \) below correspond to either \( F_{\leq n-1}^\omega \) or \( F_n^\omega\), and the functions \( v,w \) below correspond to either \( w_{\leq n-1} \) or \( w_n \). To simplify the notation, recall from Section \ref{sec:function_spaces} that
\begin{equation*}
\begin{aligned}
\| u \|_{\YN} = \| u \|_{\YN([0,T])}&:= \| \langle \nabla \rangle^\nu u \|_{(\XNeta \bigcap \XlN)([0,T])} + \| \langle \nabla \rangle^{\nu-1} \partial_t u \|_{(\XNeta \bigcap \XlN)([0,T])} \\
							 &~+\| \langle \nabla \rangle^{\sigma} u \|_{(\SNeta \bigcap \SlN)([0,T])}~. 
\end{aligned}
\end{equation*}

\begin{lem}[Low-High Interactions]\label{nl:lem_low_high}
Assume that \( \nu> 2 \), \( \sigma=\nu-1- \), \( s> 1 \), \( \sigma^\prime > 1 \), and \( \eta >0  \). Let \( D\geq D_0(s,\nu,\sigma^\prime,\sigma,\eta) \) and \( D^\prime\geq D_0^\prime(s,\nu,\sigma^\prime,\sigma,\eta,D) \) be sufficiently large. Then, we have for any \( 0 < T \leq 1 \) that  
\begin{align}
\|  \Plh(G,F) \|_{\YN}  &\lesssim T^{\frac{1}{2}} N^{\nu-s+1-\sigma^\prime} \| \langle \nabla \rangle^{\sigma^\prime} G \|_{\SNprime} \| \langle \nabla \rangle^s F \|_{\XNprime}~, \label{nl:eq_low_high_FF}\\
\|  \Plh(P_{>N^\gamma} G,F) \|_{\YN} &\lesssim T^{\frac{1}{2}} N^{\nu-s+\gamma(1-\sigma^\prime)} \| \langle \nabla \rangle^{\sigma^\prime} G \|_{\SlNprime} \| \langle \nabla \rangle^s F \|_{\XNprime}~, \label{nl:eq_low_high_PFF}~,\\
\| \Plh(P_{>N^\gamma} v , F) \|_{\YN} & \lesssim T^{\frac{1}{2}} N^{(1-\gamma)(\nu-1)+1-\sigma^\prime} \| \langle \nabla \rangle^{\nu} v \|_{\XlN} \| \langle \nabla \rangle^{\sigma^\prime} F \|_{\SNprime} \label{nl:eq_low_high_PwF}~, \\
\| \Plh(G,w) \|_{\YN} &\lesssim  T^{\frac{1}{2}} \| \langle \nabla \rangle^{\sigma^\prime} G \|_{\SlNprime} \| \langle \nabla \rangle^{\nu} w \|_{\XNeta \bigcap \XlN} \label{nl:eq_low_high_Fw}\\
\| \Plh(v,w) \|_{\YN} &\lesssim T^{\frac{1}{2}} \| \langle \nabla \rangle^{\sigma} v\|_{\SlN} \| \langle \nabla \rangle^{\nu} v \|_{\XNeta \bigcap \XlN}~. \label{nl:eq_low_high_ww}
\end{align}
\end{lem}

\begin{lem}[High-Low Interactions]\label{nl:lem_high_low}
Assume that \( \nu> 2 \), \( \sigma=\nu-1- \), \( s> 1 \), \( \sigma^\prime > 1 \), and \( \max( \nu-\sigma^\prime, \sigma-1 ) < \eta < \nu - 1 \). Let \( D\geq D_0(s,\nu,\sigma^\prime,\sigma,\eta) \) and \( D^\prime\geq D_0^\prime(s,\nu,\sigma^\prime,\sigma,\eta,D) \) be sufficiently large. Then, we have for any \( 0 < T \leq 1 \) that  
\begin{align}
\| \Phl(G,F) \|_{\YN} &\lt N^{\nu-s+1-\sigma^\prime} \|\langle \nabla \rangle^{\sigma^\prime} G \|_{\SlNprime} \| \langle \nabla \rangle^s F\|_{\XNprime}~\label{nl:eq_high_low_FF}~, \\
\| \Phl(v,F) \|_{\YN} &\lt N^{1-\sigma^\prime} \| \langle \nabla \rangle^{\nu} v \|_{\XlN} \| \langle \nabla \rangle^{\sigma^\prime} F \|_{\SNprime} \label{nl:eq_high_low_wF}~,\\
\| \Phl(G,w) \|_{\YN} &\lt  \| \langle \nabla \rangle^{\sigma^\prime} G \|_{\SlNprime} \| \langle \nabla \rangle^\nu w \|_{\XNeta}~, \label{nl:eq_high_low_Fw}\\
\| \Phl(v,w) \|_{\YN} &\lt \| \langle \nabla \rangle^{\nu}  v \|_{\XlN} \| \langle \nabla \rangle^{\sigma} w \|_{\SNeta}~. \label{nl:eq_high_low_ww}
\end{align}
\end{lem}

\begin{lem}[High-High Interactions]\label{nl:lem_high_high}
Assume that \( \nu> 2 \), \( \sigma=\nu-1- \), \( s> 1 \), \( \sigma^\prime > 1 \), and \( 0 < \eta < \nu - 1 \).  Let \( D\geq D_0(s,\nu,\sigma^\prime,\sigma,\eta) \) and \( D^\prime\geq D_0^\prime(s,\nu,\sigma^\prime,\sigma,\eta,D) \) be sufficiently large. Then, we have for any \( 0 < T \leq 1 \) that  
\begin{align}
\| \Phh(G,F) \|_{\YN} &\lt N^{\nu-s+1-\sigma^\prime} \| \langle \nabla \rangle^{\sigma^\prime} G \|_{\SlN} \| \langle \nabla \rangle^s F \|_{\XNprime}~, \label{nl:eq_high_high_FF}\\
\| \Phh(v,F) \|_{\YN} &\lt N^{1-\sigma^\prime} \| \langle \nabla \rangle^{\nu} v \|_{\XlN} \| \langle \nabla \rangle^{\sigma^\prime} F \|_{\SNprime} ~, \label{nl:eq_high_high_wF} \\
\| \Phh(G,w) \|_{\YN} &\lt \| \langle \nabla \rangle^{\sigma^\prime} G \|_{\SlNprime} \| \langle \nabla\rangle^{\nu} w \|_{\XNeta}~, \label{nl:eq_high_high_Fw} \\
\| \Phh(v,w) \|_{\YN} &\lt \| \langle \nabla \rangle^{\sigma} v \|_{\SlN} \| \langle \nabla \rangle^{\nu} w \|_{\XNeta \bigcap \XlN} \label{nl:eq_high_high_ww}~.  
\end{align}
\end{lem}
Before we begin with the proof of the inequalities \eqref{nl:eq_low_high_FF}-\eqref{nl:eq_high_high_ww}, we use them to control the contribution of \( \nabla P_{>N^\gamma} u_{n-1} \cdot \nabla F_n^\omega \). Under certain conditions on the parameters, this term will be smoother than the adapted linear evolution \( F_n^\omega \). This shows that we removed the unfavorable low-high interaction described in the introduction. Since the low-high interaction is the principal obstacle in the control of the nonlinear component \( w_n \), this is the main step in the proof of Proposition \ref{nl:prop_wn}. 
\begin{cor}[Control of \( \nabla P_{>N^\gamma} u_{n-1}\cdot \nabla F_n^\omega \)]\label{nl:cor_inhomogeneous_contribution}
Assume that \( \nu> 2 \), \( \sigma=\nu-1- \), \( s> 1 \), \( \sigma^\prime > 1 \), and \( \max( \nu-\sigma^\prime, \sigma-1 ) < \eta < \nu - 1 \). Let \( D\geq D_0(s,\nu,\sigma^\prime,\sigma,\eta) \) and \( D^\prime\geq D_0^\prime(s,\nu,\sigma^\prime,\sigma,\eta,D) \) be sufficiently large. Then, we have for any \( 0 < T \leq 1 \) that  

\begin{equation}\label{nl:eq_inhomogeneous_contribution}
\begin{aligned}
&\Big\| \duh \theta_{F,w;\leq n-1}(s) \nabla P_{>N^\gamma} u_{n-1} \nabla F_n^\omega \ds \Big\|_{\YN} \\
&\lt \left( N^{\nu-s+\gamma(1-\sigma^\prime)} + N^{(1-\gamma) (\nu-1)+1-\sigma^\prime} \right) \left( \| \bra^s F_n^\omega \|_{\XNprime} + \| \bra^{\sigma^\prime} F_n^\omega \|_{\SNprime} \right)~. 
\end{aligned}
\end{equation}
\end{cor}
\begin{proof}
We split \( u_{n-1} =  F_{\leq n-1}^\omega + w_{\leq n-1} \). \\
Using \eqref{nl:eq_low_high_PFF}, \eqref{nl:eq_high_low_FF}, and \eqref{nl:eq_high_high_FF}, we have that 
\begin{align*}
&\Big \| \duh \theta_{F,w;\leq n-1}(s) \nabla P_{>N^\gamma} F_{\leq n-1}^\omega \nabla F_n^\omega ~ \ds \Big \|_{\YN} \\
&\lt N^{\nu-s+\gamma(1-\sigma^\prime)} \| \theta_{F,w;\leq n-1} \langle \nabla \rangle^{\sigma^\prime} F_{\leq n-1}^\omega \|_{\SlNprime} \| \bra^s F_n^\omega \|_{\XNprime} \\
&\lt N^{\nu-s+\gamma(1-\sigma^\prime)}\| \bra^s F_n^\omega \|_{\XNprime} ~. 
\end{align*}
Using \eqref{nl:eq_low_high_PwF}, \eqref{nl:eq_high_low_wF}, and \eqref{nl:eq_high_high_wF}, we have that 
\begin{align*}
&\Big \| \duh \theta_{F,w;\leq n-1}(s) \nabla P_{>N^\gamma} w_{\leq n-1} \nabla F_n^\omega ~ \ds \Big \|_{\YN} \\
&\lt N^{(1-\gamma)(\nu-1)+1-\sigma^\prime} \| \theta_{F,w;\leq n-1} \langle \nabla \rangle^{\nu} w_{\leq n-1} \|_{\XlN} \| \bra^{\sigma^\prime} F_n^\omega \|_{\SNprime} \\
&\lt N^{(1-\gamma)(\nu-1)+1-\sigma^\prime} \| \bra^{\sigma^\prime} F_n^\omega \|_{\SNprime} ~. 
\end{align*}
\end{proof}
\begin{rem}
Because of the importance of the term \( \nabla P_{>N^\gamma} u_{n-1} \cdot \nabla F_n^\omega \), we  informally justify \eqref{nl:eq_inhomogeneous_contribution} and describe the motivation behind the estimate. \\
The first power of \( N \) comes from the contribution of \( \nabla P_{N^\gamma} F_{\leq n-1}^\omega \cdot \nabla F_{n}^\omega \). It is bounded by
\begin{equation*}
\| \langle  \nabla \rangle^{\nu} \duh \nabla P_{N^\gamma} F_{\leq n-1}^\omega \cdot \nabla F_n^\omega \|_{L_t^\infty L_x^2} \lesssim
T^{\frac{1}{2}} N^{\nu-1} \| \nabla P_{N^\gamma} F_{\leq n-1}^\omega \|_{L_t^2L_x^\infty}  \| \nabla F_n^\omega \|_{L_t^\infty L_x^2}~. 
\end{equation*}
Thus, the resulting power is 
\begin{equation}\label{lin:cond_rough_rough}
N^{\nu-s-\gamma(\sigma^\prime-1)} = \underbrace{N^{\nu-1}}_{\text{derivatives}}  \cdot \underbrace{N^{-\gamma(\sigma^\prime-1)}}_{\text{derivatives on }F_{\leq n-1}^\omega} ~\cdot \underbrace{N^{1-s}}_{\text{derivatives on }F_n^\omega}.
\end{equation}
The second power of \( N \) comes from the contribution  \( \nabla P_{N^\gamma} w_{\leq n-1} \cdot \nabla F_n^\omega \). It is bounded by 
\begin{equation*}
\| \langle  \nabla \rangle^{\nu} \duh \nabla P_{N^\gamma} w_{\leq n-1} \cdot \nabla F_n^\omega \|_{L_t^\infty L_x^2}\lesssim T^{\frac{1}{2}} N^{\nu-1} \| P_{N^\gamma} \nabla w_{\leq n-1} \|_{L_t^\infty L_x^2}   \| \nabla F_n^\omega \|_{L_t^2 L_x^\infty}
\end{equation*}
Thus, the resulting power is
\begin{equation}\label{lin:cond_smooth_rough}
N^{(1-\gamma) (\nu-1)+1-\sigma^\prime}= \underbrace{N^{\nu-1}}_{\text{derivatives}}  \cdot \underbrace{N^{-\gamma(\nu-1)}}_{\text{derivatives on }w_{\leq n-1}}\cdot \underbrace{N^{1-s}}_{\text{derivatives on }F_n^\omega}~.
\end{equation}
This estimate may seem counterintuitive, since the term with the higher frequency is placed in \( L_t^2 L_x^\infty \). However, this our only option to capitalize on the randomness, which enters through the probabilistic Strichartz estimate \eqref{lin:eq_probabilistic_strichartz}. In fact, switching the roles of \( w_{\leq n-1} \) and \( F_n^\omega \) above would not allow us to go below the deterministic restriction \( s>2 \). \\
\end{rem}

\begin{proof}[Proof of Lemma \ref{nl:lem_low_high}]~\\
First, we prove the estimates \eqref{nl:eq_low_high_FF} and \eqref{nl:eq_low_high_PFF}. Let \( H \in \{ P_{>N^\gamma} G, G\} \). Then, we for all \( M \geq 1 \) that
\begin{align*}
&\| \bra^\nu P_M \Plh(H,F) \|_{L_t^\infty L_x^2} + \| \bra^{\nu-1} \partial_t P_M \Plh(H,F) \|_{L_t^\infty L_x^2}+ \| \bra^\sigma P_M \Plh(H,F) \|_{L_t^2L_x^\infty} \\
&\lesssim M^{\nu-1} \sum_{1\leq L \ll M} \sum_{K\sim M} \| \nabla P_L H \cdot \nabla P_K F \|_{L_t^1 L_x^2} \\
&\lt M^{\nu-s} \sum_{1\leq L \ll M} L^{1-\sigma^\prime} \| \langle \nabla \rangle^{\sigma^\prime} P_L H \|_{L_t^2 L_x^\infty} \sum_{K\sim M} \| \bra^s P_K F \|_{L_t^\infty L_x^2} ~.
\end{align*}
After multiplying by \( c_{N,D}( M) \) and summing in \( M\), we obtain for all \( D^\prime > 2D \) and \( D>\eta \) that 
\begin{align*}
& \|\Plh(H,F) \|_{\YN} \\
&\lt \sum_{M\geq 1} \sum_{1\leq L \ll M} M^{\nu-s} L^{1-\sigma^\prime} \max \left( \frac{N}{M} , \frac{M}{N} \right)^{-D}  \| \bra^{\sigma^\prime} P_L H \|_{L_t^2 L_x^\infty} \| \bra^s F \|_{\XNprime}~.
\end{align*}

We now distinguish the two different possibilities for \( H \). If \( H= G \), then 
\begin{align*}
&\sum_{M\geq 1} \sum_{1\leq L \ll M} M^{\nu-s} L^{1-\sigma^\prime} \max \left( \frac{N}{M} , \frac{M}{N} \right)^{-D}  \| \bra^{\sigma^\prime} P_L H \|_{L_t^2 L_x^\infty}\\
&\lesssim \sum_{M\geq 1} \sum_{1\leq L \ll M} M^{\nu-s} L^{1-\sigma^\prime} \max \left( \frac{N}{M} , \frac{M}{N} \right)^{-D}  \max \left( \frac{N}{L} , \frac{L}{N} \right)^{-D^\prime} \| \bra^{\sigma^\prime} G \|_{\SNprime} \\
&\lesssim N^{\nu-s+1-\sigma^\prime}  \| \bra^{\sigma^\prime} G \|_{\SNprime} ~. 
\end{align*}
If \( H= P_{>N^\gamma} G \), then 
\begin{align*}
&\sum_{M\geq 1} \sum_{1\leq L \ll M} M^{\nu-s} L^{1-\sigma^\prime} \max \left( \frac{N}{M} , \frac{M}{N} \right)^{-D}  \| \bra^{\sigma^\prime} P_L H \|_{L_t^2 L_x^\infty}\\
&\lesssim \sum_{M\geq N^\gamma} \sum_{N^\gamma\leq L \ll M} M^{\nu-s} L^{1-\sigma^\prime} \max \left( \frac{N}{M} , \frac{M}{N} \right)^{-D}  ~ \| \bra^{\sigma^\prime}  G \|_{\SlNprime}\\
&\lesssim N^{\nu-s+\gamma(1-\sigma^\prime)} \| \bra^{\sigma^\prime}  G \|_{\SlNprime}~. 
\end{align*}
Second, we prove \eqref{nl:eq_low_high_PwF}. For any \( M \geq 1 \), we have that 
\begin{align*}
&~\| \bra^\nu P_M \Plh(P_{>N^\gamma }v,F) \|_{L_t^\infty L_x^2} +\| \bra^{\nu-1} \partial_t P_M \Plh(P_{>N^\gamma }v,F) \|_{L_t^\infty L_x^2} \\
&+ \| \bra^\sigma P_M \Plh(P_{>N^\gamma} v,F) \|_{L_t^2L_x^\infty} \\
&\lesssim M^{\nu-1} \sum_{N^\gamma\leq L \ll M} \sum_{K\sim M} \| \nabla P_L v \cdot \nabla P_K F \|_{L_t^1 L_x^2}\\
&\lt M^{\nu-\sigma^\prime} \sum_{N^\gamma \leq L \ll M} L^{1-\nu} \| \bra^\nu P_L v \|_{L_t^\infty L_x^2} ~ \sum_{K\sim M} \| \bra^{\sigma^\prime} P_K F \|_{L_t^2 L_x^\infty}~.
\end{align*}
After multiplying with \( c_{N,D}(M) \) and summing in \( M \), we obtain for all \( D^\prime> 2D \) and \( D>\eta \) that 
\begin{align*}
&\|  \Plh(P_{>N^\gamma }v,F) \|_{\YN} \\
&\lt \sum_{M\geq 1} \sum_{N^\gamma \leq L \ll M} M^{\nu-\sigma^\prime} L^{1-\nu} \max \left( \frac{N}{M} , \frac{M}{N} \right)^{-D} \| \bra^\nu v \|_{\XlN} \| \bra^{\sigma^\prime} F \|_{\XNprime} \\
&\lt N^{(1-\gamma)(\nu-1)+1-\sigma^\prime}\| \bra^\nu v \|_{\XlN} \| \bra^{\sigma^\prime} F \|_{\XNprime}~.
\end{align*}
Third, we prove \eqref{nl:eq_low_high_Fw} and \eqref{nl:eq_low_high_ww}. For any \( M \geq 1 \), we have that 
\begin{align*}
&\| \bra^\nu  \Plh(G,w) \|_{L_t^\infty L_x^2} +\| \bra^{\nu-1} \partial_t  \Plh(G,w) \|_{L_t^\infty L_x^2}+ \| \bra^\sigma \Plh(G,w) \|_{L_t^2 L_x^\infty} \\
&\lesssim M^{\nu-1} \sum_{1\leq L \ll M} \sum_{K\sim M} \| \nabla P_L G \cdot \nabla  P_K w \|_{L_t^1L_x^2}\\
&\lt  \sum_{L\geq 1} L^{1-\sigma^\prime} \| \bra^{\sigma^\prime} P_L G \|_{L_t^2 L_x^\infty} \sum_{K\sim M} \| \bra^\nu P_K w \|_{L_t^\infty L_x^2} \\
&\lt \| \bra^{\sigma^\prime} G \|_{\SNprime}  \sum_{K\sim M} \| \bra^\nu P_K w \|_{L_t^\infty L_x^2} ~. 
\end{align*}
After multiplying by \( c_{N,\eta}(M)+c_{\leq N,D}(M) \) and summing in \( M \geq 1 \), we obtain \eqref{nl:eq_low_high_Fw}. The estimate \eqref{nl:eq_low_high_ww} follows from exactly the same argument. \\
This finishes the proof of the low-high bilinear estimates.

\end{proof}

\begin{proof}[Proof of Lemma \ref{nl:lem_high_low}]~\\
First, we prove \eqref{nl:eq_high_low_FF}. For any \( M \geq 1 \), we have for all sufficiently large \( D^\prime >0 \) that
\begin{align*}
&\| \langle \nabla \rangle^{\nu} P_M \Phl(G,F) \|_{L_t^\infty L_x^2}+\| \langle \nabla \rangle^{\nu-1} \partial_t P_M \Phl(G,F) \|_{L_t^\infty L_x^2}  + \| \langle \nabla \rangle^{\sigma} P_M \Phl(G,F) \|_{L_t^2 L_x^\infty}\\
&\lesssim M^{\nu-1} \sum_{L\sim M} \sum_{K\ll M} \| \nabla P_L G \cdot \nabla P_K F \|_{L_t^1 L_x^2} \\
&\lt M^{\nu-\sigma^\prime} \max \left( 1, \frac{M}{N} \right)^{-D^\prime} \sum_{K \ll M} K^{1-s} \max \left( \frac{N}{K}, \frac{K}{N} \right)^{-D^\prime} \| \bra^{\sigma^\prime} G \|_{\SlNprime} \| \bra^s F \|_{\XNprime} \\
&\lt M^{\nu-\sigma^\prime} \Big( \Is{M\lesssim N} M^{1-s} \left( \frac{N}{M} \right)^{-D^\prime} + \Is{M\gg N} N^{1-s} \left( \frac{M}{N} \right)^{-D^\prime} \Big)\| \bra^{\sigma^\prime} G \|_{\SlNprime} \| \bra^s F \|_{\XNprime}\\
&\lt N^{\nu-s+1-\sigma^\prime} \max \left( \frac{M}{N}, \frac{N}{M} \right)^{-2D} \| \bra^{\sigma^\prime} G \|_{\SlNprime} \| \bra^s F \|_{\XNprime}~. 
\end{align*}
After multiplying by \( c_{N,D}(M) \) and summing in \( M \geq 1 \), this yields an acceptable contribution. \\
Second, we prove \eqref{nl:eq_high_low_wF}. For any \( M \geq 1 \), we have that
\begin{align*}
&\| \langle \nabla \rangle^{\nu} P_M \Phl(v,F) \|_{L_t^\infty L_x^2}+\| \langle \nabla \rangle^{\nu-1}  P_M \partial_t \Phl(v,F) \|_{L_t^\infty L_x^2}  + \| \langle \nabla \rangle^{\sigma} P_M \Phl(v,F) \|_{L_t^2 L_x^\infty}\\
&\lesssim M^{\nu-1} \sum_{L\sim M} \sum_{K\ll M} \| \nabla P_L v \cdot \nabla P_K F \|_{L_t^1 L_x^2} \\
&\lt  \Big( \sum_{K\ll M} K^{1-\sigma^\prime} \max \left( \frac{N}{K} , \frac{K}{N} \right)^{-D^\prime} \Big) \Big(\sum_{L\sim M} \| \bra^\nu P_L v \|_{L_t^\infty L_x^2} \Big) \| \bra^{\sigma^\prime} F \|_{\SNprime} \\
&\lt \Big( \Is{M\lesssim N} M^{1-\sigma^\prime} \Big( \frac{N}{M} \Big)^{-D^\prime} + \Is{M\gg N} N^{1-\sigma^\prime} \Big)\Big(\sum_{L\sim M} \| \bra^\nu P_L v \|_{L_t^\infty L_x^2} \Big) \| \bra^{\sigma^\prime} F \|_{\SNprime} ~. 
\end{align*}
After multiplying with \( c_{N,D}(M) \) and summing in \( M \geq 1 \), it follows that 
\begin{align*}
&\|  \Phl(v,F) \|_{\YN} \\
&\lt \Big( \sum_{1\leq M\lesssim N} M^{1-\sigma^\prime} \Big( \frac{N}{M} \Big)^{D-D^\prime} \Big)  \| \bra^\nu  v \|_{\XlN} \| \bra^{\sigma^\prime} F \|_{\SNprime}\\
 &+ T^{\frac{1}{2}}N^{1-\sigma^\prime} \sum_{M\gg N} \sum_{L\sim M} c_{N,D}(M)  \| \bra^\nu P_L v \|_{L_t^\infty L_x^2}  \| \bra^{\sigma^\prime} F \|_{\SNprime} \\
 &\lt  N^{1-\sigma^\prime} \| \bra^\nu  v \|_{\XlN} \| \bra^{\sigma^\prime} F \|_{\SNprime}
\end{align*}
Third, we prove \eqref{nl:eq_high_low_Fw}. For any \( M \geq 1 \), it follows from \( \eta < \nu-1 \) that
\begin{align*}
&\| \langle \nabla \rangle^{\nu} P_M \Phl(G,w) \|_{L_t^\infty L_x^2} +\| \langle \nabla \rangle^{\nu-1}  P_M \partial_t \Phl(G,w) \|_{L_t^\infty L_x^2} + \| \langle \nabla \rangle^{\sigma} P_M \Phl(G,w) \|_{L_t^2 L_x^\infty}\\
&\lesssim M^{\nu-1} \sum_{L\sim M} \sum_{K\ll M} \| \nabla P_L G \cdot \nabla P_K w \|_{L_t^1 L_x^2} \\
&\lt M^{\nu-\sigma^\prime} \max\left( 1, \frac{M}{N} \right)^{-D^\prime}  \Big( \sum_{K\ll M} K^{1-\nu} \max\left( \frac{N}{K}, \frac{K}{N} \right)^{-\eta} \Big) \| \bra^{\sigma^\prime} G \|_{\SlNprime} \| \bra^\nu w \|_{\XNeta} \\
&\lt M^{\nu-\sigma^\prime} \max\left( 1, \frac{M}{N} \right)^{-D^\prime} N^{-\eta} \| \bra^{\sigma^\prime} G \|_{\SlNprime} \| \bra^\nu w \|_{\XNeta}~. 
\end{align*}
After multiplying with \( c_{N,\eta}(M) + c_{\leq N,\eta}(M) \) and summing in \( M \geq 1 \), the total contribution is bounded by 
\begin{align*}
&T^{\frac{1}{2}} \Big( \sum_{1\leq M\lesssim N} M^{\nu-\sigma^\prime-\eta} + \sum_{M\gg N} M^{\nu-\sigma^\prime} N^{-\eta} \Big( \frac{M}{N} \Big)^{D-D^\prime} \Big) \| \bra^{\sigma^\prime} G \|_{\SlNprime} \| \bra^\nu w \|_{\XNeta} \\
&\lt  \| \bra^{\sigma^\prime} G \|_{\SlNprime} \| \bra^\nu w \|_{\XNeta}~,
\end{align*}
where we used that \( \eta > \nu - \sigma^\prime \). \\
Finally, we prove \eqref{nl:eq_high_low_ww}, where we argue as in the proof of \eqref{nl:eq_high_low_wF}. For any \( M \geq 1 \), it holds that
\begin{align*}
&\| \langle \nabla \rangle^{\nu} P_M \Phl(v,w) \|_{L_t^\infty L_x^2}+\| \langle \nabla \rangle^{\nu-1} P_M \partial_t\Phl(v,w) \|_{L_t^\infty L_x^2} + \| \langle \nabla \rangle^{\sigma} P_M \Phl(v,w) \|_{L_t^2 L_x^\infty}\\
&\lt  \Big( \sum_{K\ll M} K^{1-\sigma} \max \left( \frac{N}{K} , \frac{K}{N} \right)^{-\eta} \Big) \Big(\sum_{L\sim M} \| \bra^\nu P_L v \|_{L_t^\infty L_x^2} \Big) \| \bra^{\sigma^\prime} w \|_{\SNeta} \\
&\lt \Big( \Is{M\lesssim N} M^{1-\sigma} \Big( \frac{N}{M} \Big)^{-\eta} + \Is{M\gg N} N^{1-\sigma} \Big)\Big(\sum_{L\sim M} \| \bra^\nu P_L v \|_{L_t^\infty L_x^2} \Big) \| \bra^{\sigma} w \|_{\SNeta} ~. 
\end{align*}
In the last line, we used that \( \eta > \sigma-1 \). After multiplying with \( c_{N,\eta}(M)+c_{\leq N,D}(M) \), the total contribution is bounded by
\begin{align*}
&\|  \Phl(v,w) \|_{\YN} \\
&\lt \Big( \sum_{1\leq M\lesssim N} M^{1-\sigma}  \Big)  \| \bra^\nu  v \|_{\XlN} \| \bra^{\sigma^\prime} w \|_{\SNeta}\\
 &+ T^{\frac{1}{2}}N^{1-\sigma} \sum_{M\gg N} \sum_{L\sim M} c_{N,D}(M)  \| \bra^\nu P_L v \|_{L_t^\infty L_x^2}  \| \bra^{\sigma^\prime} w \|_{\SNeta} \\
 & \lt \| \bra^\nu  v \|_{\XlN} \| \bra^{\sigma^\prime} w \|_{\SNeta}~. 
 \end{align*}
 This finishes the proof of the high-low bilinear estimates. 
\end{proof}

\begin{proof}[Proof of Lemma \ref{nl:lem_high_high}]
We begin with the proof of \eqref{nl:eq_high_high_FF}. For any \( M \geq 1 \), we have that 
\begin{align*}
&\| \langle \nabla \rangle^{\nu} P_M \Phh(G,F) \|_{L_t^\infty L_x^2}+\| \langle \nabla \rangle^{\nu-1} \partial_t P_M \Phh(G,F) \|_{L_t^\infty L_x^2}  + \| \langle \nabla \rangle^{\sigma} P_M \Phh(G,F) \|_{L_t^2 L_x^\infty}\\
&\lesssim M^{\nu-1} \sum_{L\sim K \gg M} \| \nabla P_L G\cdot \nabla P_K F \|_{L_t^1L_x^2}\\
&\lt M^{\nu-1} \left( \sum_{L\sim K\gg M} L^{1-\sigma^\prime} K^{1-s} \max \Big( 1,\frac{L}{N} \Big)^{-D^\prime} \max\Big( \frac{N}{K}, \frac{K}{N} \Big)^{-D^\prime} \right) \| \bra^{\sigma^\prime} G \|_{\SlNprime} 
\| \bra^s F \|_{\XNprime}\\
&\lt M^{\nu-1} \left(\Is{M\lesssim N} N^{2-\sigma^\prime-s} + \Is{M\gg N} M^{2-\sigma^\prime-s} \Big( \frac{M}{N} \Big)^{-2D^\prime} \right)\| \bra^{\sigma^\prime} G \|_{\SlNprime} 
\| \bra^s F \|_{\XNprime} \\
&= T^{\frac{1}{2}} N^{\nu-s+1-\sigma^\prime} \left( \Is{M\lesssim N} \Big( \frac{M}{N} \Big)^{\nu-1} + \Is{M\gg N} \Big( \frac{M}{N} \Big)^{-2D^\prime+\nu-s+1-\sigma^\prime} \right) \| \bra^{\sigma^\prime} G \|_{\SlNprime} 
\| \bra^s F \|_{\XNprime}~. 
\end{align*}
Since \( \eta < \nu-1 \), we may multiply by \( c_{N,\eta}(M) + c_{\leq N,D}(M) \) and sum in \( M \geq 1\). \\
Next, we proof \eqref{nl:eq_high_high_wF}. For any \( M \geq 1 \), we have that 
\begin{align*}
&\| \langle \nabla \rangle^{\nu} P_M \Phh(v,F) \|_{L_t^\infty L_x^2}+ \| \langle \nabla \rangle^{\nu-1} P_M \partial_t \Phh(v,F) \|_{L_t^\infty L_x^2}  + \| \langle \nabla \rangle^{\sigma} P_M \Phh(v,F) \|_{L_t^2 L_x^\infty}\\
&\lesssim M^{\nu-1} \sum_{L\sim K \gg M} \| \nabla P_L v\cdot \nabla P_K F \|_{L_t^1L_x^2}\\
&\lt M^{\nu-1} \left( \sum_{L\sim K\gg M} L^{1-\nu} K^{1-\sigma^\prime} \max \Big( 1,\frac{L}{N} \Big)^{-D} \max\Big( \frac{N}{K}, \frac{K}{N} \Big)^{-D^\prime} \right) \| \bra^{\nu} v \|_{\XlN} 
\| \bra^{\sigma^\prime} F \|_{\SNprime}\\
&\lt M^{\nu-1} \left( \Is{M\lesssim N} N^{2-\nu-\sigma^\prime} + \Is{M\gg N} M^{2-\nu-\sigma^\prime} \Big( \frac{M}{N} \Big)^{-D-D^\prime} \right) \| \bra^{\nu} v \|_{\XlN}  \| \bra^{\sigma^\prime} F \|_{\SNprime} \\
&\lt N^{1-\sigma^\prime} \left( \Is{M\lesssim N} \Big( \frac{M}{N} \Big)^{\nu-1} + \Is{M\gg N} \Big( \frac{M}{N} \Big)^{-D-D^\prime+1-\sigma} \right)  \| \bra^{\nu} v \|_{\XlN} \| \bra^{\sigma^\prime} F \|_{\SNprime}~. 
\end{align*}
Since \( \eta < \nu-1 \), we may multiply by \( c_{N,\eta}(M) + c_{\leq N,D}(M) \) and sum in \( M \geq 1\). \\
Finally, we prove \eqref{nl:eq_high_high_Fw} and \eqref{nl:eq_high_high_ww}. For any \( M \geq 1 \), we have that 
\begin{align*}
&\| \langle \nabla \rangle^{\nu} P_M \Phh(G,w) \|_{L_t^\infty L_x^2}+\| \langle \nabla \rangle^{\nu-1} P_M \partial_t \Phh(G,w) \|_{L_t^\infty L_x^2} + \| \langle \nabla \rangle^{\sigma} P_M \Phh(G,w) \|_{L_t^2 L_x^\infty}\\
&\lesssim M^{\nu-1} \sum_{L\sim K \gg M} \| \nabla P_L G\cdot \nabla P_K w \|_{L_t^1L_x^2}\\
&\lt M^{\nu-1} \sum_{L\sim K\gg M} L^{1-\sigma^\prime} K^{1-\nu} \max\Big( 1,\frac{L}{N} \Big)^{-D} \max\Big( \frac{N}{K},\frac{K}{N} \Big)^{-\eta} \| \bra^{\sigma^\prime} G \|_{\SlN} \| \bra^\nu v \|_{\XNeta} \\
&\lt M^{1-\sigma^\prime} \left( \Is{M\lesssim N} \Big( \frac{M}{N} \Big)^{\eta} + \Is{M\gg N} \Big ( \frac{M}{N} \Big)^{-D-\eta} \right) \| \bra^{\sigma^\prime} G \|_{\SlN} \| \bra^\nu v \|_{\XNeta}
\end{align*}
In the evaluation of the sum, we have used that \( \eta < \nu-1 < \nu + \sigma^\prime - 2 \). After multiplying by \( c_{N,\eta}(M)+ c_{\leq N,D}(M) \) and summing in \( M \geq 1 \), we see that 
\begin{equation*}
\| \Phh(G,w) \|_{\YN} \lt    \| \bra^{\sigma^\prime} G \|_{\SlNprime} \| \bra^\nu v \|_{\XNeta}~. 
\end{equation*}
Since we only used \( \sigma^\prime > 1 \), the same argument also yields \eqref{nl:eq_high_high_ww}. This finishes the proof of the high-high bilinear estimates. 
\end{proof}

\subsection{Control of the nonlinear component \( w_n \)}


\begin{proof}[Proof of Proposition \ref{nl:prop_wn}:]~\\
We begin by showing the a-priori estimate for \( w_n \), which forms the main part of the proof. Afterwards, we will use contraction mapping to prove the existence and uniqueness of \( w_n \). This step could potentially be replaced by a soft argument, since all involved functions are smooth (with norms growing in \(N\)). 
\paragraph{A-priori bounds:}
We separate the proof into six cases, corresponding to the different terms in \eqref{nl:eq_truncated_wn}.\\

\emph{Case 1: Contribution of \( |\nabla F_n^\omega|^2 \).} Using \eqref{nl:eq_low_high_FF}, \eqref{nl:eq_high_low_FF}, and \eqref{nl:eq_high_high_FF}, we have that
\begin{equation*}  
\Big \|  \duh  \theta_{F;n}(s) \nabla F_n^\omega \cdot \nabla F_n^\omega \Big \|_{\YN} 
\lt N^{\nu-s+1-\sigma^\prime} \| \bra^s F_n^\omega \|_{\XNprime} ~. 
\end{equation*}


\emph{Case 2:  Contribution of \( \nabla F_n^\omega \nabla w_n \).} Using \eqref{nl:eq_low_high_Fw}, \eqref{nl:eq_high_low_Fw}, and \eqref{nl:eq_high_high_Fw}, we have that
\begin{equation*}  
\Big \|  \duh  \theta_{F;n}(s) \nabla F_n^\omega \cdot \nabla w_n \Big \|_{\YN} 
\lesssim T^{\frac{1}{2}}  \| w_n \|_{\YN}~. 
\end{equation*}


\emph{Case 3:  Contribution of \( |\nabla w_n|^2 \).} Using \eqref{nl:eq_low_high_ww}, \eqref{nl:eq_high_low_ww}, and \eqref{nl:eq_high_high_ww}, we have that
\begin{equation*}
\Big \|  \duh \theta_{w;n} \nabla w_n \cdot \nabla w_n \ds \Big \|_{\YN} 
\lesssim T^{\frac{1}{2}} 
 \| w_n \|_{\YN} ~. 
\end{equation*}

\emph{Case 4:  Contribution of \(  \nabla F_{\leq n-1}^\omega \nabla w_n\).} Using \eqref{nl:eq_low_high_Fw}, \eqref{nl:eq_high_low_Fw}, and \eqref{nl:eq_high_high_Fw}, we have that
\begin{equation*}  
\Big \|  \duh  \theta_{F,w;\leq n-1}(s) \nabla F_{\leq n-1}^\omega \cdot \nabla w_n \Big \|_{\YN} 
\lesssim T^{\frac{1}{2}}  
 \| w_n \|_{\YN}~.
\end{equation*} 

\emph{Case 5: Contribution of \(  \nabla w_{\leq n-1} \nabla w_n\).} Using \eqref{nl:eq_low_high_ww}, \eqref{nl:eq_high_low_ww}, and \eqref{nl:eq_high_high_ww}, we have that
\begin{equation*}
\Big \| \duh \theta_{F,w;\leq n-1} \nabla w_{\leq n-1} \cdot \nabla w_n \ds \Big \|_{\YN} 
\lesssim T^{\frac{1}{2}} 
 \| w_n \|_{\YN} ~. 
\end{equation*}

\emph{Case 6: Contribution of \( \nabla P_{>N^\gamma} \nabla u_{n-1} \nabla F_n^\omega\).} This term was already estimated in Corollary \ref{nl:cor_inhomogeneous_contribution}. We have that
\begin{align*}
&\Big \| \duh \theta_{F,w;\leq n-1}(s) \nabla P_{>N^\gamma} u_{n-1} \nabla F_n^\omega \ds \Big \|_{\YN} \\
&\lt \left( N^{\nu-s+\gamma(1-\sigma^\prime)} + N^{(1-\gamma) (\nu-1)+1-\sigma^\prime} \right) \left( \| \bra^s F_n^\omega \|_{\XNprime} + \| \bra^{\sigma^\prime} F_n^\omega \|_{\SNprime} \right)~. 
\end{align*}

Combining the estimates above, we obtain that 
\begin{equation*}
\| w_n \|_{\YN} \lesssim T^{\frac{1}{2}} \left( N^{\nu-s+\gamma(1-\sigma^\prime)} + N^{(1-\gamma) (\nu-1)+1-\sigma^\prime} \right) \left( \| \bra^s F_n^\omega \|_{\XNprime} + \| \bra^{\sigma^\prime} F_n^\omega \|_{\SNprime} \right) + T^{\frac{1}{2}} \| w_n \|_{\YN}~. 
\end{equation*}
Then, the a-priori bound follows by choosing \( T_0 >0 \) sufficiently small. 
\paragraph{Contraction Mapping:} ~\\
Due to the cutoffs, we may work on the whole space \( \YN \). We set
\begin{align*}
\Gamma w(t)&:=
\duh  \theta_{F;n}(s) |\nabla F_n^\omega|^2\ds  
		+2 \duh \theta_{F;n}(s) \nabla F_n^\omega \nabla w \ds \\
		&~~+ \duh \theta_{w}(s) |\nabla w|^2\ds
		+ 2 \duh \theta_{F,w;\leq n-1}(s) \nabla F_{\leq n-1}^\omega \nabla w \ds \\
		&~~+ \duh \theta_{F,w;\leq n-1}(s) \nabla w_{\leq n-1} \nabla w \ds \\
		&~~ + \duh \theta_{F,w;\leq n-1}(s) \nabla P_{>N^\gamma} \nabla u_{n-1} \nabla F_n^\omega \ds ~. 
\end{align*}
Here, the cutoff \( \theta_w(s) \) is defined by replacing \( w_n \) in the definition of \( \theta_{w;n}(s) \) with \( w \), see \eqref{it:eq_theta_wn}. The same arguments that led to the a-priori bound show that 
\begin{equation*}
\| \Gamma w \|_{\YN} \lesssim  T^{\frac{1}{2}} \left( N^{\nu-s-\gamma(\sigma^\prime-1)}+ N^{(1-\gamma)(\nu-1)+1-\sigma^\prime} \right) \left( \| \langle \nabla \rangle^s F_n^\omega \|_{\XNprime}+ \| \langle \nabla \rangle^{\sigma^\prime} F_n^\omega \|_{\SNprime}\right) + T^{\frac{1}{2}} \| w \|_{\YN}~. 
\end{equation*}
In particular, \( \Gamma \) maps \( \YN \) into \( \YN \). Thus, it suffices to prove for all \( v,w\in \YN \) that 
\begin{equation*}
\| \Gamma v - \Gamma w \|_{\YN} \lesssim T^{\frac{1}{2}} \| v - w \|_{\YN}~. 
\end{equation*}
For the linear terms in \( v \) and \( w \), this follows from the estimates above. Thus, it remains to control the quadratic term \( \theta_{v} |\nabla v|^2 - \theta_{w} |\nabla w|^2 \). We use a similar method as in the proof of \cite[Proposition 3.1]{BD99}. We define
\begin{equation*}
t_v := \sup\{ 0 \leq t \leq T\colon \| \langle \nabla \rangle^\nu v \|_{(\XNeta \bigcap \XlN)([0,t])} + \| \langle \nabla \rangle^{\sigma} v \|_{(\SNeta \bigcap \SlN)([0,t])}\leq 2 \}~. 
\end{equation*}

The time \( t_w \) is defined analogously.  Due to the continuity statement \eqref{prelim:eq_continuity}, we have that 
\begin{equation*}
\| 1_{[0,t_v]} ~ \langle \nabla \rangle^\nu v \|_{(\XNeta \bigcap \XlN)([0,T])}+ \quad \| 1_{[0,t_v]} ~ \langle \nabla \rangle^\sigma v \|_{(\SNeta \bigcap \SlN)([0,T])} \leq 2~.
\end{equation*}
To avoid confusion, we point out that the continuity statement \eqref{prelim:eq_continuity} is not enforced solely by the \( \XNT \)-norm, but comes from the definition of the space in \eqref{prelim:eq_spaces}. \\
Without loss of generality, we assume that \( t_v \leq t_w \). Using \eqref{nl:eq_low_high_ww}, \eqref{nl:eq_high_low_ww}, and \eqref{nl:eq_high_high_ww}, we have that 
\begin{align*}
&\Big\|  \duh \big( \theta_{v}(s) \ |\nabla v|^2-  \theta_{w}(s) |\nabla w|^2 \big) \ds\Big\|_{\YN}\\
&\leq \Big \| \duh 1_{[0,t_v]}(s) (\theta_v(s)-\theta_w(s)) |\nabla v|^2 \ds\Big \|_{\YN}  \\
&+ \Big\| \duh 1_{[0,t_v]}(s) \theta_w(s) \left( |\nabla v|^2 - |\nabla w|^2 \right) \ds\Big \|_{\YN} \\
&+ \Big\| \duh 1_{(t_v,t_w]}(s) (\theta_v(s)-\theta_w(s)) |\nabla w|^2 \ds \Big\|_{\YN} \\
&\lesssim T^{\frac{1}{2}} \| \theta_v - \theta_w \|_{L_t^\infty} \left( \| 1_{[0,t_v]} \bra^\nu v \|_{\XNeta \bigcap \XlN } + \| 1_{[0,t_v]} \bra^\sigma v \|_{\SNeta \bigcap \SlN} \right)^2\\
 &+ T^{\frac{1}{2}} \bigg( \| 1_{[0,t_v]} \bra^\nu v \|_{\XNeta \bigcap \XlN } + \| 1_{[0,t_v]} \bra^\sigma v \|_{\SNeta \bigcap \SlN}+ \| 1_{[0,t_v]} \bra^\nu w \|_{\XNeta \bigcap \XlN } \\
 &~~~+ \| 1_{[0,t_v]} \bra^\sigma w \|_{\SNeta \bigcap \SlN} \bigg) \| v -w \|_{\YN} \\
&+ T^{\frac{1}{2}} \| \theta_v - \theta_w \|_{L_t^\infty} \left( \| 1_{(t_v,t_w]} \bra^\nu w \|_{\XNeta \bigcap \XlN } + \| 1_{(t_v,t_w]} \bra^\sigma w \|_{\SNeta \bigcap \SlN} \right)^2\\
&\lesssim T^{\frac{1}{2}} \| v -w \|_{\YN}~. 
\end{align*}
Hence, \( \Gamma \) is a contraction on \( {\YN} \), and \( w_n \) can be defined as the unique fixed point of \( \Gamma \). 
\end{proof}

\section{Proof of the Main Theorem}\label{sec:final}
As in Section \ref{sec:lin}, any question regarding the (strong) measurability of the solutions is addressed in the appendix.
Before we begin with the proof of the main theorem, we collect all conditions on the parameters.
\paragraph{Parameter Conditions:} First, we have the basic conditions
\begin{equation}\label{fin:eq_basic_conditions}
\nu > 2 > s > 1~, \quad \sigma= \nu - 1 -, \quad \sigma^\prime > \sigma, \quad \text{and} ~ \quad \gamma \in (0,1)~. 
\end{equation}
In order to use Proposition \ref{lin:prop_probabilistic_strichartz}, Proposition \ref{nl:prop_wn}, and Corollary \ref{nl:cor_inhomogeneous_contribution}, we require the major conditions
\begin{equation}\label{fin:eq_major_conditions}
\begin{aligned}
\sigma^\prime -s + 1 - \gamma (\sigma-1) - \frac{1}{2} (1-\gamma)&<0~,\\
\nu - s - \gamma (\sigma^\prime-1) &<0 ~, \\
(1-\gamma) (\nu-1) + 1- \sigma^\prime &<0~. 
\end{aligned}
\end{equation}
Because of \eqref{lin:eq_consistency_condition_strichartz} and \eqref{lin:eq_probabilistic_strichartz}, we also require the minor conditions
\begin{equation}\label{fin:eq_minor_conditions}
\nu < \frac{5}{2} \quad \text{and} \quad s> \sigma^\prime~. 
\end{equation}
In particular, if \eqref{fin:eq_basic_conditions}, \eqref{fin:eq_major_conditions}, and \eqref{fin:eq_minor_conditions} are satisfied, we can find an \( \eta \) that satisfies the conditions of Proposition \ref{nl:prop_wn}. \\

To complete the proof of the main theorem, we now have to prove the convergence of the iterates \( u_n \), remove the truncation in \eqref{it:eq_truncated_Fn} and \eqref{it:eq_truncated_wn} by choosing a small random time \( T(\omega)>0 \), and optimize the parameters. 

\begin{proof}[\textbf{Proof of the main theorem:}]~\\
First, we show the convergence of the iterates \( u_n \). Assuming that the parameters satisfy  \eqref{fin:eq_basic_conditions}, \eqref{fin:eq_major_conditions}, and \eqref{fin:eq_minor_conditions}, we prove that there exists a random function \( {u\colon \Omega \times [0,T_0] \times \rthree \rightarrow \mathbb{R} }\) s.t.
\begin{equation}\label{fin:eq_convergence}
\begin{aligned}
u_n \rightarrow u \quad &\text{in} \quad L_\omega ^2 C_t^0 H_x^s(\Omega \times [0,T_0]\times \rthree) ~ \text{and} ~L_\omega^2 L_t^2 W_x^{\sigma,\infty}(\Omega \times [0,T_0]\times \rthree)~,\\
\partial_t u_n \rightarrow \partial_t u \quad &\text{in} \quad  L_\omega ^2 C_t^0 H_x^{s-1}(\Omega \times [0,T_0]\times \rthree) ~. 
\end{aligned}
\end{equation}
Here, \( T_0 > 0 \) is as in Proposition \ref{nl:prop_wn}.

Let \( \epsilon > 0 \) be sufficiently small depending on the parameters above. We show the convergence of the series \( \sum_{m=0}^{\infty} F_m^\omega \) and \( \sum_{m=0}^\infty w_m \) in  \( L_\omega ^2C_t^0 H_x^s \) and \( L_\omega^2 L_t^2 W_x^{\sigma,\infty} \). The convergence of the time-derivatives follows from a similar argument. \\
Let \( 0 \leq n_- < n_+ < \infty \) be arbitrary. Using Lemma \ref{lin:lem_frequency_Fnomega} and writing \( M=2^m \), we have that
\begin{align*}
&\Big \| \sum_{m=n_-}^{n_+} \langle \nabla \rangle^s F_m^\omega \Big \|_{L_\omega^2 L_t^\infty L_x^2} 
\lesssim \Big \| \sum_{m=n_-}^{n_+} \langle \nabla \rangle^s P_N F_m^\omega \Big \|_{L_\omega^2 L_t^\infty \ell_N^2 L_x^2}
\lesssim  \Big \| \sum_{m=n_-}^{n_+} \langle \nabla \rangle^s P_N F_m^\omega \Big \|_{L_\omega^2 \ell_N^2 L_t^\infty  L_x^2}\\
&\lesssim \Big \| \sum_{m=n_-}^{n_+} \| \langle \nabla \rangle^s P_N F_m^\omega \|_{L_t^\infty L_x^2} \Big \|_{L_\omega^2 \ell_N^2} 
\lesssim \Big \| \sum_{m=n_-}^{n_+} \max\left( \frac{N}{M}, \frac{M}{N} \right)^{-D^\prime} \| \langle \nabla \rangle^s  F_m^\omega \|_{\XMprime} \Big \|_{L_\omega^2 \ell_N^2} \\
&\lesssim  \Big \| \sum_{m=n_-}^{n_+} \max\left( \frac{N}{M}, \frac{M}{N} \right)^{-D^\prime} \| ( \widetilde{P}_M f_0^\omega , \widetilde{P}_M f_1^\omega) \|_{H_x^s\times H_x^{s-1}} \Big \|_{L_\omega^2 \ell_N^2}\\ 
&\lesssim \Big \| \left ( \sum_{m=n_-}^{n_+} \| (\widetilde{P}_M f_0^\omega, \widetilde{P}_M f_1^\omega) \|_{H_x^s\times H_x^{s-1}}^2 \right)^{\frac{1}{2}} \Big \|_{L_\omega^2} 
\lesssim \left(  \sum_{m=n_-}^{n_+} \| (\widetilde{P}_M f_0, \widetilde{P}_M f_1) \|_{H_x^s\times H_x^{s-1}}^2 \right)^{\frac{1}{2}} ~. 
\end{align*}
This proves that the series \( \sum_{m=0}^\infty F_m^\omega \) is Cauchy in \( L_\omega^2 C_t^0 H_x^s \). From Proposition \ref{lin:prop_probabilistic_strichartz}, we have that
\begin{equation}\label{fin:eq_Fm_Strichartz}
\| \langle \nabla \rangle^{\sigma} F_m^\omega \|_{L_\omega^2 L_t^2 L_x^\infty} \lesssim \| \langle \nabla \rangle^{\sigma^\prime} F_m^\omega \|_{L_\omega^2 \SMprime} \lesssim M^{-\epsilon}  \| (\widetilde{P}_M f_0, \widetilde{P}_M f_1) \|_{H_x^s\times H_x^{s-1}}~. 
\end{equation}
This proves the convergence of \( \sum_{m=0}^\infty F_m^\omega \) in \( L_\omega^2 L_t^2 W_x^{\sigma,\infty} \). \\
From Proposition \ref{nl:prop_wn}, we have that 
\begin{equation*}
\| \langle \nabla \rangle^{\nu} w_m \|_{L_t^\infty L_x^2} + \|\langle \nabla \rangle^{\sigma} w_m \|_{L_t^2 L_x^\infty} \lesssim
T^{\frac{1}{2}} M^{-\epsilon} \left( \| \langle \nabla \rangle^{s} F_m^\omega \|_{\XMprime} + \| \langle \nabla \rangle^{\sigma^\prime} F_m^\omega \|_{\SMprime} \right)~. 
\end{equation*}
After taking moments in \( \omega \), the convergence then follows from Lemma \ref{lin:lem_frequency_Fnomega}  and Proposition \ref{lin:prop_probabilistic_strichartz}. \\

Second, we show that there exist random times \( T(\omega) \) s.t. \eqref{intro:eq_u_duhamel} holds. To eliminate the cutoff, it suffices to choose \( T(\omega)>0 \) s.t.
\begin{equation*}
\sum_{m=0}^{\infty} \Big( \| \langle \nabla \rangle^{\sigma^\prime} F_{m}^\omega \|_{\SMprime([0,T(\omega)])} + \| w_m \|_{\YM([0,T(\omega)])} \Big)
\leq 1~. 
\end{equation*}
Using the continuity statement \eqref{prelim:eq_continuity} and the estimate \eqref{fin:eq_Fm_Strichartz}, we have for a.e. \( \omega \in \Omega \) that
\begin{equation*}
t \in [0,T_0] \mapsto \sum_{m=0}^\infty \| \langle \nabla \rangle^{\sigma^\prime} F_m^\omega \|_{\SMprime([0,t])} 
\end{equation*}
is continuous and equals zero at \( t= 0 \). As a consequence, the random time
\begin{equation*}
T_1(\omega) := \sup\bigg\{ 0 \leq t \leq T_0 \colon  \sum_{m=0}^\infty \| \langle \nabla \rangle^{\sigma^\prime} F_m^\omega \|_{\SMprime([0,t])}\leq \frac{1}{2} \bigg\} 
\end{equation*}
is almost surely positive. To control the nonlinear components \( w_m \), we recall from Proposition \ref{nl:prop_wn} that
\begin{equation*}
\| w_m \|_{\YM([0,T])}\lesssim T^{\frac{1}{2}} M^{-\epsilon} \left( \| \langle \nabla \rangle^s F_m^\omega \|_{\XMprime([0,T])}+ \| \langle \nabla \rangle^{\sigma^\prime} F_m^\omega \|_{\SMprime([0,T])}\right)
\end{equation*}
Using Lemma \ref{lin:lem_frequency_Fnomega} and Proposition \ref{lin:prop_probabilistic_strichartz}, we have almost surely that
\begin{equation*}
\sum_{m=0}^\infty M^{-\epsilon}  \left( \| \langle \nabla \rangle^s F_m^\omega \|_{\XMprime([0,T_0])}+ \| \langle \nabla \rangle^{\sigma^\prime} F_m^\omega \|_{\SMprime([0,T_0])}\right) < \infty~.
\end{equation*}
Thus, the random time 
\begin{equation*}
T_2(\omega) := \sup \bigg \{ 0\leq t \leq T_0\colon \sum_{m=0}^\infty \| w_m \|_{\YM} \leq \frac{1}{2} \bigg\}
\end{equation*}
is almost surely positive. Setting \( T(\omega) = \min( T_1(\omega), T_2(\omega) ) \), we obtain for a.e. \( w\in \Omega \) that 
\begin{equation*}
u_n(t)= W(t) (f_0^\omega, f_1^\omega) + \duh |\nabla u_n(s)|^2 \ds \qquad  \forall n\geq 0 ~ \text{and} ~\forall t \in [0,T(\omega)]   ~ . 
\end{equation*} 
Then, \eqref{intro:eq_u_duhamel} follows from the convergence of the iterates \( u_n \). \\

Third, we have to determine nearly optimal parameters \( (s,\nu,\sigma^\prime,\gamma) \). We discretized the parameter \( \gamma \in (0,1) \) and used a linear programming solver to find the remaining parameters  \( (s,\nu,\sigma^\prime ) \) with an almost optimal value of \( s \). This leads to 
\begin{equation}\label{fin:eq_parameters}
(s,~ \nu,~ \sigma^\prime,~ \gamma ) = (1.9840,~ 2.1001,~ 1.13205,~ 0.88 )~.
\end{equation}
\end{proof}\vspace{-4ex}

\appendix
\section{Appendix}
\subsection*{Strong measurability of \( F_n^\omega \) and \( w_n \)}

In this section, we prove the strong measurability of the iterates. As before, let \( (\Omega, \mathscr{F},\mathbb{P} ) \) be the given probability space. We recall the following definition from the theory of Bochner-integration.
\begin{definition}
Let \( E \) be a Banach space. A function \( v\colon \Omega \rightarrow E \)  is called simple if there exist measurable sets \( F_i \in \mathscr{F} \) and vectors \( x_i \in E \), \( i=1,\hdots,k \), such that
\begin{equation*}
v= \sum_{i=1}^k 1_{F_i}(\omega) ~ x_i ~. 
\end{equation*}
A function \( v\colon \Omega \rightarrow E \)  is called strongly measurable (or strongly \( \mathscr{F}\)-measurable) if it can be written as the pointwise limit of simple functions. Finally, a function \( v\colon \Omega \rightarrow E \) is called strongly \( \mathbb{P}\)-measurable if there exists a strongly measurable function \( \widetilde{v} \colon \Omega \rightarrow E \) such that \( v (\omega)= \widetilde{v}(\omega) \) holds \( \mathbb{P}\)-almost surely. 
\end{definition}

The following two properties follow directly from the definition. 
\begin{lem}Let \( E,E_1,E_2, \) and \( F \) be Banach spaces. \label{appendix:lem_properties}
\begin{enumerate}[itemsep=0ex]
\item If \( v\colon \Omega \rightarrow E \)  is strongly measurable and \( \phi\colon E \rightarrow F \) is continuous (but possibly nonlinear),  then the composition \( \phi\circ v\colon \Omega \rightarrow F \) is strongly measurable.
\label{appendix:enum_cont} 
\item If \( v_i \colon \Omega \rightarrow E_i \), \( i=1,2 \), are strongly measurable, then \( (v_1,v_2) \colon \Omega \mapsto E_1 \times E_2 \) is strongly measurable. \label{appendix:enum_prod}
\end{enumerate}\vspace{-4ex}
\end{lem}

We are now ready to prove the main proposition of this section. Recall the definition of the sigma-algebra \( \mathscr{F}_n := \sigma( g_l \colon \| l\|_2 < 2^n) \), where \( n\in \mathbb{N}_0 \).
\begin{prop}\label{appendix:prop_measurability}
Let \( F_n^\omega\), \( F_{n,k} \), and \( w_n\) be as in \eqref{it:eq_truncated_Fn}, \eqref{it:eq_truncated_Fnk}, and \eqref{it:eq_truncated_wn}. Furthermore, let 0 \( \leq T \leq T_0 \) be as in Theorem \ref{main_thm}. Then, we have for all \( n \geq 0 \) that 
\begin{enumerate}[itemsep=0ex]
\item the functions \( \omega \mapsto \bra^s F_n^\omega \in \XNprime  \), \( \omega \mapsto \bra^{s-1} \partial_t F_n^\omega \in \XNprime \), and \( \omega \mapsto \bra^{\sigma^\prime} F_n^\omega \in \SNprime \) are strongly \( \mathscr{F}_n \)-measurable, \label{appendix:enum_Fn} 
\item the functions \( \omega \mapsto \nabla F_{n,k} \in C_t^0 \Bkprime \),   \( \omega \mapsto \partial_t F_{n,k} \in C_t^0 \Bkprime \), and  \( \omega \mapsto  F_{n,k} \in C_t^0 \Bkprime \)  are strongly \( \mathscr{F}_{n-1}\)- measurable, \label{appendix:enum_Fnk}
\item the function \( \omega \mapsto w_n \in \YN \) is strongly \( \mathscr{F}_n \)-measurable. \label{appendix:enum_wn}
\end{enumerate}
Furthermore, let \( u \) be the solution from Theorem \ref{main_thm}. Then, the maps \( \omega \mapsto u\in C_t^0 H_x^s \medcap L_t^2 W_x^{\sigma,\infty} \) and \( \omega \mapsto \partial_t u \in C_t^0 H_x^{s-1} \) are strongly \( \mathbb{P}\)-measurable. 
\end{prop}

Before we prove the proposition, we need the following lemma which proves the measurability of the cutoff. 

\begin{lem}\label{appendix:lem_cutoff}
If \( \omega \in \Omega \mapsto v^\omega \in \XN([0,T]) \) is strongly \( \mathscr{F}_n \)-measurable, then the map 
\begin{equation*}
(\omega,\tau) \in \Omega \times [0,T] \rightarrow \| v^\omega \|_{\XN([0,\tau])} \in \mathbb{R}_{\geq 0} 
\end{equation*}
is measurable with respect to the product sigma-algebra \( \mathscr{F}_n \bigotimes \mathscr{B}([0,T]) \). Here, \( \mathscr{B}([0,T]) \) denotes the Borel sigma-algebra. \\
An analogous statement also holds for \( \XlN([0,T]), ~ \SN([0,T]), ~ \text{and} ~ \SlN([0,T]) \). 
\end{lem}
\begin{proof}
Since \( v^\omega \) is strongly \( \mathscr{F}_n \)-measurable, it suffices to prove the statement for simple functions. Thus, we may assume that there exists pairwise disjoint measurable sets \( F_i \in \mathscr{F}_n \) and (deterministic) functions \( v_i \in \XN([0,T]) \), \( i=1,\hdots, k \), such that 
\begin{equation*}
v^\omega  = \sum_{i=1}^k 1_{F_i}(\omega) v_i ~. 
\end{equation*} 
It follows that \begin{equation*}
\| v^\omega \|_{\XN([0,\tau])} = \sum_{i=1}^k 1_{F_i}(\omega) \| v_i \|_{\XN([0,\tau])}~. 
\end{equation*}
Thus, Lemma \ref{appendix:lem_cutoff}  follows from the continuity statement \eqref{prelim:eq_continuity}.
\end{proof}

\begin{proof}[Proof of Proposition \ref{appendix:prop_measurability}]
We prove the proposition by induction on \( n\). Since the base case \( n=0 \) and induction step follow from the same argument, we may assume directly that \ref{appendix:enum_Fn}-\ref{appendix:enum_wn} hold for all \( {m=0,\hdots, n-1} \). \\
Due to \ref{appendix:enum_Fn}, \ref{appendix:enum_wn}, and Lemma \ref{appendix:lem_properties}.\ref{appendix:enum_cont}, we see that \( \omega \mapsto  u_{n-1} \in L_t^2 W_x^{\sigma,\infty} \) is strongly \(\mathscr{F}_{n-1} \)-measurable. Similarly, using \ref{appendix:enum_Fn}, \ref{appendix:enum_wn}, and Lemma \ref{appendix:lem_cutoff}, we obtain the measurability of the cutoff \( \theta_{F,w;\leq n-1} \). Since the proof of  Proposition \ref{lin:prop_freq_envelope} leads to a contraction mapping argument, we see that the solution \( F_k \) of \eqref{lin:eq_Fk} depends continuously on \( \phi \in L_t^2 W_x^{\sigma,\infty} \). Therefore, we obtain \ref{appendix:enum_Fnk} from Lemma \ref{appendix:lem_properties}.\ref{appendix:enum_cont}. For any sufficiently large \( D^{\prime\prime} >0 \), we have the continuous embeddings \( C_t^0 \Bkprime \hookrightarrow \XNprime \) and \( C_t^0 \Bkprime \hookrightarrow \SNprime\), where the norm of the embedding may depend on \( N \). Since \begin{equation*}
F_n^\omega = \sum_{N/2 \leq \| k\|_2 < N} g_k(\omega) F_{n,k}~,
\end{equation*}
this proves \ref{appendix:enum_Fn}.  Since the proof of Proposition \ref{nl:prop_wn} consists of a contraction mapping argument, \( w_n \in \YN \) depends continuously on \( F_m^\omega\), \( w_m \), where \( m=0,\hdots, n-1\), and \( F_n^\omega\), all in their respective norms. Thus, \ref{appendix:enum_wn} follows from \ref{appendix:enum_Fn} with \( m=0,\hdots,n \), \ref{appendix:enum_wn} with \( m=0,\hdots,n-1\),  Lemma \ref{appendix:lem_properties}, and Lemma \ref{appendix:lem_cutoff}. \\
Finally, the strong \( \mathbb{P} \)-measurability of \( u \) follows from the convergence of the iterates, see \eqref{fin:eq_convergence}.
\end{proof}



\bibliography{Library_Wiki}
\bibliographystyle{hplain}

\Addresses


\end{document}