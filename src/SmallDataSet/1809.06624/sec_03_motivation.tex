%----------------------------------------------------------%

\section{Isolating SDN Control Traffic Overhead \\ with 6TiSCH Tracks}
\label{sec_motivation}

In this section we firstly examine how the controller overhead generated by SDN can have an adverse affect on regular network traffic flows. We introduce $\mu$SDN, our lightweight SDN implementation written for Contiki OS, in order to characterize SDN control traffic, and show the effect of different control traffic types on regular network traffic. We then propose that, in order to address the effects of SDN overhead within IEEE 802.15.4-2015 TSCH networks, 6TiSCH tracks can be utilized to provide an isolated network slice to SDN control traffic: delivering low latency SDN controller communication with minimal jitter, and minimizing disruption to the rest of the network.

\begin{figure}[ht]
\centering
	\includegraphics[width=0.6\columnwidth]{images/stack-6tisch-usdn.pdf}
  \caption{$\mu$SDN and 6TiSCH protocol stack.}
  \label{fig:stack_6tisch_usdn}
\end{figure}

$\mu$SDN incorporates features for minimizing controller overhead, whilst supporting a fully programmable SDN flowtable, and is integrated with the Contiki IEEE 802.15.4-2015 stack using our own 6TiSCH implementation. It has been tested in Cooja using TI's exp5438 platform with MSP430F5438 CPU and CC2420 radio. Figure \ref{fig:stack_6tisch_usdn} shows the integration of $\mu$SDN within the 6TiSCH protocol stack. 

Traffic generated with $\mu$SDN, using the $\mu$SDN protocol exhibits different behavior depending on the packet type, and the effect that SDN overhead has on regular network traffic is related to this behavior. $\mu$SDN nodes introduce two main forms of SDN control overhead. Firstly, a Node State Update (NSU )message is periodically sent from a node to the SDN controller, and carries information about the state of that node, such as energy and buffer congestion, as well as its local network state. These messages are used to update the controller on the state of the overall network, and are sent on a timer that is configured as the node joins the SDN network. They will continue to be sent until the controller reconfigures this timer. Secondly, Flowtable Query (FTQ) messages are sent to the controller in response to a flowtable miss (i.e. the SDN process checks the flowtable for instructions on how to handle a packet but is unable to find a matching entry). These packets contain information about the packet that caused the miss. The timing of FTQ messages therefore depends on the number of 'new' packets seen by a node, as well as the flowtable entry lifetime period. Once a flowtable entry expires, the node has to query the controller for new instructions about what to do with that packet type; this process is analogous to the \textit{PacketIn}/\textit{PacketOut} process in OpenFlow. If a number of nodes have similar expiry times for flowtable entries this can result in bursts of queries being sent towards the controller. In addition, successive packets arriving at a node could cause multiple queries to be sent whilst the node waits for a response from the controller. 

\begin{figure}[ht]
\centering
  \includegraphics[width=0.8\columnwidth]{images/overhead-rate-comparison.pdf}
  \caption{Effect of SDN control overhead on application-layer traffic.}
  \label{fig:nsu_1min}
\end{figure}


The SDN control traffic generated from the controller update and query processes can therefore be both \textit{periodic} as well as \textit{intermittent} and \textit{bursty}, as summarized in Table \ref{table:usdn_traffic_characterization}. However, by allocating dedicated slices for control traffic, tracks can mitigate the effects of SDN overhead on the normal operation of the network, as well as providing dedicated resources for low-latency controller communication.

\begin{table}[ht]
	\renewcommand{\arraystretch}{1.0}
	\caption{Traffic Categorization of $\mu$SDN Packet Types} 
    \label{table:usdn_traffic_characterization}
	\centering
    \begin{tabular}{ |l|l|l| }
    \hline
      	\bfseries Packet Type & \bfseries Behavior & \bfseries Effect\\ \hline
		NSU (Update) & Periodic & Increase Delay \\ \hline
        FTQ (Query) & Bursty & Increase Jitter and Packet Loss \\ 
    \hline
    \end{tabular}
\end{table}


