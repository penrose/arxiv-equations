We evaluate our system in indoor and outdoor scenarios, including an inventory mission in an active warehouse. A video showing autonomous mission execution and reactive obstacle avoidance can be found on our website\footnote{\scriptsize{\url{http://www.ais.uni-bonn.de/videos/IROS_2018_InventAIRy}}}. Here, we also publish recorded datasets, tools, and parts of our pipeline.

% Courtyard
First, we test the robustness of the localization and control pipeline with an experiment that involves fast flight between alternating waypoints in an obstacle free courtyard over a distance of \SI{25}{\meter}.
The localization in an allocentric map of the courtyard and state estimation of the MAV was solely based on the onboard 3D lidar and the IMU; no GNSS feedback was used.
Between the acceleration and deceleration phases of the flight, the MAV reached a maximum velocity of \SI{7.8}{\meter\per\second}, measured by the onboard DJI GPS -- considered as ground truth.
The laser localization was running at \SI{20}{\hertz} to account for the large velocities. It was able to robustly track the MAV pose during the whole flight.
Despite strong wind, the maximum deviation from the straight line connection between both waypoints was only \SI{49}{\centi\meter} during all 11 alternations. The maximum overshoot recorded during the experiment was \SI{1.2}{\meter}.

% Garage
In a second experiment, the MAV flew a figure eight around two pillars in a garage to additionally test the influence of turbulences close to structures and the ground.
Due to the high accelerations of approximately \SI{0.85}{\meter\per\second\squared} in the curved segments of the trajectory, the maximum velocity in these runs was reduced to yield a feasible, collision-free trajectory.
Still, the MAV reached velocities up to \SI{1.75}{\meter\per\second} in this indoor environment.
The laser localization tracked the MAV pose with \SI{10}{\hertz} and was able to keep the MAV localized in the map of the garage at all times.
\reffig{fig:trajectory_garage_image} shows the resulting trajectory in the map of the garage. It can be seen that our method yields robust repeatability in four consecutive flights despite turbulences.
Nevertheless, it can be seen that the MAV spirals out of the curved segments as it cannot accurately track the moving waypoint.

% Warehouse
In a third experiment, our integrated system, including laser-based localization, planned navigation, obstacle avoidance, and acquisition of information about stock positions, was demonstrated in a warehouse with a building area of \SI[product-units = single]{100 x 60}{\meter} with \SI{1.3}{\kilo\meter} shelf (approximately \SI{12000}{\meter\squared} storage front).
As described in \refsec{sec:3D_Mapping}, we built an initial laser-based SLAM map of the environment with a manual flight, shown in \reffig{fig:mapping}.
This map is aligned with the semantic map containing storage units from the WMS.
For the demonstration of autonomous inventory, a mission containing the complete inventory of one shelf row and the inspection of a single storage unit in another row was specified in the WMS.
The MAV executed this mission autonomously multiple times while avoiding static obstacles, \eg the shelves and stock protuding from the shelves.
In \reffig{fig:trajectory_image} we visualize the trajectory of two consecutive flights in the warehouse. The MAV reaches velocities up to \SI{2.1}{\meter\per\second}.
Although faster flight is possible (as shown in the previous experiments), we used the ability of our MPC to limit the maximum velocity a) to account for the acceleration/deceleration distance needed by the MAV and b) to reduce motion blur in the cameras (see \reffig{fig:detections}):

The closed loop dynamics of MAV and MPC dictate the dynamic behavior of the system. Even under the assumption that the MAV is able to perfectly track the trajectories generated by the MPC and without any perception- or communication delay, an acceleration/deceleration distance of \SI{12.1}{\meter} is necessary with a maximum velocity of \SI{7.8}{\meter\per\second} (with parameters $a_{max} = \SI{3.5}{\meter\per\second\squared}, j_{max} = \SI{4.0}{\meter\per\second\cubed}$).

Furthermore, due to the artificial lighting in the warehouse, the camera exposure time had to be set to at least \SI{4}{\milli\second} for acceptable image quality. The used AprilTags have an edge length of \SI{16}{\centi\meter} that results in a patch size of \SI[product-units = single]{2 x 2}{\centi\meter}. Thus, the Nyquist frequency limits the relative velocity to \SI{10}{\meter\per\second} under ideal conditions. This velocity, however, would require special signal reconstruction techniques to preprocess the image for the AprilTag detector.
Also roll, pitch, and yaw motion superimpose the linear MAV velocity and generate relative motion between tag and MAV.
High-frequency vibrations generated by the propellers provoke additional blur. Therefore, we conservatively constrained the linear velocity in favor of robust detections in the warehouse experiment.

In contrast to the visual detection pipeline, the RFID reader did not limit the inventory speed since it is able to read up to 750 tags per second. We throttled the speed to 20 reads/second which was enough for our experiments and allowed for a higher detection range.

Every view pose is reached with a mean deviation of only \SI{9.65}{\centi\meter} respectively \SI{5.78}{\centi\meter} in both flights.
As no dynamic obstacles above the MAV were to be expected in this demonstration, we neglected the planning with visibility constraints in favor of faster mission execution.

During both flights, AprilTags on the sides of the aisle and RFID tags of the specified shelf row were captured and sent to the WMS.
\reffig{fig:detections} and \reftab{tab:results} show the result of the two flights. It can be seen that except for Tag 6, and 11, all tags are reliably detected (Tags 8 and 12 were not used). Our method was unable to detect Tag 11 due to a shadow that partially covered the tag on a disadvantageously positioned stock. Tag 6 was not attached properly and was flipped by turbulent air from the MAV. Not a single false positive detection happened during the experiment. It can be seen that only minimal scattering occurs and thus the relative detection error is small.

\begin{table}[t]
  \caption{Statistics of AprilTag detections for two flights. }
\small
 \setlength{\tabcolsep}{1.5mm}
  \vspace{-1ex}
\centering
  \begin{tabular}{@{}l@{\hspace{1pt}}ccccccccccc@{}}
  \toprule
  Tag ID               & 0              & 1                      & 2                       & 3                      & 4                       & 5                      & 7                      & 9                      & 10                     & 13                     & 14\\
  \midrule
  $n_{1}$              & 3              & 7                      & 6                       & 3                      & 1                       & 64                     & 2                      & 10                     & 6                      & 4                      & 6\\
  $n_{2}$              & 7              & 7                      & 6                       & 3                      & 1                       & 41                     & 3                      & 5                      & 3                      & 4                      & 5\\
  \midrule
  $\sigma_{1}$    & 10.4           & 3.2 & 4.7  & 2.7 & -                       & 4.7 & -                      & 3.3 & 4.2 & 3.3 & \phantom{0}3.9\\
  $\sigma_{2}$    & \phantom{0}3.9 & 4.3 & 5.0  & 3.4 & -                       & 3.8 & 3.3 & 3.5 & 2.1 & 2.8 & \phantom{0}4.8\\
  $|\mu_{1-2}|$  & 28.6           & 2.9 & 8.9  & 5.5 & 3.3  & 2.1 & 1.2 & 3.2 & 1.0 & 2.1 & 10.7\\
  \bottomrule    
\end{tabular}
\vspace{0.5ex}
\footnotesize 
   \rule[-.3\baselineskip]{0pt}{\baselineskip} \hspace*{-2ex} $n_i$ is the number of detections per flight, $\sigma_i$ the deviation of the\\ 
    detections in cm. $|\mu_{1-2}|$ is the distance of the means $\mu_i$ in cm.
  %\vspace{-2ex}
  \label{tab:results}
\end{table}

% Dynamic Obstacle
After the executed inventory mission, the MAV hovered at a height of \SI{2}{\meter} above the ground.
A person approached the MAV, which avoided the dynamic obstacle by means of our reactive obstacle avoidance, shown in \reffig{fig:hindernisvermeidung_screenshot}.
Furthermore, a person stepped into the way of the MAV while it approached a waypoint.
The MAV stopped at a safe distance in all cases.

% Future work
As shown in the experiments, the limiting factor for faster inventory is motion blur in the cameras caused by the large exposure time due to bad lighting conditions. In future work, we want to oppose this bottleneck either by illuminating the scene ourselves (by using a flash on the MAV) or by using special equipment like \eg event based cameras.

In the current setup, the MAV continuously records images with \SI{3}{\hertz}. The AprilTags however only cover less than \SI{3}{\percent} of the area (\SI{0.0256}{\meter\squared} tag size vs. \SI{0.96}{\meter\squared} storage unit front). A more targeted strategy would reduce the generated data.
Furthermore, we also plan to extend our vision pipeline to not only detect AprilTags, but also other visual indicators like, \eg barcodes, QR codes, and human readable text, commonly found on stock. This would further enhance the versatility of the system.
We also plan to integrate multiple MAVs into the mission planner for simultaneous inventory to speed up the process even more.

One might also think of eliminating the first manual flight in favor of an automated SLAM process, but that a manual flight is more robust in this crucial map building phase. Furthermore, in comparison to the service life of such a system, the map building phase only causes a small fixed effort, since the map is reusable. After the manual flight, the operators can check the map for possible artifacts and misregistrations.
