\documentclass[preprint,12pt]{article}
%\documentclass{elsart3-1}

\usepackage{setspace}
\usepackage{pdfpages}
%\doublespacing

\usepackage{amsthm,amsmath,amssymb,amsfonts,amscd,amsbsy}%,latexsym,epsfig,subfig}
%\usepackage{graphicx,psfrag,verbatim,wasysym}
\usepackage{graphicx}
\usepackage[ruled,noline,linesnumbered]{algorithm2e}
%\usepackage{bm}
\usepackage[tight]{subfigure}
\usepackage{fullpage}
\usepackage{color}
\usepackage{url}
\usepackage{mdframed}
%\usepackage{setspace}
%\usepackage[tight]{subfigure}
%\usepackage{listings}
\newcommand{\doi}[1]{\textsc{doi:} \textsf{#1}\xspace}

\DeclareGraphicsExtensions{.jpg,.png,.pdf}
\DeclareGraphicsRule{*}{mps}{*}{}

\graphicspath{{./fig/}}

\def\calA{\mathcal{A}}
\def\calB{\mathcal{B}}
\def\calC{\mathcal{C}}
\def\calD{\mathcal{D}}
\def\calE{\mathcal{E}}
\def\calF{\mathcal{F}}
\def\calG{\mathcal{G}}
\def\calH{\mathcal{H}}
\def\calI{\mathcal{I}}
\def\calK{\mathcal{K}}
\def\calL{\mathcal{L}}
\def\calM{\mathcal{M}}
\def\calN{\mathcal{N}}
\def\calO{\mathcal{O}}
\def\calP{\mathcal{P}}
\def\calQ{\mathcal{Q}}
\def\calR{\mathcal{R}}
\def\calS{\mathcal{S}}
\def\calT{{\mathcal{T}}}
\def\calU{\mathcal{U}}
\def\calV{\mathcal{V}}
\def\calW{\mathcal{W}}
\def\calX{\mathcal{X}}
\def\calY{\mathcal{Y}}
\def\calZ{\mathcal{Z}}

\def\AA{\mathbb{A}}
\def\BB{\mathbb{B}}
\def\DD{\mathbb{D}}
\def\EE{\mathbb{E}}
\def\MM{\mathbb{M}}
\def\CC{\mathbb{C}}
\def\FF{\mathbb{F}}
\def\LL{\mathbb{L}}
\def\NN{\mathbb{N}}
\def\PP{\mathbb{P}}
\def\RR{\mathbb{R}}
\def\SS{\mathbb{S}}
\def\TT{\mathbb{T}}
\def\UU{\mathbb{U}}
\def\VV{\mathbb{V}}
\def\XX{\mathbb{X}}
\def\ZZ{\mathbb{Z}}

\newcommand{\ttt}[1]{\texttt{#1}}


\newcommand{\bmat}[1]{\left(\begin{array}{#1}}
\newcommand{\emat}{\end{array}\right)}

\newcommand{\pp}[2]{\frac{\partial #1}{\partial #2}}

\def\noi{\noindent}



\title{Project Artie: An Artificial Student for Disciplines Informed by Partial Differential Equations\thanks{Submitted to Computer Applications in Engineering Education}}
\date \today
\author{Anthony T Patera \thanks{Department of Mechanical Engineering, Massachusetts Institute of Technology \newline \mbox{}\hspace{.21in} Room 3-266, MIT, 77 Massachusetts Avenue, Cambridge, MA 02139 USA \newline \mbox{}\hspace{.20in} patera@mit.edu} }

\begin{document}

\maketitle

\begin{abstract}

%We present an Artificial Student, ``Artie,'' for engineering science disciplines in which the governing mathematical model is a partial differential equation (PDE); in this first embodiment, Artie considers the particular case of undergraduate-level steady heat conduction. Artie accepts problem statements as posed in natural language to our (actual) students. Artie provides a mixed symbolic-numeric approximate problem solution: the PDE field --- here, the temperature; scalar Quantities of Interest (QoI), expressed  as functionals of the PDE field --- here (say), the heat transfer rate over a prescribed surface. The problem statement will typically not provide explicit guidance as to the particular equation or approximations which should be invoked: the student, and Artie, must deduce an appropriate approximate solution approach from the stated geometry, boundary conditions, physical properties, and QoI. We also present Artie+, who further provides the finite element solution to the PDE: the exact solution to the problem within a prescribed (tight) tolerance as enforced by an asymptotic {\it a posteriori} error estimator.
%
%Artie comprises four principal technical ingredients: 1) Natural Language Processing. We proceed in two stages: in the the first stage we apply the general (domain-independent) Google Natural Language syntax analyzer; in the second stage we apply a frame-specific ``conduction parser'' of our own conception. 2) PDE Template. The form of the governing PDE is exploited by  the conduction parser  to extract geometry, boundary conditions, coefficients, and sources to yield a well-posed problem; any subsequent approximation can be deduced from information included in this ground-truth description. 3) Problem Classes and Geometry Classes; Components and Systems. A problem class is defined by requirements on spatial domain, boundary conditions, properties, and QoI; associated to each problem class are several parametrized geometry classes. A component is realized by instantiation of the geometry class for prescribed values of the geometric and PDE parameters; a particular system (associated to a problem statement) is represented as an assembly of components connected at compatible faces. 4) Variational Formulation and Approximation. We consider the weak statement and associated minimization principle (for our coercive symmetric PDE) both to formulate the PDE and also to develop suitable approximations and in some cases QoI bounds; implementation proceeds through static condensation and direct stiffness assembly over component port degrees of freedom.
%
%We describe a prototype implementation of Artie and Artie+, and we provide several examples of problem statements and associated approximate (and exact) problem solutions. 

We present an Artificial Student, ``Artie,'' for engineering science disciplines in which the mathematical model is a partial differential equation (PDE); Artie considers here the particular case of steady heat conduction. Artie accepts problem statements posed in natural language. Artie provides a symbolic-numeric approximate solution: the PDE field; scalar Quantities of Interest (QoI), expressed  as functionals of the field. The problem statement will typically not provide explicit guidance as to the equation or approximations which should be invoked. We also present Artie+, who provides the finite element solution to the PDE: the exact solution to within a prescribed  tolerance controlled by an {\it a posteriori} error estimator.

Artie comprises four technical ingredients: Natural Language Processing: We proceed in two stages:  domain-independent Google Natural Language syntax analyzer followed by frame-specific conduction parser. PDE Template: The PDE is exploited by the conduction parser  to extract geometry, boundary conditions, and coefficients; subsequent approximations are deduced from this ground-truth description. Problem Classes, Geometry Classes; Components, Systems: A problem class places requirements on spatial domain, boundary conditions, properties, and QoI; associated to each problem class are several geometry classes. A component is an instantiation of the geometry class for prescribed  geometric and PDE parameters; a system is represented as an assembly of connected components. Variational Formulation: We consider the weak statement and minimization principle to formulate the PDE and develop suitable approximations; implementation proceeds through static condensation and direct stiffness assembly over component ports.

We describe and illustrate a prototype implementation of Artie and Artie+.
\end{abstract}

\section{Motivation}

We present the software in the guise of an artificial student, and it is thus appropriate to briefly summarize our perspective on undergraduate education  in the engineering sciences.

\subsection{Pedagogical Background}
\label{subsec:PB}

Traditional fields of continuum mechanics, such as conduction heat transfer or linear elasticity, are well understood: constitutive laws and conservation principles yield a well-posed governing partial differential equation (PDE). Nevertheless, education in these  disciplines remains challenging: the abundance of behaviors for different geometries, materials, and boundary conditions is matched only by the paucity of corresponding closed-form (hence indisputable) solutions.

Engineers and engineering students hence typically --- or at least classically --- replace the governing PDE with a look-up table: the dictionary comprises a few canonical problem classes and associated closed-form approximation procedures; any given problem is modified such that the perturbed problem conforms to a ``nearest'' canonical problem class; the perturbed problem is then amenable to closed-form approximate solution. We can view the process conceptually as a Voronoi tesselation of problem space in which the canonical problem classes serve as generators. The resulting (symbolic closed-form) approximate problem solutions are very useful for prediction and design: transparent, accessible, and rapidly evaluated.

Numerical solution of the governing PDE of course offers much higher accuracy in particular for problems far from any canonical problem class. Not suprisingly, PDE computation now plays the central role in industry research, development, and design. Yet we continue to teach (implicitly) the Voronoi approach in most of our undergraduate engineering subjects, and for good reason: re-purposed closed-form approximations retain relevance even in the digital era, most notably for (i) conceptual and preliminary design --- motivated by the expense and relatively slow response of PDE solvers, and  (ii) verification of numerical computations --- motivated by the opacity of PDE solvers and associated software. Closed-form approximations are particularly valuable for the detection of blunders committed in the preprocessing, processing, or postprocessing stages of simulation.

\subsection{The Artificial Student} 

We introduce in this paper an Artificial Student, ``Artie,'' for engineering science disciplines in which the governing mathematical model is a partial differential equation; in this first embodiment, Artie considers the particular case of undergraduate-level steady heat conduction. 
\begin{itemize}
\item[] Artie accepts problem statements as posed in natural language to our (actual) students. {\em In this first paper, we consider natural language processing but not image processing; the problem statement text is suitably elaborated to provide geometric information.} 
\item[] Artie provides an approximate problem solution: (i)  the PDE field --- here, the temperature; (ii) scalar Quantities of Interest (QoI), expressed as functionals of the PDE field  --- here (say), the heat transfer rate over a prescribed surface. {\em In this first paper we shall only consider problem statements which conform to some problem class, and hence we eliminate the source of error associated with problem perturbation.}
 \end{itemize}
Note we consider only analysis (or forward) problems, not inverse or design problems.
 
We emphasize that the problem statement will typically not provide explicit guidance as to the particular equation or approximations which should be invoked: the student, and Artie, must deduce an appropriate approximation procedure based on the stated geometry, boundary conditions, physical properties, and QoI. We also note that the actual student and Artie need not, and typically will not, pursue the same inference or approximation procedure; however, we do require that the actual student and Artie will each arrive ultimately at a valid solution --- and typically the same solution --- to any given problem within a particular problem class.

At present Artie is a rather rudimentary prototype. But ultimately, upon subsequent development and elaboration, we envision several roles for Artie:

\begin{enumerate}
\item (education) Artie can provoke. A successful Artie, subject to the same grading standards as our actual students, should receive an ``A'' in an undergraduate engineering science subject. Should we thus reconsider what we teach our students? Or how we assess our students?

\item (education) Artie can provide learning assistance. Artie can reveal to students intermediate results in symbolic form. The student can thus identify not only a flaw in their approach but also possibly the source of the error. (Artie can not necessarily provide the student with {\em procedural guidance}, since the actual student and Artie may pursue different albeit equivalent approximation approaches.)  

\item (education) Artie can provide teaching assistance. Artie can serve as ``surrogate student'' in the development of problems for assignments and exams. In particular, Artie can detect problem statements which are ambiguous or misleading; Artie can also identify (underlying) mathematical models which are ill-posed.

\item (professional practice) Artie can provide blunder detection. Artie's closed-form approximate solutions can serve to identify blunders in PDE numerical solution procedures: incorrect input specifications, insufficiently refined discretizations, and errors in interrogation. (We emphasize ``blunders'':  the error in Artie's approximate solution must be sufficiently less than the error in the numerical solution.)

\end{enumerate}
Artie can also serve, through the development process, in the abstraction of approximation procedures.

In this paper we also describe Artie+. Given a problem statement, Artie+ provides Artie's closed-form symbolic-numeric approximate problem solution but also the finite element (FE) solution to the PDE. The latter is the exact solution to the problem statement within a prescribed tolerance as enforced by an asymptotic {\it a posteriori} error estimator: we control the error in the PDE field in an appropriate energy norm; we also directly control  the error in the QoI. In this paper we impose tight tolerances such that the FE solution is effectively exact.

Our prototype of Artie+ is even more rudimentary than our protype of Artie, and in particular we address at present only problems defined over two-dimensional domains (hence infinite or insulated in the third direction). However, ultimately, we envision several roles for Artie+:

\begin{enumerate}
\item (education) Artie+ can provide supplemental  information and perspective. Artie+ can confirm --- or refute --- the relevance of Artie's closed-form approximate solution through direct comparison with the exact (PDE numerical) solution. Artie+'s rendering of the PDE field can furthermore highlight the source of any substantial discrepancies.

\item (professional practice) Artie+ can provide more flexible and more rigorously certified engineering analyses. Artie's closed-form approximations can serve for rapid evaluation; Artie+'s PDE numerical solution can be invoked, as necessary, for confirmation. (Note that since the closed-form prediction and the PDE numerical solution share data and code, Artie+ is not necessarily reliable as a blunder detector.)
\end{enumerate}
We also note that, in some sense, Artie+ provides a natural language pre-processor for PDE numerical solution.
\section{Technical Approach}

\subsection{Principal Ingredients and Related Work}

There are four main ingredients to Artie and Artie+:

\begin{enumerate}
\item {\em Natural Language Processing}. We consider a two-step approach to syntax analysis similar to the architecture described in \cite{twopart_parser}: in the first stage we apply the general (domain-independent) Google Natural Language syntax analyzer \cite{Google_NLP} to identify tokens, parts-of-speech, and verb tense; in the second stage we apply a domain-specific syntax analyzer of our own conception, which we denote a (steady) {\em conduction parser}. The latter treats any given problem statement in terms of a heat conduction ``frame''  of reference \cite{frames1,frames}. 

At present, Artie has no real intelligence: the conduction parser learns from instances only through the intermediary of the code developer, who expands and modifies the parsing algorithm as needed for each new problem statement encountered; in principle, increasingly few (and ultimately no) modifications are required as the training suite is expanded. In actual fact, our interest is less in natural language than in non-unnatural language: Artie can still well serve the purposes cited above even if we restrict syntax, and in particular discourage complicated sentence structures apt to confuse; the latter are arguably poor scientific and pedagogical practice in any event.

\item {\em PDE Template}. The conduction parser is certainly informed by the particular context and language of conduction heat transfer \cite{Lienhard}. But more generally the conduction parser is informed by the underlying PDE which describes heat conduction: we take the viewpoint that any rational approximation must ultimately derive from the ground truth; we thus (implicitly) complete a PDE template as the first and crucial step in the approximate problem solution process. The PDE Template corresponds to data structures which characterize the problem spatial domain $\Omega$, the type of boundary conditions over different parts of the spatial domain boundary $\partial \Omega$, the surface source terms, any volumetric source terms, the coefficients which appear in the weak statement of the PDE, and the type of QoI; these specifications are collectively constrained to ensure a well-posed problem \cite{QandV}. We note that in most cases the template considers the abstract mathematical form such that Artie would require relatively little modification to treat different physical disciplines governed by (currently) second-order elliptic PDEs. For example, the PDE Template boundary condition types Dirichlet, Neumann, and Robin correspond respectively to the problem statement specifications of ``temperature,'' ``flux,'' and ``heat transfer coefficient.''

In this context we mention two earlier related efforts. We first discuss the FEniCS project \cite{FEniCS}. The goals of Artie+ are similar to the goals of FEniCS: a natural language interface for description and ultimately solution of PDEs. However, Artie considers the natural language of engineers, a modestly expanded form of English, whereas FEniCS considers the natural language of mathematicians --- a precise mapping from symbols to interpretation. Conversely, in \cite{alg_wordproblems}, the authors do consider solution of mathematical problems posed in (truly) natural language: \cite{alg_wordproblems} addresses word problems which yield small systems of algebraic equations. In fact, perhaps surprisingly, Artie's task is simpler than the goals pursued in \cite{alg_wordproblems}: the identity of our variables is readily extracted from the specific (heat conduction) context.


\item {\em Problem Classes and Geometry Classes; Components and Systems}. As described earlier, Artie and Artie+ will only accept a problem statement which belongs to an anticipated {\em problem class}: the criteria for membership involve the geometry and topology, the boundary conditions, the physical properties, and the QoI. Associated to each problem class we define several parametrized {\em geometry classes}. 

The problem class and geometry class are coupled through the notion of a {\em component}: a component is an instantiation of the geometry class for prescribed values of the geometric parameters and the (local restriction of the) PDE parameters. Components provide several advantages: components are a convenient fashion by which to construct systems with discontinuous coefficients, as arise frequently in engineering analysis; components often refer to actual building blocks encountered in engineering practice. Components are endowed with (i) faces for assignment of boundary conditions and connection with other components, and (ii) ports for representation of degrees of freedom. A {\em system} associated to a particular problem statement is an assembly of components connected at compatible faces. 

At present, Artie treats two problem classes, which indeed are sufficient to treat a majority of undergraduate steady heat conduction questions:

\begin{itemize}
\item[I] Quasi-1d Systems.  This {\em problem class} places requirements on geometry and parameters: geometry in (the axial coordinate) $x$ is homogeneous such that the cross-sectional area $A$ and perimeter $P$ are constant; the Biot number, Bi $\equiv h_{\max} (A/P)/k_{\min}$, is small compared to unity, where $h_{\max}$ and $k_{\min}$ correspond to the maximum heat transfer coefficient and minimum thermal conductivity of the system; all standard boundary conditions are supported; the QoI may take the form of temperature anywhere in the domain, and flux or heat transfer on either axial face of $\partial \Omega$.  We associate to this problem class a single {\em geometry class}: right cylinder with specified cross-section shape; at present, we consider only rectangular cross section, with two geometric parameters, however more cases can readily be included by minor expansion of the class definition. A {\em component} has (i) three faces, two rectangular axial faces and a lateral face, and (ii) two ports, which correspond to the axial faces. This problem class can be readily extended to geometry which is no longer homogeneous in $x$ but only slowly varying in $x$, however we must then relax ``closed-form'' to include numerical solution of {\em ordinary} differential equations.

Most notably, quasi-1d systems are relevant to one-dimensional walls and, more importantly, thermal fins \cite{Lienhard}; the latter constitute a highly relevant example of extended-surface heat transfer which is furthermore within undergraduate reach as regards analysis and application.

\item[II] Generalized Walls. This {\em problem class} places requirements on geometry, boundary conditions, and QoI: the spatial domain $\Omega$ must be represented as a conforming union of bricks; all exposed surfaces are insulated except for the faces at $x_1 = x_1^{\rm left}$ and $x_1 = x_1^{\rm right}$ exposed to respective (heat transfer coefficient, fluid temperature) pairs ($h^{\rm left}$, $T^{\rm left}$) and ($h^{\rm right}$, $T^{\rm right}$); the QoI is a nondimensional heat transfer rate over all  faces at $x_1 = x_1^{\rm left}$. Here $x_1^{\rm left} = \min_{x_1 \in \bar{\Omega}}$ and $x_1^{\rm right} = \max_{x_1 \in \bar \Omega}$ define the smallest slab (in $x_1$) which contains  $\Omega$. We associate to this problem class a single {\em geometry class}: a parallelpiped with specified dimensions in $x_1$, $x_2$, and $x_3$. A {\em component} has (i) six rectangular faces, and (ii) two rectangular ports, which correspond to the $x_1 = \text{ const}$ faces, which we will denote simply ``$x_1$ faces''; we introduce only two ports and hence two degree of freedom per component in anticipation of the subsequent closed-form (1d) approximation. We note that this problem class can be readily extended to consider appropriate combinations of temperature and flux boundary conditions at $x_1 = x_1^{\rm left}$ and $x_1 = x_1^{\rm right}$.

Most notably, generalized walls are relevant to actual walls in particular intended to insulate. The generalization to brick construction then permits heterogeneous materials and composites, voids, and more complex geometries, and furthermore illustrates the effects of geometric contraction and expansion on heat flow.

\end{itemize}
We can certainly view Artie as an expert system, or more precisely a collection of expert systems associated with respective problem classes. Artie differs from current heat transfer expert systems (typically focused on heat exchangers, for example \cite{Afgan,Cochran}) in a number of ways: Artie admits inputs in natural language; Artie's problem classes are defined by a very high-dimensional parameter space; Artie's set of ``rules'' is  relatively complicated and often implicit; and finally, Artie relies relatively weakly on a knowledge or experiential database (we discuss the latter in Section \ref{subsec:CP}). But we must also emphasize that Artie treats academic undergraduate problems which are very simple compared to the industrial heat exchanger problems considered in \cite{Cochran,Afgan}.

\item {\em Variational Formulation and Approximation}. We consider the weak statement and associate minimization principles both to formulate the PDE and also to develop suitable approximations \cite{QandV}.  The variational formulation offers several advantages: natural boundary conditions are very easily accommodated; systematic approximation is possible by consider of appropriate (sub)spaces; the minimization principle offers in many cases useful lower and upper bound constructions \cite{bounds_us,bounds_Elrod}, as discussed further in Section \ref{sec:Ex} example \texttt{wall-3d}; and finally, we can pursue direct stiffness assembly \cite{Bab} at the component level to automatically form the system matrix. We note that prior to direct stiffness assembly we perform (symbolically) static condensation \cite{Wi_IJNME} to restrict the degrees of freedom to the ports.

As indicated earlier, actual students and Artie will typically follow different procedures to derive the temperature field and associated QoIs. In particular, whereas actual students might invoke the strong form, thermal resistance concepts, conservation of energy (network Kirchoff Laws), and equivalent resistance, Artie pursues a variational formulation with suitable approximation spaces. It would be possible to enhance Artie to translate variational results (after the fact) into ``student form.'' 
\end{enumerate}

\subsection{Conduction Parser}
\label{subsec:CP}

We provide a few details of the conduction (PDE) parser, in particular to demonstrate the relatively generic nature of the procedure. We recall that the conduction parser takes as input the Google Natural Language syntax analysis: the tokens, parts-of-speech, and verb tenses associated with the problem statement. In actual practice, we call the Google Natural Language parser several times in order to disambiguate the original problem statement.

\begin{enumerate}
\item {\em Syntax Preparation}. The conduction parser first modifies the Google syntax analysis to reflect the technical nature of our frame. As but one example, ``normal'' can be a noun, not just an adjective, in engineering science analysis. We also introduce our own escape characters to signal symbol and equation delimiters and also symbol and equation tokens.

\item {\em Entities, Snippets, and Attributes}. We next identify {\em entities}, compound nouns (typically) followed by adpositional phrases. We then isolate {\em snippets}: simple subject-predicate-object (or -complement) sequences expressed in terms of entities and a present-tense verb. We then derive, from the snippets, {\em attributes} associated with each entity. We also perform some disambiguation operations, in particular given that we must perform the syntax analysis for the entire problem statement and not just individual sentences;  for example, ``handle of the spoon" and ``spoon handle'' refer to the same entity and are coalesced. 

\item {\em Commonsense Incorporation}. We then perform a similar syntax analysis not on the problem statement but rather on a ``commonsense database'' which serves all problems in all problem classes.  The latter includes information, largely non-technical, which we would expect an actual student to know, for example ``air is a gas", or ``a spoon is a solid object". We then append commonsense (entity, entity attributes) as problem statement entity attributes according to certain subset and  inclusion rules; for example, commonsense sentence ``A spoon is a solid object.'' is incorporated as problem statement (entity = ``spoon'', attribute = ``a solid object''); the commonsense syntax tree is then discarded. 

\item {\em Solid and Fluid Entities}. We can now identify solid entities and fluid entities, a subset of which will ultimately define our components and then degrees of freedom. In fact, Artie can also deduce solid or fluid ``state'' from other diagnostics, such as connection; in future we will take advantage of multiple predictors to provide more robust inference.

\item {\em Inheritance and Instantiation}. We next look for ``inheritance'' words to identify parent-child relationships; for example, ``a spoon consists of a head and handle'' yields inheritance pairs ``spoon-handle'' and ``spoon-head''. These inheritance pairs then serve to pass attributes from parent to children; for example, ``head'' and ``handle'' are now labeled as solid objects even though ``head'' in isolation is not evidently a solid object. 

In a similar fashion we search for ``instantiation'' indicators, for example ``each'', to identify archetypes. We then perform a second attribute incorporation procedure now over the problem statement entities; ``spoon'' and ``spoon geometry'' are separate entities, however we associate ``spoon geometry'' attributes to ``spoon'' as well; even more importantly, attributes of (say) archetype ``brick'' --- for example, geometry class as identified by particular keywords --- are then automatically adopted by instantiations ``brick 1'', ``brick 2'', ``brick 3''. 

In engineering analysis we often construct systems through assembly and instantiation, and it would be laborious or even impossible --- and distinctly un-natural --- to  individually specify attributes clearly shared by many parts.

\item{\em Connection}. We next search for ``connection'' words, for example ``a wall {\em separates} inside air and outside air'', to construct a system graph: the nodes are entities, and the edges constitute possible heat transfer paths. In future the system graph might be derived from a figure and image processing, however at present we rely on text analysis. 

\item {\em Component Identification}. A  component is defined as an entity which satisfies the following four conditions: (i) the entity --- interpreted as a node of our connection graph --- is of degree $> 0$, (ii) the entity is solid, (iii) the entity is {\em not} an ``insulator'' or made of insulating material, and (iv) the entity is {\em neither} an inheritance parent entity {\rm nor} an instantiation archetype entity.

\item {\em Coordinate Variables and Spatial Domains}. We next extract coordinate variables and the spatial domains associated with each component; we can then construct the system spatial domain through component face connectivity information. Note all processing is symbolic, and hence there are no issues related to numerical precision in the identification of topology and geometry. We can subsequently develop the usual local-to-global mapping from component and local port to unique global degree of freedom (associated to coincident local ports). In future, spatial domains may be deduced or corroborated from a figure, but at present we are limited to text analysis.

\item {\em Boundary Conditions and QoI}. We then identify boundary conditions  through structured searches of entity attributes and snippets informed by appropriate keywords --- ``temperature'', ``heat flux'', and ``heat transfer coefficient''. Note a boundary condition such as heat transfer coefficient requires not just a heat transfer coefficient but also a fluid temperature, each of which might appear in different sentences within the problem statement.  QoI, such as temperature, heat flux, and heat transfer rate, may be found in a similar fashion with additional criteria related to ``find'' words.

\item {\em Physical Properties}. For steady conduction the only relevant physical property is thermal conductivity. We can identify thermal conductivity either directly (as in the examples in this paper) or through attributes which contain ``material'' information; in future, we will also include a material properties database in the same fashion as the commonsense database.

\item {\em Parameter Values}. Finally, we search for equalities which do not involve the coordinate variables in order to isolate the numerical parameter values associated to our symbolic variables; the former are important (i) to determine parameter regime and hence problem class, quite independent of numeric evaluation of the solution, but also (ii) to instantiate numerical solutions for particular physical systems of interest. 

\end{enumerate}
At this stage all information is available for approximate solution of the PDE (Artie) and FE solution of the PDE (Artie+). Note the approximate solution is expressed in terms of a (static-condensation) basis directly deduced from the problem class and component specifications; the corresponding stiffness matrix and load vector are then easily evaluated symbolically, and the system matrix formed from the local-to-global mapping. 


We conclude with several remarks. First, with only few and small exceptions, most of the preceding steps are not overly specific to heat conduction, and could be readily extended to other continuum disciplines. Second, as already indicated, the connection graph and component spatial domains would typically be deduced by (actual) students from a figure. Absent image processing at present, we resort to a verbal explanation. In fact, Artie prepares from the verbal explanation figures with the connection graph and geometry, and thus we could reserve the graph and geometry text solely for Artie as pre-processor and then supply the students with the corresponding figures. Ultimately, however, Artie must graduate to include image processing and character recognition as well. Third, and finally, in this first embodiment of Artie we do not explicitly include units: all quantities are given implicitly in SI units (note temperature is in degrees Celsius, not Kelvin). In the future, we will include explicit units in the problem statements, however in this first embodiment we prefer to test Artie without the rather heavy-handed clues provided by units; Artie's performance will only improve once we can include units in our inference procedure. In fact, units are often only associated with numerical values, and hence units should only serve as corroboration.  


\section{Examples}
\label{sec:Ex}

We precede the examples with a few details related to implementation. Artie and Artie+ are implemented in MATLAB \cite{matlab}: MATLAB supports the string, symbolic, and numeric processing required by Artie and Artie+. A Google Natural Language client is hosted within MATLAB \cite{urlread2,PhGNL} to permit ready and interactive query. In principle Artie can provide symbolic results for the PDE field and QoI; in practice, the MATLAB-produced symbolic expressions are not tractable or enlightening for more than a single component, and thus we typically resort to numerical evaluation of our symbolic component stiffness matrices prior to assembly and inversion. The finite element (FE) procedures of Artie+ are implemented within MATLAB by the \texttt{fem2d} template package \cite{Masafem2d} which in turn relies on the \texttt{DistMesh} \cite{Per} mesh generator; we consider  simple two-level (asymptotic)  error estimators \cite{Bab} which inform a \texttt{fem2d} Nearest Vertex Bisection (NVB) adaptive refinement procedure.\footnote{The Artie and Artie+ software is available ``as-is'' free-of-charge under an open-source license. All required third-party packages except the Google Natural Language Processor are also available free-of-charge under an open-source license; the Google Natural Language Processor, for Artie's modest requirements, is very inexpensive. We emphasize that Artie is a research code not intended (in the present version)  for production within the context of actual subject deployment.}

We now turn to three examples. The first two examples exercise problem class Quasi-1d; the third example illustrates problem class Generalized Wall. In all cases we consider either heat transfer coefficient or zero-flux boundary conditions, though Artie is also instrumented to treat temperature and non-zero flux boundary conditions.  We present here the problem statements in latex format, but for full disclosure we include in the appendix the plain-text (verbatim) input files actually processed by Artie; the latter are adorned with our particular escape characters and any special allowances related to the Google Natural Language Processor or subsequent MATLAB symbolic manipulation.

%All types of boundary conditions are not yet exhaustively tested and hence we restrict attention to heat transfer coefficient boundary conditions and zero flux (insulated).  QoIs are not yet exhaustively tested, and hence we do not emphasize the QoI portion of Artie's solution. Artie can still be confused by certain compound subject-predicate clauses, and we thus prefer independent clause separate by semi-colon. Finally, we present here the questions in  The FE approach is...

\subsection{Example 1: \texttt{wall-1d}}. 

We present the problem statement:

\begin{itemize}
\item[]
A composite wall separates inside air and outside air. The inside air is maintained at temperature $T_{\rm in}$; the outside air is maintained at temperature $T_{\rm out}$. The composite wall comprises three layers: a fir layer, a pine layer, and a cedar layer; the pine layer is in between the fir layer and the cedar layer. The fir layer is exposed to the inside air; the cedar layer is exposed to the outside air. Figure 1 is Artie's depiction of the heat transfer paths.

The composite wall is a right cylinder with rectangular cross-section of dimensions $a$ and $b$; the coordinate through the wall is $x$. The fir layer is of length $L_{\rm f}$; the pine layer is of length $L_{\rm p}$; the cedar layer is of length $L_{\rm c}$. The spatial domain of the fir layer is $0 < x < L_{\rm f}$; the spatial domain of the pine layer is $L_{\rm f} < x < L_{\rm f} + L_{\rm p}$; the spatial domain of the cedar layer is $L_{\rm f} + L_{\rm p} < x < L_{\rm f} + L_{\rm p} + L_{\rm c}$. Figure 2 is Artie's construction of the wall geometry.

Let $h_{\rm in}$ denote the heat transfer coefficient from inside air to fir layer prescribed over the  face at $ x = 0 $; let $h_{\rm out}$ denote the heat transfer coefficient from cedar layer to outside air prescribed over the face $ x = L_{\rm f} + L_{\rm p} + L_{\rm c} $. The fir layer, pine layer, and cedar layer are insulated on the lateral faces.

The thermal conductivity of the fir layer is $k_{\rm f}$; the thermal conductivity of the pine layer is $k_{\rm p}$; the thermal conductivity of the cedar layer is $k_{\rm c}$.

Plot the temperature distribution as a function of $x$. You may use the following parameter values: $T_{\rm in} = 23$, $T_{\rm out} = 0$, $a = 0.1$, $b = 0.1$, $L_{\rm f} = 0.05$, $L_{\rm p} = 0.1$, $L_{\rm c} = 0.05$, $h_{\rm in} = 10$, $h_{\rm out} = 100$, $k_{\rm f} = 0.2$, $k_{\rm p}= 0.1$, and $k_{\rm c} = 0.05$. $\Box$
\end{itemize}
We observe that this problem is in problem class Quasi-1d, and in fact the problem as stated is exactly one-dimensional. The geometry class is right cylinder with rectangular cross-section; in practice, we can provide aliases and hence refer to  ``slab'' rather than right cylinder. (The geometry description is mathematically informal, but context (problem class) serves to fill in the blanks.) We note that ``plot'' is mandatory within problem class Quasi-1d, and hence the first sentence of the last paragraph is parsed by Artie but no inference is required. We do not request a QoI.

Artie's solution is summarized by (i) Figure 1 and Figure 2 referenced in the question statement (but generated by Artie), here respectively Figure \ref{fig:wall-1d_fig1} and Figure \ref{fig:wall-1d_fig2}, and (ii) the plot of the temperature field, here Figure \ref{fig:wall-1d_fig3}. Note this problem has three components corresponding to the three layers of wood; the (children) layers inherit the geometry class of the (parent) wall. The problem is relatively simple both for the students and also for Artie. Note, however, that the actual students and Artie would proceed in different fashions: the student would typically consider a network of convection and conduction resistances in series and apply voltage divider relations; Artie considers the weak formulation with piecewise linear functions within each component.

\begin{figure}[h!]
\begin{center}
 \includegraphics[scale=0.24]{wall1d_fig1.png}
\end{center}
\caption{Artie's depiction of the heat transfer paths for problem \texttt{wall-1d}.}
\label{fig:wall-1d_fig1}
\end{figure}

\begin{figure}[h!]
\begin{center}
 \includegraphics[scale=0.3]{wall1d_fig2}
\end{center}
\caption{Artie's construction of the geometry for problem \texttt{wall-1d}.}
\label{fig:wall-1d_fig2}
\end{figure}

\begin{figure}[h!]
\begin{center}
 \includegraphics[scale=0.3]{wall1d_fig3}
\end{center}
\caption{Artie's prediction of the temperature field for problem \texttt{wall-1d}.}
\label{fig:wall-1d_fig3}
\end{figure}

\subsection{Example 2: \texttt{spoon}}. 

We present the problem statement:

\begin{itemize}
\item[]
A cup contains tea. Air is in contact with the tea; also air surrounds the cup. The tea is maintained at temperature $T_{\rm liq}$; the air is maintained at temperature $T_{\infty}$. The tea - air interface is at $ x = 0 $. A spoon comprises two parts: a head connected to a handle. The head rests in the cup. The head is immersed in the tea; the handle is exposed to the air. Figure 1 is Artie's depiction of the heat transfer paths.

The spoon geometry is approximated as a right cylinder with rectangular cross-section for dimensions $a$ and $b$; the axial coordinate is $x$. The head is of length $L_1$; the handle is of length $L_2$. For our coordinate system, the head extends from $ x = - L_1 $ to $ x = 0 $, and the handle extends from $ x = 0 $ to $ x = L_2 $. Figure 2 is Artie's construction of the spoon geometry.


Let $h_1^{\rm bot}$ denote the heat transfer coefficient from head to tea prescribed over the axial face at $ x = - L_1$; let $h_2^{\rm top}$ denote the heat transfer coefficient from handle to air prescribed over the axial face $x = L_2$; let $h_1^{\rm lat}$ denote the heat transfer coefficient from head to tea prescribed over the head lateral face; let $h_2^{\rm lat}$ denote the heat transfer coefficient from handle to air prescribed over the handle lateral face.

The head has thermal conductivity $k_1$; the handle has thermal conductivity $k_2$; the cup is an insulator.

Plot the temperature as a function of the axial coordinate. You may use the following numerical values: $ a = 0.002$, $b = 0.01 $, $ L_1 =  0.05 $, $L_2 = 0.12$, $h_1^{\rm bot} = 10.0$, $h_2^{\rm top} = 5$, $h_1^{\rm lat} = 10.0$, $h_2^{\rm lat} = 5.0$, $ k_1 = 50 $, $ k_2 = 50$, $T_{\rm liq} = 90$, and $T_\infty = 23$. $\Box$
\end{itemize}
We observe that this problem is in problem class Quasi-1d; the geometry class is right cylinder with rectangular cross-section. We recall that ``plot'' is mandatory within problem class Quasi-1d, and hence the first sentence of the last paragraph is parsed by Artie but no inference is required; the command is primarily for the actual students. We do not request a QoI since in this case the most relevant output --- the temperature of the spoon at $x = L_2$ (as sensed by a finger) --- can be deduced from the field plot.

Artie's solution is summarized by (i) Figure 1 and Figure 2 referenced in the question statement (but generated by Artie), here respectively Figure \ref{fig:spoon_fig1} and Figure \ref{fig:spoon_fig2}, and (ii) the plot of the temperature field, here Figure \ref{fig:spoon_fig3}; note Artie confirms that the Biot number is small, as required by the problem class. This problem has two components corresponding to the head and the handle; the (children) head and handle inherit the geometry class of the (parent) spoon. This problem is less straight-forward: we do not indicate which equation or which limit should be considered. (In fact, most undergraduate students could only solve this problem for the case in which $h_1^{\rm lat} = h_2^{\rm lat}$.)  Again the actual students and Artie would proceed in different fashions: the student would typically invoke the already-derived solution of the 1d fin equation as provided for example in \cite{Lienhard}; Artie considers the weak formulation with appropriate $\sinh$ and $\cosh$ (statically condensed) basis functions, as well as a bubble function, within each component.

\begin{figure}[h!]
\begin{center}
 \includegraphics[scale=0.24]{spoon_fig1}
\end{center}
\caption{Artie's depiction of the heat transfer paths for problem \texttt{spoon}.}
\label{fig:spoon_fig1}
\end{figure}

\begin{figure}[h!]
\begin{center}
 \includegraphics[scale=0.3]{spoon_fig2}
\end{center}
\caption{Artie's construction of the geometry for problem \texttt{spoon}.}
\label{fig:spoon_fig2}
\end{figure}

\begin{figure}[h!]
\begin{center}
 \includegraphics[scale=0.3]{spoon_fig3}
\end{center}
\caption{Artie's prediction of the temperature field for problem \texttt{spoon}.}
\label{fig:spoon_fig3}
\end{figure}

\subsection{Example 3: \texttt{wall-3d}}. 

We present the problem statement:

\begin{itemize}
\item[] A wall separates inside air and outside air. The wall consists of four bricks: brick 1, brick 2, brick 3, and brick 4. Brick 1 connects to brick 2; brick 2 also connects to brick 3 and brick 4. Brick 1 is in communication with inside air at temperature $T_{\rm in}$; Brick 2, Brick 3, and Brick 4 are in communication with outside air at temperature $T_{\rm out}$. Figure 1 is Artie's depiction of the heat transfer paths.

The coordinates are denoted $x_1$, $x_2$, and $x_3$; $x_1$ corresponds to distance through the wall. Each brick is a parallelepiped of rectangular cross-section of dimensions $a$ (in $x_2$), $b$ (in $x_3$), $L$ (in $x_1$). The spatial domain of Brick 1 is $0 < x_1 < L, 0 < x_2 < a, 0 < x_3 < b$; the spatial domain of Brick 2 is $L < x_1 < 2L, 0 < x_2 < a, 0 < x_3 < b$; the spatial domain of Brick 3 is $L < x_1 < 2L, 0 < x_2 < a, b < x_3 < 2b$; the spatial domain of Brick 4 is $L < x_1 < 2L, 0 < x_2 < a, - b < x_3 < 0$.  Figure 2 is Artie's construction of the wall geometry.

Each brick has thermal conductivity $k_{\rm b}$.

Brick 1 is exposed to inside air over the face at $x_1 = 0$ through heat transfer coefficient $h_{\rm in}$. Brick 2, Brick 3, and Brick 4 are exposed to outside air over the faces at $x_1 = 2L$ through heat transfer coefficient $h_{\rm out}$. The remainder of the boundary is insulated.

We introduce a nondimensional heat transfer rate $H$ given by $Q/(k_b(T_{\rm in}-T_{\rm out})a)$; here $Q$ denotes the heat transfer rate into Brick 1 over the face at $x_1 = 0$. Develop a lower bound and also an upper bound for $H$. You may use the following parameter values: $T_{\rm in} = 23$, $T_{\rm out} = 0$, $a = 0.1$, $b = 0.1$, $L = 0.05$, $h_{\rm in} = 10$, $h_{\rm out} = 100$, and $k_b = 0.5$. $\Box$
\end{itemize}
We observe that this problem is the problem class Generalized Wall; the geometry class is parallelepiped. The QoI $H$ and associated lower and upper bounds are mandatory with the problem class, so the first two sentences of the last paragraph are parsed by Artie but no inference is required; the QoI request is primarily for the actual students.

Artie+'s solution is summarized by (i) Figure 1 and Figure 2 referenced in the question statement (but generated by Artie), here respectively Figure \ref{fig:wall-3d_fig1} and Figure \ref{fig:wall-3d_fig2}, (ii) the display of the lower and upper bounds for $H$, here provided in Figure \ref{fig:wall-3d_fig4}, and (iii) the plot of the FE temperature field, here Figure \ref{fig:wall-3d_fig3}. As regards the latter, the geometry is 3d but insulated in $x_2$ and hence the temperature field is in fact 2d and thus amenable to our (currently) two-dimensional finite element treatment. This problem has four components corresponding to Brick $n$, $1 \le n \le 4$; Brick $n$, $1 \le n \le 4$, are instantiations of the archetype brick and inherit from the archetype the geometry class and also conductivity (any conductivities provided explicitly in the problem statement for a particular Brick $n'$ would take precedence).\footnote{Note the \texttt{wall-3d} problem statement is somewhat inconsistent as regards capitalization of proper bricks; Artie is not confused.} This problem is not too difficult for students {\em if} we also include a hint --- as we would typically do in practice --- for the bound construction: for the lower bound, insulator cuts are inserted on $\bar \Omega \cap \{x_3 = 0\}$ and $\bar \Omega \cap \{x_3 = b\}$, whereas for the upper bound, a superconductor cut is inserted on $\bar \Omega \cap \{x_1 = L\}$; in both cases, the resulting heat flow is 1d and may be treated by resistances in series and parallel. (Note $\{x_1 = \text{ const}\}$ is shorthand for $\{ x \in \RR^3 \,| \, x_1 = \text{ const}\}$.) We need not provide these hints to Artie; Artie can deduce the bound construction from the connectivity information.

We elaborate upon Artie's approach and in particular the variational formulation. In this analysis we assume that $h_{\rm in} > 0$ and $h_{\rm out} > 0$, and also, as always, $k_{\rm b} > 0$. We also introduce several standard function spaces: $L^2(\Omega)$ is the space of functions which are (Lebesgue) square integrable over $\Omega$; $H^1(\Omega) \equiv \{ v \in L^2(\Omega)\,|\, |\nabla v| \in L^2({\Omega})\}$. For our problem (and more generally problem class Generalized Wall) we can evaluate $H$ as
\begin{align}
H = \frac{1}{k_{\rm b}(T_{\rm in} - T_{\rm out})a}\int_{\partial \Omega_0} h_{\rm in} (T_{\rm in} - T)\,dA \;,  \label{H1}
\end{align}
where $T$ refers to the solution of our PDE for the prescribed boundary conditions; note $\partial \Omega_0 \equiv \{ x \in \partial \Omega \, | \, x_1 = 0\}$ and, for future reference, $\partial \Omega_{2L} \equiv \{ x \in \partial \Omega \, | \, x_1 = 2L\}$. We can also express $H$ as a minimum:
\begin{align}
H = C_1(\min_{w \in X} 2 J(w) + C_2)\; , \label{H2}
\end{align}
where $C_1$ and $C_2$ are positive constants,
\begin{align}
C_1 = \frac{1}{k_{\rm b}(T_{\rm in} - T_{\rm out})^2 a}\; , \label{C1}
\end{align}
\begin{align}
C_2 =  \int_{\partial \Omega_0} h_{\rm in} T_{\rm in}^2 + \int_{\partial \Omega_{2L}} h_{\rm out} T_{\rm out}^2 \; , \label{C2}
\end{align}
$X = H^1(\Omega)$, and $J: X \times X \rightarrow \RR$ is the functional
\begin{align}
J(w) = \frac{1}{2} \calA(w,w) - \calF(w) \; ; \label{J}
\end{align}
here $\calA$ and $\calF$ are respectively the (symmetric, coercive) bilinear and linear forms associated to our PDE,
\begin{align}
\calA(w,v) \equiv \int_{\partial \Omega_0} h_{\rm in} w v\, dA + \int_{\partial \Omega_{2L}} h_{\rm out} w v dA + \int_\Omega k_{\rm b} \nabla w \cdot \nabla v\,dV  \; , \label{calA}
\end{align}
and
\begin{align}
\calF(v) \equiv \int_{\partial \Omega_0} h_{\rm in} T_{\rm in} v\, dA + \int_{\partial \Omega_{2L}} h_{\rm out} T_{\rm out} v\, dA\; . \label{calF}
\end{align}
We note that $T \in X$ satisfies the weak form: $\calA(T,v) = \calF(v), \forall v \in X$.

Artie now proceeds to identify the $x_1$ locations of the brick (component) $x_1$ faces in the system, $\{x_{1}^j\}_{j = 1,\ldots,m_1}$, and associated domain slices $\calS^j \equiv \text{ (the interior of) } \bar \Omega \cap \{x_1 = x_1^j\}$; for \texttt{wall-3d},  $m_1 = 3$ with $x_1^j = (j-1)L, 1 \le j \le m_1$, and $\calS^1 \equiv \{x_1 = 0, 0 < x_2 < a, 0 < x_3 < b\}$, $\calS^2 \equiv \{x_1 = L, 0 < x_2 < a, -b < x_3 < 2b\}$, $\calS^3 \equiv \{x_1 = 2L, 0 < x_2 < a, -b < x_3 < 2b\}$. We also identify a minimal set of parallelepipeds, $\{\calP^j\}_{j = 1,\ldots,m_{\rm ppd}}$, such that $\bar \Omega = \cup_{j = 1}^{m_{\rm ppd}} \bar \calP^j$: each parallelepiped is a (sub)assembly of bricks with connections only on brick (component) $x_1$ faces; for \texttt{wall-3d}, $m_{\rm ppd} = 3$ with $\calP^1 \equiv \{0 < x_1 < 2L, 0 < x_2 < a, 0 < x_3 < b\}$, $\calP^2 \equiv \{L < x_1 < 2L, 0 < x_2 < a, b < x_3 < 2b\}$, $\calP^3 \equiv \{L < x_1 < 2L, 0 < x_2 < a, -b < x_3 < 0\}$. We then introduce two additional spaces: 
\begin{align}
X^{\rm UB}= \{ v \in H^1(\Omega)\,| \, v|_{\calS^j} = \text{ Const}^j, j = 1,\ldots,m_1\} \; , \label{XUB}
\end{align}
and
\begin{align}
X^{\rm LB} = \{ v \in L^2(\Omega)  \, | \, v|_{\calP^j}  \in H^1(\calP^j), j = 1,\ldots,m_{\rm ppd}\} \; . \label{XLB}
\end{align}
We make two remarks on the upper-bound space, \eqref{XUB}. First, for any given $j$, the constant Const$^j \in \RR$ that appears in the constraint is {\em part} of the solution, hence unknown {\it a priori} --- the constraint imposes uniformity over $\calS^j$, {\em not} a prescribed (Dirichlet) value. Second, if any $\calS^j$ is disconnected --- not the case for \texttt{wall-3d} --- we can improve the upper bound space: we permit not a single constant Const$^j$ associated to the entire slice $\calS^j$ but rather a different constant for each {\em connected part} of slice $\calS^j$. We also make a remark on the lower-bound space, \eqref{XLB}: for any parallelepiped $\calP$ with no Robin  faces we must additionally require that the field satisfy a zero-mean condition over $\calP$. We can then construct the quantities
\begin{align}
H^{\rm UB} = C_1(\min_{w \in X^{\rm UB}}2J(w) + C_2) \; , \label{HUB}
\end{align}
and
\begin{align}
H^{\rm LB} = C_1(\min_{w \in X^{\rm LB}} \sum_{j = 1}^{m_{\rm ppd}} 2 J|_{\calP^{j}}(w) + C_2) \; , \label{HLB}
\end{align}
which, from simple variational arguments, satisfy $H^{\rm LB} \le H \le H^{\rm UB}$. Note that $J|_{\calP^j}$ refers to $J$ restricted to $\calP^j$.

The implementation of the bounds is very simple, and thus within Artie's reach even without any knowledge of variational methods. We discuss the upper bound; the lower bound admits even simpler treatment. The exact minimizer associated with \eqref{HUB} is in fact piecewise-linear in $x_1$ (and independent of $x_2$ and $x_3$); Artie can thus deduce the discrete equations directly from the weak form over $X^{\rm UB}$ applied to basis functions which are piecewise linear over each component. The only subtlety, easily addressed, is the constraint in $X^{\rm UB}$: for each $j \in \{1,\ldots,m_1\}$, the global degrees of freedom associated to all ports on  (component) $x_1 = x_1^j$ faces are simply coalesced into a single degree of freedom --- a revised local to global mapping from component and local port to unique global degree of freedom --- prior to direct stiffness assembly over components.





\begin{figure}[h!]
\begin{center}
 \includegraphics[scale=0.24]{wall3d_fig1}
\end{center}
\caption{Artie's depiction of the heat transfer paths for problem \texttt{wall-3d}.}
\label{fig:wall-3d_fig1}
\end{figure}

\begin{figure}[h!]
\begin{center}
 \includegraphics[scale=0.3]{wall3d_fig2}
\end{center}
\caption{Artie's construction of the geometry for problem \texttt{wall-3d}.}
\label{fig:wall-3d_fig2}
\end{figure}

\begin{figure}[h!]
\vspace{.2in}
\begin{mdframed}
\begin{center}
 \includegraphics[scale=0.7]{Artie_wall3d_QoI}
\end{center}
\end{mdframed}
\vspace{-.1in}
\caption{Artie's lower and upper bound for $H$ for problem \texttt{wall-3d}; Artie+'s FE prediction for $H$ and associated {\it a posteriori} error estimator.}
\label{fig:wall-3d_fig4}
\end{figure}

\begin{figure}[h!]
\begin{center}
 \includegraphics[scale=0.3]{wall3d_fig3}
\end{center}
\caption{Artie+'s FE prediction of the temperature field for problem \texttt{wall-3d} over any $x_2 = \text{ const}$ plane.}
\label{fig:wall-3d_fig3}
\end{figure}

\section{Future Work}

There are a number of short-term targets of opportunity: improvement of Artie's parsing skills in particular for compound predicates; generation of more examples including different boundary conditions and various QoI; development of new geometry classes such as (for problem class Quasi-1d) axisymmetric and spherical configurations, and more generally small-curvature filaments; extension to other fields of heat transfer (unsteady conduction, radiation), to linear elasticity and structural mechanics, and to fluid dynamics and heat convection.

We also envision more challenging tasks. First, we would like to extend Artie to consider {\em problem perturbation} as discussed in Pedagogical Background, Section \ref{subsec:PB}: given a problem statement, find a ``closest'' problem class and associated parameters; the latter is best conducted in concert with the development of {\em image processing} techniques to deduce connections and geometry. Important considerations include the choice of norm and the scoring algorithm; QoI bounds can in certain cases play an important role. Second, we would like to initiate a first simple form of learning, {\em commonsense learning}: Artie can pose questions (``Is tea a solid?") the answer to which can then be incorporated into the commonsense database for use in future problems. Third, we would like to incorporate component-based {\em model-order reduction} techniques \cite{Enu} into Artie+'s PDE numerical solution capabilities in particular to reduce response time for truly three-dimensional problems.

Finally, in the much longer term, we would need to develop an uber-Artie who could develop Artie software for a particular discipline from associated textbook material: {\em textbook learning}. We would continue to rely on the PDE Template but now at a higher level: the PDE Template would serve to guide Artie's knowledge acquisition and subsequent development of discipline-specific parsers. In the even longer term, Artie could incorporate archival material for example related to material properties, heat transfer coefficient correlations, and other empirical information.

\section{Acknowledgments}

I would like to thank Dr Phuong Huynh of Akselos, SA, for fruitful exchanges about Project Artie and in particular for his development and generous distribution of the \texttt{GNLRequest} function for direct client access to the Google Natural Language Processor through MATLAB;  the latter greatly facilites syntax analysis tightly integrated with symbolic and numeric computation. I would like to thank Professor Masayuki Yano of University of Toronto for his pedagogical and mathematical insights into Project Artie and furthermore for the development and generous distribution of his \texttt{fem2d}  templates for  adaptive finite element solution of partial differential equations. I would also like to thank Professor Ed Greitzer of MIT for his comments on Artie in education, and Professor Reza Malek-Madani of the US Naval Academy for his encouragement to pursue Project Artie. Project Artie is supported by MIT through a Ford Professorship.




%We cite this article \cite{Enu, Wi_IJNME}.

\footnotesize
\bibliographystyle{plain}	%
\bibliography{bib_folder/Artie}




\newpage
\appendix
\section{Problem Statements: Inputs to Artie+}

\includepdf[pages=1,pagecommand=\subsection{Example 1: \texttt{wall-1d}}, offset = 0 -3cm]{wall-1d_problem.pdf}
\includepdf[pages=1,pagecommand=\subsection{Example 2: \texttt{spoon}}, offset = 0 -3cm]{spoon_problem.pdf}
\includepdf[pages=1,pagecommand=\subsection{Example 3: \texttt{wall-3d}}, offset = 0 -3cm]{wall-3d_problem.pdf}

%\includepdf[pages=2-,pagecommand={}]{wall-1d_problem.pdf}

%\subsection{Example 1: \texttt{wall-1d}}
%
%\includepdf{wall-1d_problem.pdf}


\end{document}

\begin{center} {\LARGE {White Paper}}\\{\Large (Confidential)}\\[1em] {\Large {\bf The Artifical Student}}\\[1.2em]
% {\Large Reza Malek-Madani, USNA and ONR}\\ 
{\Large Anthony T Patera, MIT}\\[.5em] {\large \today} \end{center}\mbox{}\\[1em]

\noindent Acknowledgments: I would like to thank Professor Reza Malek-Madani (United States Naval Academy) and Professor M Yano (University of Toronto) for helpful discussions on related topics. 

\begin{abstract}

Traditional fields of continuum mechanics, such as conduction heat transfer or linear elasticity, are well understood: constitutive laws and conservation principles yield a well-posed governing partial differential equation (PDE). Nevertheless, education in these  disciplines --- and also engineering application --- remains challenging: the abundance of behaviors for different geometries, materials, and boundary conditions is matched only by the paucity of corresponding closed-form (hence indisputable) solutions.

Engineers and engineering students hence typically --- or at least historically --- replace the governing PDE with a look-up table: a few important closed-form scenarios comprise the canon; any given problem is formulated as a modification to a nearest canonical scenario; intuition serves to estimate the effect of the problem perturbation on Quantities of Interest. We can view the process conceptually as a Voronoi tesselation of problem space in which the canonical scenarios serve as generators. The resulting (symbolic) mathematical models are transparent, accessible, and rapidly evaluated, and hence very useful for prediction and design. 

However, numerical solution of the governing PDE offers much higher accuracy in particular far from canonical scenarios. Not suprisingly, PDE computation now plays a central role in industry research, development, and design. Yet we continue to teach the classical canon in most of our undergraduate engineering subjects, and (largely fortuitously) {\em for good reason}: re-purposed lower-fidelity models retain relevance even in the digital era, most notably for verification of numerical computations largely opaque to the engineer. Can we provide a more rigorous foundation and also software embodiment for this important Voronoi framework?

Recent advances in numerical computation are matched or surpassed by parallel progress in artificial intelligence and related classification schemes. In principle, the latter could well replicate the cognitive processes associated with the classical Voronoi approach to engineering analysis: semantic parsing of the question statement followed by identification and appropriate modification of the nearest canonical scenario. We would then be in a position to place (artificial) neural and digital engineering computation on a common platform --- and hence consider powerful synergistic approaches.

Any software embodiment of the Voronoi framework must match the performance of our undergraduate students. As a concrete and perhaps provocative first step we ask the following question:
\begin{itemize}
\item[] Can we develop software --- an Artificial Student --- which would earn an A in an undergraduate heat conduction subject, or more generally in any given MIT undergraduate subject in a basic engineering science discipline?
\end{itemize}
The Artificial Student would be trained independently and offline from (and prior to) the natural students but would be assessed based on the same materials and in the same fashion as the natural students.

The implications of the answer are potentially profound: 
\begin{enumerate}
\item How would an affirmative answer change education --- what we teach our students, and how we assess our students?
\item How would an affirmative answer change professional profiles and practices?
\item How could we combine the comparative advantages of (artificial) neural and digital processing in novel ways to improve the engineering analysis process and in particular reduce uncertainties?
\end{enumerate} 
The last question suggests research topics which would incorporate natural language processing, artificial intelligence algorithms, computational geometry, the mathematics of PDEs, numerical methods and numerical analysis for PDEs, model order reduction, and engineering design. 

%As one example, we might consider the development of a neural network trained on PDE numerical solutions  to predict the sign of the error in a lower-fidelity model Quantity of Interest. In some situations we can prove the sign of the error through variational arguments, but in even more cases we can intuit the result. A digital-neural approach could supplement intuition with rigor and lead to very rapid and transparent predictive approaches.   


\end{abstract}

\section{Project Objective}
\label{sec:objective}

\subsection{Motivation}
\label{subsec:motiv}

This research project is focused on the development of integrated analysis systems for engineers. The concepts and methods developed should be equally applicable across many engineering science disciplines: solid mechanics and structures, fluid dynamics and heat transfer, acoustics and vibration. However, for purposes of concreteness, this white paper considers a particular (and particularly simple) topic: conduction heat transfer. 

Although our project is intended to impact engineering practice, our goals are best motivated in the context of current engineering undergraduate education --- the preparation our students receive for ultimate engineering practice. We consider a particular actual (sub)subject: 2.051C is the conduction portion of a broader undergraduate heat transfer course taught at MIT over the last few years. Note here that ``course'' denotes the syllabus, teaching materials, and assessment.\\

\noindent {\em Conduction Heat Transfer}. For the purposes of 2.051C, we define conduction heat transfer rather narrowly: prediction of the steady or time-dependent temperature distribution and associated heat fluxes and heat transfer rates in (and on the surface of) a solid body. We emphasize two restrictions in scope:
\begin{enumerate}
\item We consider only conduction; we do not address either convection or radiation. Effects due to convection and radiation are included in (linear, or linearized) boundary conditions on the solid body boundary with all relevant constitutive constants provided to the student. The student is responsible for the thermophysical properties of the solid body:  thermal conductivity, density, and specific heat (or thermal diffusivity).\footnote{In 2.051C we consider  temperature-dependent properties very superficially: the properties are assumed uniform over the solid body; the latter is reasonably accurate for smaller temperature differences.}
\item We consider only prediction: analysis of the forward problem; we do not address design synthesis, or inverse problems more generally. We do note, however, that the analysis exercises are often relevant to --- in support of --- design or operations, for example the evaluation of Quantities of Interest (QoI) related to performance objectives and safety contraints.

\end{enumerate}
Conduction heat transfer, in particular steady heat conduction, enjoys special energy principles (and also maximum principles) which can be exploited, explicitly or implicitly; we take advantage of these principles, also available say in linear elasticity, but in a limited fashion: much of the methodology presented should be broadly applicable, perhaps with some loss of rigor.

Prediction comprises two ingredients: a mathematical model; the solution, or approximate solution, of the mathematical model. These two ingredients are of course related, in that a mathematical model is only relevant to the extent that (sufficiently accurate approximate) solutions can be obtained with reasonable resources.\\

\noindent {\em Exact Mathematical Model: Partial Differential Equation (PDE)}. Within the designated scope we can reasonably define an ``exact" mathematical model associated to any given problem: the First Law of Thermodynamics and Fourier's Law translates into a linear coercive symmetric elliptic (respectively, parabolic) partial differential equation for the steady (respectively, time-dependent) temperature field; the typical boundary conditions of temperature, heat flux, and heat transfer coefficient translate into Dirichlet, Neumann, and Robin conditions (we also consider a few less standard conditions); continuity of temperature and heat flux translate into the standard interface conditions; we admit as sources distributed volumetric heat generation; for time-dependent problems, specification of the temperature provides the requisite initial condition. Note that in 2.051C we do invoke intuition derived from tensor conductivity considerations, but we do not formally develop the associated PDE.

We can view this mathematical model, for any given problem $\calP$, as a map $\calM_\calP: \calD_\calP \rightarrow T(t) \in V(\Omega), F(t) \in \RR^{N_{\rm QoI}}$. Here $\calD_\calP$, the problem specification, represents the spatial domain $\Omega$, the properties and related coefficient functions, the boundary conditions, and (for time-dependent problems) the initial conditions; $T(t)$ is the temperature field and $F(t)$ is a vector of $N_{\rm QoI}$ scalar Quantities of Interest (QoI) which are evaluated as prescribed functionals of $T(t)$; $t$ is time in some interval $(0,t_{\rm final}]$ (to be disregarded in the steady case); and $V(\Omega)$ is the (affine) function space in which the solution $T(t)$ resides. We consider a weak formulation in space in order to readily include discontinuous thermal conductivity. Note  that $\Omega$ is the spatial domain associated with the mathematical model, and may thus be either two-dimensional or three-dimensional even if the actual physical system is always three-dimensional. 

We shall restrict attention to problems which are well-posed: question statements which, in conjunction with readily available material property data, contain sufficient information to completely specify a problem $\calP$ such that $\calM_\calP$ represents a well-posed PDE: $T(t)$ exists and is unique.\footnote{Within the 2.051 context, the only worrisome case is steady conduction for all-Neumann boundary conditions, however the solvability conditions fortunately admits a simple interpretation in terms of energy conservation.} As stated, we consider QoI functionals at any given time $t$, however we may also consider QoI which correspond to space-time functionals of the temperature. The latter can resolve certain classical issues related to short-time regularity of the heat equation for ``rough'' initial conditions.

Numerical solution for this particular class of PDEs is very well developed, and adaptive techniques in conjunction with {\it a posteriori} error estimation provide both efficiency and rigor. We shall denote the numerical approximation to the temperature and QoI by $T_\delta(t) \in V_\delta(\Omega) \subset V(\Omega)$ (where $V_\delta(\Omega)$ is an affine subspace of $V(\Omega)$) and $F_\delta(t)$, respectively; here $\delta$ refers to the discretization parameter, for example the diameter of the underlying finite element (FE) spatial triangulation and time step of an associated finite difference temporal grid. We presume convergence (in appropriate norms) as $\delta \rightarrow 0$  perhaps subject to stability conditions.

Numerical PDE solution does not currently play a role in 2.051C: the requisite computational methods are beyond student capabilities; access to third-party PDE packages remains cumbersome; and the necessary software training is of little general educational value. Instead, 2.051C emphasizes prediction based on lower-fidelity mathematical models: these models require less mathematical preparation and in particular permit more transparent --- symbolic --- derivation and justification; symbolic results are also more readily integrated in higher-level (design) engineering analyses. 

In the future, numerical solution of PDEs should be, and can be, better integrated into 2.051C and other subjects in continuum mechanics. However, these more advanced analysis techniques for PDE mathematical models will not displace the current 2.051C emphasis on more transparent analysis of lower-fidelity mathematical models.  Indeed, the availability of PDEs in 2.015C will only re-inforce the importance of lower-fidelity models, albeit in some different contexts (and learning outcomes) than originally anticipated in the pre-computational era.\\

\noindent {\em Lower-Fidelity Mathematical Models}. Closed-form solution of the PDE is only possible in very special separable or similarity geometries or particular asymptotic limits of the parameters. We can take advantage of these closed-form (symbolic) solutions to develop lower-fidelity models. At present we include in 2.051C the following exact, or asymptotically exact, closed-form solutions, which we denote canonical scenarios:
\begin{enumerate}
\item Composite Wall (steady), expressed as thermal resistances in series.
\item Thermal Fin (steady, asymptotic).
\item Lumped Body (unsteady, asymptotic).
\item Semi-Infinite Body (unsteady).
\item Dunking Transient (unsteady).
\item Dimensional Scaling (steady, unsteady).
\end{enumerate}
In each case we can determine the conditions on $\calD_\calP$ such that $\calP$ belongs to a particular scenario $\SS$, and derive the associated (nondimensional) map $\calM_\SS$ from $\calD_\calP$ to temperature field and fluxes (in terms of which we can then evaluate any desired QoI). This short list of scenarios is reasonably comprehensive at least in part due to the design contexts to which engineers and nature respond.

Nevertheless, it is rare that any given problem exactly satisfies any of the scenario criteria. We thus require a crucial pre-processing step, in which we approximate, more precisely modify, our problem specification to conform to a ``nearby'' scenario. In particular, given any problem $\calP$ and associated specifications $\calD_\calP$, our lower-fidelity mathematical model can be expressed as a map $\calM_{\tilde \calP}: \calD_{\tilde \calP} \rightarrow \tilde T(t) \in \tilde V(\tilde \Omega), \tilde F(t) \in \RR^{N_{\rm QoI}}$. Here $\calD_{\tilde \calP}$ represents the geometry (denoted by $\tilde \Omega$), properties, and boundary conditions for a problem $\tilde \calP$ perturbed from problem $\calP$, $\tilde T(t)$ and $\tilde F(t)$ denote the associated lower-fidelity model prediction for the temperature and QoI, respectively, and $\tilde V(\tilde \Omega)$ is the new space for the temperature and (in homogeneous form) test function in the weak form; here $\tilde V (\tilde \Omega)$ incorporates the form for the lower-fidelity temperature distribution.\footnote{In fact, in 2.051C, the solution to the lower-fidelity model in fact satisfies the exact PDE over $V(\tilde \Omega)$. However, for some problem modifications $\calD_\calP \rightarrow \calD_{\tilde \calP}$ the temperature perforce resides in a subpace  $\tilde V(\tilde\Omega)$. For example, an infinite conductivity in a particular spatial coordinate implies temperature dependence only in the orthogonal spatial coordinates. Hence from the student perspective $\tilde V(\tilde\Omega)$ is not related to model definition but rather to model solution; in this way we avoid notions of projection.The space $\tilde V(\tilde \Omega)$ also plays a role, behind the scenes, in rigorous proof of QoI error information.} In many cases it is possible to argue convincingly from thermal resistance concepts, or to invoke rigorously energy principles (in the steady case), to deduce the sign of (some components of) $F - \tilde F$, which we shall refer to as ``QoI (error) sign information.'' 

We emphasize the goal: given the problem of interest, $\calP$, we wish to find a modified (or ``perturbed'') problem $\tilde \calP$ such that 
\begin{itemize}
\item[(i)] $\calD_{\tilde \calP}$ is somehow close to the original problem specification, $\calD_\calP$; {\em and}
\item[(ii)] $\tilde \calP$ conforms to the requirements of one of the canonical scenarios; 
\end{itemize}
the first requirement addresses accuracy; the second requirement addresses tractability. As regards the latter, we note that  $\calD_{\tilde \calP}$ and $\tilde V(\tilde \Omega)$ conspire to ensure that we can easily obtain a relatively simple closed-form solution for $\tilde T$ and $\tilde F$ and, in many important cases, also develop a rigorous derivation of QoI sign information.

In general the canonical scenarios in 2.051C are sufficiently different that, for any $\calP$, the most relevant scenario, $\SS$, is readily identified. However, there are many choices for $\calD_{\tilde \calP}$. For selection purposes we can assign scores in two fashions: 
\begin{itemize}
\item[(a)] the specification-distance score, given by $\| \calD_{\calP} - \cal D_{\tilde \calP} \|_{\rm spec}$; or 
\item[(b)] for cases in which we have QoI sign information, the $\tilde F$-value score, which is simply the value of (some components of) $\tilde F$. 
\end{itemize}
For the specification-distance score, lower is better; for the $\tilde F$-value score, lower (respectively, higher) is better for $F - \tilde F$ non-positive (respectively, non-negative). Here $\| \cdot \|_{\rm spec}$ is an appropriate distance metric on the problem specification: spatial domain, properties, and boundary conditions; a good norm will reflect the induced errors (say, in the steady case) $$\| T - \tilde T \|_{V(\Omega \cap \tilde \Omega)},\; \| F - \tilde F \|_{\ell^\infty}\; .$$
For elliptic and parabolic problems, the difference between $\Omega$ and $\tilde \Omega$ is perhaps reasonably measured by the Hausdorff distance; the bests norms for properties and boundary conditions will be informed by standard PDE theory, though limits of zero and infinite conductivity require special attention. We shall discuss a third scoring option, more supervised, in a subsequent section. 

Lastly, it remains to address the ways in which reasonable perturbations $\calD_{\calP} \rightarrow  \calD_{\tilde \calP}$ might be effected. Geometry modification is most challenging, but with many options:  find best fits to standard geometries (spheres, right cylinders) which appear in the canonical scenarios; eliminate small features, as motivated by Saint Venant; exploit separation of length scales; apply small-curvature limits; reduce spatial dimensionality, for example based on aspect ratio arguments or cross-section averaging. Property modification most often takes the form of either (i) homogenization, or (ii) superconductor and insulator volumetric material or manifolds (``cuts''). Boundary condition modification often takes the form of heat transfer coefficient lower or upper limits. We refer to these different approaches collectively as problem modification procedures: functions which take $\calD_\calP$ as input and yield $\calD_{\tilde \calP}$ as output.\\

\noindent {\em Assessment: Modeling Questions}. We can categorize questions posed to the students (in recitation and lecture, homework assignments, and exams) as either ``mechanical questions'' or ``modeling questions.''  Mechanical questions  specify the lower-fidelity mathematical model and the student is asked to exercise the associated closed-form symbolic solution; we shall not consider mechanical questions further here. Modeling questions present a physical system and associated environmental conditions and the student must himself/herself {\em develop} and assess an appropriate lower-fidelity mathematical model --- effectively $\calD_{\tilde \calP}$ ---  and only then proceed to apply the associated closed-form symbolic solution in context. We always ask for justification of assumptions, approximations, and derivations; the latter is summarized in the usual ``show your work." (In many cases we grant the student assumptions which reduce the problem complexity.) In 2.015C we often choose systems and artifacts related to everyday experience so that students can easily perform informal companion experiments.

We provide a short example (hence not much engineering context) of a modeling question of average difficulty: ``A spoon is placed in hot tea: find the temperature of the tip of the spoon handle. You may assume that the tea remains uniformly at temperature $T_{\rm tea}$ in space and time over the period of interest, that the part of the spoon exposed to air is subject to a uniform natural convection heat transfer coefficient of 5 ${\rm Watts}/({\rm m}^2{\rm -}{}^\circ{\rm C})$, and that the temperature distribution in the spoon reaches steady state. Consider both a stainless steel spoon and also a silver spoon. Provide a symbolic answer and also a numerical answer for $T_{\rm tea} = 90{}^\circ {\rm C}$. Show your work and justify all assumptions. In the event that you have access to a silver spoon, confirm experimentally your qualitative conclusions.'' The geometry of the spoon (of assumed uniform rectangular cross-section) would be provided to the students, perhaps with an attendant sketch.

Modeling questions, as compared to mechanical questions, are more difficult for students to solve but also more difficult for instructors to grade. It does not suffice to require only that all assumptions are justified: the latter ultimately requires some quantification, or very simple but grossly inaccurate solutions can not be rejected. In this sense, though not currently implemented in 2.051C explicitly, we should prescribe an error tolerance for the QoIs. The latter could be verified by the instructor based on $T_\delta$ and $F_\delta$ (and associated {\it a posteriori error estimators}), and thus in principle modeling questions would admit rigorous assessment. Alternatively, and more consistently with the current 2.051C emphasis, a poor lower-fidelity model (poor choice, say, of geometry modification) can be unmasked by comparison with a better model and, if possible, appeal to QoI sign information; the latter is an application of the $F$-value score. In actual practice most students --- or most students who attend lecture and recitation --- provide the anticipated solution: the solution for which the grading template has been prepared.\\

\noindent {\em Lecture Content}. We summarize the lecture content as an indication of the teaching materials available to the natural student:
\begin{enumerate}
\item The First Law of Thermodynamics and Fourier's Law: assumptions; statement; associated thermophysical properties.
\item Newton's Law of Cooling: heat transfer coefficient; phenomenology; orders of magnitude; approximate application to radiation.
\item The governing PDEs: derivation from the First Law of Thermodynamics and Fourier's Law; boundary conditions (temperature, flux, heat transfer coefficient); interface conditions (continuity of temperature and heat flux, contact resistance); initial conditions; local conservation property. 
\item Uniqueness of solutions: application to dimensionality reduction for insulated boundaries or large-aspect-ratio spatial domains.
\item Thermal resistance: conduction and convection one-dimensional resistance elements; series and parallel connections; networks of thermal resistances; connection to PDE and QoIs; equivalent resistance; methods for the derivation of lower and upper bounds for the equivalent resistance and related QoIs. We provide intuitive justification for the latter in 2.051C; in fact, rigorous justification is often possible based on energy arguments. We note that thermal resistance networks are equivalent to finite-volume or finite-difference approximations of the governing equations, although this interpretation is not emphasized in 2.051C.
\item Canonical scenarios: problem statement; requirements on $\calD_\calP$; exact (or asymptotic) solution for temperature and fluxes; non-dimensional form.
\item Low-Fidelity Models: examples (``supervised learning'') to demonstrate identification of scenario, problem modification and hence choice of $\tilde \calD_{\tilde \calP}$, development of QoI sign information, and (informally) scoring to assess error. 

In practice we introduce and develop these models for the students by (control volume) consideration of the First Law of Thermodynamics and Fourier's Law; the latter are essentially the strong form associated to our weak lower-fidelity mathematical model. Most often this strong statement takes the form of an algebraic system of equations or an ordinary differential equation, though occasionally a particularly simple PDE is retained. We emphasize that neither the weak form nor the energy principle ever appears in the 2.051C material; conservation of energy, the strong form of the PDE, and intuition (based on resistance concepts) suffices.
\end{enumerate}
In actual practice the course groups content first for steady problems and then for time-dependent problems.\\

\noindent {\em Learning Outcomes; Professional Practices}. We restrict ourselves to the primary intended learning outcome of the course:
\begin{itemize}
\item[] Student can successfully and completely respond to a modeling question: identify the  problem $\calP$ and associated specifications $\calD_{\calP}$ for the relevant physical system and context; summarize the exact mathematical model; select a nearby canonical scenario $\SS$; choose problem modification strategies to develop a perturbed problem $\tilde \calP$ and associated specifications $\calD_{\tilde \calP}$; exercise the lower-fidelity model $\calM_{\tilde \calP}$ to obtain symbolic representations for  temperature field and QoI; evaluate the latter for particular numerical instances requested; confirm conservation of energy; verify units for consistency; provide as possible QoI sign information; discuss the relevance of the latter to any stated design or operation objectives, for example conservative estimate of a safety margin; justify all approximations and assumptions.
\end{itemize}
In short, the student should be able to apply conduction heat transfer concepts to real (albeit, in this undergraduate class, relatively simple) artifacts.

In the longer term, we anticipate that Student will be prepared for the following professional practices:
\begin{enumerate}

\item Student can reduce a design space to a plausible set of parameters in a compact set. Here we suppose that a design objective and constraints are specified and involve QoI related to conduction heat transfer.
    
Identification of the design space is informed by QoI predictions from lower-fidelity models; once a restricted design space is constructed, numerical PDE software can in principle be engaged.

\item Student can assess the result of a PDE numerical approximation to confirm an absence of blunders; the latter is a weak form of verification.
     
Assessment is informed by (i) visualization of the simulated field --- to confirm conservation principles, boundary conditions, and asymptotic features, and (ii) comparison of 
QoI predictions from the simulation with QoI predictions from lower-fidelity mathematical models
--- to confirm consistency.

\item Student can execute an engineering decision either in the design or operations context.
     
The decision is informed by characterization of uncertainties and the associated effect on QoIs and hence engineering objectives and constraints; lower-fidelity models and QoI sign information (and in particular conservative estimates) can play a large role, as can verified PDE numerical approximations.

\end{enumerate}

The emphasis on our single learning outcome is justified: the learning outcome is a prerequisite for most subsequent professional practices. Nevertheless, it would also be of interest to assess some professional practices directly in 2.051C (for example, Professional Practice 2); the latter would certainly be facilitated by the introduction of PDE software.


\subsection{Research Question}

We now pose the fundamental research question: 
\begin{itemize}
\item[] Can we develop software --- an Artificial Student --- which would earn an A in 2.051C, or more generally in any given MIT undergraduate subject in a basic engineering science discipline?
\end{itemize}
The Artificial Student would be trained independently and offline from (and prior to) the natural students but would be assessed based on the same materials and in the same fashion as the natural students. Sufficient archival curricular materials --- train and test --- exist to answer this research question based solely on historic data, but new deployments can serve to more rigorously remove the effects of grader variability. For succinctness, the Artificial Student will be denoted ``Artie."

The implications of an affirmative answer are profound: 
\begin{enumerate}
\item How would an affirmative answer change education --- what we teach our students, and how we assess our students?
\item How would an affirmative answer change professional profiles?
\item How could a professional version of Artie, Artie+, transform engineering practice? Artie+ would incorporate more advanced capabilities, such as  numerical PDE solution, and in particular combine neural and digital processing in novel ways. We propose several research tasks in Section \ref{sec:artie+}.
\item How, ultimately,  would an affirmative answer change the product of the engineering enterprise in terms of reduced uncertainty and improved system performance, safety, and sustainability?
\end{enumerate}
Engineers no longer create graphics, or construction drawings, by hand. Engineers no longer solve linear systems of equations by hand. Engineers (will soon) no longer create prototypes or physical models by hand. Perhaps soon engineers will no longer develop mathematical models by hand. 

\section{A First Architecture}
\label{sec:artie}

We describe here a first architecture and associated technical components. It is unlikely that most or even any of the technical aspects of this sketch will persist in subsequent actual embodiments, however a concrete example serves to better illustrate the objectives, constraints, and implications of the project. We describe the architecture in terms of the sequence of steps which would be executed by Artie upon presentation of a new modeling question. Note many of the steps involve inference engines and function (method) libraries which would be trained offline prior to Artie's enrollment in 2.051C.

\begin{enumerate}
\item {\em Problem Identification}. Artie applies a semantic parser to the question statement and a geometric (image) parser to associated graphics to identify problem $\calP$. 

A problem data structure \ttt{probname} encodes the problem specifications $\calD_\calP$: geometry (spatial domain and boundary parts), properties, boundary conditions (type and coefficients), sources, and initial conditions; sources and initial conditions are typically quite simple in heat conduction problems. The structure \ttt{probname} also contains fields related to the QoIs and the particular question posed (physical system and artifacts, engineering context). As regards geometry description, we may consider Boolean operations on standard primitives, equivalent representations by signed distance functions, and ultimately --- as derived from the latter --- full two-dimensional triangulations or three-dimensional tetrahedral meshes. We note that component descriptions can also prove useful: the spatial domain is described as an assembly of components from a library associated with a particular family of problems related to a given canonical scenario. 

At present we envision a frame-based semantic parser. However, current frames (say) in FrameNet may not contain sufficient disambiguation of technical terms, and indeed the few existing frames related to ``heat'' contain incorrect connotations; we thus envision an enhanced set of frames tailored to the characterization of heat transfer problems. The current frames may suffice for the more lay engineering terms associated with context and artifacts. We shall also consider more general natural language processing approaches. In any event, the goal is identification of entities, part-of-speech tagging, and syntax (parsing) trees, from which $\calD_\calP$ and hence \ttt{probname} may be extracted.

\item {\em Identification of Lower-Fidelity Model}. Artie passes the structure \ttt{probname} to a modeling algorithm which classifies \ttt{probname} as one of our canonical scenarios and proposes an appropriate modified problem as the lower-fidelity model.  Note the modeling algorithm takes as inputs the data structure \ttt{probname} --- geometry, properties, boundary conditions, initial conditions, and QoIs all play a role  --- as well as a general materials property database; the latter is also available to the natural students. There are three stages: 

\begin{enumerate}
\item {\em Feature Extraction}. Artie is equipped with a library of functions to calculate geometric features from the geometric descriptions of $\Omega$. Features (and associated functions) are related to scale, aspect ratio, rotation and center (origin), homogeneity, and symmetries. In this first stage, Artie applies all relevant functions to evaluate features associated with the geometry of problem $\calP$; introduces an appropriate nondimensionalization and associated nondimensional spatial coordinates, temporal coordinate (Fourier number), and parameters (notably, the Biot number, as well as geometric parameters); and includes this derived information in the data structure \ttt{probname}. 

\item {\em Scenario Selection}. Artie is equipped with a library of the conditions on $\calD_{\calP}$ associated with each of the canonical scenarios $\SS$. In this second stage, based on the elaborated structure \ttt{probname}, Artie identifies the best scenario by which to describe $\calP$. In general the geometric features, properties (and numerical values of the properties, as reflected in nondimensional parameters), boundary conditions, and initial conditions should suffice to identify the most appropriate scenario. (In some cases two scenarios could be pursued, for example a Lumped Body and Dunking Transient for a low-Biot sphere; we can readily extend the algorithm to consider several scenarios.)

It might be the case for simple problems that this second-stage selection will identify an exact fit with a particular scenario $\SS$: $\calD_\calP$ satisfies the criteria for scenario $\SS$, and we thus take $\calM_\SS$ as our low-fidelity (in fact, high-fidelity) model. Note that for asymptotic canonical scenarios, for example the Thermal Fin for low Biot number, we consider the fit exact if a certain threshold criterion is satisfied. More generally, no exact fit will be found, and the third stage will be necessary.

\item {\em Model Identification}. Artie is equipped with a library of functions which implement the problem modification procedures described in Section \ref{subsec:motiv}; for example, a function may replace different conductivities with a single conductivity, the minimum, maximum, or average; another function may replace $\tilde \Omega$ as the best fit to the geometry associated with the relevant scenario. Note in all cases these functions return a specification $\calD_{\tilde \calP}$ which conforms to the scenario criteria. In this third stage, Artie applies all relevant functions to generate a suite of modified problems $\tilde \calP$ characterized by $\calD_{\tilde \calP}$; these proposals are then ranked according to either the specification-distance score or the $F$-value score. Artie may pursue several low-fidelity mathematical models in particular to obtain complementary QoI sign information: two models for which respectively $F - \tilde F$ is non-positive and $F - \tilde F$ is non-negative (for a given component of $F$).

\end{enumerate}

\item {\em Analysis.} Artie passes the elaborated \ttt{probname} to an analyzer algorithm. The analyzer instantiates the relevant scenario for the particular problem specification(s) $\calD_{\tilde \calP}$, solves (symbolically) any necessary sets of equations for the QoI, and prepares a report. The report states and justifies any assumptions associated with the selected lower-fidelity model; provides the symbolic form for $\tilde T$ and $\tilde F$ (for each lower-fidelity model retained); evaluates the model for particular numerical values provided; discusses available QoI sign information; confirms global energy conservation; verifies dimensions and units; and finally, addresses any specific questions related to the engineering context. 

Artie submits his report. The report is graded based on the same template and procedure applied to a natural student.

\item {\em Grade Maximization}. Artie shows up at office hours and complains about its grade.
\end{enumerate}
It is clear that the above summary leaves many unanswered questions, however successes in related fields would suggest feasibility.

We might ask why we could not apply a much more general supervised learning approach: submit a large number of heat transfer modeling questions for which complete solutions are available; train a classifier. There are several reasons this is most likely infeasible: there are relatively few modeling questions in the archives (relative to ``big data''); the possible modeling questions and associated variation include a wide range of possibilities, especially related to geometry; the requisite outputs, whether symbolic of numerical, reside in a very large space, in particular since we require both QoI but also field; the requirement to show work in natural mathematical language --- arguably important in engineering decisions which involve safety ---  would be difficult to extract from (say) internal layers. In contrast, incorporation of some knowledge of the map from problem statement to PDE problem specification and PDE solution --- as we propose here --- dramatically reduces scope.

\section{Artie+: Next Steps}
\label{sec:artie+}

We indicate here some next steps which would permit Artie ($\rightarrow$ Artie+) to perform useful service in the professional context.\\


\noindent {\em Expanded Heat Conduction Canonical Scenarios}. There are other examples of closed-form solutions to the PDEs of heat conduction, for example related to uniform, line, or point sources, which can prove quite useful to analysis. These scenarios, and associated relevant problem modification functions, can be readily included in Artie. There is a large classical literature from which to draw.\\

\noindent {\em Expanded Disciplines in Continuum Mechanics and Transport}. We will consider additional topics in heat transfer: convection heat transfer; radiation heat transfer. More ambitiously, we can consider integrated treatment of conduction, convection, and radiation.

We will also consider other disciplines in mechanical engineering: prime candidates include linear elasticity and structural mechanics, acoustics, and incompressible fluid mechanics. Although the general framework developed in this white paper should be broadly applicable --- and indeed our undergraduate and even some graduate subjects follow the Voronoi template --- different disciplines will emphasize different aspects of the framework.

The notions of components and systems is a very well-developed branch of engineering analysis: components are characterized by a constitutive relation between appropriate dual variables (or ``through'' and ``across'' variables) defined over component ports; assembly is described by the connection of local ports at nodes; system equations are derived by continuity and conservation considerations (Kirchoff's Laws, suitably interpreted for the particular discipline) applied at nodes.

Component and system analysis methods are relevant to disciplines for which important engineering applications are well described by component assemblies with low-dimensional connectivity. Classical examples which  furthermore appear in undergraduate curricula include truss and beam systems (stiffness matrices) in structures, duct systems (frequency-domain transfer matrices) in acoustics, and piping and hydraulic systems (friction factors) in fluid dynamics.\footnote{Of course the best example is electrical circuits, but we restrict attention to mechanical engineering.} In these situations we must consider not just canonical components (e.g., trusses) but also canonical systems (e.g., systems of trusses); the latter require different identification, justification, and verification approaches, and hence a considerable expansion of the framework described in this white paper.\\

\noindent {\em QoI Sign Information}. There are a number of ways in which rigorous QoI sign information can be extracted, at least for problems with some underlying energy principle. It would be important to catalog the approaches and subsequently consider implementation in terms of problem modification functions.\\ 

\noindent {\em Parametrized PDE Models}. In typical design problems, and inverse problems more generally, a mathematical model is exercised over some parameter domain; parameters might relate to geometry, properties, or boundary conditions. As described above, Artie provides a symbolic expression for the solution, but validity is confirmed only for the numerical values provided in the modeling question. It is not difficult to extend Artie to address parametric variation: the modeling question would now include a parameter domain rather than a single value;  the model identification algorithm would now consider a worst-case ranking over the parameter domain. In actual practice, for larger parameter domains (say, for heat conduction, Biot number both small and large compared to unity), composite models would be required.\\

\noindent{\em Generalized Canonical Scenario.} We might also consider a relaxation of the requirements on a canonical scenario. In particular, rather than demand a closed-form symbolic solution, we might permit other formats: (i) solutions which can be expressed as an integral or the solution of a (or several) ordinary differential equation initial or boundary value problem; (ii) experimental results from the archival literature (in the form of datasets, tables, or figures) which are well vetted, and (iii) parametrized model order reduction codes with {\it a posteriori} error estimators. In all cases we would retain accessibility and in particular rapid evaluation, but we would lose some degree of transparency related to justification, and we would lose the insight associated with symbolic representation. 

This step should not be taken lightly, and would require some commitment on the part of the community to maintain verification standards. The next task (directly below), numerical solution of PDEs, further extends the diagnostic capabilities. In some sense, we can lower the standards for transparency if we include predictions --- and a framework for assessment ---  associated with different analysis approaches.\\

\noindent {\em Numerical Solution of PDEs}. The incorporation of numerical solution of PDEs is, in principle, not difficult. The data structure \ttt{probname} will contain all the necessary information to construct a volumetric mesh with associated boundary groups; adaptive mesh refinement can be pursued to obtain an efficient approximation; {\it a posteriori} error estimators can serve to impose an error tolerance. Of course for large-scale engineering systems described by CAD models this process is highly nontrivial and very time-consuming, however for smaller systems the process is not difficult. Obviously it would be in principle possible to provide pre-processing and post-processing interfaces with standard software packages.

The most obvious benefit --- an extension of the previous tasks --- is that Artie+ would now represent a complete analysis system, including a natural-language interface to PDE numerical solution. Lower-fidelity models offer transparency and easy verification, rapid response, and insight into parametric variation. However, lower-fidelity models will often be deficient in terms of accuracy, in particular as we depart from canonical scenarios. Numerical solution of PDEs offers quite generally the higher accuracy associated with the exact mathematical model. However, PDE numerical solutions do not readily (or rapidly) reveal parametric dependence. Furthermore, and more crucially, PDE numerical solutions are less transparently verified, since the pre-processing, processing, and post-processing procedures are hidden from the user. In some sense, Artie+ --- neural and digital --- would provide for the comparisons and judgements we would expect of our undergraduates. (The latter again returns to the question posed earlier: how should we prepare students if indeed many skills in our undergraduate curriculum may now be automated?) 

Incorporation of PDEs carries expanded analysis techniques. But it also suggests the development of a new family of combined neural-digital approaches to prediction. We identify here two research tasks:
\begin{enumerate}
\item
{\em Supervised Lower-Fidelity Model Selection}. Armed with PDE numerical solution capabilities, we can consider a third approach to assignment of score for model selection, the ``error score'': we simply set the score to the error in the lower-fidelity model, for example (in the steady case) $\| T - \tilde T \|_{V(\Omega \cap \tilde \Omega)}$ or more easily $\| F - \tilde F \|_{\ell^\infty}$.  We would expect with this error score to identify, for any given problem $\calP$, a best lower-fidelity model.  However, the error score is very expensive to compute.

It is thus of interest to train a model selection algorithm (for any given scenario) over some large number of training problems $\{\calP^{\rm train}_i\}_i$: in the online stage, the model selection algorithm would take as input $\calP$ and provide as output the problem modification function calls which yield the best lower-fidelity model. Parametrized model order reduction, in particular component methods, can serve to effectively generate training data for large sets of problems $\calP^{\rm train}$ perturbed about a particular scenario.

It is not clear that this selection algorithm could be effectively trained: the class of problems is large, and the output is high-dimensional; however, engineers can readily identify poor lower-fidelity models, and hence artificial neural nets (ANNs) may be able to replicate this discrimination.

\item {\em Supervised QoI Sign Information}. The availability of QoI sign information transforms a lower-fidelity model from a guideline or suggestion into a viable design tool. Furthermore, the sign of the error in the QoI constitutes a simple binary classification problem. And finally, intuition (for example, based on resistance concepts) is often very effective in deducing whether a lower-fidelity model will under-predict or over-predict a QoI, which argues that an ANN might be able to perform similar tasks, and also --- given a larger training set --- identify the characteristics of anomalous situations. 

We would thus propose to modify the model selection algorithm of the previous task to also predict the sign of the error of the QoI in the identified lower-fidelity model. This could greatly extend the class of problems for which we can obtain QoI sign information, in particular from the current (restricted) class of problems which enjoy rigorous energy principles to the larger class of problems for which ``almost everywhere'' statements may be possible.

\end{enumerate}
We have described here just a few of the research possibilities associated with the integration of neural and digital processing.






\end{document}
