% 8.22.18 cyy's o1017cyy..tex
% 8.09.18 n1017cyyunitary.tex based on Rup'v6 and Cheng's m1017...
% 8.05.18 MR's shortened version, TK's revisions marked ???..???
% 7.23.18 shortened version l1017cyyunitary.tex
% 7.03.18 slightly modified marked by ???...????
% 6.23.18 i1017cyyunitary.tex   Fermi Liquid part revised
% 6.21.18 version h1017cyyunitary.tex
% 6.16.18 g1017cyyunitary.tex
% 6.14.18 f1017cyyunitary.tex new version
% 5.24.18 e1017..version
% 11.20.17 c1017cyyunitary.tex
\documentclass[twocolumn,preprintnumbers,superscriptaddress]{revtex4}
%\documentclass[twocolumn,showpacs,preprintnumbers,showkeys,superscriptaddress]{revtex4}
%\documentclass[onecolumn,showpacs,preprintnumbers,showkeys,superscriptaddress]{revtex4}
\usepackage{mathrsfs}
\usepackage{amssymb}
\usepackage{graphicx}
\usepackage{dcolumn}
\usepackage{bm}
\usepackage{graphicx}
\usepackage{float}
\usepackage{longtable}
\usepackage{amsmath}
\usepackage{multirow}
\usepackage{setspace}
\usepackage{color}

\newfont{\largemi}{cmmi10}
\baselineskip=4mm
\newcommand{\SD}{$S\!-\!D~$}
\newcommand{\sd}{$s\!-\!d~$}
\newcommand{\larges}{{\hbox{\largemi s}}}
\newcommand{\larger}{{\hbox{\largemi r}}}
\newfont{\smallmi}{cmmi6}
\newcommand{\smallN}{{\hbox{\smallmi N}}}
\newcommand{\CG}{C^{L_{\smallN}~M_{\smallN}}_{J_{\smallN}^{}M_{\smallN}^{'},~t\sigma}}
\def\baselinestretch{1.1}
\draft


\topmargin=-50mm

\def\eqref#1{Eq.~(\ref{eq:#1})}
\def\eqlab#1{\label{eq:#1}}
\def\figref#1{Fig.~(\ref{fig:#1})}
\def\figlab#1{\label{fig:#1}}
\def\tabref#1{Table \ref{tab:#1}}
\def\tablab#1{\label{tab:#1}}
\def\secref#1{Section~\ref{sec:#1}}
\def\seclab#1{\label{sec:#1}}
\newcommand{\vslash}[1]{#1\hspace{-0.5em}/}

\voffset 3.5cm


\begin{document}

\title{Unitary limit and linear scaling of neutrons in\\
 harmonic trap with tuned CD-Bonn and square-well interactions}

\author{Yi-Yuan Cheng \footnote{yycheng@phy.ecnu.edu.cn}}
\affiliation{Department of Physics, East China Normal University, Shanghai 200241, China}

\author{T. T. S. Kuo \footnote{kuo@tonic.physics.sunysb.edu}}
\affiliation{Department of Physics and Astronomy, Stony Brook University, New York 11794-3800, USA}

\author{R. Machleidt \footnote{machleid@uidaho.edu}}
\affiliation{Department of Physics, University of Idaho, Moscow, ID 83844, USA}

\author{Y. M. Zhao \footnote{ymzhao@sjtu.edu.cn}}
\affiliation{School of Physics and Astronomy, Shanghai Jiao Tong University, Shanghai 200240, China}

\date{\today}


\begin{abstract}
We study finite numbers of neutrons in a harmonic trap at the unitary limit.
Two very different types of neutron-neutron interactions
are applied, namely, the
meson-theoretic CD-Bonn potential and hard-core square-well interactions, all
tuned to posses infinite scattering length.
The potentials are renormalized
to equivalent, scattering-length preserving low-momentum potentials,
$V_{{\rm low}-k}$,
with which the particle-particle hole-hole
ring diagrams are summed to all orders
to yield the ground-state energy $E_0$ of the finite neutron systems.
We find the ratio
$E_0/E_0^{\rm free}$ (where $E_0^{\rm free}$ denotes
the ground-state energy of the corresponding non-interacting system)
to be remarkably independent from variations of the
harmonic trap parameter, the number of neutrons,
the decimation momentum $\Lambda$ of $V_{{\rm low}-k}$,
and the choice of the unitarity potential.
Our results support a special virial linear scaling relation.
Certain properties of Landau's quasi-particles
for trapped  neutrons at the unitary limit are also discussed.
\end{abstract}

%\pacs{}

\vspace{0.2in}

\maketitle


{\bf Introduction.}
The scenario of ``unitary limit'' was originally formulated by Bertsch in 1999,
asking what will be the ground-state properties of a spin-1/2 fermion system
with an interaction of zero range and infinite scattering length \cite{Bertsch}.
With impressive advances of cold-atom experimental techniques,
such unitary Fermi systems became experimentally accessible at the atomic level,
and have attracted intensive attention.
This limit is also of great interest to nuclear physics,
because the $^1S_0$ channels
of the realistic nucleon-nucleon potentials all have
fairly large scattering lengths,
such as -18.97 fm in the meson-theoretic CD-Bonn potential \cite{CDBonn},
and have much smaller interaction ranges.


For a Fermi gas at the unitary limit, the ground-state energy $E_0$
is expected to be proportional to the energy of corresponding free gas $E_0^{\rm free}$,
i.e., $E_0=\xi E_0^{\rm free}$ with $\xi$ an universal constant shared by all unitary Fermi gases.
This can be intuitively understood as the Fermi momentum is the only relevant length scale in the system.
There have been huge efforts devoted to describing universal behaviors of unitary Fermi gases
in regard to the ground-state properties, as well as collective excitations, thermodynamic properties,
and non-equilibrium aspects, see e.g.,
Refs.~\cite{Jr1999,Heiselberg2001,Carlson2003,Perali2004,Bulgac2005,Nishida2006,Haussmann2007,Schaefer2007,Siu2008,Dong2010,Schaefer2010,exp1,exp2,exp3,exp4,exp5,exp6,exp7,exp8,exp9}.
Our previous results of unitary neutron matter \cite{Siu2008,Dong2010}
by summing the low-momentum particle-particle hole-hole
ring diagrams to all orders,
with very different unitarity potentials, give $\xi$
values all closely equal to 0.44,
which is also consistent with the quantum
Monte Carlo calculation in Ref.~\cite{Carlson2003}.
Very recently, extensive studies on few-nucleon systems
\cite{Konig2017,Kievsky2018} and neutron stars \cite{Kievsky2018}
from perspective of the unitary limit
have been carried out.

A unitary system of finite-number fermions confined in a small trap is more challenging.
Here we mention theoretical works of Refs.~\cite{Werner2006,virial,Chang2007,Blume2007}.
In Ref.~\cite{Chang2007}, the unitary systems with small particle numbers in a harmonic trap were studied
using Green's function Monte Carlo method, where the trapped small systems exhibit evident differences from extensive homogeneous gases.
Of great interest and importance is then the fundamental question whether or not
the trapped finite-number systems at the unitary limit are universally related to corresponding non-interacting systems.
In this work we find the ratio $E_0/E_0^{\rm free}$ for the trapped finite-number system at the unitary limit
is remarkably invariant with respect to the harmonic trap parameter,
the number of fermions, the decimation momentum $\Lambda$ of the low-momentum interaction
$V_{{\rm low}-k}$ \cite{vlowk1,vlowk2,vlowk3,vlowk4,vlowk5,vlowk6,vlowk7},
and the choice of the unitarity potentials.
In particular, our results support a linear scaling relation for unitary systems in a harmonic trap,
which is a consequence of the virial theorem of Ref.~\cite{virial}.
Certain regularities of trapped unitary neutrons in terms of Landau's quasi-particles \cite{MBT}
are also discerned and discussed.

\vskip 0.5cm
{\bf Formalism.}
We consider a system of neutrons trapped
in a harmonic trap with Hamiltonian $H=T+U_{\rm osc}+V$ where
$T$ represents the kinetic energy, $U_{\rm osc}$ the oscillator potential
$\sum_i \frac{1}{2}m\omega ^2 r_i^2$, and $V$ a unitary neutron-neutron
interaction which will be described later.
We shall consider both closed- and open-shell systems.
Let us consider the former first.
The ground-state (g.s.)
energy $E_0$ of this system
will be calculated
using a linked-diagram expansion \cite{Goldstone,Kuo1971,Kuo1990} of the form
\begin{eqnarray}
E_0  &=& E_0^{\rm free} + \Delta E_0~, \nonumber \\
 \Delta E_0 &=&
     \frac{ \langle\Phi_0|V U(0,t')|\Phi_0\rangle}
         { \langle\Phi_0| U(0,t')|\Phi_0\rangle}|_{t'\rightarrow -\infty}~,
     \\
%   &=&  D^{(1)} +\cdots +  D^{(n)} +\cdots \nonumber \\
 E_0^{\rm free}&=& \langle\Phi_0|U_{\rm osc}|\Phi_0\rangle
+\langle\Phi_0|T|\Phi_0\rangle \equiv U_0 + T_0~,  \nonumber
\end{eqnarray}
where $ U(t,t')$ is the time evolution operator. $E_0^{\rm free}$ is
the non-interacting g.s.~energy, and $\Delta E_0$ is given by the sum of
all the linked diagrams generated by the interaction $V$ (see Ref.~\cite{Siu2008} for details).
$\Phi _0$ is a closed-core wave function. For example, it is a shell-model
state with the shells $(0s,0p,0d1s,0f1p)$
all filled when a system of 40 neutrons is considered.
Denoting the oscillator single particle (s.p.) energy as
$\epsilon ^{(0)}_a=(2n_a + l_a +3/2)\hbar \omega$ and the top filled
s.p.~orbit of $\Phi_0$ as $k_F$,  we have
\begin{equation}
 T_0=U_0= \sum\limits_{a\leq k_F} \frac{1}{2} \epsilon^{(0)}_a~.
\end{equation}

We define the trap ratio $R_t$ for neutrons in a harmonic trap as
\begin{equation}
R_t = \frac{E_0}{E_0^{\rm free}}=\frac{U_0+T_0+\Delta E_0}{U_0+T_0}~. \label{Rt}
\end{equation}
We consider also a different ratio, an intrinsic ratio $R_i$
defined as
\begin{equation}
R_i =\frac{T_0-K_{\rm cm}+\Delta E_0}{T_0-K_{\rm cm}}~, \label{Ri}
\end{equation}
where $K_{\rm cm}$ is the kinetic energy of the center-of-mass (cm)
motion of the nucleons confined in $\Phi _0$.
We have  $K_{\rm cm}= \frac{3}{4}\hbar\omega$, since $\Phi_0$
is a shell-model closed-core state.
This ratio is based on the intrinsic kinetic energy of the neutrons,
excluding that from their cm motion, and the interaction energy  generated
by the inter-neutron interaction $V$.
In fact the ratios $R_t$ and $R_i$ are simply related by $2R_t= (1+R_i)$, if we neglect
$K_{\rm cm}$ which is generally small compared with $T_0$ when the number
of neutrons is large such as 20 or larger.
We shall study these ratios at the unitary limit, namely
as $a_s\rightarrow \infty$ where $a_s$ is the scattering length of $V$.


Let us now briefly describe how we prepare
the unitary interactions used in this work.
We start from the high-precision meson-exchange CD-Bonn potential
\cite{CDBonn}, with its $^1S_0$ scattering length $a_s$
tuned by adjusting its  meson-exchange parameters.
Note that $a_s$ of the original
CD-Bonn potential for realistic nuclear systems is already
fairly large, i.e., -18.97 fm.
By slightly tuning the mass of the $\sigma$ meson, we have obtained
a family of CD-Bonn neutron-neutron potentials with various $a_s$ values,
including those approaching $\pm\infty$~\cite{Siu2008}.
For example, we shall employ  a CD-Bonn  potential with $a_s$=-12070 fm,
which is obviously enormous compared to any length
scales in the nuclear system.

We shall also employ a family of hard-core square-well (HCSW)
interactions in our work. They are of the form
\begin{equation}
V(r)=V_c~(r\le r_c),~~V_b~(r_c < r \le r_b),~~0~(r > r_b),
\end{equation}
where $V_c,r_c,V_b$ and $r_b$ are parameters.
Since the scattering lengths of these interactions can be given analytically \cite{Dong2010},
exact unitary HCSW interactions are readily obtained.
The four potentials employed in this work
(denoted as HCSW-1, -2, -3, -4) all have their $V_c$=3000 MeV,
and their other parameters  $r_c$=(0.30, 0.15, 0.30, 0.50) fm,
$V_b$=(-50, -20, -30, -50) MeV  and $r_b$=(1.61, 2.31, 2.03, 1.81) fm
respectively for the potentials HCSW(-1, -2, -3, -4).
At the unitary limit, the ratios $R_t$ and $R_i$ of Eqs.~(\ref{Rt}, \ref{Ri})
given by different
unitary interactions should each be invariant. We shall check
if this requirement is satisfied by our calculated ratios
using the above CD-Bonn and HCSW interactions.

In our calculations of these ratios, a first step is to obtain
a low-momentum interaction $V_{{\rm low}-k}$
from the above unitary interactions.
A renormalization procedure where the high-momentum
components of such potentials are integrated out is enacted
applying well-established procedures~\cite{Siu2008,vlowk1,vlowk2,vlowk3,vlowk4,vlowk5,vlowk6,vlowk7}.
Note that $V_{{\rm low}-k}$ preserves the scattering length.
In the present work, for most cases, we use the decimation momentum $\Lambda$=2.0 fm$^{-1}$
as used in a number of nuclear structure studies \cite{vlowk3}.
We will also check whether or not the ratios are invariant with respect to the choice of $\Lambda$.


Next we shall calculate
the energy shift $\Delta E_0$ using a ring-diagram method
\cite{Siu2008,pphh1,pphh2}, where the particle-particle hole-hole
($pphh$) ring diagrams are summed to all orders.
The low-order $pphh$ ring diagrams of $\Delta E_0$ can be readily
calculated  as seen below.
We define the $n$-th order g.s.~energy shift as
\begin{equation}
\Delta E_0^{(n)}= D^{(1)}+D^{(2)}+ \cdots  +D^{(n)},
\end{equation}
where $D^{(n)}$ denotes the $n$-th order $pphh$ ring diagram.
The first-order diagram is
\begin{equation}
D^{(1)} =  \frac{1}{2}\sum\limits_{ab} \langle ab|V_{{\rm low}-k}|ab
\rangle n_a n_b~,   \label{D1}
\end{equation}
where
$n_a$ is the occupation number of the state $a$,
given by $n_a=1$ if $a \leq k_F$ and $=0$ if otherwise.

Similarly the second-order diagram is
\begin{equation} D^{(2)}=
 \sum\limits_{abcd}
\frac{\langle ab|V_{{\rm low}-k}|cd \rangle^2 n_a n_b (1-n_c)(1-n_d)}
{4\times [\bar\epsilon_a + \bar \epsilon_b -\bar \epsilon_c -\bar \epsilon_d]}~,
  \label{D2}
\end{equation}
where $\bar\epsilon$ is a shell-model Hartree-Fock (smHF) s.p.~energy.
We dress the single-particle and single-hole lines
with self-energy, or HF insertions.
We shall include such insertions to all orders.
Sample diagrams illustrating the general structure
of the $pphh$ ring diagrams and their self-energy
insertions have been given
in e.g. Refs.~\cite{Siu2008,pphh1}.
The net effect of  such all-order insertions
is the replacement of the oscillator
s.p.~energy $\epsilon^{(0)}$ of the undressed ring diagrams
by a smHF s.p.~energy $\bar \epsilon$  given by
\begin{eqnarray}
\bar \epsilon_a= \epsilon ^{(0)}_a + \sum\limits_{b} \langle ab|V_{{\rm low}-k}|ab \rangle n_b~.
\end{eqnarray}
Unless specified otherwise, we shall use $\bar \epsilon$
in all the calculations reported in this work.

The oscillator s.p.~wave functions ($a$, $b$ ...), which are dependent
on the oscillator parameter $\hbar \omega$, are used for
calculating the shell-model matrix elements
of $V_{{\rm low}-k}$.  Also the oscillator s.p.~energies $\epsilon^{(0)}$ are dependent on $\hbar \omega$.
Thus $\Delta E_0$ in general has a complicated dependence on $\hbar \omega$.
As will be discussed in Section III, we have found,
rather unexpectedly, that our calculated ratios $R_t$ and $R_i$ at the unitary limit
are practically independent of $\hbar \omega$.

The energy shift calculated by summing the $pphh$ ring diagrams to all orders, denoted as $\Delta E_0^{\rm pp}$,
corresponds to $\Delta E_0^{(n)}$ of Eq.~(6) with $n \rightarrow \infty$.
An initial step is to solve the following RPA equation
\cite{Siu2008,pphh1}
\begin{eqnarray}
&&\sum\limits_{cd\in P} [ (\bar \epsilon_a+\bar \epsilon _b ) \delta_{ab,cd} + \lambda (1-n_a-n_b)  \nonumber\\
&& ~ \langle ab|V_{{\rm low}-k}|cd \rangle ] \times Y_m(cd,\lambda) =
\omega_m Y_m(ab,\lambda)~.  \label{RPAeq}
\end{eqnarray}
where $\lambda$ is a strength parameter. As described later,
we shall integrate $\lambda$ from 0 to 1.
The indices $(a,b)$ can be either both particles $(p,p')$ or
both holes $(h,h')$ and so are $(c,d)$. The subscript $P$
refers to the ranges for particles and holes used in the
calculation. As an  example, for  the 40 neutron case
we have $(h,h')$ in the core composed of the
$(0s,0p,0d1s,0f1p)$ shells
and $(p,p')$ in the two major shells above the core.
The above equation has 2 sets of eigen values:
$\omega ^+_n$ corresponding to $(E_n(A+2)-E_0(A))$
and $\omega ^-_m$ corresponding to  $(E_0(A)-E_m(A-2))$
where $A$ is the core mass number \cite{Kuo1994}.
The eigenfunction for $\omega^+_n$ is
$Y_n(ij)=\langle\Psi_n^{A+2}|a^{\dagger}_ia^{\dagger}_j |\Psi _0^A\rangle$.
Similarly the  one for $\omega ^-_m$ is
$Y_m(ij)=\langle\Psi_m^{A-2}|a_ja_i |\Psi _0^A\rangle$.
Usually $Y_n$ and $Y_m$ are dominated respectively
by their $(p,p')$ and $(h,h')$ components. The energy shift given
by the all-order $pphh$ ring diagrams is given by \cite{pphh1}
\begin{eqnarray}
&&\Delta E_0^{\rm pp} = \int_0^1 d\lambda
 \sum\limits_m \sum\limits_{abcd\in P} Y_m(ab,\lambda) \nonumber \\
&& \times Y^*_m(cd,\lambda) \langle ab|V_{{\rm low}-k}|cd \rangle~. \label{eigen-vec}
\end{eqnarray}
This $\Delta E_0^{\rm pp}$ has contributions from
$\langle \Psi_m^{A-2}|a_pa_{p'} |\Psi _0^A\rangle$ where $(p,p')$ are both
particles. To have such contributions, $\Psi_0$ must have
particle-hole excitations. The above method will be
referred to as the eigen-vector method for calculating
$\Delta E_0^{pp}$.

This g.s.~energy shift can also be calculated using a different
method \cite{pphh2} which gives
\begin{eqnarray}
\Delta E_0^{\rm pp}= -  \sum_m \omega_m^-
 + \sum_{ab} \left( \bar \epsilon _a n_a + \bar \epsilon _b n_b \right)~, \label{eigen-val}
\end{eqnarray}
where $\omega_m^-$ is obtained by solving Eq.~(\ref{RPAeq}) with $\lambda=1$.
This method will be referred to as the eigen-value method, and  is simpler
than the previous eigen-vector method. We shall carry out our calculations
using both methods to cross check our numerical results.




\begin{table*}
\centering
\caption{\label{table1} The trap ratio $R_t$ and the
intrinsic ratio $R_i$ given by summing up $pphh$ ring diagrams
to all orders, denoted as $R^{({\rm all})}$, as well as those
by the 1st- and 2nd-order approximations, denoted respectively
by $R^{(1)}$ and $R^{(2)}$. Two closed-shell systems with $A$=40 and 70
are considered.}
\begin{spacing}{1.3}
\begin{tabular}{ccccccccccccccccccccccccccccccccccccccccccccccccccccccccccccccccccccccccccccccccccccccccccccccccccccccccccccccccccc}
\hline \hline

           & \multirow{2}{0.3cm}{$\Lambda$} & \multirow{2}{0.5cm}{$\hbar\omega$}
                                    &&     \multicolumn{3}{c}{40 neutrons}        &&    \multicolumn{3}{c}{70 neutrons}        \\ \cline{5-7} \cline{9-11}
           &&                       &&$R_t^{(1)}/R_i^{(1)}$&$R_t^{(2)}/R_i^{(2)}$&$R_t^{({\rm all})}/R_i^{({\rm all})}$
                                    &&$R_t^{(1)}/R_i^{(1)}$&$R_t^{(2)}/R_i^{(2)}$&$R_t^{({\rm all})}/R_i^{({\rm all})}$     \\ \hline

CD-Bonn   ~&~ 2.0 ~&~7.5  ~&&~ 0.763/0.521 ~&~ 0.758/0.510 ~&~ 0.756/0.508 ~&&~ 0.754/0.505 ~&~ 0.752/0.502 ~&~ 0.752/0.501    ~\\

CD-Bonn   ~&~ 2.0 ~&~8.5  ~&&~ 0.760/0.515 ~&~ 0.755/0.506 ~&~ 0.754/0.504 ~&&~ 0.752/0.502 ~&~ 0.751/0.499 ~&~ 0.750/0.498    ~\\

CD-Bonn   ~&~ 2.0 ~&~9.5  ~&&~ 0.759/0.512 ~&~ 0.754/0.504 ~&~ 0.753/0.502 ~&&~ 0.752/0.502 ~&~ 0.751/0.499 ~&~ 0.750/0.498    ~\\

CD-Bonn   ~&~ 2.0 ~&~10.5 ~&&~ 0.758/0.511 ~&~ 0.754/0.504 ~&~ 0.754/0.502 ~&&~ 0.753/0.503 ~&~ 0.752/0.501 ~&~ 0.751/0.500    ~\\

CD-Bonn   ~&~ 2.0 ~&~12.0 ~&&~ 0.759/0.513 ~&~ 0.756/0.507 ~&~ 0.755/0.505 ~&&~ 0.755/0.508 ~&~ 0.754/0.506 ~&~ 0.754/0.506    ~\\

CD-Bonn   ~&~ 1.8 ~&~10.5 ~&&~ 0.758/0.510 ~&~ 0.754/0.503 ~&~ 0.753/0.501 ~&&~ 0.752/0.502 ~&~ 0.751/0.500 ~&~ 0.751/0.499    ~\\

CD-Bonn   ~&~ 2.3 ~&~10.5 ~&&~ 0.758/0.511 ~&~ 0.754/0.504 ~&~ 0.754/0.502 ~&&~ 0.753/0.503 ~&~ 0.752/0.501 ~&~ 0.751/0.500    ~\\

HCSW-1    ~&~ 2.0 ~&~10.5 ~&&~ 0.760/0.516 ~&~ 0.754/0.504 ~&~ 0.753/0.501 ~&&~ 0.750/0.498 ~&~ 0.748/0.494 ~&~ 0.748/0.493    ~\\ %r3bm5

HCSW-2    ~&~ 2.0 ~&~10.5 ~&&~ 0.754/0.503 ~&~ 0.751/0.497 ~&~ 0.750/0.495 ~&&~ 0.750/0.498 ~&~ 0.749/0.496 ~&~ 0.749/0.495    ~\\ %r15bm2

HCSW-3    ~&~ 2.0 ~&~10.5 ~&&~ 0.753/0.502 ~&~ 0.750/0.494 ~&~ 0.749/0.493 ~&&~ 0.749/0.495 ~&~ 0.748/0.493 ~&~ 0.747/0.492    ~\\ %r3bm3

HCSW-4    ~&~ 2.0 ~&~10.5 ~&&~ 0.753/0.500 ~&~ 0.749/0.493 ~&~ 0.748/0.491 ~&&~ 0.749/0.496 ~&~ 0.748/0.493 ~&~ 0.748/0.493    ~\\ %r5bm5

\hline
\end{tabular}
\end{spacing}
\end{table*}


\begin{figure}
\centering
\includegraphics[width = 0.38\textwidth]{unitfig1.eps}
\caption{\label{fig2} Energy per neutron (denoted as $E_0/A$)
given by the all-order sum of the $pphh$ ring diagrams,
versus the $\hbar\omega$ value. See text for further explanations.}
\end{figure}



\begin{figure}
\centering
\includegraphics[width = 0.440\textwidth]{unitfig2.eps}
\caption{\label{fig6} Summation of $e^{\rm int}_{\rm qp}(a)=e_{\rm qp}(a)-\frac{1}{2}\epsilon _a ^{(0)}$ up to the $n$-th orbit,
for two closed-shell systems with $A$=40 and 70, respectively.
The results are based on the unitary CD-Bonn interaction combined with $\hbar\omega$=8.5 MeV
and the unitary HCSW-2 interaction combined with $\hbar\omega$=10.5 MeV, respectively.
The values over $\hbar\omega$ are presented here.}
\end{figure}


\begin{figure}
\centering
\includegraphics[width = 0.38\textwidth]{unitfig3.eps}
\caption{\label{fig7} Quasi-particle energy $e_{\rm qp}$ of the $n$-th orbit,
for two closed-shell systems with (a) $A$=40 and (b) $A$=70, respectively.
The results are based on the unitary CD-Bonn interaction combined with $\hbar\omega$=8.5 MeV
and the HCSW-2 interaction with $\hbar\omega$=10.5 MeV, respectively.
The values over $\hbar\omega$ are presented here.}
\end{figure}

\vskip 0.4cm
{\bf Results and discussions.}
In this section we discuss the calculated g.s.~properties
for a finite-number system confined in a harmonic trap
interacting with unitary neutron-neutron interactions.
In Table \ref{table1} we present the trap ratio $R_t$ connecting
the total energy of the unitary system and that of the non-interacting system,
as well as the intrinsic ratio $R_i$ connecting
the two intrinsic energies, for two closed-shell systems with $A$=40 and 70.
The ratios with the superscript ``all'' are given
by the g.s.~energy shifts calculated by
summing up the $pphh$ ring diagrams
to all orders, using both the eigenvector method of Eq.~(\ref{eigen-vec})
and the eigenvalue method of Eq.~(\ref{eigen-val}). In fact, the results given by both are
identical (to fourth decimal), providing a check of
our numerical calculations.
We also present the ratios given
by the first- and second-order approximations, of Eqs.~(\ref{D1}, \ref{D2}),
with their results denoted by the superscripts ``1'' and ``2'', respectively.

In Table \ref{table1} one can see that, for each case,
starting from the 1st-order approximation,
the ratios decrease slightly by including the 2nd-order diagram.
The ratios given by the 2nd-order approximation are further decreased,
but only very slightly, by summing up the diagrams to all orders.
These very small differences indicate that  our  linked-diagram expansion
provides  a rapidly converging framework for calculating
the g.s energy of
trapped closed-shell unitary neutron systems.
As also shown in Table \ref{table1}, of great interest is that
the $R_t^{\rm (all)}$ (and $R_i^{\rm (all)}$) values
are remarkably invariant in regard to variations of the harmonic trap parameter,
the decimation momentum $\Lambda$ of $V_{{\rm low}-k}$,
and the choice of the unitarity potential.


Below we study a connection between
our results and the unitary-limit virial theorem
of Werner and Castin \cite{virial}. According to  this thorem,
the g.s.~energies of unitary Fermion systems in a harmonic trap
satisfy the relation
\begin{equation}
\langle \Psi_0|[T +U_{\rm osc} +V]|\Psi _0   \rangle=E_0
=2 \langle \Psi_0|U_{\rm osc}|\Psi _0   \rangle \,,
\end{equation}
where $T$ is the kinetic energy,  $U_{\rm osc}$ the oscillator potential
($\sum_i \frac{1}{2}m\omega ^2 r_i^2$) and $V$ an unitary neutron-neutron
interaction. $E_0$ and $\Psi_0$ are respectively the g.s.~energy
and wave function of the system.

From the above equation we readily have
\begin{equation}
\omega \frac{d}{d\omega}E_0=E_0~,
\end{equation}
which gives
\begin{equation}
 E_0=\alpha \hbar \omega ~, \label{linear}
\end{equation}
where $\alpha$ is a constant independent of $\hbar \omega$.
The above is a special virial linear scaling relation for the g.s.~energy $E_0$
of a unitary Fermion system trapped in a harmonic trap.
We shall check if our calculated
g.s.~energies are consistent with this scaling relation.

In Fig.~\ref{fig2}(a) we present our ring-diagram
g.s.~energies using the fine-tuned CD-Bonn potential
and $\hbar\omega$=7.5, 8.5, 9.5, 10.5, 12.0 MeV.
Three closed-shell systems with $A$=20, 40, and 70 are considered.
Plotted are the results for the energy per neutron (dots) ,
versus the $\hbar\omega$ value,
as well as the corresponding linear-fitting results (line).
One sees all lines fit the data nearly perfectly,
and moreover   they all converge to
the origin $(E,\hbar\omega)=(0,0)$
also near perfectly. Clearly our results are in highly
satisfactory agreement with the virial linear scaling
of Eq.~(\ref{linear}).

We also generalize our study to open-shell systems,
with valence neutrons occupying the major shell just
above the closed core.
Because of space consideration, let us only briefly
outline our procedures in carrying out such  calculations.
The one-body and two-body effective interactions for valence neutrons
are derived
using a $\hat{Q}$-box folded-diagram method \cite{vlowk3,Kuo1990},
with its $\hat{Q}$-box vertex function  composed of
1st- and 2nd-order irreducible valence-linked diagrams,
including the core-polarization ones.
The unitary interactions used are the same as those
for the closed-shell systems.
We then calculate and diagonalize the Hamiltonian matrix
in the many-body model space of valence particles,
using a standard shell-model code \cite{nushellx}.

In Fig.~\ref{fig2}(b)
we exemplify the results of open-shell systems
using the systems
with $A$=26, 36, 46
and $\hbar\omega$=7.5, 8.5, 9.5, 10.5, 12.0 MeV.
One sees for open-shell systems, the energy per
neutron is also remarkably proportional to the energy scale $\hbar\omega$,
again in very good agreement with the scaling relation of Eq.~(\ref{linear}).
According to our calculation, this scaling relation also
holds perfectly for odd-$A$ systems.

Returning to Table 1, the ratios $R_t$
presented there are practically
invariant with the variation of $\hbar \omega$.
This invariance is actually
a consequence of the above scaling relation.
$R_t$ is given by $E_0/E_0^{\rm free}$
where the denominator is proportional to  $\hbar \omega$.
The g.s.~energy $E_0$
is also proportional to $\hbar\omega$ according to
the scaling relation. Thus $R_t$ should be invariant in regard to
$\hbar \omega$, as we have found.


At last we study the unitary ratio guided by Landau's Fermi liquid (FL) theory \cite{MBT}.
In this way, a simple relation between the unitary ratio
and the Fermi-liquid quasi-particles can be obtained.
For a system of neutrons in a harmonic trap, the quasi-particle
energy $e_{\rm qp}$ and the g.s.~energy $E_0$ are related as
\begin{eqnarray}
e_{\rm qp}(a) &=& \frac{\delta E_0}{\delta n_a}~, \nonumber \\
E_0 &=& \frac{1}{2}[\sum _a \epsilon_a ^{(0)}n_a + \sum _a e_{\rm qp}(a) n_a]~.
\end{eqnarray}
The unitary ratio of Eq.~(\ref{Rt}) then becomes
\begin{equation}
\sum _a [e_{\rm qp}(a)-\frac{1}{2}\epsilon _a ^{(0)} ]n_a
= 2(R_t-\frac{3}{4})E_0^{\rm free}~.
\end{equation}
To have $R_t$=0.75, the FL quasi-particles should satisfy the sum rule
\begin{equation}
\sum\limits_a [e_{\rm qp}(a)-\frac{1}{2}\epsilon _a^{(0)}] n_a = 0~. \label{sumrule}
\end{equation}
We also define an intrinsic quasi-particle energy $e^{\rm int}_{\rm qp}(a)=e_{\rm qp}(a)-\frac{1}{2}\epsilon _a^{(0)}$,
and the intrinsic ratio of Eq.~(\ref{Ri}) is then determined by $\sum _a e^{\rm int}_{\rm qp}(a) n_a = 2(R_i-\frac{1}{2})T_0$
if $K_{\rm cm}$ is neglected. One easily sees the sum rule also gives $R_i=0.5$.

As indicated by Table \ref{table1}, the g.s.~energy given by low-order approximation is closely equal to that including high-order diagrams,
we have thus approximated $e_{\rm qp}$ by
\begin{equation}
 e_{\rm qp}^{(2)}(a)
 = \epsilon _a^{(0)} +\frac{\delta}{\delta n_a}[D^{(1)}+D^{(2)}]~.
\end{equation}
Then $e_{\rm qp}^{(2)}$ is obtained by solving the above equation in a self-consistent manner.
We exemplify the sum rule of Eq.~(\ref{sumrule}) in Fig.~\ref{fig6}, using the closed-shell systems with $A$=40 and 70.
The results are based on the unitary CD-Bonn interaction
combined with $\hbar\omega$=8.5 MeV
and the unitary HCSW-2 interaction combined with
$\hbar\omega$=10.5 MeV, respectively.
In Fig.~\ref{fig6} one can see that the above sum rule is indeed well satisfied by our results.
We also present the quasi-particle energy $e_{\rm qp}$
in Fig.~\ref{fig7}, where one sees that
the $e_{\rm qp}$ values for orbits within one harmonic oscillator
shell are nearly degenerate,
and moreover the spacing of the $e_{\rm qp}$ between two
neighboring shells is about 1.5$\hbar\omega$,
i.e., about 1.5 times larger than that of
s.p.~energies in the non-interacting system.
In ordinary nuclear systems the s.p.~energy levels
of each major shell are generally non-degenerate, such as the non-degenerate
($0f_{7/2},0f_{5/2},1p_{3/2},1p_{1/2}$) levels of $^{40}$Ca.
Our results indicate that these levels should become
nearly degenerate in the unitary limit. It will be
of much interest to check if this drastic change does take place experimentally.


\vskip 0.4cm
{\bf Summary.}
We have studied finite numbers of neutrons in a harmonic
trap at the unitary limit.
Two very different types of neutron-neutron interactions have
been applied, namely, the
meson-theoretic CD-Bonn potential and hard-core square-well interactions, all
tuned to posses infinite scattering lengths. The potentials were renormalized
to equivalent, scattering-length preserving
low-momentum potentials, $V_{{\rm low}-k}$,
with which the particle-particle hole-hole
ring diagrams are summed to all orders
to yield the ground-state energy $E_0$ of the finite neutron systems.
The eigen-vector and eigen-value methods were both employed
for the ring-diagram $E_0$ calculation,
giving practically identical results.
We find the ratio
$R_t \equiv E_0/E_0^{\rm free}$ (where $E_0^{\rm free}$ denotes
the ground-state energy of the corresponding non-interacting system)
to be remarkably invariant in regard to variations of the harmonic
trap parameter,
the number of neutrons,
the decimation momentum $\Lambda$ of $V_{{\rm low}-k}$,
and the choice of the unitarity potential.
Resulting $R_t$'s
are all very close to 0.75 as seen in Table~I.
Our results also support a special virial linear scaling relation.
At the unitary limit, we find that Landau's
quasi-particles for trapped neutrons satisfy a sum rule
and have a major-shell degeneracy behavior.

\begin{acknowledgments}
We thank Ismail Zahed for many helpful discussions.
The works by T.T.S.K. and R.M. were supported in part by the U.S. Department of Energy
under Award Numbers DE-FG02-88ER40388 and DE-FG02-03ER41270, respectively.
Y.Y.C. and Y.M.Z. thank the National Natural Science Foundation of China (Grant Nos.~11875134, 11505113 and 11675101),
and the Program of Shanghai Academic/Technology Research Leader (Grant No.~16XD1401600) for financial supports.
\end{acknowledgments}

\begin{thebibliography}{50}

\bibitem{Bertsch} R. F. Bishop, Int. J. Mod. Phys. B {\bf 15}, iii (2001), Many-Body Challenge Problem by G. F. Bertsch.
\bibitem{CDBonn} R. Machleidt, Phys. Rev. C {\bf 63}, 024001 (2001).
\bibitem{Jr1999} G. A. Baker, Jr., Phys. Rev. C {\bf 60}, 054311 (1999).
\bibitem{Heiselberg2001} H. Heiselberg, Phys. Rev. A {\bf 63}, 043606 (2001).
\bibitem{Carlson2003} J. Carlson, S. Y. Chang, V. R. Pandharipande, and K. E. Schmidt, Phys. Rev. Lett. {\bf 91}, 050401 (2003).
\bibitem{Perali2004} A. Perali, P. Pieri, and G. C. Strinati, Phys. Rev. Lett. {\bf 93}, 100404 (2004).
\bibitem{Bulgac2005} A. Bulgac and G. F. Bertsch, Phys. Rev. Lett. {\bf 94}, 070401 (2005).
\bibitem{Nishida2006} Y. Nishida and D. T. Son, Phys. Rev. Lett. {\bf 97}, 050403 (2006).
\bibitem{Haussmann2007} R. Haussmann, W. Rantner, S. Cerrito, and W. Zwerger, Phys. Rev. A {\bf 75}, 023610 (2007).
\bibitem{Schaefer2007} T. Sch\"{a}fer, Phys. Rev. A {\bf 76}, 063618 (2007).
\bibitem{Siu2008} L. W. Siu, T. T. S. Kuo, and R. Machleidt, Phys. Rev. C {\bf 77}, 034001 (2008).
\bibitem{Dong2010} H. Dong, L. W. Siu, T. T. S. Kuo, and R. Machleidt, Phys. Rev. C {\bf 81}, 034003 (2010).
\bibitem{Schaefer2010} T. Sch\"{a}fer, Phys. Rev. A {\bf 82}, 063629 (2010).

\bibitem{exp1} K. M. O'Hara, S. L. Hemmer, M. E. Gehm, S. R. Granade, and J. E. Thomas, Science {\bf 298}, 2179 (2002).
\bibitem{exp2} T. Bourdel, L. Khaykovich, J. Cubizolles, J. Zhang, F. Chevy, M. Teichmann, L. Tarruell, S. J. J. M. F. Kokkelmans, and C. Salomon, Phys. Rev. Lett. {\bf 93}, 050401 (2004).
\bibitem{exp3} J. Kinast, A. Turlapov, J. E. Thomas, Q. Chen, J. Stajic, and K. Levin, Science {\bf 307}, 1296 (2005).
\bibitem{exp4} G. B. Partridge, W. Li, R. I. Kamar, Y. A. Liao, and R. G. Hulet, Science {\bf 311}, 503 (2006).
\bibitem{exp5} J. T. Stewart, J. P. Gaebler, C. A. Regal, and D. S. Jin, Phys. Rev. Lett. {\bf 97}, 220406 (2006).
\bibitem{exp6} L. Luo and J. E. Thomas, J. Low Temp. Phys. {\bf 154}, 1 (2009).
\bibitem{exp7} S. Nascimb\`{e}ne, N. Navon, K. J. Jiang, F. Cevy, C. Salomon, Nature {\bf 463}, 1057 (2010).
\bibitem{exp8} C. Cao, E. Elliott, J. Joseph, H. Wu, J. Petricka, T. Sch\"{a}fer, J. E. Thomas, Science {\bf 331}, 58 (2011).
\bibitem{exp9} M. J. H. Ku, A. T. Sommer, L. W. Cheuk, M. W. Zwierlein, Science {\bf 335}, 563 (2012).

\bibitem{Konig2017} S. K\"{o}nig, H. W. Grie{\ss}hammer, H. W. Hammer, and U. van Kolck, Phys. Rev. Lett. {\bf 118}, 202501 (2017).
\bibitem{Kievsky2018} A. Kievsky, M. Viviani, D. Logoteta, I. Bombaci, and L. Girlanda, Phys. Rev. Lett. {\bf 121}, 072701 (2018).


\bibitem{Werner2006} F. Werner and Y. Castin, Phys. Rev. Lett. {\bf 97}, 150401 (2006).
\bibitem{virial} F. Werner and Y. Castin, Phys. Rev. A {\bf 74}, 053604 (2006).
\bibitem{Chang2007} S. Y. Chang and G. F. Bertsch, Phys. Rev. A {\bf 76}, 021603(R) (2007).
\bibitem{Blume2007} D. Blume, J. von Stecher, and C. H. Greene, Phys. Rev. Lett. {\bf 99}, 233201 (2007).

\bibitem{vlowk1} S. K. Bogner, T. T. S. Kuo, and L. Coraggio, Nucl. Phys. A {\bf 684}, 432c (2001).
\bibitem{vlowk2} S. K. Bogner, T. T. S. Kuo, L. Coraggio, A. Covello and N. Itaco, Phys. Rev. C {\bf 65}, 051301(R) (2002).
\bibitem{vlowk3} L. Coraggio, A. Covello, A. Gargano, N. Itaco, T. T. S. Kuo, D. R. Entem, and R. Machleidt, Phys. Rev. C {\bf 66}, 021303(R) (2002).
\bibitem{vlowk4} A. Schwenk, G. E. Brown, and B. Friman, Nucl. Phys. A {\bf 703}, 745 (2002).
\bibitem{vlowk5} J. D. Holt, T. T. S. Kuo, and G. E. Brown, Phys. Rev. C {\bf 69}, 034329 (2004).
\bibitem{vlowk6} S. K. Bogner, T. T. S. Kuo, and A. Schwenk, Phys. Rep. {\bf 386}, 1 (2003).
\bibitem{vlowk7} T. T. S. Kuo, J. W. Holt, and E. Osnes, Phys. Scr. {\bf 91}, 033009 (2016).

\bibitem{MBT} G. E. Brown, {\it Many-Body Theory} (North-Holland Publishing Company, Amsterdam Holland, 1972).

\bibitem{Goldstone} J. Goldstone, Proc. R. Soc. London, Ser. A {\bf 239}, 267 (1957).
\bibitem{Kuo1971} T. T. S. Kuo, S. Y. Lee, and K. F. Ratcliff, Nucl. Phys. A {\bf 176}, 65 (1971).
\bibitem{Kuo1990} T. T. S. Kuo and E. Osnes, Lect. Notes Phys. {\bf 364}, 1 (1990).

\bibitem{pphh1} S. D. Yang, J. Heyer, and T. T. S. Kuo, Nucl. Phys. A {\bf 448}, 420 (1986).
\bibitem{pphh2} Y. H. Tzeng and T. T. S. Kuo, Nucl. Phys. A {\bf 485}, 85 (1988).
%\bibitem{Song1987} H. Q. Song, S. D. Yang, and T. T. S. Kuo, Nucl. Phys. A {\bf 462}, 491 (1987).
\bibitem{Kuo1994} T. T. S. Kuo and Y. Tzeng, Int. Jour. Mod. Phys. E {\bf 3}, 523 (1994).

\bibitem{nushellx} B. A. Brown, W. D. M. Rae, E. McDonald, and M. Horoi, NuShellX@MSU.

\end{thebibliography}

\end{document}

