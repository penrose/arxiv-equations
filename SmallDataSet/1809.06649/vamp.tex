\documentclass{twocol}
\usepackage{flushend}
\twocolumn
\lat



\title{Cryogenic differential amplifier for NMR applications}

\rtitle{Cryogenic differential amplifier for NMR applications}

\sodtitle{Cryogenic differential amplifier for NMR applications}

\author {
  V.~V.~Zavjalov\/\thanks{email: vladislav.zavyalov@aalto.fi},
  A.~M.~Savin
  P.~J.~Hakonen
}
\address{Low Temperature Laboratory,
Department of Applied Physics, Aalto University,
PO Box 15100, FI-00076 AALTO, Finland}
\dates{18 Sep 2018}{*}

\rauthor{V.~V.~Zavjalov, A.~M.~Savin, P.~J.~Hakonen}

\sodauthor{Zavjalov, Savin, Hakonen}

\abstract{
We have designed and characterized  a cryogenic amplifier for
use in $^3$He NMR spectrometry. The amplifier, with a power consumption
of  $\sim 2.5$~mW, works at temperatures down to 4~K. It has a
hi-impedance input for measuring a signal from NMR resonant circuit, and
a 50~$\mathrm{\Omega}$ differential input which can be used for pick-up
compensation and gain calibration. At 4.2~K, the amplifier has a voltage
gain of 45, output resistance 146~$\mathrm{\Omega}$ and a 4.4~MHz
bandwidth starting from DC. At 1~MHz, the voltage and current noise
amount to 1.3~\nVsHz{}  and 12~\fAsHz{}, respectively, which yields an
optimal source impedance of $\sim 100$~k$\mathrm{\Omega}$. }

\PACS{}


%\graphicspath{{pics/}}
\newcommand{\image}[3]{
\begin{figure}[#1]
\begin{center}
\includegraphics{full_#2.eps}
\caption{\small#3}
\label{image:#2}
\end{center}
\end{figure}
}


\def\nVsHz{%
$\mbox{nV}/\sqrt{\mbox{Hz}}$}
\def\pAsHz{%
$\mbox{pA}/\sqrt{\mbox{Hz}}$}
\def\fAsHz{%
$\mbox{fA}/\sqrt{\mbox{Hz}}$}


\begin{document}

\maketitle

Keywords: cryogenic amplifier; HEMT; low-noise; differential input

%%%%%%%%%%%%%%%%%%%%%%%%%%%%%%%%%%%%%%%%%%%%%%%%%%%%%%%%%%
\image{h!}{amp1}{Schematics and a photo of component layout.}
%%%%%%%%%%%%%%%%%%%%%%%%%%%%%%%%%%%%%%%%%%%%%%%%%%%%%%%%%%

\section{Introduction}

Many scientific and research applications benefit from the use of
cryogenically cooled amplifiers. Reduction of the amplifier operation
temperature leads to significant improvement of the signal-to-noise
ratio, which finally improves sensitivity of measurements.

Nowadays almost all commercially available transistors and operational
amplifiers use silicon technology, and they do not work below 10~K. On
the other hand most cryostats have a 4~K plate or liquid-helium bath
where cold amplifiers can be placed. With silicon-based amplifiers,
overheating with temperature stabilization can be
used~\cite{2009_hayashi} for normal operation. Many non-silicon
transistors used previously for cold amplifiers~\cite{1985_bloyet} are
not available anymore. Now there are a few types of low-noise non-silicon
transistors produced commercially which are designed for GHz frequencies.
They include GaAs HEMT-technology FET transistors made by Avago (ATF
series), CEL (CE series) or TriQuint (TGF series) and SiGe:C npn bipolar
transistors made by Infineon (BFP series). These transistors can be used
not only at high
frequencies~\cite{2005_pospieszalski,2004_roschier,2005_hu} but also for
low frequency
applications~\cite{2000_koivuniemi,2004_oukhanski,2004_robinson,2006_dicarlo,2008_mathur,2012_beev,2013_arakawa,2017_hirata}.
The performance of cooled low-frequency  amplifiers has also been
promoted by special, foundry-made transistors~\cite{2014_dong}.

In this paper, we present a cryogenic amplifier based on easily-available, commercial Avago ATF33143 HEMT components
and Infineon BFP640 npn transistors. The goal has been to design 
a preamplifier for NMR experiments which operate around 1 MHz 
 and utilize tank circuits with high impedance ($\sim 100$~k$\mathrm{\Omega}$).



\section{Design}

The schematics and a photograph of the amplifier are shown in Fig.~1. The
differential cascode input stage consists of four HEMT transistors
$T_1-T_4$. The stage is powered by a current supply made of two npn
transistors $T_5-T_6$. The current is set by the bias voltage $V_0$ and a
19.1~k$\mathrm{\Omega}$ resistor. Cascode design together with non-symmetric
inputs are used to reduce the Miller effect~\cite{miller}. This effect takes
place when the output voltage couples to the input through transistor's
parasitic capacitance, and then, after amplification, decreases the total
gain. A 50~$\mathrm{\Omega}$ resistor on input~2 forms a divider with the
parasitic capacitance which reduces the effect. In equilibrium, the
current in both arms is half of the total current, and the output of
the amplifier is biased near $V_0/2$. A follower with npn
transistor $T_7$ is used to reduce the output resistance of the
amplifier. All transistors have 100~$\mathrm{\Omega}$ resistors connected to their
bases/gates. These resistors are needed to damp high-frequency resonances
caused by parasitic inductances of the circuit and to improve the stability of the amplifier. 
At the input terminals, the 100~$\mathrm{\Omega}$ resistors are shunted
by 24~nH inductors at the relevant signal frequencies. The bias voltage
line is connected through an RC-filter with a 100~$\mathrm{\Omega}$ resistor, a
150~$\mu$F tantalum capacitor and a 9.1~nF ceramic capacitor.

For the input stage we use Avago ATF33143 HEMTs. We have tested
comparable transistors with smaller gate areas (ATF34143, ATF35143)  and
obtained similar results. For the current supply and the output follower
we use Infineon BFP640 npn SiGe transistors. Passive components for
cryogenic applications need to demonstrate both mechanical and electrical
stability at low temperatures. We use ceramic C0G
capacitors~\cite{2010_teyssandier}, tantalum
capacitors~\cite{2006_teverovsky}, and metal-film resistors.

With a bias voltage $V_0=4$~V the current source gives $\sim 0.2$~mA and
the total power consumption of the device amount to~$\sim 2.5$~mW. Low power
consumption has certain disadvantages, namely high output resistance and
low slew rate. This makes the amplifier performance to be dependent on
the output circuit. If the amplifier is connected to the
output line with capacitance, two effects are observed at high
frequencies: the gain drops because the line capacitance forms an RC-filter
with the output resistance; at high amplitudes the signal is distorted
because the output current of the amplifier is limited and this restricts
the charging rate of the line capacitance. Thte second effect can be estimated
using the known parameters of the amplifier: the current limit is determined by the bias
voltage and the resistor in the output follower, $I_{\mbox{max}} \sim
0.4$~mA. The charging rate is constrained as~$\dot V = 2\pi f\,V =
I_{\mbox{max}} / C_{\mbox{line}}$. For~$C_{\mbox{line}}=500$~pF
and at~$f=1$~MHz, the distortion starts at the output amplitude
of~$V\sim 130$~mV. This estimation agrees with measurements.


\section{Measurements}

To obtain output parameters of the amplifier (gain, bandwidth and the
output resistance), we measured the gain as a function of frequency for a
few different configurations: amplifier connected directly to a
measurement device (lock-in amplifier); 1~nF capacitor added to the
output of the amplifier; a 60~$\mathrm{\Omega}$ grounding resistor added after the
capacitor. All the data were fitted simultaneously with a simple
electrical model with four parameters: gain, bandwidth, the output
resistance and the cable capacitance. Measurements were done at room
temperature (300~K), temperature of liquid nitrogen (77~K) and liquid
helium (4.2~K). Typical cable capacitance in our setup is $\sim580$~pF
and temperature-independent. Other parameters are presented in Table~\ref{table}.
In all measurements presented here we used input~1 while input~2 was
shunted to ground by 50~$\mathrm{\Omega}$ resistor. With swapped input
connections, we obtained very similar gain characteristics as expected
for a differential amplifier. The measured gain of the amplifier (without
the additional capacitor and grounding resistor on the output) is plotted
as a function of bias voltage~$V_0$ (at 1~MHz) and frequency
(at~$V_0=4$~V) in Fig.~2a and~2b, respectively. The measured cut-off
frequency~$\sim 1$~MHz in Fig.~2b is determined mostly by capacitance of
the output cable which forms an RC filter with the output resistance of
the amplifier. The obtained intrinsic gain of the amplifier
extends over DC--4~MHz and it is illustrated in Fig.~2b by dashed curves.

%%%%%%%%%%%%%%%%%%%%%%%%%%%%%%%%%%%%%%%%%%%%%%%%%%%%%%%%%%
\image{h}{gain}{ {\bf (a)} Measured AC gain at 1~MHz as a function of the bias
voltage~$V_0$. {\bf (b)} Measured AC gain at $V_0 = 4$~V as a function of
frequency. The 1~MHz cut-off frequency here is determined by the output
resistance of the amplifier ($\sim$150~$\mathrm{\Omega}$) and the capacitance of
output cables ($\sim$580~pF). The obtained intrinsic bandwidth~$\sim4$~MHz
is shown by dashed lines. Measurements were done at $T=300$~K, 77~K, and 4.2~K.}
%%%%%%%%%%%%%%%%%%%%%%%%%%%%%%%%%%%%%%%%%%%%%%%%%%%%%%%%%%

%%%%%%%%%%%%%%%%%%%%%%%%%%%%%%%%%%%%%%%%%%%%%%%%%%%%%%%%%%
\image{h}{noise}{ Model for noise measurements {\bf (a)}, measured
voltage {\bf (b)} and current {\bf (c)} noise of the amplifier.
Measurements were done at 300~K, 77~K and 4.2~K using bias $V_0=4$~V.}
%%%%%%%%%%%%%%%%%%%%%%%%%%%%%%%%%%%%%%%%%%%%%%%%%%%%%%%%%%

A model used for our noise measurement analysis is shown in Fig.~3a. The amplifier
consists of a noiseless amplifier with voltage gain~$K$, voltage input
noise source $U_{\mbox{in}}$, and current input noise
source~$I_{\mbox{in}}$. The input of the amplifier is connected to a
resistor~$R_S$ which also works as a source of thermal noise~$I_R^2 =
4k_BT/R$, where $T$ is the temperature of the resistor. Parasitic
capacitance~$C_S$ of the input circuit is also taken into account. Output of the
amplifier is connected to a digital oscilloscope which is used to record
noise spectra. The noise source~$U_{\mbox{out}}$ represents the noise of
the oscilloscope.

If the noise sources are not correlated, we can make a square sum of average
noise voltages produced by each source. Then the square of the measured
noise spectral density becomes
\begin{equation}\nonumber
U^2 = U_{\mbox{out}}^2 + K^2 \left[ U_{\mbox{in}}^2
   + \left(I_{\mbox{in}}^2 + I_R^2\right) |Z_S|^2 \right],
\end{equation}
$$
|Z_S|^2 = \frac{R_S^2}{1+(\omega R_S C_S)^2}.
$$
Here $Z_S$ is the total impedance of the source ($R_S$ and $C_S$ in parallel)
and~$\mathrm{\Omega}$ denotes the angular frequency.

It is easy to find~$U_{\mbox{out}}$ and~$U_{\mbox{in}}$ by measuring two
noise spectra: one with the amplifier powered off ($K=0$) and another
with shorted input ($R_S=0$). For finding~$U_{\mbox{in}}$, we used the
measured gain to convert the output voltage noise spectra to the input.

To determine the input current noise~$I_{\mbox{in}}$, we recorded noise
spectra with non-zero source resistance~$R_S$. The noise with~$R_S=0$ was
subtracted (removing the effects of $U_{\mbox{out}}$ and~$U_{\mbox{in}}$) and
the result was converted to amplifier input. This gives us a source-dependent
component of the input noise, $(I_{\mbox{in}}^2 + I_R^2) |Z_S|^2$. For
small~$R_S$, the main contribution is due to thermal noise. A cut-off frequency $1/R_S
C_S$ is clearly seen on the corresponding noise spectra, which allows us
to obtain an estimate for the capacitance at the input $C_S=6.7$~pF. Knowing $C_S$, we can evaluate the impedance~$Z_S(\omega)$ and
extract the current noise. We did this using $R_S=10$~M$\mathrm{\Omega}$ at which
the thermal current noise generator $I_R$ is smaller. The obtained values for~$U_{\mbox{in}}$
and~$I_{\mbox{in}}$ are displayed in Fig.~3b and~3c.

The voltage noise determination described above is well defined and
straightforward, but the measurement of the current noise is more tricky:
a large thermal  noise part has to be subtracted from the result.
To decrease the thermal noise $I_R$ one has to use huge source resistances,
but this limits the bandwidth because of the $RC$-filtering by the input
capacitance of the amplifier. In addition, the parasitic input resistance could
affect the result if it were comparable with the source resistance.

In order to improve the accuracy of the current-noise measurements, we
made another experiment utilizing a resonant circuit as a high-impedance
source, similarly to the work in Ref.~\cite{2000_koivuniemi}. The circuit
was formed by a niobium coil with $L=100$~$\mu$H and a capacitor
with~$C=200$~pF to obtain a resonance frequency near 1~MHz. We used an
American Technical Ceramics porcelain capacitor (100B series) which has
very low dissipation. Any resistive metal near the coil leads to extra
rf-dissipation as well as thermal noise. To avoid this, we used a
superconducting wire without normal-metal matrix. The circuit was placed
in a solder-covered copper box which worked as a superconducting
enclosure. We concluded that if a small piece of normal metal inside the
enclose does not decrease the $Q$-value below a few thousand, then it has
no substential effect on the noise.  This confirms that the measured
noise is originated from the amplifier.

The output voltage noise was recorded as described above. The gain of the amplifier was
calibrated using an external noise source connected to input~2. Near
1~MHz, the equivalent input noise contains a constant voltage noise
($\approx1.3$~nV/$\sqrt{\mbox{Hz}}$) and a Lorentzian-shaped noise peak,
produced by current noise imposed on the impedance of the tank circuit.

With all the precautions, we managed to reach $Q\approx1.2\cdot10^6$,
which corresponds to the maximum impedance $Z=0.9$~G$\mathrm{\Omega}$ and
an effective dissipative resistance 0.5~m$\mathrm{\Omega}$ in series with
the coil. This confirms that the input impedance of the amplifier is
higher than~$0.9$~G$\mathrm{\Omega}$. It is hard to make good noise
measurements with such a high $Q$-value circuit because of the long time
needed for good frequency resolution and frequency drifts during this
time. For avoiding these problems in our tank-circuit noise measurements, 
we added a small resistor to reduce the $Q$-value to $40\cdot10^3$
($Z=28$~M$\mathrm{\Omega}$). The noise curve measured with such a circuit
is depicted in Fig.~4. According to the obtained data, the current noise
of the amplifier is 11.8~fA/$\sqrt{\mbox{Hz}}$, which is in a good
agreement with the above-described measurements with a
10~M$\mathrm{\Omega}$ resistive source.

%%%%%%%%%%%%%%%%%%%%%%%%%%%%%%%%%%%%%%%%%%%%%%%%%%%%%%%%%%
\image{h!}{tank_circuit}{Noise measured using a hi-Q tank circuit at the
input of the amplifier. The curve is a sum of small input voltage noise 1.3~nV/$\sqrt{\mbox{Hz}}$
and input current noise 11.8~fA/$\sqrt{\mbox{Hz}}$ imposed on the impedance of the tank circuit. }
%%%%%%%%%%%%%%%%%%%%%%%%%%%%%%%%%%%%%%%%%%%%%%%%%%%%%%%%%%



\begin{table}
\begin{tabular}{|l|c|c|c|}\hline
Temperature $T$,~K           & 300  & 77   & 4.2 \\\hline
Gain $K_0$                   & 21.3 & 38.7 & 44.5 \\
$R_{\mbox{out}}$,~$\mathrm{\Omega}$   & 233  & 128  & 146  \\
Cut-off frequency $f_c$, MHz & 3.23 & 4.02 & 4.41\\[5pt]
Voltage noise, nV/$\sqrt{\mbox{Hz}}:$&&&\\
at 10~kHz                    & 35.2 & 14.9 & 6.0 \\
at 100~kHz                   &  8.4 &  3.6 & 2.1 \\
at 1~MHz                     &  3.8 &  1.6 & 1.3 \\
Current noise, fA/$\sqrt{\mbox{Hz}}:$&&&\\
at 10~kHz                    & 17.5 &  6.2 &  6.4 \\
at 100~kHz                   & 30.9 &  9.3 &  7.7 \\
at 1~MHz                     & 83.3 & 21.4 & 14.5 \\
Optimal source impedance, $\mathrm{\Omega}$:&&&\\
at 10~kHz                    & 2.0M & 2.4M & 930k \\
at 100~kHz                   & 270k & 390k & 270k \\
at 1~MHz                     &  45k &  74k &  89k \\
\hline
\end{tabular}
\caption{Characteristics of the amplifier measured at three different
temperatures. Bias voltage~$V_0=4$~V has been  employed in all
measurements. Current noise at all frequencies is measured with resistive
source. More accurate measurement with a resonant circuit
gives~11.8~fA/$\sqrt{\mbox{Hz}}$ at 1~MHz.}
\label{table}
\end{table}

We made and tested a few similar amplifiers. Altogether, it was easy to make a
working device and to get the expected characteristics. The gain is determined by
the input HEMT transistors, the characteristics of which may vary. We observed about 15\% scatter in
the gain value between different devices. The balancing of the transistors is
also important: one has to check that the DC response is symmetric and
replace HEMT transistors if needed. We did not observe any noticeable
change of the amplifier characteristics after $10-20$ fast cooling and heating
cycles. Floating of input~1 can cause electrostatic damage to the input HEMT
transistor. To avoid this one can ground input~1 by a 1~M$\mathrm{\Omega}$ resistor,
but depending on the impedance of the signal source, the noise contribution of
this resistor can be significant. Noise coming from input~2 can increase
both the current and voltage noise of the amplifier. One has to incorporate a sufficient amount 
attenuation at low $T$ when connecting input~2 to room-temperature devices.

%%%%%%%%%%%%%%%%%%%%%%%%%%%%%%%%%%%%%%%%%%%%%%%%%%%%%%%%%%
\image{h}{appl}{Application example.}
%%%%%%%%%%%%%%%%%%%%%%%%%%%%%%%%%%%%%%%%%%%%%%%%%%%%%%%%%%

\section{Application example}

We use the amplifier for NMR experiments in superfluid~$^3$He. Schematics
of the NMR spectrometer is shown in Fig.~5. The $^3$He sample is placed
in a magnetic field of $20-30$~mT. A pair of crossed coils is used to
apply an RF excitation from a generator and to pick up a signal from
precessing magnetization. The pick-up coil ($L\sim50\mu$H) together with
a capacitor forms a tank circuit with a resonant frequency $\sim 1$~MHz,
$Q \sim 200$ and an on-resonance resistance of $\sim
70$~k$\mathrm{\Omega}$. This source resistance is close to the optimal
impedance of the amplifier, which means that effects of voltage and
current noise are comparable. Using the values from Table~1, one obtains
the total input noise~2.3~nV/$\sqrt{\mbox{Hz}}$.

Even outside the nuclear magnetic resonance condition, there is a coupled
signal on the amplifier due to the mutual inductance and capacitance
between the excitation and pick-up coils. We use the second input of the
amplifier to compensate for this signal. This compensation is not
affected by possible drifts of the amplifier gain and can be used also
for calibration of the NMR signal amplitude.

\section{Conclusions}

A differential cryogenic amplifier has been developed and used for
high-resolution NMR measurements. The amplifier consumes only 2.5~mW of
power, which facilitates installation of many such amplifiers on a 4~K
stage of a regular dilution refrigerator. The input current noise of the
amplifier is 12~fA/$\sqrt{\mbox{Hz}}$ at 1~MHz, which renders our
amplifier excellent for high-$Q$, tuned NMR probes up to impedances of a
few hundred k$\mathrm{\Omega}$.  For the intended NMR applications, it is also
important to be able to compensate for the cross-talk input signal on the
first amplifier stage, and to measure the amplifier gain in-situ using
the differential input. Our design can be operated even at a smaller
power consumption level, but with a slight loss in the amplifier gain. In
future, after optimization for lower bias voltage operation, we plan to
position the amplifier on the still plate of a
dilution refrigerator, which would further help in minimizing possible
external interferences and parasitic impedance problems in cryogenic
input circuitry.

This work was performed as part of the ERC QuDeT project (670743). It was
also supported by Academy of Finland grants 310086 (LTnoise) and 312295
(CoE QTF). This research project made use of the Aalto University
OtaNano/LTL infrastructure which is part of European Microkelvin
Platform.

A commercial version of the amplifier is available from Aivon OY~\cite{aivon}.

\begin{thebibliography}{}

% Si device, 10K
\bibitem{2009_hayashi}
%A current to voltage converter for cryogenics using a CMOS operational amplifier
K.Hayashi, K.Saiton, Y.Shibayama, K.Shirahama,
\textit{J. of Phys.: Conference series} {\bf 150}, 012016 (2009)

% old FETs
\bibitem{1985_bloyet}
%Very-low-noise amplifier for low-temperature pulsed NMR experiments,
D. Bloyet, J. Lepaisant, E. Varoquaux,
\textit{Rev. Sci. Instrum.} {\bf 56}, 1763 (1985)


% self-made HEMT
\bibitem{2014_dong}
%Ultra-low noise high electron mobility transistors for high-impedance and low-frequency deep cryogenic readout electronics,
Q. Dong, Y. X. Liang, D. Ferry, A. Cavanna, U. Gennser, L. Couraud, Y. Jin,
\textit{Appl. Phys. Lett.} {\bf 105}, 013504 (2014)


% high frequency
\bibitem{2005_pospieszalski}
%Extremely low-noise amplification with cryogenic FETs and HFETs: 1970-2004
M. W. Pospieszalski,
\textit{Microwave Magazine} {\bf 6}, 62 (2005)

% high frequency
\bibitem{2004_roschier}
%Design of cryogenic 700 MHz amplifier
L. Roschier, P. Hakonen,
\textit{Cryogenics} {\bf 44}, 783 (2004)

% high frequency
\bibitem{2005_hu}
%A cryogenic voltage amplifier with 36MHz bandwidth using discrete GaAs metal-semiconductor field-effect transistors,
B. H. Hu, C. H. Yang,
\textit{Rev. Sci. Instrum.} {\bf 76}, 124702-1 (2005)


% noise voltage: 2.2~nV/sqrt(Hz)
% noise voltage: 4.6~fA/sqrt(Hz)
% optimal source resistance: 230 kOhm
% dissipation at 4K: 15 mW
% frequency: 699, 916 kHz
% tank circuit measurement!
\bibitem{2000_koivuniemi}
%Noise temperature of cryogenic FET amplifier with high-Q resonator,
J. Koivuniemi, M. Krusius,
\textit{Phys. B: Cond. Matter} {\bf 284-288}, 2149 (2000)

%\bibitem{2003_oukhanski}
%N. Oukhanski, M. Grajcar, E. Ilichev, and H.-G. Meyer
%Low noise, low power consumption high electron mobility transistors amplifier, for temperatures below 1 K
%Review of Scientific Instruments {\bf 74}, 1145 (2003)

% frequency: 100kHz - 35 MHz
% working temperature 380~mK
% voltage noise at 1MHz: 0.2~nV
% current noise at 1MHz: 10~fA (1~fA at 3-5MHz)
% power dissipation: 0.1-0.6 mW
\bibitem{2004_oukhanski}
%Ultrasensitive radio-frequency pseudomorphic high-electron-mobility-transistor readout for quantum devices
N. Oukhanskia, E. Hoenig,
\textit{Appl. Phys. Lett.} {\bf 85}, 2956 (2004)

% 0.7 nV/sqrt(Hz)
% 25 fA/sqrt(Hz)
\bibitem{2004_robinson}
%Cryogenic amplifier for $\sim 1$~MHz with a high input impedance using a commercial pseudomorphic high electron mobility transistor,
A. M. Robinson, V. I. Talyanskii,
\textit{Rev. Sci. Instrum.} {\bf 75}, 3169 (2004)

% cross correlation!
% voltage noise 0.4 nV
% current noise 5 fA
\bibitem{2006_dicarlo}
%System for measuring auto- and cross correlation of current noise at low temperatures,
L. Di Carlo, Y. Zhang, D. T. McClure, C. M. Marcus, L. N. Pfeiffer, K. W. West,
\textit{Rev. Sci. Instrum.} {\bf 77}, 073906 (2006)

% gain: 250
% freq: DC-850kHz
% voltage noise @ 100kHz 13 nV/sqrt(Hz)
% voltage noise @ 1MHz    2.2 nV/sqrt(Hz)
% dissipation at 4K: 10~mW
% input impedance: 10M
\bibitem{2008_mathur}
%A Low-Noise Broadband Cryogenic Preamplifier Operated in a High-Field Superconducting Magnet,
R. Mathur and R. W. Knepper and P. B. O'Connor,
\textit{Trans. Appl. Supercond.} {\bf 18}, 1781 (2008)

% temperature: 10K
% frequency 10hZ-10MHz
% voltage noise @1MHz 0.3 nV
% current noise @1MHz 120 fA
% optimal resistance  2.5 kOhm
\bibitem{2012_beev}
%Note: Cryogenic low-noise dc-coupled wideband differential
%amplifier based on SiGe heterojunction bipolar transistors
N. Beev, M. Kiviranta,
\textit{Rev. Sci. Instrum.} {\bf 83}, 066107 (2012)

% frequency 0.2 - 4MHz
% voltage noise: 1.5 nV
\bibitem{2013_arakawa}
%Cryogenic amplifier for shot noise measurement at 20 mK,
T. Arakawa, Y. Nishihara, M. Maeda, S. Norimoto, K. Kobayashi,
\textit{Appl. Phys. Lett.} {\bf 103}, 172104 (2013)


% freq: 0.2MHz - 6MHz
% voltage noise @ 100kHz  0.8 nV/sqrt(Hz)
% voltage noise @ 1MHz    0.8 nV/sqrt(Hz)
\bibitem{2017_hirata}
%Development of Cryogenic Enhancement-Mode Pseudomorphic High-Electron-Mobility Transistor Amplifier,
T. Hirata, T. Okazaki, K. Obara, H. Yano, O. Ishikawa,
\textit{J. Low Temp. Phys.} {\bf 187}, 596 (2017)

% % frequency: 20-50 MHz
% \bibitem{2018_johansen}
% D. H. Johansen, J. D. Sanchez-Heredia, J. R. Petersen, T. K. Johansen, V. Zhurbenko, J. H. Ardenkjr-Larsen,
% Cryogenic Preamplifiers for Magnetic Resonance Imaging,
% Transactions on Biomedical Circuits and Systems {\bf 12}, 202 (2018)

% Miller effect
\bibitem{miller}
P. Horowitz, W. Hill,
The Art of Electronics, 3rd edn., (Cambridge University Press, 2015), pp.~113--115 %section 2.4.5

% capacitors
\bibitem{2010_teyssandier}
%Commercially Available Capacitors at Cryogenic Temperatures,
F. Teyssandier, D. Prele,
\textit{Ninth International Workshop on Low Temperature Electronics - WOLTE9, Jun 2010, Guaruja, Brazil}

% capacitors
\bibitem{2006_teverovsky}
A. Teverovsky,
Performance and reliability of solid tantalum capacitors at cryogenic conditions,
(NASA GSFC, 2006).

\bibitem{aivon}
Aivon Oy, Valimotie 13A FI-00380, Helsinki, Finland,
http://aivon.fi/



%\bibitem{1995_Hastings}
%Characterization of a complementary metal-oxide semiconductor operational amplifier from 300 to 4.2 K,
%J.T. Hastings, K.-W. Ng,
%Review of Scientific Instruments {\bf 66}, 3691 (1995)

% Si devices

%A cryogenic amplifier for direct measurement of high-frequency signals from a single-electron transistor,
%S. L. Pohlen, R. J. Fitzgerald, J. M. Hergenrother, M. Tinkham,
%Applied Physics Letters {\bf 74}, 2884 (1999)

% old FETs ?
%frequency 1kHz - 1MHz
% not many details

%A cryogenic voltage amplifier with 36MHz bandwidth using discrete GaAs metal-semiconductor field-effect transistors,
%B. H. Hu, C. H. Yang,
%Review of Scientific Instruments {\bf 76}, 124702-1 (2005)

% high frequency

\end{thebibliography}

\end{document}


