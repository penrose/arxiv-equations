\newpage
%%%%%%%%%%%%%%%%%%%%%%%%%%%%%%%%%%
\section{Simulation and performance}
\label{subsec:SimAndPerf}
%%%%%%%%%%%%%%%%%%%%%%%%%%%%%%%%%%

The following paragraphs describe the performance of \gls{fatalic} in terms of noise, linearity and radiation
tolerance, based on both simulation and experimental measurements with 24 prototype units. It is noted however 
that the current All-in-One cards, used to accomodate \gls{fatalic} for the purposes of these studies, are not
adequate to support the slow channel. Specifically, due to the particularly large gain of this channel, the resistance 
range of the respective on-board potensiometer is not sufficient for the adjustment of the pedestal within the input 
dynamic range. As a workaround, it was decided to reduce the biasing current in the input stage, from the nominal value 
of \SI{0.5}{mA} to \SI{0.2}{mA}. This however affected the performance of the fast channels, causing dynamic amplification 
of the signal depending on its amplitude, with significant impact on the linearity. This effect would normally be eliminated 
by equipping the All-in-One cards with potentiometers of larger resistance range or by implementing slow control 
functionalities in \gls{fatalic}.

%---------------------------------
\subsection{Noise measurements}
\label{subsec:ped}
%---------------------------------

The dominant noise introduced in fast channels is white noise from the input stage expected, from simulation results, 
to be approximately \SI{8}{fC} in the high-gain channel. On the other hand, the dominant noise in the slow channel is 
$\sfrac{1}{f}$ noise from the input stage, estimated at the level of \SI{7}{nA}. This substantially exceeds the 
specification of \SI{0.25}{nA} (see Table\,\ref{tab:f5_specs}) initially defined for \gls{fatalic} because the lowest
region of the noise frequency spectrum was not properly included in the simulations used to define the design.
%due to overlooking the lower region of the noise frequency spectrum.
The spectral density of the noise for the fast and slow channels is given in Fig.\,\ref{fig:gnoise}.

%
\begin{figure}[!bt]
\centering
\subfloat[][]{\includegraphics[width=0.48\textwidth]{figures/performance/simu_hg_noise}\label{fig:gnoise_a}}
\subfloat[][]{\includegraphics[width=0.48\textwidth]{figures/performance/simu_slow_noise}\label{fig:gnoise_b}}
\caption{Simulation of the noise spectral density in the (a) fast channels (high gain) and (b) slow channel.}
\label{fig:gnoise}
\end{figure}

Experimental measurements of the noise were taken with the All-in-One cards connected to \glspl{pmt} under high voltage, 
without signal. Fig.\,\ref{fig:ped_sum_c},\,\ref{fig:ped_sum_d} present the mean and standard deviation of the pedestal
distribution in the fast channels, obtained by fit of a gaussian. The noise, estimated from the standard deviation and 
using the fC/ADC conversion factors of Section~\ref{subsec:linearity}, averages to $(6.1\pm 0.9)$\,fC in the high-gain 
channel, $(26.5\pm 3.1)$\,fC in the medium-gain channel and $(260.0\pm 21.1)$\,fC in the low-gain channel. Respective 
results for the case of the slow channel are presented in Fig.\,\ref{fig:ped_sum_e},\,\ref{fig:ped_sum_f}. In this case 
the average noise is found $(26.2\pm 0.7)$\,counts which, considering the design ratio of 0.25\,nA/count, corresponds to 
$(6.6\pm 0.8)$\,nA.

\begin{figure}[!t]
  \centering
  %\subfloat[][]{\includegraphics[width=0.4\textwidth]{figures/development/Pedestal_CH2_HighGain}\label{fig:ped_sum_a}}
  %\subfloat[][]{\includegraphics[width=0.4\textwidth]{figures/development/CorrMatrix}\label{fig:ped_sum_b}}\\
  \subfloat[][]{\includegraphics[width=0.48\textwidth]{figures/performance/ped_mean}\label{fig:ped_sum_c}}
  \subfloat[][]{\includegraphics[width=0.48\textwidth]{figures/performance/ped_rms}\label{fig:ped_sum_d}}\\
  \subfloat[][]{\includegraphics[width=0.48\textwidth]{figures/performance/integrator_ped_mean}\label{fig:ped_sum_e}}
  \subfloat[][]{\includegraphics[width=0.48\textwidth]{figures/performance/integrator_ped_rms}\label{fig:ped_sum_f}}
  \caption{Pedestal measurements, based on 24 prototype \gls{fatalic} units. (a) Mean and (b) standard deviation 
           of the pedestal in the high-, medium- and low-gain channels. (c) Mean and (d) standard 
           deviation of the pedestal in the slow channels.\label{fig:ped_sum}}
\end{figure}

%---------------------------------
\subsection{Linearity measurements}
\label{subsec:linearity}
%---------------------------------

Fig.\,\ref{fig:glin_a}, \ref{fig:glin_b} and \ref{fig:glin_c}, present the deviation from linearity, obtained by
simulation, of the analog pulse peak amplitude (in millivolt) as a function of the input charge. The non-linearity, 
defined as the deviation from a linear fit over the maximum channel response (\SI{1}{V}), does not exceed 0.3\% in 
the input charge range up to \SI{850}{pC}, while at higher charge values it increases to approximately 0.6\% at 
\SI{1.2}{nC}. In the case of the slow channel, the deviation from linearity is expected to be less than 0.2\% over 
the entire input current range, as shown in Fig.~\ref{fig:glin_d}.

\begin{figure}[!tbh]
  \centering
  \subfloat[][]{\includegraphics[width=0.48\textwidth]{figures/performance/simu_hg_linearity}\label{fig:glin_a}}
  \subfloat[][]{\includegraphics[width=0.48\textwidth]{figures/performance/simu_mg_linearity}\label{fig:glin_b}}\\
  \subfloat[][]{\includegraphics[width=0.48\textwidth]{figures/performance/simu_lg_linearity}\label{fig:glin_c}}
  \subfloat[][]{\includegraphics[width=0.48\textwidth]{figures/performance/simu_slow_linearity}\label{fig:glin_d}}
  \caption{Simulation results, showing the deviation from linearity in the (a) high-gain, (b) medium-gain, 
           (c) low-gain and (d) slow channel.\label{fig:glin}}
\end{figure}

Experimental studies of the linearity are carried out using the \gls{cis}. Since the linearity of the CIS has not been
verified, the response in this case is obtained from the sum of the selected digitised samples, which is less sensitive 
to the shape of the injected pulse. Fig.\,\ref{fig:lin_ex_a}, \ref{fig:lin_ex_b} and \ref{fig:lin_ex_c} present the measured response 
as a function of the injected charge for one unit. The maximum deviation from linearity (average from the tested prototype 
units, relative to the maximum response) in the high-gain channel is $(0.3\pm 0.1)\%$ above \SI{2}{pC}, increasing to 
$(1.3\pm 0.2)\%$ for lower charge values. In the medium-gain channel the deviation is $(1.4\pm 0.6)\%$, while in the low-gain 
channel it is $(0.4\pm 0.1)\%$, below \SI{800}{pC}, reaching $(3.5\pm 0.7)\%$ at \SI{1.2}{nC}. The fC/ADC conversion factors, 
summarised in Fig.\,\ref{fig:lin_ex_d}, are finally obtained from the slope of the linear interpolation and average to 
$(2.46 \pm 0.03)$\,fC/ADC, $(20.4 \pm 0.7)$\,fC/ADC and $(211.3 \pm 6.4)$\,fC/ADC, respectively.

\begin{figure}[!tbh]
  \centering
  \subfloat[][]{\includegraphics[width=0.48\textwidth]{figures/performance/Linearity_CH2_MD1_high_FF}\label{fig:lin_ex_a}}
  \subfloat[][]{\includegraphics[width=0.48\textwidth]{figures/performance/Linearity_CH2_MD1_medium_FF}\label{fig:lin_ex_b}}\\
  \subfloat[][]{\includegraphics[width=0.48\textwidth]{figures/performance/Linearity_CH2_MD1_low_FF}\label{fig:lin_ex_c}}
  \subfloat[][]{\includegraphics[width=0.48\textwidth]{figures/performance/QperADC_FF}\label{fig:lin_ex_d}}
  \caption{ Linearity measurements using the \gls{cis}. Response as a function of injected charge for one \gls{fatalic} unit
            in the (a) high, (b) medium and (c) low gain channel. (d) fC/ADC conversion factors for the 24 prototype units.
            \label{fig:lin_ex}}
\end{figure}

%---------------------------------
\subsection{Radiation Tolerance}
\label{subsec:rad}
%---------------------------------

The \SI{130}{nm} GF CMOS technology is recommendated by ATLAS due to its high radiation tolerance in terms of Total Ionising 
Dose (TID). It is suitable for the development of radiation-hard chips, up to at least \SI{100}{Mrad} with, however, a peak of
leakage current at $\sim$\SI{1}{Mrad}. The expected radiation level, based on both Monte Carlo and in-situ measurements is \SI{50}{krad}
(including conservative safety factors), well below the \SI{1}{Mrad} peak. Hence, no further assessment for TID or Non Ionising 
Energy Loss (NIEL) is deemed necessary for \gls{fatalic} (it is necessary though for the associated cards). 
%Single Event Effects (SEE) still need to be tested for hadron fluxes up to $\SI{6.45e12}{particles/cm^2}$
%(typically using \SI{200}{MeV} protons), which includes a safety factor of 2 to account for the various types of SEEs.

\clearpage

%%%%%%%%%%%%%%%%%%%%%%%%%%%%%%%%%%
\section{Energy reconstruction}
\label{sec:EnergyReco}
%%%%%%%%%%%%%%%%%%%%%%%%%%%%%%%%%%

As described in Section\,\ref{sec:strategy}, the fast channel shapers deliver an asymmetric pulse, the amplitude $A$ 
of which has to be reconstructed from the set of digitised samples (typically using seven samples), collected after
trigger decision. At the \SI{40}{MHz} sampling rate, the second sample is expected to coincide with the pulse peak. 
Small time-shifts $\tau$, of the order of a few nanoseconds, are however possible. Both $A$ and $\tau$ are reconstructed 
by Optimal Filtering~\cite{Fullana:2005dwa}, a weighted sum of the digitised samples with minimum sensitivity to both
correlated (e.g. pile-up) and uncorrelated noise.

\begin{figure}[!tb]
  \centering
  \subfloat[][]{\includegraphics[width=0.48\textwidth]{figures/energyreco/fatalic_sampraw}\label{fig:pulses_b}}
  \subfloat[][]{\includegraphics[width=0.48\textwidth]{figures/energyreco/fatalic_sampeff}\label{fig:pulses_c}}
  \caption{(a) \gls{adc} measurements from a \SI{100}{GeV} electron beam event. (b) Selected samples, normalised to the high-gain 
  channel scale, used to reconstruct the analog pulse by Optimal Filtering (superimposed).
  \label{fig:pulses}}
\end{figure}

%-------------------------------------------------
\subsection{Selection of digitised samples}
\label{sec:samples}
%-------------------------------------------------

The dynamic gain switch of \gls{fatalic} is exploited in order to ensure the best possible resolution for the acquisition of 
each digitised sample $S_i$. If the high-gain channel does not saturate, the \gls{adc} measurement is acquired from the alternative 
gain output. Otherwise, the switch turns to low gain, in which case the medium gain output is preferred. If, however, the medium-gain 
channel also saturates, then the alternative, low-gain measurement is used. It is noted that, once the high-gain channel saturates, 
the switch remains to low gain for the next seven samples (low-gain block). This is imposed through the Mainboard FPGAs to allow the 
high-gain channel to recover from the saturation state.
%The same treatment is followed by the reconstruction algorithm for the medium gain channel; in case it saturates, the 
%alternative (low) gain readout is exclusively used for the next seven samples.
Once the \gls{adc} value has been acquired, the pedestal $p_i$ of the selected channel is subtracted and the sample is normalised to 
the high-gain channel scale. Fig.\,\ref{fig:pulses} demonstrates a characteristic case in which digitised samples from different
channels are selected.

%-------------------------------------------------
\subsection{Simulated effects}
\label{sec:simeffects}
%-------------------------------------------------

Simulation studies have been carried out in order to test the impact of different experimental effects on the energy resolution, 
defined as the standard deviation of the $(E_{\text{reco}}-E_{\text{true}})/E_{\text{true}}$ distribution, where $E$ refers to 
the Tile cell energy corresponding to one of two \glspl{pmt} and the label \textit{true} (\textit{reco}) refers to the input
(reconstructed) energy. In these studies, the Optimal Filter is calibrated with respect to the measured output pulse shape of a 
prototype unit, while, to simulate charge-injection, this reference pulse is given random amplitudes and is subsequently sampled 
and digitised, as described in Section~\ref{sec:samples}. Random noise is then added to each digitised sample, based on measurements 
with the same prototype unit; \SI{1.5}{ADC} counts for the low- and medium-gain channels (which correspond to \SI{8.4}{fC} and 
\SI{32}{fC}, respectively), and \SI{3.5}{ADC} counts for the high-gain channel (\SI{256}{fC}). As shown in Fig.\,\ref{fig:simu1}, 
the expected resolution is kept below 2\% in the input range above \SI{2}{pC}, while for lower values it increases up to $\sim$7\%.

\begin{figure}[t]
  \centering
  \subfloat[][]{\includegraphics[width=0.48\textwidth]{figures/energyreco/simu1}\label{fig:simu_a}}
  \subfloat[][]{\includegraphics[width=0.48\textwidth]{figures/energyreco/simu2}\label{fig:simu_b}}\\
  \subfloat[][]{\includegraphics[width=0.48\textwidth]{figures/energyreco/simu_pulses}\label{fig:simu_pulses}}
  \subfloat[][]{\includegraphics[width=0.48\textwidth]{figures/energyreco/simu3}\label{fig:simu_c}}
  \caption{(a,b) Energy resolution as a function of the true energy in different scenarios probing the impact of various
           experimental effects. (c) Variation of the fast channel shaper output with the input charge and (d) its impact
           on the energy resolution.\label{fig:simu1}}
\end{figure}

\paragraph*{Electronic noise and gain saturation:} The reference scenario described above is first compared
     to the ideal case where no electronic noise and/or no low-gain block is applied. The results are presented in 
     Fig.\,\ref{fig:simu_a}. The impact of noise on the intrinsic 18-bit resolution is about one order of
     magnitude, whereas the cost of the low gain block is less than 1\% over the entire input range.

\paragraph*{Phase variations:} The arrival time of the pulse, with respect to the digitising clock, may vary 
      by a few nanoseconds. To simulate this effect, the generated pulses are shifted by a random phase 
      $\tau\in[-8,8]$\,ns. The results (Fig.\,\ref{fig:simu_a}) show that such time-shifts are accounted for 
      by Optimal Filtering, affecting the resolution by $\sim$1\%.

%\paragraph*{Pedestal variations:} Pileup interactions introduce a continuous signal to the \glspl{pmt}, which 
%      is added to the pedestals, altering the relative gain of the fast channels. To simulate this effect, each pedestal
%      is convoluted with a gaussian distribution of mean 0 and standard deviation 0.20. The observed impact on the 
%      resolution is $\sim$3\% (Fig.\,\ref{fig:simu_a}).

\paragraph*{Gain variations:} The impact of gain variations is tested by shifting the gain of each channel by 5\% 
      or 10\%. Such variations do not affect the resolution (Fig.\,\ref{fig:simu_b}) but rather introduce a shift to 
      the reconstructed energy, which can be recovered by calibration using the \gls{cis}.

\paragraph*{Pulse shape variations:} Imperfections of the electronics are found to distort the output pulse shape
      depending on the input charge. Fig.\,\ref{fig:simu_pulses} compares simulated pulses obtained with different
      injected charges. These variations of the pulse shape have less than 0.5\% impact on the resolution 
      (Fig.\,\ref{fig:simu_c}). However, they introduce a shift to the reconstructed energy (of less than 1\%), which 
      can be accounted for by applying a scale factor to the gain as a function of the input charge.

\paragraph*{Pile-up:} Inelastic $pp$ interactions taking place in the same (in-time pile-up) and adjacent (out-of-time pile-up) 
      bunch crossings introduce parasitic pulses, which contaminate the signal of the actual hard scattering. In the years 
      2015-2017 the average number of inelastic interactions per bunch-crossing was measured $\langle\mu\rangle=32$. In the 
      HL-LHC the nominal expected rate is $\langle\mu\rangle=140$, but it is foreseen to increase as much as $\langle\mu\rangle\approx 200$. 
      To test the performance of \gls{fatalic} in the presence of pile-up background, random in-time and out-of-time pile-up 
      pulses are added to the signal according to the energy spectrum of minimum-bias events, obtained after full simulation 
      of the detector for different values of $\langle\mu\rangle$. The impact on the resolution is prominent in the low energy 
      range, as presented in Fig.\,\ref{fig:simu_pu} for the readout cell A13 (the most exposed to radiation from $pp$ collisions) 
      and cell D1 (the least exposed). It can however be reduced by calibrating the Optimal Filter against the correlation matrix 
      of the pile-up background~\cite{Fullana:2005dwa}.

%-------------------------------------------------
\subsection{Two-gain scenario}
\label{sec:2gain}
%-------------------------------------------------

In order to quantify the benefit of having three gains, the scenario of using two gains with a gain ratio of 32
is explored, based on the simulation described above. Considering digitisation with 12-bit \glspl{adc}, the effective 
output range in this case is 17-bits. The noise is assumed to be \SI{1.8}{ADC} counts for both channels. As seen in 
Fig.\,\ref{fig:simu_2gain} the resolution drops by $\sim$8\% in the intermediate range (\SI{40}{MeV}-\SI{180}{MeV})
which, in the case of \gls{fatalic}, is recovered by the medium-gain channel.

\clearpage

\begin{figure}[h]
  \centering
  \subfloat[][]{\includegraphics[width=0.48\textwidth]{figures/energyreco/simu_pu}\label{fig:simu_pu}}
  \subfloat[][]{\includegraphics[width=0.48\textwidth]{figures/energyreco/simu4}\label{fig:simu_2gain}}
  \caption{(a) Energy resolution as a function of the true energy for $\langle\mu\rangle=30$,140 and 200, in Tile cells 
           D1 and A13. (b) Energy resolution achieved with \gls{fatalic}, as a function of the true energy, compared to 
           the scenario of using only two fast channels with a gain ratio of 32.\label{fig:simu2}}
\end{figure}
