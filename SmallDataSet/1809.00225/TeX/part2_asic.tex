%%%%%%%%%%%%%%%%%%%%%%%%%%%%%%%%%%%%%%%%%%%%%%%%%%%%%%%%%%%%%%%%%%%%%%
\section{The FATALIC Architecture}
\label{sec:strategy}
%%%%%%%%%%%%%%%%%%%%%%%%%%%%%%%%%%%%%%%%%%%%%%%%%%%%%%%%%%%%%%%%%%%%%%

\begin{figure}[b]
\centering
  \includegraphics[height=0.25\textheight]{figures/asic/fatalic5.jpg}
  \caption{Photograph of a prototype \gls{fatalic} unit.\label{fig:fatalic}}
\end{figure}

\gls{fatalic} (Fig.\,\ref{fig:fatalic}) is an ASIC, based on \SI{130}{nm} GlobalFoundries (GF) Complementary Metal-Oxide-Semiconductor 
(CMOS) technology, proposed to replace the current discrete \gls{fe} readout electronics of the \gls{TileCal}. The proposal of an ASIC, 
embedding both analog signal processing and digitisation, is motivated by its significant advantages in terms of simplicity, radiation 
tolerance, power consumption and production cost for a large number of units. On the other hand, since the operating low voltage 
(\SI{1.6}{V}) is dictated by the technology, \gls{fatalic} must rely on a current-driven, rather than a voltage-driven architecture in 
order to handle the large input dynamic range. Hence, an input stage employing current conveyors is implemented to adjust the impedance 
and distribute the signal to the different channels.

\begin{figure}[tb]
  \centering
  \includegraphics[width=0.8\textwidth]{figures/asic/fatalic_5}
  \caption{Block diagram of \gls{fatalic}.}
  \label{fig:fatalic5}
\end{figure}

The specifications of \gls{fatalic} are listed in Table\,\ref{tab:f5_specs}, while a block diagram summarising the architecture is given 
in Fig.\,\ref{fig:fatalic5}. In order to handle the large input range (up to \SI{1.2}{nC}), \gls{fatalic} provides three fast channels to
process the \gls{pmt} signal at the \SI{40}{MHz} LHC clock rate, with relative amplification $\times$1 (low gain), $\times$8 (medium gain)
and $\times$64 (high gain). This three-gain scheme is chosen, instead of using two gains, in order to improve the energy resolution (at the
Tile cell level) across the input range, as shown by the related study in Section\,\ref{sec:2gain}. In parallel, the signal is routed to an 
additional, slow channel for integration over a large ($\SI{100}{\mu s}$) time constant.

\begin{table}[t]
	\begin{center}
		\begin{tabular}{l c}
			\toprule
			Technology & 130\,nm CMOS GF \\
			Number of channels per ASIC & 1 \\
			Polarity & negative \\
			Fast channel dynamic range in charge & \SI{25}{fC} - \SI{1.2}{nC}\\
			Fast channel dynamic range in current peak& \SI{1.25}{\mu A} - \SI{60}{mA}\\
			Rise time of the current peak & \SI{4}{ns}\\
			Fall time of the current peak & \SI{36}{ns} \\
			Fast channel noise (rms) & <\SI{12.5}{fC} \\
			Slow channel dynamic range in average current & \SI{0.5}{nA} - \SI{1}{\mu A} \\
			Slow channel Noise (rms) & 0.25\,nA \\
			Power consumption & $\sim$\SI{200}{mW} \\
			Power supply      & 1.6\,V \\
			Output            & 12-bit words \\
			\bottomrule
		\end{tabular}
		\caption{Specifications of \gls{fatalic}.\label{tab:f5_specs}}
	\end{center}
\end{table}

%---------------------------------
\subsection{Fast channels}
%---------------------------------
In the fast channels the signal is read by three current conveyors with different input impedances in order to define the respective gain ratios. 
Current integration and current-to-voltage conversion follow at the transimpedance-shaping stage by three identical shapers, composed by a 
differential amplifier with RC feedback loop. A \SI{25}{ns} time constant is used to produce an asymmetric output pulse with \SI{25}{ns} 
peaking time, as shown in Fig.\,\ref{fig:pulse}. The digitisation is carried out in the ASIC, by 12-bit \SI{40}{MS/s} \glspl{adc}, to avoid 
degradation of the signal during transmission to an external converter. This establishes an effective 18-bit dynamic range and defines the
range of each channel as: $\SI{25}{fC}-\SI{20}{pC}$ (high gain), $\SI{200}{fC}-\SI{164}{pC}$ (medium gain) and $\SI{1.6}{pC}-\SI{1.3}{nC}$ (low gain).

\begin{figure}[!t]
  \centering
  \includegraphics[width=0.6\textwidth]{figures/energyreco/fatalic_pulse}
  \caption{Output analog pulse from the fast channel shapers. At the \SI{40}{MHz} clock rate, the 2$^\text{nd}$ sample coincides with the pulse peak.\label{fig:pulse}}
\end{figure}

%---------------------------------
\subsection{Slow channel}
%---------------------------------
The slow channel is designed to integrate low amplitude currents in the range from \SI{0.5}{nA} to $\SI{1}{\mu A}$ with minimum contamination
from $\sfrac{1}{f}$ noise, induced by the input stage. A current conveyor with low input impedance is therefore used to drive approximately 87\% 
of the \gls{pmt} signal to the slow channel, while current integration and current-to-voltage conversion are carried out by a differential 
amplifier with large time-constant ($\SI{100}{\mu s}$) RC feedback. The integrated signal is then sampled by a 12-bit \SI{833}{kS/s} \gls{adc}. 
Finally, to minimise the contamination from white noise in the measurement of such low amplitude currents, the data are obtained after averaging
over a time interval of \SI{10}{ms}, as described in Section\,\ref{sec:cards}.

%---------------------------------
\subsection{Outputs}
\label{sec:output}
%---------------------------------

The 12-bit fast channel samples are read-out from 12 respective output pins. In order to comply with the bandwith of the
uplink to the back-end electronics (see Section\,\ref{sec:cards}), \gls{fatalic} is restricted to provide the data of only 
two of the three channels; the 12-bit data of the medium-gain channel are always delivered upon the rising edge of the \SI{40}{MHz} 
clock, while the data of either the low- or the high-gain channel (alternative gain) are delivered upon the falling edge. 
The selection of the alternative gain is made by the embedded digital block, depending on the saturation of the most sensitive 
channel (dynamic gain switch), and is declared by an output flag. The dynamic gain switch can however be forced to provide 
only low gain readout by a dedicated input bit. On the other hand, slow channel data are delivered serially, at \SI{10}{Mbps} 
(\SI{833}{kHz} for the 12 bits), through a single readout pin.

\clearpage



%%%%%%%%%%%%%%%%%%%%%%%%%%%%%%%%%%%%%%%%%%%%%%%%%%%%%%%%%%%%%%%%%%%%%%
\section{Development of the integrated circuit}
\label{sec:ASICdev}
%%%%%%%%%%%%%%%%%%%%%%%%%%%%%%%%%%%%%%%%%%%%%%%%%%%%%%%%%%%%%%%%%%%%%%

%~~~~~~~~~~~~~~~~~~~~~~~~~~~~~~~~~~~~~~~~~~~~~~~~~
\subsection{Current conveyor system}
\label{subsec:Conveyor}
%~~~~~~~~~~~~~~~~~~~~~~~~~~~~~~~~~~~~~~~~~~~~~~~~~

The current conveyor system, schematically depicted in Fig.~\ref{iconv}, consists of one input stage and four output stages, 
which provide differential signal to each of the \gls{fatalic} channels. At the input stage, a current mirror sets the biasing 
current to the nominal value of \SI{0.5}{mA}. The \gls{pmt} signal is read at the source of four common-gate NMOS\footnote{
%------------------------------------------
Negative channel Metal Oxide Semiconductor}
%------------------------------------------
with the same length (L) but with different widths: $\mathrm{W_{fast}/1}$ (high gain), $\mathrm{W_{fast}}/8$ (medium gain), $\mathrm{W_{fast}}/64$ 
(low gain) and $\mathrm{W_{slow}=8\,W_{fast}}$ (slow channel). A PMOS\footnote{
%------------------------------------------
Positive channel Metal Oxide Semiconductor}
%------------------------------------------
current mirror is then used to replicate the current from the drain of each NMOS onto the respective output stage, while a replica of 
the input stage (dummy) is implemented to return just the biasing current to be subtracted from the final output.

In terms of noise, the performance of such current-conveyor is modest compared to a charge preamplifier, but well adequate, 
given the large dynamic range of input stage. The input impedance is kept below $\SI{70}{\Omega}$ for the entire input range, 
as presented in Fig.\,\ref{fig:Zin_vs_Idc}. Finally, to account for the open-loop configuration of the input stage, which 
induces large dispersion among the pedestal values, a tuning system with a \gls{dac} is also added to the above design.

\begin{figure}[b]
	\begin{center}
		\includegraphics[width=\textwidth]{figures/asic/ConvF5_v2.pdf}
		\caption{The current conveyor system, consisting of the input stage, the four output stages and a replica 
        (dummy) of the input stage in order to subtract the biasing current from the differential output.}
		\label{iconv}
	\end{center}
\end{figure}

\begin{figure}[t]
	\begin{center}
		\includegraphics[width=0.6\textwidth]{figures/performance/simu_impd}
		\caption{Input impedance as a function of the input charge for the nominal biasing current, $\Ibias=0.5$\,nA, 
        and for a test value of $\Ibias=0.2$\,nA.}
		\label{fig:Zin_vs_Idc}
	\end{center}
\end{figure}

%~~~~~~~~~~~~~~~~~~~~~~~~~~~~~~~~~~~~~~~~~~~~~~~~~
\subsection{Signal shaping}
\label{subsec:Shaping}
%~~~~~~~~~~~~~~~~~~~~~~~~~~~~~~~~~~~~~~~~~~~~~~~~~

The shaper (Fig.\,\ref{fig:amp_a}) is a differential amplifier with RC feedback loop, with a time constant of \SI{25}{ns} 
($R=\SI{5}{k\Omega}$, $C=\SI{5}{pF}$) in the fast channels and $\SI{100}{\mu s}$ ($R=\SI{500}{k\Omega}$, $C=\SI{200}{pF}$) 
in the slow channel. The amplifier is based on a folded-cascode boosted architecture, depicted in Fig.\,\ref{fig:amp_b}, 
with two differential input pairs, one NMOS and one PMOS, and Common Mode FeedBack (CMFB) to control the output common mode. 
Its open-loop characteristics are listed in Table\,\ref{tab:opamp}. 

\begin{table}[h]
	\centering
	\begin{tabular}{l c}
		\toprule
		Open-loop gain       & \SI{84}{dB}\\
		-3db bandwidth       & \SI{38}{kHz}\\
		Unity-gain bandwidth & \SI{650}{MHz}\\
		Slew rate            & $\SI{220}{V/\mu s}$\\
		Phase margin         & 85$^\circ$\\
		Supply voltage       & \SI{1.6}{V}\\
		Consumption          & \SI{2.3}{mA}\\
		\bottomrule
	\end{tabular}
\caption{\label{tab:opamp}Open-loop characteristics of the shaper with the \gls{adc} input capacitive load of \SI{1.6}{pF}.}
\label{amp_charac}
\end{table}	

\begin{figure}[t]
  \centering
  \subfloat[][]{\includegraphics[width=0.6\textwidth]{figures/asic/shap}\label{fig:amp_a}}\\
  \subfloat[][]{\includegraphics[width=0.8\textwidth]{figures/asic/Ampli}\label{fig:amp_b}}
  \caption{(a) Schematic of the shaper, consisting of a differential amplifier with RC feedback loop. 
           (b) Architecture of the differential amplifier.\label{fig:amp}}
\end{figure}

%~~~~~~~~~~~~~~~~~~~~~~~~~~~~~~~~~~~~~~~~~~~~~~~~~
\subsection{Signal digitisation}
\label{subsec:ADC}
%~~~~~~~~~~~~~~~~~~~~~~~~~~~~~~~~~~~~~~~~~~~~~~~~~

The shaper output is sampled by three 12-bit \SI{40}{MS/s} \glspl{adc} in the fast channels and one 12-bit \SI{833}{kS/s} 
\gls{adc} in the slow channel. The \gls{adc} design by LPC is based on the pipeline architecture
%, which is well adequate for the particular resolution and sampling rate, 
with 1.5-bit-per-stage resolution.
%. is chosen to avoid degradation of the performance due to the comparator offsets. 
Among the various efficient \gls{adc} architectures developed and improved over the last ten years, the pipelined \gls{adc} 
has been adapted for high resolution, speed and dynamic range with relatively low power consumption and low component count. 
In CMOS technologies, resolutions in the range of 10-14 bits with a sampling frequency up to \SI{100}{MS/s} are typically 
achieved with power consumption lower than \SI{100}{mW}.

Fig.\,\ref{fig:adc} presents the block diagram of the pipelined ADC, with two output bits per stage, displaying the architecture
of one of the stages. Each stage receives a differential input voltage, in the range $\pm$\SI{500}{mV}, which is read by a 
sample-and-hold (S/H) and a 2-bit flash \gls{adc}. The flash \gls{adc} compares the input to two threshold voltages ($\pm$\SI{125}{mV}) 
and outputs a 2-bit word, while a \gls{dac} and a residue amplifier provide the input to the next stage, as presented in 
Table\,\ref{tab:adc_stage}. Although a 2-bit word is delivered, the effective resolution is 1.5-bit since the combination \texttt{11} is 
avoided. This bit redundancy limits the degradation of the Integral Non-Linearity\footnote{
%------------------------------------------------------------
The INL is defined as the deviation, in Least-Significant-Bits (LSBs), of the output code from the ideal transfer-function.}
%------------------------------------------------------------
(INL) due to variations of the residue amplifier gain or the comparator offsets (the INL is unaffected by variations of the offset 
voltage for up to $\pm$12.5\% of the full-scale input voltage). The complete architecture includes 12 cascading stages and is 
followed by the digital correction block (see Section\,\ref{subsec:DigitalBlock}), which delivers the final digital code. The 
estimated power consumption of the \gls{adc} is \SI{48}{mW}.

\begin{table}[h]
  \centering
  \begin{tabular}{rcl c c c}
  \toprule
                      & $V_\mathrm{in}^i$  &  & $[b_2b_1]$  & DAC          & $V_\mathrm{in}^{i+1}$\\
  \midrule
  $ V_\mathrm{ref}/4$ &-& $ V_\mathrm{ref}$   & [10] & $V_\mathrm{ref}/2$  & $2V_\mathrm{in}^i - V_\mathrm{ref}$\\
  $-V_\mathrm{ref}/4$ &-& $ V_\mathrm{ref}/4$ & [01] & 0                   & $2V_\mathrm{in}^i$\\
  $-V_\mathrm{ref}$   &-& $-V_\mathrm{ref}/4$ & [00] & $-V_\mathrm{ref}/2$ & $2V_\mathrm{in}^i + V_\mathrm{ref}$\\
  \bottomrule
  \end{tabular}
  \caption{Each stage $i$ of the pipeline converts the input voltage $V_\mathrm{in}^{i}$, in the dynamic range $\pm V_\mathrm{ref}$, 
           into a 2-bit word $[b_2b_1]$ and feeds the amplified residual $V_\mathrm{in}^{i+1}=2(V_\mathrm{in}^i-\mathrm{DAC})$ to the 
           next stage.\label{tab:adc_stage}}
\end{table}

\begin{figure}[tb]
	\centering
	\includegraphics[width=0.9\textwidth]{figures/asic/adc}
	\caption{Block diagram of the 12-bit pipelined ADC with two output bits per stage.}
	\label{fig:adc}
\end{figure}

%~~~~~~~~~~~~~~~~~~~~~~~~~~~~~~~~~~~~~~~~~~~~~~~~~
\subsubsection*{The comparator}

The architecture of the comparator is presented in Fig.\,\ref{fig:comp_adc}. The transconductance input stage is fully differential, 
comparing the differential input signal to the differential threshold voltage, while isolating the input from kick-back noise, 
induced by the switching of the subsequent latched stage. The latched stage performs the comparison when the switch-transistors are 
OFF and a reset when they are ON. The state of the comparator, when it is latched, is memorised thanks to the bistable third stage. Lastly, 
two NOT gates perform the final digital shaping of the output signal. The main characteristics of the comparator are given in 
Table\,\ref{tab:comp}.

%The latch state of the comparator is memorised by a bi-stable third stage.
\begin{figure}[t]
  \centering
  \includegraphics[width=0.9\textwidth]{figures/asic/Comp}
  \caption{Architecture of the comparator (single-ended representation for simplicity).}
  \label{fig:comp_adc}
\end{figure}

%~~~~~~~~~~~~~~~~~~~~~~~~~~~~~~~~~~~~~~~~~~~~~~~~~
\subsubsection*{The \gls{dac}}

The \gls{dac} employs a set of switches, controlled by the comparators, to select the differential reference voltage to 
be applied on the feedback capacitor of the residue amplifier. Given the $\pm$\SI{500}{mV} input dynamic range, the 
reference voltages are set to $\pm$\SI{250}{mV}.

\begin{table}[h]
\centering
	\vspace{0.5cm}
	\begin{tabular}{l c}
	\toprule
	Clocking frequency & 40\,MHz\\
	Sensitivity        & 50\,$\mu$V\\
	Power supply       & 1.6\,V\\
	Power consumption  & 2.1\,mW\\
	\bottomrule
	\end{tabular}
	\caption{\label{tab:comp}Characteristics of the comparator.}
\end{table}	

%~~~~~~~~~~~~~~~~~~~~~~~~~~~~~~~~~~~~~~~~~~~~~~~~~
\subsubsection*{The gain-2 residue amplifier}

The residue amplifier, displayed in Fig.\,\ref{fig:adc}, is a differential amplifier with capacitive feedback. Capacitive, 
rather than resistive feedback is used for better component matching, which is crucial for the accuracy of the amplification 
and, therefore, the linearity of the \gls{adc}. A capacitance of \SI{800}{fF} is sufficient to minimise both the thermal 
noise ($kT/C$) and the component mismatch ($\sim$$1/\sqrt{C}$). On the other hand, the design requires a small die surface 
and low supply current for dynamic performance. To match both the sampling ($C_\mathrm{S}$) and feedback ($C_\mathrm{F}$) 
capacitance to \SI{800}{fF}, with an accuracy better than 0.1\%, an array of four \SI{1600}{fF} MIM capacitor unit cells is 
drawn in common-centroid layout, while dummy switches are added to counterbalance parasitic capacitors introduced by the 
reset switches. Lastly, the timing sequence, controlling the sample, hold/amplification and reset phases of each stage, has 
been carefully defined to prevent charge disruption.

%\begin{figure}[bt]
%\centering
%\includegraphics[width=0.8\textwidth]{figures/asic/etage_adc}
%\caption{One stage of the pipelined ADC (single-ended representation for simplicity).}
%\label{fig:ADC_stage}
%\end{figure}

%~~~~~~~~~~~~~~~~~~~~~~~~~~~~~~~~~~~~~~~~~~~~~~~~~
\subsection{Digital block}
\label{subsec:DigitalBlock}
%~~~~~~~~~~~~~~~~~~~~~~~~~~~~~~~~~~~~~~~~~~~~~~~~~

\begin{figure}[tb]
\centering
\includegraphics[width=0.9\textwidth]{figures/asic/Fatalic5_digital}
\caption{Architecture of the digital block.}
\label{bloc_digital2}
\end{figure}

Each \gls{adc} delivers twelve 2-bit words at the \SI{40}{MHz} clock rate. The result of the complete conversion is therefore 
a 24-bit word with 12 redundant bits. The digital block (Fig.\,\ref{bloc_digital2}) synchronises the outputs using shift 
registers to compensate for the delay due to the position of each stage in the pipeline. The total latency of this process is
equal to eight clock cycles (\SI{200}{ns}) and it is fixed for every power cycle. Digital summation (with carry-save) of the
most-significant bit of each stage with the least-significant bit of the previous stage is then performed to the final 12-bit 
code every \SI{25}{ns} for the fast channels and $\SI{1.2}{\mu s}$) for the slow channel. The second function of the digital block 
is to select the alternative gain output between the high-gain and the low-gain channel data (see Section\,\ref{sec:output}). 
This selection is made according to the output of the low-gain channel; if it is equal or higher (lower) than \SI{600}{ADC} counts, 
then the low-gain (high-gain) channel data are delivered.

%~~~~~~~~~~~~~~~~~~~~~~~~~~~~~~~~~~~~~~~~~~~~~~~~~
\subsection{The Floorplan}
%~~~~~~~~~~~~~~~~~~~~~~~~~~~~~~~~~~~~~~~~~~~~~~~~~

The floorplan of \gls{fatalic} (Fig.\,\ref{fig:fatalic_floor}) is optimized for minimum surface, while preserving signal 
integrity. Two main regions are distinguished; the region hosting the input stage and the shapers (analog block), and 
the region hosting the \glspl{adc} and the digital block. The two regions are isolated by a high impedance (BFMOAT) layer 
to reduce the coupling. The analog power and reference voltage rails are also decoupled by embedded large capacitors. The 
total surface of the chip measures $\SI{6.3}{mm^2}$, while the core area is limited to $\SI{2.3}{mm^2}$.

\begin{figure}[tb]
\bigskip
\centering
  \includegraphics[width=0.9\textwidth]{figures/asic/F5.png}
  \caption{The floorplan of \gls{fatalic}. Two main regions are distinguished; the region hosting the analog blocks
  (left) and the region hosting the \glspl{adc} and the digital block.\label{fig:fatalic_floor}}
\end{figure}

