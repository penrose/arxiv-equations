\subsection{Toric Code}

\begin{figure}
	\includegraphics[width=0.5\linewidth]{unit_cell}
	\caption{An illustration of the lattice and a $B_p$ stabilizer check. The $2\times 2$ unit cells are marked by the red color. We also give an example of a coarse-grained edge, which locates at the top-right and contains two edges of the original lattice.}
	\label{fig:lattice}
\end{figure}

First we give a brief introduction of toric code.
Consider a $L\times L$ square lattice with periodic boundary condition, where a qubit lives on each edge.
The stabilizer group of toric code is generated by two types of operator $A_s$ and $B_p$
\begin{equation}
A_s = \bigotimes_{q\in n(s)} X_q, \quad B_p = \bigotimes_{q\in n(p)} Z_q,
\end{equation}
where $s$ and $p$ is any site and plaquette respectively, and $n(\cdot)$ consists of the 4 qubits neighboring $s$ or $p$.
The logical-$Z$ operators have the form
\begin{equation}
\bar{Z}_i = \bigotimes_{q\in l_i} Z_q ,
\end{equation}
where $l_{1,2}$ are two shortest inequivalent non-contractible loop.
The toric code has a distance $d=L$.

In this paper, we will focus on the bit-flip noise model, i.e. only $X$ errors can happen.
We will also assume perfect measurements.
Under this restriction, the quantum states will stay in the $+1$ eigenspace of $A_s$.
Therefore, we only need to consider the expectation values of $B_p$ and $\bar{Z}_i$.
For simplicity, let us suppose in the beginning $\langle Z_i \rangle = +1$.
And then a set of $X$ errors happened, which leads to the syndrome $s=\{\langle B_p \rangle \}$.
The goal of a decoder is to apply $X$ to the qubits, such that $\langle B_p \rangle$ and $\langle Z_i \rangle$ return to $+1$.
Without going to detail, we claim it is enough to know the parity of the number of $X$ errors happened on the loops $l_{1,2}$.
These two parities will be the final training target for our neural decoder.
We will refer to the two parities as logical correction.

\subsection{Renormalization Group Decoder}

Let us first set up some notation.
We will use $e$ to denote an edge.
When we say $e$ is a coarse-grained edge, we mean $e$ is an edge of a unit cell which consists of two (or more) edges of the original lattice.
We use $x(e) = 1$ to denote an $X$ error happened on edge $e$, and otherwise $x(e)=0$.
When $e$ is a coarse-grained edge consists of edges $\{e_i\}$, $x(e) = \sum_i x(e_i) \mod 2$.
Lastly, the marginal probability distribution of error on edge $e$ is denoted by $p_e$.
For example, the error rate of edge $e$ is $p_e(1)$, and $p_e(0) = 1-p_e(1)$.

One renormalization stage consists of the following:
\begin{enumerate}
	\item Divide the lattice into $m\times m'$ unit cells, where in this work $m=m'=2$.
	\item The outputs of the renormalization step are $\{p_e\}$ for each coarse-grained edge $e$ that is a border of a unit cell.
	They are computed by belief propagation, which is a heuristic procedure for computing marginal probabilities (see \aref{appendix:bp}).
	These $\{p_e\}$ are treated as the error rate of the coarse-grained edge $e$ for the next renormalization stage.
\end{enumerate}

At the end of renormalization process, we obtain $p_e$ for $e$ being either of the two non-contractible loops.
For simplicity, we assume the two non-contractible loops are $l_{1,2}$.
Thus, we get an approximation of the marginal probability distribution for logical correction.
