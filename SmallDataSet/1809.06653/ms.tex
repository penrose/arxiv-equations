\documentclass[journal]{IEEEtran}

\pdfoutput=1

\usepackage{color}
\usepackage{amsmath}
\usepackage{graphicx}
\usepackage{tabularx}
\usepackage{amssymb} 
\usepackage{gensymb}
\usepackage{multirow}
\usepackage{subfig}
\usepackage[table]{xcolor}
\usepackage{array}

\usepackage[bookmarks=false,colorlinks=true,citecolor=blue]{hyperref}

\def\fmd{{f_\text{mD}}}
\def\fdmax{{f^\mathrm{D}_\text{max}}}


\begin{document}

% paper title
\title{Toward Unobtrusive In-home Gait Analysis \\ Based on Radar Micro-Doppler Signatures}

% author names and IEEE memberships
\author{Ann-Kathrin~Seifert,~\IEEEmembership{Student Member,~IEEE,}
        Moeness~G.~Amin,~\IEEEmembership{Fellow,~IEEE,}
        and~Abdelhak~M.~Zoubir,~\IEEEmembership{Fellow,~IEEE} 
        \thanks{
Ann-Kathrin Seifert is a member of the Signal Processing Group at Technische Universit\"at Darmstadt, Germany. E-Mail: seifert@spg.tu-darmstadt.de

Moeness G.~Amin is the Director of the Center for Advanced Communications at Villanova University, PA, USA. E-Mail: moeness.amin@villanova.edu

Abdelhak M.~Zoubir is the Head of the Signal Processing Group at Technische Universit\"at Darmstadt, Germany. E-Mail: zoubir@spg.tu-darmstadt.de
        }}% <-this % stops a space
\markboth{Submitted for publication}{}
% make the title area
\maketitle

\begin{abstract}
\textit{Objective:} In this paper, we demonstrate the applicability of radar for gait classification with application to home security, medical diagnosis, rehabilitation and assisted living. Aiming at identifying changes in gait patterns based on radar micro-Doppler signatures, this work is concerned with solving the intra motion category classification problem of gait recognition. 
\textit{Methods:} New gait classification approaches utilizing physical features, subspace features and sum-of-harmonics modeling are presented and their performances are evaluated using experimental K-band radar data of four test subjects. Five different gait classes are considered for each person, including normal, pathological and assisted walks. 
\textit{Results:} The proposed approaches are shown to outperform existing methods for radar-based gait recognition which utilize physical features from the cadence-velocity data representation domain as in this paper. The analyzed gait classes are correctly identified with an average accuracy of 93\%, where single gait classes reach 99\% correct detection rates.
\textit{Conclusion:} Radar micro-Doppler signatures and their Fourier transforms are well suited to capture changes in gait. Five different walking styles are recognized with high accuracy. 
\textit{Significance:} Radar-based sensing of human gait is an emerging technology with multi-faceted applications in security and health care industries. We show that radar, as a contact-less sensing technology, can supplement existing gait diagnostic tools with respect to long-term monitoring and reproducibility of the examinations.
\end{abstract}

% Note that keywords are not normally used for peerreview papers.
\begin{IEEEkeywords}
assisted living, biomedical monitoring, Doppler radar, gait recognition, radar signal processing
\end{IEEEkeywords}

\IEEEpeerreviewmaketitle

%-----------------------------------------------------------------------------------
%	Paper CONTENTS
%-----------------------------------------------------------------------------------

\section{Introduction}

% why radar
\IEEEPARstart{R}{ecently}, radar has received much attention in civilian applications, most notably, in automotive and health care industries. Specifically, the applications of radar technology in home security, elderly care, and medical diagnosis have emerged to be front and center in indoor human monitoring \cite{Che14,Ami16,Ami17}. These include fall motion detection, classifications of daily human activities, and vital sign monitoring. The considerable rise in radar indoor applications and smart homes is credited to its safety, reliability, and ability to serve as an effective device for contact-less motion monitoring of subjects in the surrounding settings and environments, while preserving privacy. Other non-wearable sensing modalities for indoor human monitoring include infrared reflective light, refractive light, video cameras, and in-ground force platforms \cite{Sim04,Mur14}. However, visual perception or video recordings of human motions can easily be disturbed by occlusions, lighting conditions and clothing. 

% prior work / other approaches
Low-cost Doppler radars have widely been used for detection \cite{Wan14,Su15,Kim15,Cle15,Jok18}, classification \cite{Kim09,Bjoe15,Jok17}, and recognition of human motions \cite{Ric15}. However, most of these works are concerned with the discrimination between different classes of motions, and as such, they consider inter motion category classification problems. A nominal example is discriminating between running, walking, crawling, creeping, sitting, bending and falling. Yet, little thought has been given to study the intra motion category classification problem, i.e., discerning variations within one motion class.

% radar-based gait recognition
In this paper, we focus on classifying human gait within its class. Human gait analysis plays a key role in medical diagnosis, biomedical engineering, sports medicine, physiotherapy and rehabilitation \cite{Mur14}. Constant monitoring of changes in gait aids in assessing recovery from body injury. Further, it enables early diagnosis of different diseases, including multiple sclerosis, Parkinson's and cardiopathies, and facilitates studying the course of disease for designing adequate treatment \cite{Mur14}. For these reasons, it is important to detect gait abnormalities and monitor alterations in walking patterns over time.
However, detailed gait analysis and proper assessments of walking aids can prove difficult for physicians, health care providers and nursing staff. Thorough clinical gait studies are often time-consuming, costly and lack reproducibility \cite{Sim04}. That is why we seek a contact-less sensing technology to empower, and not necessarily replace, naked eye gait examination, with the goal of achieving an expedited, more accurate and more efficient gait diagnosis. 

% assistive walking decives
Gait abnormalities also include using assisted walking devices. It is noted that a great number of seniors resort to assistive walking devices, such as a cane or a walker, in order to compensate for decrements in balance, gain mobility and overcome the fear of falling. In 2011, 8.5 million U.S.~seniors aged 65 and older reported having assistive walking devices, with a cane being most commonly used by two thirds of the elderly \cite{Gel15}. In this regard, a correct use of mobility devices becomes essential to guarantee optimal support and avoid postural deformities injuries or physical impairments with the purpose of re-establishing a normal gait. 

% this work
Using electromagnetic sensing modality, we consider classifying different walking styles, and thus demonstrate the effectiveness of radar in detecting subclasses of gait abnormalities. We show that radar can present a viable, convenient and contact-less supplement or alternative to using wearable sensors (for an overview see e.g.~\cite{Tao12}), which have widely been use to study human gait (for recent works see e.g.~\cite{Pha17,Ren17,Bro15}). As opposed to prior radar-based human gait classification methods, which consider walks with and without arm swinging \cite{Mob09,Tiv10,Tiv15}, or different speeds of walking \cite{Cle15,Ric15}, we focus on detecting differences in the lower limbs kinematics.

For this purpose, we devise a new approach based on predefined features for classifying gaits, where normal, pathological and assisted walks are considered. Here, two types of limping gait are analyzed, where one or both legs are not swinging normally. Further, we consider two different synchronization styles between the cane and the legs, as detection of walking aids has gained increased interest in the latest past \cite{Gur17,Sey18}. 

In addition to physical-based feature extractions, the paper considers automatic learning via subspace analysis. Applied in the cadence domain, we show that features based on principal component analysis (PCA) lead to desirable results that outperform those involving sum-of-harmonics modeling. Considering five different gait classes, a normal walking is correctly detected in 93\% of the cases. Although viewed in the same category as neural networks in unsupervised feature learning, PCA does not demand the same level of computations as deep learning approaches \cite{Jok16,Kim16,Sey18}, neither does it necessitate a very large number of (training) samples.  

In order to proof the applicability of the proposed method for detection of gait asymmetries, we collected radar data of four individuals with different diagnosed gait disorders. The corresponding radar data representations reveal characteristic features that indicate gait disorders. Applying the proposed classification method, the gait asymmetry is correctly detected with an accuracy of 100\% for three of the four test subjects.

% paper organization
The remainder of the paper is organized as follows. For analyzing backscattered radar data from human motions, Section~\ref{sec:signalrepresentations} briefly motivates and outlines different radar data representations, which can be utilized for feature extraction. In Section~\ref{sec:featextraction}, we propose feature extraction techniques for gait classification. Corresponding results are presented and discussed in Section~\ref{sec:results}. Conclusions are given in Section~\ref{sec:conclusion}. % Introduction
\section{Representations of Human Radar Signatures}\label{sec:signalrepresentations}

% Time-domain
The human walk is periodic by nature, i.e., after taking two steps, the course of motions is repeated. One period of the human walk is referred to as gait cycle and a single stride makes up for half a gait cycle \cite{Che11}. One would expect the time-domain radar return signal from a walking person to be periodic with each gait cycle. However, the periodicity information cannot be directly accessed in time-domain, because it is 'hidden' in the observed frequency of the signal. The radar return signal contains multiple time-varying Doppler shifted versions of the transmitted signal. These Doppler components are periodic with each gait cycle, i.e., we observe a periodically frequency-modulated signal.

% Spectrogram (Section ii.)
A classical tool to reveal periodicities in a signal is the Fourier transform (FT). However, the FT does not depict the local frequency behavior, and as such, is not the proper analysis tool for studying the instantaneous frequency and time-dependent Doppler and micro-Doppler signal components. The individual components and their power distribution over time and frequency become visible when utilizing joint time-frequency representations (TFRs). For a walking person, the TFR of the radar return signal depicts the main Doppler shift due to the torso's motion along with the micro-Doppler components due to swinging arms and legs. The spectrogram, which is the energetic representation of the signal's short-time Fourier transform (STFT), is the most common TFR used for analyzing radar micro-Doppler signatures.

%Cadence Velocity Domain (Section iii.)
Since the spectrogram is a windowed FT, a periodic signal will remain periodic in its TFR, with each frequency component exhibiting the same periodicity. Therefore, the periodic structure of the cyclic motion articulations of human gait persist in the time-frequency domain, with a sparser and higher power concentration compared to the time-domain description of the signals. With this property, by taking the FT along the time variable for each frequency bin in the spectrogram, we can assess how often certain Doppler shifts appear over time. The result is known as the cadence-velocity diagram (CVD) \cite{Cle15,Bjoe15,Ric15,Ote05}, where velocity is proportional to the observed Doppler shifts. For a normal human walk, the fundamental frequency in the CVD represents the stride rate or cadence.

\subsection{Radar Signal Model}
Considering a mono-static radar system that transmits a sinusoidal signal \cite{Che11}
\begin{equation} %transmitted signal
s_t(t) = \cos{\left(2 \pi f_c t\right)}
\end{equation}
with carrier frequency $f_c$, the received radar return signal is a superposition of multiple radar scatter components, i.e.,
\begin{equation}
s_r(t) = \sum_i \rho_i \cos{\left(2 \pi \left(f_c + f^\mathrm{D}_i \right) t\right)}.
\label{Eq:recSig}
\end{equation}
Here, $\rho_i$ denotes the path loss of the $i$th scatter component and $f^\mathrm{D}_i $ is the corresponding observed Doppler shift given by 
\begin{equation}
f^\mathrm{D}_i (t) \approx - f_c \frac{2 v_i(t)}{c} \cos{\theta_i}, \quad \text{for~} v_i \ll c~\forall~i,
\label{eq:Doppler}
\end{equation}
where $v_i$ is the velocity of the $i$th target component, $\theta_i$ is the corresponding angle of motion relative to the radar line of sight (LOS) and $c$ is the propagation speed of the electromagnetic (EM) wave. Note that while the radar receiver remains static, the Doppler shifts are generally time-varying as the target changes its velocity over time. After a quadrature detector at the receiver we obtain
\begin{equation}
s(t) = \sum_i \frac{\rho_i}{2} \exp{\left(-j 2 \pi f^\mathrm{D}_i  t\right)},
\end{equation}
which is the baseband representation of the multi-component radar return signal in (\ref{Eq:recSig}) with each component having a distinct Doppler frequency. For further processing, $s(t)$ is sampled at an interval of $\Delta t = 1/f_s$, such that $s(n) = s(t)|_{t=n\Delta t}$ for $n = 0, \dots ,N-1$, where $f_s$ is the sampling frequency and $N \in \mathbb{N}$ is the total number of time samples in the discretized radar signal. For further processing, the mean of the radar return signal is removed such that
\begin{equation}\label{eg:remmean}
\tilde{s}(n) = s(n) - \frac{1}{N} \sum_{n=0}^{N-1} s(n).
\end{equation}

%FT does not reveal the periodic structure because the basis functions are not localized

\subsection{Time-Frequency Representations}
As the radar return signal of a walking person is highly non-stationary, its characteristics are best revealed in a joint-variable representation, such as the time-frequency domain. For human motion analysis, the spectrogram is employed to show the signal's prower distribution over time and frequency. For a discrete-time signal $\tilde{s}(n)$, the spectrogram is given by the squared magnitude of the STFT \cite{Opp99}
\begin{equation}
\mathrm{S}(n,k) = \left| \sum_{m=0}^{M-1} w(m) \tilde{s}(n+m) \exp{\left(-j 2 \pi \frac{mk}{K}\right)}\right|^2, 
\label{eq:spectrogram}
\end{equation}
for $n = 0, \dots, N-1$, where $M$ is the length of the smoothing window $w(\cdot)$, $k$ is the discrete frequency index with $k = 0, \dots, K-1$, and $M, K \in \mathbb{N}$. 
Spectrograms for different walking styles and directions relative to the sensor are shown in Fig.~\ref{fig:specs}. 
We pointed out that normal stride signatures when moving toward the radar system are different compared to those when the radar has a back view on the target (compare Figs.~\ref{fig:specs}\subref{NW} and~\ref{fig:specs}\subref{NWa}) \cite{Sei17}. The salient micro-Doppler feature of a normal stride away from the radar is a spike, i.e., an impulsive-like behavior in the TFR. This characteristic has been overlooked in other works and was not reported in any experimental as well as simulated micro-Doppler signatures. The latter includes, for example, the approach of using a Microsoft Kinect sensor to estimate the human posture via 20 points on the skeleton \cite{Ero15}. Besides the restriction that the entire body needs to be in the field of view of the sensor, the number of discrete sensing points along the human body is not sufficient to capture fine details in human locomotion, such that strides toward the sensor look similar to those away from it. 
Using radar, we can clearly identify deviations from a normal stride, e.g., when one of the knees is not fully bent (see Figs.~\ref{fig:specs}\subref{L1} and~\ref{fig:specs}\subref{L1a}) or a cane is used (see Figs.~\ref{fig:specs}\subref{CW} and~\ref{fig:specs}\subref{CWa}). 

From the spectrogram, two other signals can be found: the envelope signal of the micro-Doppler signatures and the short-time energy signal of the radar return. Both signals are extracted from the noise-reduced spectrogram, where an adaptive thresholding technique is used to suppress the background noise in the TFR \cite{Kim09}. To extract the envelope signal an energy-based thresholding algorithm is applied \cite{Ero16}. It captures the varying maximal Doppler shifts throughout a gait cycle, irrespective of whether it corresponds to a leg or cane motion, and thus converts the time-frequency signal into a train of pulses along the time variable. In contrast, the short-time energy signal accounts for the inherent pattern of individual micro-Doppler signatures by summing over $K'$ Doppler bins in the spectrogram as
\begin{equation}\label{eq:meanEnergy}
E(n) = \frac{1}{K'} \sum_{k=0}^{K'-1} \tilde{\mathrm{S}}(n,k), \quad n = 0, \dots, N-1,
\end{equation}
where $\tilde{\mathrm{S}}$ is the noise-reduced spectrogram, and $K' < K$ is the number of relevant frequency bins corresponding to micro-Doppler shifts excluding the torso's signature.

% Example of spectrograms
\begin{figure}[!t]
	\vspace{-1em}
	\centering{
		\subfloat[Normal walk]{\includegraphics[clip, trim= 0 0 20 18, width=0.5\columnwidth]{figures/S_NW.png}%
			\label{NW}}
		%\hfill
		\subfloat[Normal walk]{\includegraphics[clip, trim= 0 0 20 18,width=0.5\columnwidth]{figures/S_NWa.png}%
			\label{NWa}}\vspace{-0.7em}}

	\centering{
		\subfloat[Limping with one leg]{\includegraphics[clip, trim= 0 0 20 18,width=0.5\columnwidth]{figures/S_ANW.png}%
			\label{L1}}
		%\hfill
		\subfloat[Limping with one leg]{\includegraphics[clip, trim= 0 0 20 18,width=0.5\columnwidth]{figures/S_ANWa.png}%
			\label{L1a}}\vspace{-0.7em}}

	\centering{
		\subfloat[Walking with a cane]{\includegraphics[clip, trim= 0 0 20 18,width=0.5\columnwidth]{figures/S_1CW.png}%
			\label{CW}}
		%\hfill
		\subfloat[Walking with a cane]{\includegraphics[clip, trim= 0 0 20 18,width=0.5\columnwidth]{figures/S_1CWa.png}%
			\label{CWa}}\vspace{-0.7em}}

	\centering{
		\subfloat[Walking with a cane out of sync]{\includegraphics[clip, trim= 0 0 20 18,width=0.5\columnwidth]{figures/S_1CWoos.png}%
			\label{CWoos}}
		%\hfill
		\subfloat[Walking with a cane out of sync]{\includegraphics[clip, trim= 0 0 20 18,width=0.5\columnwidth]{figures/S_1CWoosa.png}%
			\label{CWoosa}}}
%	\vspace{1em}
	\caption{Examples of spectrograms for walking toward (left) and away from (right) the radar system. The color indicates the energy level in dB.}
	\label{fig:specs}\vspace{-0.5em}
\end{figure}


\subsection{Cadence-Velocity Diagram}
In order to analyze the cyclic nature and inherent periodicities of a human walk, we generate another joint-variable representation that depicts the periodic pattern of certain micro-Doppler components over time. 
%A natural tool to reveal the period of each of these components is the FT. 
This representation is the CVD which is obtained by taking the FT of the spectrogram along each Doppler frequency bin as \cite{Cle15}
\begin{equation}%\mathcal{F}_n \left\{ S(n,k) \right\}
\mathrm{C}(\epsilon,k)  = \left| \sum_{n=0}^{N-1} \tilde{\mathrm{S}}(n,k\textbf{}) \exp{\left(-j 2 \pi \frac{n \epsilon}{L}\right)} \right|, 
\label{eq:CVD}
\end{equation}
where $\epsilon = 0, \cdots, L-1$ is the cadence frequency, $\tilde{\mathrm{S}}$ is the noise-reduced spectrogram, and $L \in \mathbb{N}$. Here, the velocity is directly proportional to the observed Doppler frequency as given by (\ref{eq:Doppler}). Prior to FT calculation, the mean is removed from each Doppler frequency bin to reduce the influence of stationary Doppler components. Note that, in contrast to TFRs, the CVD does not depend on the initial phase of the gait cycle, i.e., it is a time-invariant analysis method. Fig.~\ref{fig:cvds} shows the corresponding CVDs of the measurements depicted in Fig.~\ref{fig:specs}, along with the mean cadence spectra.

% Example of CVDs
\begin{figure}[!t]
	\vspace{-1em}
	\centering{
		\subfloat[Normal walk]{\includegraphics[clip, trim= 0 0 20 8,width=0.5\columnwidth]{figures/CVD_NW.png}%
			\label{CVDNW}}
		%\hfill
		\subfloat[Normal walk]{\includegraphics[clip, trim= 0 0 20 8,width=0.5\columnwidth]{figures/CVD_NWa.png}%
			\label{CVDNWa}}\vspace{-0.7em}}

	\centering{
		\subfloat[Limping with one leg]{\includegraphics[clip, trim= 0 0 20 8,width=0.5\columnwidth]{figures/CVD_ANW.png}%
			\label{CVDL1}}
		%\hfill
		\subfloat[Limping with one leg]{\includegraphics[clip, trim= 0 0 20 8,width=0.5\columnwidth]{figures/CVD_ANWa.png}%
			\label{CVDL1a}}\vspace{-0.7em}}

	\centering{
		\subfloat[Walking with a cane]{\includegraphics[clip, trim= 0 0 20 8,width=0.5\columnwidth]{figures/CVD_1CW.png}%
			\label{CVDCW}}
		%\hfill
		\subfloat[Walking with a cane]{\includegraphics[clip, trim= 0 0 20 8,width=0.5\columnwidth]{figures/CVD_1CWa.png}%
			\label{CVDCWa}}\vspace{-0.7em}}

	\centering{
		\subfloat[Walking with a cane out of sync]{\includegraphics[clip, trim= 0 0 20 8,width=0.5\columnwidth]{figures/CVD_1CWoos.png}%
			\label{CVDCWoos}}
		%\hfill
		\subfloat[Walking with a cane out of sync]{\includegraphics[clip, trim= 0 0 20 8,width=0.5\columnwidth]{figures/CVD_1CWoosa.png}%
			\label{CVDCWoosa}}}

	\caption{Examples of CVDs and mean cadence spectra for walking toward (left) and away from (right) the radar system. The corresponding spectrograms are given in Fig.~\ref{fig:specs}.}
	\label{fig:cvds}\vspace{-0.5em}
\end{figure}

The mean cadence spectrum (mCS) depicts how often certain Doppler components appear throughout a gait, independent of the components' velocities. It is obtained by summing over all Doppler frequencies in the CVD, i.e., \cite{Bjoe15,Ric15}
\begin{equation}\label{eq:meanCS}
\bar{\zeta}(\epsilon) = \frac{1}{K} \sum_{k=0}^{K-1} \mathrm{C}(\epsilon,k), \quad \epsilon = 0, \dots, L-1.
\end{equation}
The highest peak of the mCS typically represents the stride rate. For example, Fig.~\ref{fig:cvds}\subref{CVDNW} reveals a stride rate of 0.9\,Hz, which is consistent with almost six strides in the 6\,s data measurement (see Fig.~\ref{fig:specs}\subref{NW}). The step rate is an important characteristic of a walk and belongs to the group of physical features, which are easily interpretable. A detailed analysis of physical features for radar-based gait recognition will be given in Section~\ref{subsec:phyfeatures}. Note that, in medical terminology of human gait analysis, the cadence is defined as the number of steps per unit time, and thus serves as a measure of the step rate \cite{Lev12}. Here, however, we draw a distinction between the cadence frequency, as a measure of the periodicity of micro-Doppler signatures, and the stride rate, which describes the number of strides per second. Therefore, the repetition frequency of micro-Doppler signatures, hereafter, is referred to as $\fmd$, which does not necessarily relate to the stride rate in assisted or pathological gait. 

Similarly, we can find the mean Doppler spectrum by summing over all cadence frequencies for each Doppler frequency bin in the CVD, i.e.,
\begin{equation}\label{eq:meanDS}
\bar{\Gamma}(k) = \frac{1}{L} \sum_{\epsilon=0}^{L-1} \mathrm{C}(\epsilon,k), \quad k = 0, \dots, K-1.
\end{equation}
From the mean Doppler spectrum, features such as the average walking speed of a person or the minimal and maximal Doppler shifts, can be extracted \cite{Ric15,Bjoe12}.
 % Representations of Human Radar Signatures
\section{Feature Extraction and Experimental Methods}\label{sec:featextraction}

\subsection{Experimental Data and Setup}\label{sec:exp_setup}
The experimental radar data were recorded in an office environment (no absorbers) at Technische Universit\"{a}t Darmstadt, Germany. An ultra wide band radar \cite{Anc} was used in continues wave mode with a carrier frequency of 24\,GHz. The antenna feed point of the radar was placed at table height, 1.15\,m above the floor, which represents a nominal hip height. Ten volunteers (age: 23.8 $\pm$ 2.6, 8 men, 2 women) were asked to walk slowly and without arm swinging toward and away from the radar system in an $0\degree$ angle relative to the radar LOS, between approximately 4.5\,m and 1\,m from the antenna feed point. All participants provided informed consent. Note that this setup would be a practice when radar is used as a diagnostic tool for analysis of gait abnormalities, where weak micro-Doppler signatures due to large aspect angles can easily be avoided. In total, the data set contains 1000 measurements, i.e., 100 measurements per person. Five different walking styles are considered: normal walking (NW), limping with one (L1) or both legs (L2), walking with a cane in sync with one leg (CW) and out of sync with any leg (CW/oos). Here, a limping leg is simulated by a knee that cannot be bent such that the stride motion is performed in a semicircular manner. In the case of limping with both legs neither of the knees can be bent. The number of samples per class and walking direction are equal among the test subjects. Using the same experimental setup, radar data of four additional subjects (4 women) with diagnosed gait disorders were collected at Villanova University, USA. This test data set contains 13, 20, 28 (16 thereof with a cane in sync with one leg), and 28 measurements for person A, B, C, and D, respectively, i.e., 89 measurements in total.

%%%%%%%%%%%%%%%%%%%%%%%%%%%%%%%%%%%%%%%%%%%%%%%%%%%%%%%%%%%%%%%%%%%%%%%%%%%%%%%%%%%%%
% 					Phy features	    											%
%%%%%%%%%%%%%%%%%%%%%%%%%%%%%%%%%%%%%%%%%%%%%%%%%%%%%%%%%%%%%%%%%%%%%%%%%%%%%%%%%%%%%
\subsection{Physical Features based on Sum-of-Harmonics Analysis}\label{subsec:phyfeatures}
% Spectrogram and CVD
Gait characteristics manifest themselves differently depending on the data representation and the transforms adopted. For feature extractions, we consider both spectrogram and CVD. Whereas the former depicts the Doppler and micro-Doppler signatures which correspond to velocities and their time-varying natures, the latter accentuates periodicities and better describes the harmonic components of the limbs. 

% Physical features
Concerning the feature extraction mechanism, we note that physical gait features are inter-related, and depend on \cite{Bjoe15,Gur15}
\begin{itemize}
	\item data pre-processing, e.g., noise reduction in the spectrogram,% \cite{Gur15},
	\item type of radar used,% \cite{Bjoe15}, 
	\item environment,% \cite{Bjoe15}, 
	\item target characteristics.% \cite{Bjoe15}.
\end{itemize}
Since the features play an important role in classification problems, they should be chosen to be (i) relevant to the considered classification problem and (ii) accurately estimated or extracted from the micro-Doppler signature or its transforms \cite{Gur15}. Features of physical interpretations have been widely used for radar-based human activity recognition and include, but are not limited to \cite{Kim09,Ote05}: 
\begin{itemize}
	\item torso Doppler frequency,% \cite{Kim09},
	\item total Doppler bandwidth,% \cite{Kim09},
	\item offset of the total Doppler,% \cite{Kim09},
	\item Doppler bandwidth without micro-Doppler effects,% \cite{Kim09},
	\item normalized standard deviation of Doppler signal strength,% \cite{Kim09},
	\item period of the limb motion or stride rate,% \cite{Kim09},
	\item average radial velocity,% \cite{Ote05},
	\item stride length,% \cite{Ote05},
	\item radar cross-section of some moving body components (gait amplitude ratio).% \cite{Ote05}.
\end{itemize}
However, most of the above features are not descriptive for distinguishing different walking styles. For example, the target's average radial velocity is expected to be similar for the different classes of gait considered. Hence, the offerings of these features in human gait recognition need to be examined.

% Base velocity
In order to estimate the average radial velocity of a person, referred to as base velocity $v_0$, the mean Doppler spectrum is calculated as given by (\ref{eq:meanDS}). This mean value can be smoothed to minimize the influence of noise by applying a moving average filter with a span corresponding to approximately 11\,Hz in Doppler frequency or 0.07\,m/s. Next, the maximum value of the mean Doppler spectrum is determined, and used as an estimate of the average walking speed of the person by using (\ref{eq:Doppler}).

% micro-Doppler repetition frequency fmD
An important characteristic of a person's walk is the gait periodicity, which corresponds to the stride rate for a normal human walk. However, in the case of cane-assisted walks, the strides may not be periodic due to the additional cane movements (see Figs.~\ref{fig:specs}\subref{CWoos} and \ref{fig:specs}\subref{CWoosa}). Hence, we introduce the micro-Doppler repetition frequency $\fmd$, which captures the periodicity of the micro-Doppler signatures irrespective of being due to leg or cane movements. For extracting $\fmd$, the spectrogram, given by (\ref{eq:spectrogram}), is used and the upper and lower envelopes of the micro-Doppler signatures are extracted for toward and away from radar motions, respectively, by applying an energy-based thresholding technique \cite{Ero16}. Since a swinging foot or a cane motion assumes the highest Doppler shifts in human gait motions, they lead to maxima in the absolute value of the envelope signal. 
%As such, the envelope signal becomes periodic. 
Taking the FT of the envelope signal, we find $\fmd$ by determining the frequency with the maximal amplitude.

% fmD vs. v_0 and fmD vs. maxD
\begin{figure}[!t]
	\begin{minipage}{\columnwidth} %clip, trim= 0 0 30 18
		\includegraphics[width=0.9\textwidth]{figures/v0_vs_fDmax}
		\caption*{(a)\label{fig:f0_vs_v0}}
	\end{minipage}
	\begin{minipage}{\columnwidth}
		\includegraphics[width=0.9\textwidth]{figures/fmD_vs_fDmax}
		\caption*{(b)\label{fig:f0_vs_fdmax}}
	\end{minipage}
	\caption{Scatter plots of physical features: (a) base velocity $v_0$ vs.~maximal Doppler shift $\fdmax$ and (b) micro-Doppler repetition frequency $\fmd$ vs.~$\fdmax$. \label{fig:scatterPhyFeat}\vspace{-0.5em}}
\end{figure}

% maximal Doppler shift
Next, we consider the maximal observed Doppler shift in the measurement as a feature. This is motivated by the observation that a limping leg has a lower radial velocity, and thus causes smaller Doppler shifts. The envelope signals can also be utilized for this purpose. The maximal Doppler shift $\fdmax$ is estimated by use of the maximal values of the envelope signal. Here, the mean of the highest 10\% of the maximal Doppler shifts is used to be less sensitive to variations between different micro-Doppler signatures. 

In order to show relevance, Fig.~\ref{fig:scatterPhyFeat} shows scatter plots of the three described features, namely the base velocity $v_0$, the micro-Doppler repetition frequency $\fmd$ and the maximal Doppler shift $\fdmax$, for the five considered gait classes. From (a), it can be seen that the base velocity is not discriminative of the walks, because the walking speed of a person is not (necessarily) influenced by the use of an assistive walking device or gait impairments. However, the scatter plot in (b) reveals that the micro-Doppler repetition frequency increases when walking with a cane. In particular, we remark that if the cane is moved out of sync, $\fmd$ becomes notably higher compared to the other four classes. Further we note that limping with both legs has clearly the lowest maximal Doppler shift among the considered classes.

% coefficient of variation
However, except for limping with both legs, the maximal Doppler shift does not help in discriminating between the remaining gait classes, at least not in the person-generic case considered here. In this respect, we note that some walking styles exhibit different maximum Doppler shifts per leg or cane motion. In particular, from Figs.~\ref{fig:specs}\subref{L1} and \ref{fig:specs}\subref{L1a}, we observe that limping with one leg leads to a characteristic pattern of alternating high and low maximal Doppler shifts. For capturing this oscillatory behavior, we find the peaks of the envelope signal and approximate the envelope's envelope using spline interpolation. In order to quantify the variation in maximal Doppler shifts, we proceed to calculate the coefficient of variation as
\begin{equation}
c_v = \frac{\sigma}{\mu},
\end{equation}
where $\sigma$ is the standard deviation and $\mu$ is the mean of the interpolated signal. The coefficient of variation is expected to be particularly high for limping with one leg and thus serves as an indicator of abnormality.

% Gait harmonic ratio 
Further, we observe that the gait classes of normal walking (NW), limping with one leg (L1) and walking with a cane (CW) are not well separated in the feature space spanned by $\fmd$ and $\fdmax$ depicted in Fig.~\ref{fig:scatterPhyFeat}(b). That is, the micro-Doppler repetition frequency $\fmd$, by itself, does not capture the underlying regularity or irregularity of the walk. For this, we calculate the gait harmonic frequency ratio as \cite{Alz14}
\begin{equation}
\beta = \frac{f_0}{\fmd},
\label{eq:beta}
\end{equation}
where $f_0$ is the fundamental frequency (FF) of the gait. For the considered gait classes, we expect the values of $\beta$ to be:
\begin{itemize}
	\item $1$ for NW and L2 as each micro-Doppler stride signature assumes the same pattern,
	\item $1/2$ for L1 and CW as every other micro-Doppler signature appears the same,
	\item $1/3$ for CW/oos as a set of two strides and one cane movement constitutes one period.
\end{itemize}
In order to find $f_0$, we use the short-time energy signal defined in (\ref{eq:meanEnergy}) and model it as a sum-of-harmonics (SOH) \cite{Sei18}, i.e.,
\begin{equation}\label{eq:soh}
\begin{aligned}%T_\text{s} 
x(n) &= \sum_{i=1}^{q} u_i \cos(2 \pi i f_0 n ) + v_i \sin(2 \pi i f_0 n ) \\
     &= \sum_{i=1}^{q} \alpha_i \cos(2 \pi i f_0 n + \phi_i),
\end{aligned} 
\end{equation}
where $f_0$ is the fundamental frequency in Hz, $q$ is the number of harmonics (NOH), and the harmonic amplitudes and phases are $\alpha_i$ and $\phi_i$, respectively. 
Here, we use the algorithm proposed in \cite{Whi03} to estimate the FF, the NOH, and harmonic amplitudes and phases. Assuming the energy signal $E(n)$ is composed of a SOH signal $x(n)$ and an additive white Gaussian noise component $u(n)$, i.e.,
\begin{equation}\label{eq:model}
E(n) = x(n) + u(n), \quad n = 0, \dots,N-1, 
\end{equation}
the parameters are then found by minimizing the squared-error between the data and the model, i.e.,
\begin{equation}\label{eq:LS}
\xi = \sum_{n = 0}^{N-1} \left| x(n) - E(n) \right|^2,
\end{equation}
and utilizing the nonlinear least squares method for estimating $f_0$, which is augmented by a model order selection method for detecting $q$. For this, (\ref{eq:LS}) is jointly optimized over candidate FFs and candidate orders. We use $\fmd$ as an initial estimate of the FF. In a first step of the SOH algorithm, this estimate is refined by minimizing (\ref{eq:LS}) using an optimization technique. Next, candidate FFs are determined from the refined $f_0$ estimate for which the cost function defined by the NLS method is evaluated. At this point, we incorporate prior knowledge to limit computational costs in the joint-optimization for finding $f_0$ and $q$, and avoid overfitting. As described earlier, we expect $f_0$ to be $1/3 \cdot \fmd$, $1/2 \cdot \fmd$ or $1 \cdot \fmd$ given the initial FF estimate $\fmd$ is correct. Thus, the candidate FFs assume only the aforementioned values.  
Given the estimates for the FF and the NOH, the SOH model in (\ref{eq:soh}) is linear in the parameters $u_i$ and $v_i$. Thus, using the linear least-squares solution, the harmonic amplitudes $\alpha_i$ and phases $\phi_i$, $i=1,\dots,q$, can be computed in a closed-form as a function of $f_0$ and $q$. The estimated parameter vector is thus given by
\begin{equation}
\mathbf{h} = \left[ f_0~q~\alpha_1 \cdots \alpha_q ~\phi_1 \cdots \phi_q \right].
\label{eq:SOHparas}
\end{equation}

Given $f_0$, we proceed to calculate the gait harmonic frequency ratio $\beta$ using (\ref{eq:beta}). Table~\ref{tab:results_beta} shows the classification results using solely the $\beta$ feature for all considered walking styles. Clearly, $\beta$ is proving to be an important descriptive feature to characterize the analyzed walking patterns, as 68\%, 75\% and 91\% of the respective measurements show the expected gait harmonic frequency ratios $1/3$, $1/2$ and $1$, respectively. 

Based on the above results and the contributions of the various parameters, we form a physical feature vector as
\begin{equation}
\mathbf{z}^\text{phy} = \left[\fmd~\fdmax~c_v~\beta~\alpha_1 \cdots \alpha_{q_\text{max}} \right],
\label{eq:phyfeat}
\end{equation}
where again $\fmd$ is the micro-Doppler repetition frequency, $\fdmax$ is the maximal observed Doppler shift in the measurement, $c_v$ is the coefficient of variation of maximal micro-Doppler shifts, and $\beta$ is the gait harmonic frequency ratio. The harmonic amplitudes $\alpha_i$ relate to the height of the peaks in the mCS and help to discriminate different articulations of abnormality. Here, $q_\text{max}=5$ is the maximal order of the SOH model and $\alpha_i = 0~\forall~i > q$. Note that we do not include the base velocity $v_0$, as it was found not to be an appropriate discriminative feature for the motions considered.

% BETA Table
\begin{table}[!t]
	\renewcommand{\arraystretch}{1.2}\setlength{\tabcolsep}{0.3em}
	\caption{Confusion matrices for classifying three different gait patterns using the gait harmonic feature $\beta$. Numbers are given in \%.}
	\label{tab:results_beta}
	\centering
	\subfloat[both]{
		\begin{tabular}{ l | c | c | c }
			\hline
			\textbf{True / Predicted } & NW, L2 & CW, L1 & CW/oos \\
			\hline \hline
			Normal walk (NW), Limping both (L2) 			& \cellcolor{green!20} 68 & \cellcolor{orange!20} 28 & 4 \\
			\hline
			Cane - synchronized (CW), Limping one (L1) 	& \cellcolor{orange!20} 21 & \cellcolor{green!20} 75 & 4 \\
			\hline
			Cane - out of sync (CW/oos) 				& 5 & 4 & \cellcolor{green!20} 91 \\
			\hline
		\end{tabular} \label{tab:beta_both}}\\
	\subfloat[toward]{
		\begin{tabular}{ l | c | c | c }
			\hline
			\textbf{True / Predicted } & NW, L2 & CW, L1 & CW/oos \\
			\hline \hline
			Normal walk (NW), Limping both (L2) 			& \cellcolor{green!20} 68 & \cellcolor{orange!20} 28 & 4 \\
			\hline
			Cane - synchronized (CW), Limping one (L1) 	& \cellcolor{yellow!20} 11 & \cellcolor{green!20} 83 & 6\\
			\hline
			Cane - out of sync (CW/oos) 				&  4 & 5 & \cellcolor{green!20} 91 \\
			\hline
		\end{tabular} \label{tab:beta_toward}}\\
	\subfloat[away]{
		\begin{tabular}{ l | c | c | c }
			\hline
			\textbf{True / Predicted } & NW, L2 & CW, L1 & CW/oos \\
			\hline \hline
			Normal walk (NW), Limping both (L2) 			& \cellcolor{green!20} 68 & \cellcolor{orange!20} 29 & 3\\
			\hline
			Cane - synchronized (CW), Limping one (L1) 	& \cellcolor{orange!20} 31 & \cellcolor{green!20} 67 & 2\\
			\hline
			Cane - out of sync (CW/oos) 				& 5 & 5 & \cellcolor{green!20} 90\\
			\hline
		\end{tabular} \label{tab:beta_away}}
	\vspace{-1.7em}
\end{table}

%%%%%%%%%%%%%%%%%%%%%%%%%%%%%%%%%%%%%%%%%%%%%%%%%%%%%%%%%%%%%%%%%%%%%%%%%%%%%%%%%%%%%
% 					other approaches     											%
%%%%%%%%%%%%%%%%%%%%%%%%%%%%%%%%%%%%%%%%%%%%%%%%%%%%%%%%%%%%%%%%%%%%%%%%%%%%%%%%%%%%%
In this work, we desire to compare the classification performance using the above features with those used by recent works in this field, particularly \cite{Bjoe15} and \cite{Ric15}. We limit our comparison to the classification techniques that employ the cadence-velocity domain, as proposed in this paper.
%%%%%%%%%%%%%%%%%%% [Bjoe15] %%%%%%%%%%%%%%%%%%%%%%%%%%%%%%%%%%%%%%%%%%%%%%%%%%%%%%%%
Bj{\"o}rklund \textit{et.~al} \cite{Bjoe15} used a 77\,GHz radar system to discriminate between the motions crawl, creep, walk, jog and run, which were performed by three test subject. For classification they used features from the cadence-velocity domain and a support vector machine (SVM). By taking the average over all velocities in the CVD, they form the mCS, from which the three highest peaks are identified. At the corresponding cadence frequencies, denoted as $f_1$, $f_2$, and $f_3$, the velocity profiles $\Gamma_1$, $\Gamma_2$, and $\Gamma_3$ are extracted from the CVD, i.e., the energy distribution in the CVD for the given cadence frequencies as a function of the Doppler frequency. The velocity profiles are resampled to have 100 samples each.
% and normalized, while their relative magnitudes are maintained. 
Further, the base velocity $v_0$ is extracted by finding the maximum in the mean Doppler spectrum. The corresponding feature vector is given by
\begin{equation}\label{eq:Bjoefull}
\mathbf{z}^\text{B1}  = [f_1~f_2~f_3~\Gamma_1~\Gamma_2~\Gamma_3~v_0],
\end{equation}
where $\Gamma_i$ denotes the resampled velocity profile at cadence frequency $f_i$, $i=1,\dots,3$, and $v_0$ is the base velocity.
Further, they define a reduced feature vector as $\mathbf{z}^\text{B2} = [f_1~f_2~f_3~|v_0|]$, where the velocity profiles are not considered. Note that they drop the sign of the base velocity by taking the absolute value, i.e., they do take the direction of motion into account.

%%%%%%%%%%%%%%%%%%% [Ric15] %%%%%%%%%%%%%%%%%%%%%%%%%%%%%%%%%%%%%%%%%%%%%%%%%%%%%%%%
Ricci and Balleri \cite{Ric15} extracted features from the cadence-velocity domain for target recognition and identification. For that, four different subjects were walking on a treadmill in front of a 10\,GHz radar at constant speed. For discriminating the targets performing the same motion, the following features were extracted. From the mCS, they obtain an estimate of the person's stride rate. In our work, we resort to $\fmd$, which is more reliably estimated from the envelope of the micro-Doppler signatures. Next, a mean Doppler spectrum $\bar{\Gamma}_\text{mD}$ is formed around $\fmd$ by averaging over $\delta = 5$ neighboring cadence frequencies corresponding to 0.825\,Hz cadence bandwidth. The second and third feature, $f^\mathrm{D}_\text{mD,min}$ and $f^\mathrm{D}_\text{mD,max}$, are found by determining the minimum and maximum Doppler frequencies in $\bar{\Gamma}_\text{mD}$ exceeding a predefined threshold $\gamma = 0.05$. 
Thus, the first feature vector is defined as
\begin{equation} \label{eq:Ric1}
\mathbf{z}^\text{R1} = [\fmd~f^\text{D}_{\text{mD,min}}~f^\text{D}_{\text{mD,max}}].
\end{equation}
%where the superscript of the features highlights that the mean Doppler spectrum is formed around $f_\text{mD}$ only.
Second, the mean Doppler spectrum around $f_\text{mD}$ is used to define the feature vector
\begin{equation} \label{eq:Ric2}
\mathbf{z}^\text{R2} = \bar{\Gamma}_{\text{mD}}.
\end{equation}

%%%%%%%%%%%%%%%%%%%%%%%%%%%%%%%%%%%%%%%%%%%%%%%%%%%%%%%%%%%%%%%%%%%%%%%%%%%%%%%%%%%%%
% 					PCA 															%
%%%%%%%%%%%%%%%%%%%%%%%%%%%%%%%%%%%%%%%%%%%%%%%%%%%%%%%%%%%%%%%%%%%%%%%%%%%%%%%%%%%%%
\subsection{Subspace Features}\label{subsec:pcafeatures}
In PCA, the intrinsic features of the considered walking styles are automatically learned and do not necessarily bear one-to-one correspondence to human motion kinematics \cite{Sei17a}. First, the input signals $\mathbf{C}$ are vectorized row-wise, i.e., $\mathbf{c} = \text{vec}\{\mathbf{C}^\mathrm{T}\} \in \mathbb{R}^{p \times 1}$, and stacked column-wise to form a data matrix $\mathbf{Y}$, such that
\begin{equation}
\mathbf{Y} = [\mathbf{c}_1~\mathbf{c}_2~\cdots~\mathbf{c}_d ] \in \mathbb{R}^{p \times d},
\end{equation} 
where $d$ is the number of training samples. The principal components are given by the eigenvectors of the covariance matrix.
There are various methods to compute the principal components \cite{Shl14}. We apply singular value decomposition (SVD) to decompose the data matrix such that $\mathbf{Y} = \mathbf{U} \mathbf{D} \mathbf{V}^\mathrm{T}$, where the columns of $\mathbf{U}$ and $\mathbf{V}$ are the left and right eigenvectors, respectively, and the diagonal entries of the diagonal matrix $\mathbf{D}$ are the singular values. The eigenvalues are related to the singular values by $\mathbf{\Lambda} = 1/(d-1) \mathbf{D}^2$ \cite{Koc13}.
The left eigenvector that has the largest eigenvalue, i.e., explains most of the variance in the data, is the first principal component. Using $\lambda$ principal components, which span a $\lambda$-dimensional subspace of the originally $d$-dimensional data space, each vectorized training and test image, $\mathbf{c}$, is projected onto that subspace by
\begin{equation}
\mathbf{p} = \tilde{\mathbf{U}}^\mathrm{T} \mathbf{c} \in \mathbb{R}^{\lambda \times 1},
\end{equation}
where $\tilde{\mathbf{U}} \in \mathbb{R}^{p \times \lambda}$ are the eigenvectors, or eigenimages, corresponding to the first $\lambda$ eigenvalues. The resulting projections $\mathbf{p}$ form the feature vector used for classification, i.e.,
\begin{equation}
\mathbf{z}^\text{PCA} = [p_1~p_2~\cdots~p_{\lambda}]^\mathrm{T}, \quad \lambda \leq d \in \mathbb{N}.
\label{eq:pcafeature}
\end{equation}

%%%%%%%%%%%%%%%%%%%%%%%%%%%%%%%%%%%%%%%%%%%%%%%%%%%%%%%%%%%%%%%%%%%%%%%%%%%%%%%%%%%%%
% 					Input Signals													%
%%%%%%%%%%%%%%%%%%%%%%%%%%%%%%%%%%%%%%%%%%%%%%%%%%%%%%%%%%%%%%%%%%%%%%%%%%%%%%%%%%%%%
\subsection{Radar Data Representations}\label{ssec:inputsignals}
The different radar data representations and their dimension are listed in Table~\ref{tab:inputs}.

% spectrogram
Using measurements of 6\,s duration, we calculate the spectrogram using (\ref{eq:spectrogram}), where a Hamming window of approximately 0.1\,s length is applied and the STFT is evaluated at $K = 2048$ frequency points. An excerpt of the spectrogram is used with Doppler components smaller than 500\,Hz as depicted in Fig.~\ref{fig:specs}, and its amplitude is normalized to the range of $[0,1]$. To further reduce the dimensionality of the spectrogram, it is sub-sampled in the time-domain by a factor of 20 and image binning is applied, where groups of $4\times4$ pixel are averaged. Thus, the spectrogram has $90 \times 192 = 17280$ entries. 

%CVD
Next, the CVD is calculated according to (\ref{eq:CVD}), where zero-padding is used to obtain a cadence frequency resolution of approximately 0.04\,Hz. Again, the relevant part of the CVD images is extracted. In this regards, cadences up to 5\,Hz and Doppler frequencies from 0\,Hz to +500\,Hz and 0\,Hz to \mbox{-500\,Hz} for toward and away from radar motion measurements are considered, respectively, as shown in Fig.~\ref{fig:cvds}. Further, the resulting CVD image is downsampled in the Doppler domain to yield an image of dimensions $101 \times 129$ pixels, and is normalized to have values in the range of $[0,1]$. From this excerpt of the CVD, the mCS is obtained via (\ref{eq:meanCS}). 

% warped CVD
As we are particularly interested in the characteristic pattern in the CVD image, different stride rates and different maximal Doppler shifts among the measurements are compensated so as to align the CVD images. This is achieved by warping the CVD images along the cadence frequency and Doppler frequency axis using $\fmd$ and $\fdmax$, respectively. Afterwards, all CVDs assume $\fmd = 1$\,Hz and $\fdmax = 500$\,Hz. These CVDs are again considered up to 5\,Hz cadence frequency resulting in images of dimension $101 \times 129$ pixels. Hereafter, these CVDs will be referred to as pre-processed CVDs. As for the raw CVD images, we can find the pre-processed mCS and from the pre-processed CVDs using (\ref{eq:meanCS}).

%time-domain signal
Finally, one can alternatively utilize the time-domain signal itself, where the lower Doppler components due to the torso's motion are removed by high-pass filtering the signal with a cut-off frequency corresponding to $2v_0$. Taking the FT of this high-pass filtered signal, we obtain a similar representation as the mCS with peaks at the fundamental cadence and its harmonics.

\begin{table}[!t]
	\renewcommand{\arraystretch}{1.3}
	\caption{Radar data representations and their dimensions.}
	\label{tab:inputs}
	\centering
	\begin{tabular}{l c c }
		\hline
		Representation & Dimension & $p$ \\
		\hline \hline
		Spectrogram			    					& 90 $\times$ 192 & 17280 \\
		Cadence-velocity diagram (CVD)  			& 101 $\times$ 129 & 13029 \\
		Mean cadence spectrum (mCS)					& 1 $\times$ 129 & 129 \\
		Pre-processed CVD 							& 101 $\times$ 129 & 13029\\
		Pre-processed mCS 							& 1 $\times$ 129 & 129 \\
		FT of filtered time-domain signal 			& 1 $\times$ 129 & 129 \\
		\hline
	\end{tabular}
\end{table}

%%%%%%%%%%%%%%%%%%%%%%%%%%%%%%%%%%%%%%%%%%%%%%%%%%%%%%%%%%%%%%%%%%%%%%%%%%%%%%%%%%%%%
% 				Methodology	+ Parameter Opt   										%
%%%%%%%%%%%%%%%%%%%%%%%%%%%%%%%%%%%%%%%%%%%%%%%%%%%%%%%%%%%%%%%%%%%%%%%%%%%%%%%%%%%%%
\subsection{Methodology and Parameter Optimization}\label{ssec:method}
In order to assess the appropriateness of the proposed features to the intra gait motion classification problem at hand, a very simple classifier is considered, namely the nearest neighbor (NN) classifier. Each classification result is obtained using 80\% of the data for training, where stratified sampling is applied to randomly choose the training samples, and the remaining data is used for testing. Final classification numbers are obtained by averaging over 500 classification results.

Considering the different radar data representations listed in Table~\ref{tab:inputs}, we aim to find the optimal dimension of the PCA-based feature vector, as defined in (\ref{eq:pcafeature}), which depends on the number of principal components $\lambda$ used to span a subspace for data representation. In this work, we choose $\lambda$ such that the average correct classification rate is maximized. Toward this end, Fig.~\ref{fig:parapet} shows the achieved correct classification rates as a function of the number of principal components $\lambda$. We find that the classification accuracy does not significantly increase for $\lambda >$ 20 for any of the considered data representations.

In general, the NN classifier can be easily extended to the $\kappa$-NN classifier, which considers a number of $\kappa$ neighbors in the decision process. Similarly, the parameter $\kappa$ can be optimized such that the classifier achieves the highest average correct classification rate. Fig.~\ref{fig:kappa_vs_lambda} illustrates the joint optimization of the parameters $\lambda$ and $\kappa$ using pre-processed CVDs, where the color indicates the average correct classification rate. Note that we omitted the results for $\lambda <$ 10 for visual clarity, as the corresponding classification rates are significantly lower. We find that the highest average correct classification rate is achieved by using the NN classifier ($\kappa$ = 1) and $\lambda$ = 20 principal components in the PCA-based feature extraction process. Increasing $\kappa$ does not increase the average classification rate. Note that, larger values of $\kappa$ or $\lambda$ increases computation time.

% optimzing lambda
\begin{figure}[!t]
	\centering
	\includegraphics[clip, trim= 0 0 20 18,width=\columnwidth]{figures/classify_fig4_j6}
	\caption{Average correct classification rate over all classes as a function of the number of principal components used for PCA-based feature extraction based on different radar data representations. The shaded areas depict the interval of $\pm1\sigma$. \label{fig:parapet}}
	\vspace{-0.5em}
\end{figure}

% #PC vs #NN
\begin{figure}[!t]
	\centering
	\vspace{-0.5em}
	\includegraphics[clip, trim= 0 0 20 18,width=0.9\columnwidth]{figures/NN_vs_NPC.png}
	\caption{Average correct classification rates for different numbers of neighbors $\kappa$ for the classification process and different numbers of principal components $\lambda$ used for PCA-based feature extraction based on pre-processed CVDs. \label{fig:kappa_vs_lambda}}
	\vspace{-0.5em}
\end{figure}

\subsection{Features from Different Data Domains}
We highlight the importance of the CVD to the classification problem at hand by comparing the classification performance using PCA-based features from different data domains. Here, we use the spectrogram, the pre-processed CVD, the pre-processed mCS as inputs to the PCA. The subspace-based feature vectors are obtained as described in Section~\ref{subsec:pcafeatures}. The number of principal components is set to $\lambda$ = 20. Fig.~\ref{fig:compare_domains} shows the correct classification rates for each gait class using PCA-based features from the three different domains. Clearly, the spectrogram is inferior to the other representations for extracting descriptive subspace-based features. One reason for the poor classification performance is that the spectrogram is a time-dependent representation of the human gait, i.e., the measurements are not aligned in time. The CVD shows the highest classification rates. This indicates that the CVD contains key information relevant to classification but lost when calculating the mCS. Based on the above results, the CVD is used as a reference in the follow-on comparison below.

% CLASSIY: PCA-based - specs vs. cvds vs. mcd (both) 
\begin{figure}[!t]
	\centering 
	\vspace{-0.5em}
	\includegraphics[clip, trim= 0 0 20 18,width=\columnwidth]{figures/classify_fig6_j6.png}
	\caption{Classification results using subspace features ($\lambda$ = 20) obtained from the spectrogram, the pre-processed cadence-velocity diagram (CVD) and the pre-processed mean cadence spectrum (mCS).
	\label{fig:compare_domains}}
\end{figure} % Experimental Methods and Feature Extraction
\section{Experimental Results}\label{sec:results}

\subsection{Physical Features}
Table \ref{tab:Phy_results_5classes} shows the classification results using the feature vector as defined in (\ref{eq:phyfeat}). The average correct classification rates assume 82\%, 80\%, 90\%, 79\% and 93\% for NW, L1, L2, CW and CW/oos, respectively. The corresponding overall classification performance is 85\%, with a false alarm rate of 18\% and a missed detection rate of 17\%. False alarms are normal walks that are misclassified as abnormal or assisted, and missed detections denote abnormal or assisted walks that are wrongly classified as normal walk. We note that normal walks (NW) are mostly confused with walking with a cane (CW), and vice versa. This is expected in the sense that the underlying motion of walking with a cane is a normal walk, where the cane's micro-Doppler signatures superimpose every other leg micro-Doppler signature.

% [Bjoe15]
Next, Table \ref{tab:Bjoe151_results_5classes} presents the classification results for the feature vector $\mathbf{z}^{\text{B1}}$ used by Bj{\"o}rklund \textit{et. al}~\cite{Bjoe15}. For comparison, we use the same classifier as for the physical features, i.e., the NN classifier. Using the first feature vector, the overall correct classification rate assumes 78\%, with a false alarm rate of 13\% and a missed detection rate of 29\%. Despite the increased number of features, the average correct classification rate is lower compared to using physical features. Removing the velocity profiles from the feature vector, i.e., using $\mathbf{z}^{\text{B2}}$, the classification accuracy decreases to only 42\%, which shows that the cadence frequencies $f_1$, $f_2$ and $f_3$ along with the base velocity $v_0$ are not key in discriminating the considered gait classes.

% [Ric15]
Using the feature vectors $\mathbf{z}^{\text{R1}}$ and $\mathbf{z}^{\text{R2}}$ defined by Ricci and Balleri \cite{Ric15}, the results are given in Tables \ref{tab:Ric151_results_5classes} and \ref{tab:Ric152_results_5classes}, respectively. In the first case, the parameters $\Delta m$ and $\gamma$ were optimized as to achieve the highest average classification rate. We observe that a high number of abnormal (L1) and assisted walks (CW) are wrongly classified as normal gait, and vice versa. Further, limping with one leg (L1) is confused with cane-assisted walks (CW) and vice versa. Thus, the gait classes that reveal the same gait pattern are difficult to be classified correctly. The overall classification rate is found as 56\% with a false alarm rate of 60\% and a missed detection rate of 59\%. Using the mean Doppler spectrum around $\fmd$ as a feature, i.e., $\mathbf{z}^{\text{R2}}$, the correct classification rate is given by 75\%, where the false alarm and missed detection rates assume 26\% and 27\%, respectively. Here, most confusions are observed between limping with one (L1) and both legs (L2), as well as, between normal walk (NW) and cane-assisted walks (CW). Again, we note that the mean Doppler spectrum comprises more information for classification of the considered motions than single Doppler or cadence frequencies.

Hence, we conclude that physical features, such as the base velocity or the micro-Doppler repetition frequency, are not suited to discriminate between the considered gait classes, i.e., to solve the intra motion category classification problem of gait recognition. However, signals obtained from the CVD, e.g., the mean Doppler spectrum, do hold discriminative characteristics that allow to distinguish between different walking styles.

\subsection{Subspace-based Features}

Table~\ref{tab:subspace_results_cvd_5classes} shows the classification results utilizing PCA-based features of CVDs and the NN classifier. Here, $\lambda$ = 20 principal components are used. The overall correct classification rate assumes 93\%, where the false alarm rate is 8\% and the missed detection rate is 12\%. The highest classification rates are achieved for walking with a cane out of sync (99\%). The gait class CW shows the lowest classification rate (85\%) as this motion is again confused with normal walking, and vice versa.

The results demonstrate the suitability of the CVD and the effectiveness of PCA for feature extraction. Even though we only consider one motion class and a single signal domain, the proposed method classifies the gaits with a high accuracy.

\begin{table}[!t]
	\renewcommand{\arraystretch}{1.2} \setlength{\tabcolsep}{0.25em}
	\caption{Confusion matrix for classification using physical features, i.e, $\mathbf{z}^\text{phy}$, and the NN classifier. Numbers are given in \%.} 
	%myq = 5, NN, 500 MCRuns, FA: 17.84%, MD: 16.07%
	\label{tab:Phy_results_5classes}
	\centering
	\begin{tabular}{  l | >{\centering\arraybackslash}m{0.9cm} | >{\centering\arraybackslash}m{0.9cm} | >{\centering\arraybackslash}m{0.9cm} | >{\centering\arraybackslash}m{0.9cm} | >{\centering\arraybackslash}m{0.9cm} }
		\hline
		\textbf{True / Predicted } & NW & L1 & L2 & CW & CW/oos \\
		\hline \hline
		Normal walk (NW) & \cellcolor{green!20} 82 & 1 & 4 & \cellcolor{yellow!20} 11 & 2\\
		\hline
		Limping with one leg (L1) & 3 & \cellcolor{green!20} 80 & 5 & 9 & 3\\
		\hline
		Limping with both legs (L2) & 3 & 4 & \cellcolor{green!20} 90 & 3 & 0\\
		\hline
		Cane - synchronized (CW) & 9 & 7 & 4 & \cellcolor{green!20} 79 & 1 \\
		\hline
		Cane - out of sync (CW/oos) & 2 & 4 & 0 & 1 & \cellcolor{green!20} 93\\
		\hline
	\end{tabular}
\end{table}

\begin{table}[!t]
	\renewcommand{\arraystretch}{1.2} \setlength{\tabcolsep}{0.25em}
	\caption{Confusion matrix for classification using the feature vector $\mathbf{z}^{\text{B1}}$ by Bj\"orklund \textit{et al.} \cite{Bjoe15}, and the NN classifier. Numbers are given in \%.}
	%NN, 500 MCRuns, FA: 13.82%, MD: 19.93%
	\label{tab:Bjoe151_results_5classes}
	\centering
	\begin{tabular}{  l | >{\centering\arraybackslash}m{0.9cm} | >{\centering\arraybackslash}m{0.9cm} | >{\centering\arraybackslash}m{0.9cm} | >{\centering\arraybackslash}m{0.9cm} | >{\centering\arraybackslash}m{0.9cm} }
		\hline
		\textbf{True / Predicted } & NW & L1 & L2 & CW & CW/oos \\
		\hline \hline
		Normal walk (NW) & \cellcolor{green!20} 87 & 2 & 2 & 8 & 1\\
		\hline
		Limping with one leg (L1) & 5 & \cellcolor{green!20} 79 & 7 & 7 & 2\\
		\hline
		Limping with both legs (L2) & 1 & 2 & \cellcolor{green!20} 94 & 3 & 0\\
		\hline
		Cane - synchronized (CW) & \cellcolor{yellow!20} 18 & 4 & 1 & \cellcolor{green!20} 73 & 4 \\
		\hline
		Cane - out of sync (CW/oos) & \cellcolor{yellow!20} 11 & 9 & 2 & \cellcolor{yellow!20} 18 & \cellcolor{green!20} 60\\
		\hline
	\end{tabular}
\end{table}

\begin{table}[!t]
	\renewcommand{\arraystretch}{1.2} \setlength{\tabcolsep}{0.25em}
	\caption{Confusion matrix for classification using the feature vector $\mathbf{z}^{\text{R1}}$ by Ricci and Balleri \cite{Ric15}, and the NN classifier.
		Numbers are given in \%.}
	%NN, 500 MCRuns, FA: 52.45 MD: 54%
	\label{tab:Ric151_results_5classes} 
	\centering
	\begin{tabular}{  l | >{\centering\arraybackslash}m{0.9cm} | >{\centering\arraybackslash}m{0.9cm} | >{\centering\arraybackslash}m{0.9cm} | >{\centering\arraybackslash}m{0.9cm} | >{\centering\arraybackslash}m{0.9cm} }
		\hline
		\textbf{True / Predicted } & NW & L1 & L2 & CW & CW/oos \\
		\hline \hline
		Normal walk (NW) & \cellcolor{green!20} 40 & \cellcolor{yellow!20} 17 & 9 & \cellcolor{orange!20} 27 & 7\\
		\hline
		Limping with one leg (L1) & \cellcolor{yellow!20} 14 & \cellcolor{green!20} 44 &  \cellcolor{yellow!20} 16 &  \cellcolor{orange!20} 20 &  6\\
		\hline
		Limping with both legs (L2) & \cellcolor{yellow!20} 12 &  \cellcolor{yellow!20} 10 & \cellcolor{green!20} 74 & 3 & 1\\
		\hline
		Cane - synchronized (CW) & \cellcolor{orange!20} 26 & \cellcolor{yellow!20} 18 & 3 & \cellcolor{green!20} 43 &  \cellcolor{yellow!20} 10 \\
		\hline
		Cane - out of sync (CW/oos) & 5 & 6 & 0 & 9 & \cellcolor{green!20} 80\\
		\hline
	\end{tabular}
\end{table}

\begin{table}[!t]
	\renewcommand{\arraystretch}{1.2} \setlength{\tabcolsep}{0.25em}
	\caption{Confusion matrix for classification using the feature vector $\mathbf{z}^{\text{R2}}$ by Ricci and Balleri \cite{Ric15}, and the NN classifier.
		Numbers are given in \%.}
	%NN, 500 MCRuns, FA: 29.45 MD: 39.93%
	\label{tab:Ric152_results_5classes}
	\centering
	\begin{tabular}{  l | >{\centering\arraybackslash}m{0.9cm} | >{\centering\arraybackslash}m{0.9cm} | >{\centering\arraybackslash}m{0.9cm} | >{\centering\arraybackslash}m{0.9cm} | >{\centering\arraybackslash}m{0.9cm} } %44 FA 45 , MD 41
		\hline
		\textbf{True / Predicted } & NW & L1 & L2 & CW & CW/oos \\
		\hline \hline
		Normal walk (NW) & \cellcolor{green!20} 74 & 5 & 4 & \cellcolor{yellow!20} 15 & 2\\
		\hline
		Limping with one leg (L1) & 6 & \cellcolor{green!20} 74 & \cellcolor{yellow!20} 10 &  5 &  5\\
		\hline
		Limping with both legs (L2) & 4 & \cellcolor{yellow!20} 12 & \cellcolor{green!20} 80 & 4 & 0\\
		\hline
		Cane - synchronized (CW) & \cellcolor{yellow!20} 12 & 4 & 3 & \cellcolor{green!20} 74 &  7 \\
		\hline
		Cane - out of sync (CW/oos) & 6 & 6 & 1 &  \cellcolor{yellow!20} 12 & \cellcolor{green!20} 75\\
		\hline
	\end{tabular}
\end{table}

\begin{table}[!t]
	\renewcommand{\arraystretch}{1.2} \setlength{\tabcolsep}{0.25em}
	\caption{Confusion matrix for classification using the feature vector $\mathbf{z}^{\text{PCA}}$ ($\lambda$ = 20), and the NN classifier. Numbers are given in \%.}
	\label{tab:subspace_results_cvd_5classes}
	%NN, lambda = 18, 500 MCRuns, FA: 7.06 MD: 13.10%
	\centering
	\begin{tabular}{  l | >{\centering\arraybackslash}m{0.9cm} | >{\centering\arraybackslash}m{0.9cm} | >{\centering\arraybackslash}m{0.9cm} | >{\centering\arraybackslash}m{0.9cm} | >{\centering\arraybackslash}m{0.9cm} } % j8 results: 73 % (FA 22 MD 21) %j7 results: 69% (FA 27,MD 26)
		\hline
		\textbf{True / Predicted } & NW & L1 & L2 & CW & CW/oos \\
		\hline \hline
		Normal walk (NW) & \cellcolor{green!20} 93 & 1 & 0 & 6 & 0\\
		\hline
		Limping with one leg (L1) & 0 & \cellcolor{green!20} 95 & 0 & 5 & 0\\
		\hline
		Limping with both legs (L2) & 3 & 0 & \cellcolor{green!20} 95 & 2 & 0\\
		\hline
		Cane - synchronized (CW) & \cellcolor{yellow!20} 10 & 5 & 0 & \cellcolor{green!20} 85 & 0 \\
		\hline
		Cane - out of sync (CW/oos) & 0 & 1 & 0 & 0 & \cellcolor{green!20} 99\\
		\hline
	\end{tabular}
\end{table}

\begin{table}[!t]
	\renewcommand{\arraystretch}{1.2}
	\caption{Comparison of different gait recognition algorithms. Numbers are given in \%.}
	\label{tab:all_results}
	\centering
	\begin{tabular}{ l | >{\centering\arraybackslash}m{0.4cm} | >{\centering\arraybackslash}m{0.4cm} | >{\centering\arraybackslash}m{0.4cm} | >{\centering\arraybackslash}m{0.4cm} | >{\centering\arraybackslash}m{0.4cm}} 
		\hline
		& Phy & B1 & R1 & R2 & PCA\\
		\hline \hline
		Correct classification rate & 85 & 78 & 56 & 75 & \textbf{93}\\
		\hline
		False alarm rate 		& 18 & 13 & 60 & 26 & \textbf{8}\\
		\hline
		Missed detection rate 	& 17 & 29 & 59 & 27 & \textbf{12}\\
		\hline
	\end{tabular}
\end{table}

\subsection{Discussion}

We used radar measurements that contain a representative portion of the gait to classify five different walking styles, including abnormal and assisted gait. Table~\ref{tab:all_results} summarizes the results of all presented gait classification methods. The subspace feature extraction method utilizing PCA of CVD images achieves the highest correct classification rate (93\%), while the false alarm (8\%) and missed detection (12\%) rates are kept low. Thus, we conclude that (i) subspace-based features are superior to physical features in classifying different gaits, and (ii) the CVD comprises more information on the gait than, e.g., the spectrogram.
It is pointed out that the proposed method works reasonably good for all gait classes, despite of the relatively small number of 1000 measurements of ten different test subjects. This is certainly one benefit over popular deep learning approaches, which require a very large (training) data set and are computationally costly \cite{Sey18,Kim16}.

In order to underscore the relevance of the acquired radar data, we also conducted experiments for radar data acquisition involving four test subjects with gait disorders due to different medical conditions. Examples of spectrograms for these subjects are shown in Fig.~\ref{fig:specs_threesubjects}. Figs.~\ref{fig:specs_threesubjects}\subref{LA}, \subref{LD} and \subref{LB} clearly show the same characteristic as the spectrogram of abnormal walking in Fig.~\ref{fig:specs}\subref{L1a}. Here, every other micro-Doppler stride signature has a lower maximal Doppler shift, which indicates an asymmetrical gait. In fact, as a result of a stroke at young age, Person A suffers from generalized dystonia affecting multiple muscle groups on one side of the body. Person B also experienced a stroke which caused a different gait disorder. In the case of Person C, due to the relative strength of the left side of the body, the asymmetry of the gait manifests itself in the knees' motions, rather than in different swinging velocities of the feet. Still, the spectrogram, as shown in Fig.~\ref{fig:specs_threesubjects}\subref{LC}, evidently reveals the gait asymmetry: on the onset of every other micro-Doppler stride signature, we can observe higher energy levels due to the altered stride motions (see arrows). Fig.~\ref{fig:specs_threesubjects}\subref{LCCane} shows a spectrogram of Person C walking with a cane, where the cane's signatures is overlapping with every other stride signature, similar to Fig.~\ref{fig:specs}\subref{CW}. The fourth person (D) has a congenital hip dislocation and suffers from a hip osteoarthritis on one body side due to it.

When applying the proposed classification method, i.e., using subspace-based features of pre-processed CVDs, we can correctly identify the gait as abnormal in 100\% (13/13), 100\% (20/20), 83\% (10/12), and 100\% (28/28) of the cases for Person A, B, C, and D, respectively. The cane is correctly detected for Person C in 81\% (13/16) of the cases. Here, the test data set included the measurements of one individual at a time, and the data set of ten individuals as described in Section~\ref{sec:exp_setup} was used for training. Even though the observation time is only 6\,s per measurement, we can detect the asymmetry of the gait with very high accuracies. These results, which are based on Doppler radar data representations of subjects with diagnosed gait disorders, are very promising and will serve as a basis for more extensive studies.

% Example of spectrograms
\begin{figure}[!t]
	\vspace{-1em}
	\centering{
		\subfloat[Person A]{\includegraphics[clip, trim= 0 0 20 18, width=0.5\columnwidth]{figures/PersonA_A.png}%
			\label{LA}}
		%\hfill
		\subfloat[Person D]{\includegraphics[clip, trim= 0 0 20 18,width=0.5\columnwidth]{figures/PersonD_A.png}%
			\label{LD}}\vspace{-0.7em}}
	
	\centering{
		\subfloat[Person B]{\includegraphics[clip, trim= 0 0 20 18,width=0.5\columnwidth]{figures/PersonB_T.png}%
			\label{LB2}}
		%\hfill
		\subfloat[Person B]{\includegraphics[clip, trim= 0 0 20 18,width=0.5\columnwidth]{figures/PersonB_A.png}%
			\label{LB}}\vspace{-0.7em}}

	\centering{
		\subfloat[Person C]{\includegraphics[clip, trim= 0 0 20 18,width=0.5\columnwidth]{figures/PersonC_T_arrows.png}%
			\label{LC}}
		%\hfill
		\subfloat[Person C walking with a cane]{\includegraphics[clip, trim= 0 0 20 18,width=0.5\columnwidth]{figures/PersonC_CW_T.png}%
			\label{LCCane}}}
	\caption{Examples of spectrograms of four subjects with diagnosed gait disorders. The color indicates the energy level in dB.}
	\label{fig:specs_threesubjects}
\end{figure}
 % Results
\section{Conclusion}\label{sec:conclusion}

In this paper, different human walking styles were analyzed based on radar micro-Doppler signatures and their Fourier transforms. We showed the capability of radar to serve as a diagnostic tool and long-term monitoring device of human gait. Methods were presented to perform gait recognition utilizing the cadence-velocity domain, where five different walking styles were distinguished including limping and walking with a cane. Here, subspace-based features were superior to standard physical features in solving the intra motion category classification problem of discerning different human gaits. % Conclusions

%-----------------------------------------------------------------------------------
%	ACKNOWLEDGEMENT
%-----------------------------------------------------------------------------------

\section*{Acknowledgment}
This work is supported by the Alexander von Humboldt Foundation, Bonn, Germany.

%----------------------------------------------------------------------------------------
%	REFERENCE LIST
%----------------------------------------------------------------------------------------
\bibliographystyle{IEEEtran}
\bibliography{ref_Journal2017}

%----------------------------------------------------------------------------------------
\end{document}


