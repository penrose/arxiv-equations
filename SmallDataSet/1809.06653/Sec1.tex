\section{Introduction}

% why radar
\IEEEPARstart{R}{ecently}, radar has received much attention in civilian applications, most notably, in automotive and health care industries. Specifically, the applications of radar technology in home security, elderly care, and medical diagnosis have emerged to be front and center in indoor human monitoring \cite{Che14,Ami16,Ami17}. These include fall motion detection, classifications of daily human activities, and vital sign monitoring. The considerable rise in radar indoor applications and smart homes is credited to its safety, reliability, and ability to serve as an effective device for contact-less motion monitoring of subjects in the surrounding settings and environments, while preserving privacy. Other non-wearable sensing modalities for indoor human monitoring include infrared reflective light, refractive light, video cameras, and in-ground force platforms \cite{Sim04,Mur14}. However, visual perception or video recordings of human motions can easily be disturbed by occlusions, lighting conditions and clothing. 

% prior work / other approaches
Low-cost Doppler radars have widely been used for detection \cite{Wan14,Su15,Kim15,Cle15,Jok18}, classification \cite{Kim09,Bjoe15,Jok17}, and recognition of human motions \cite{Ric15}. However, most of these works are concerned with the discrimination between different classes of motions, and as such, they consider inter motion category classification problems. A nominal example is discriminating between running, walking, crawling, creeping, sitting, bending and falling. Yet, little thought has been given to study the intra motion category classification problem, i.e., discerning variations within one motion class.

% radar-based gait recognition
In this paper, we focus on classifying human gait within its class. Human gait analysis plays a key role in medical diagnosis, biomedical engineering, sports medicine, physiotherapy and rehabilitation \cite{Mur14}. Constant monitoring of changes in gait aids in assessing recovery from body injury. Further, it enables early diagnosis of different diseases, including multiple sclerosis, Parkinson's and cardiopathies, and facilitates studying the course of disease for designing adequate treatment \cite{Mur14}. For these reasons, it is important to detect gait abnormalities and monitor alterations in walking patterns over time.
However, detailed gait analysis and proper assessments of walking aids can prove difficult for physicians, health care providers and nursing staff. Thorough clinical gait studies are often time-consuming, costly and lack reproducibility \cite{Sim04}. That is why we seek a contact-less sensing technology to empower, and not necessarily replace, naked eye gait examination, with the goal of achieving an expedited, more accurate and more efficient gait diagnosis. 

% assistive walking decives
Gait abnormalities also include using assisted walking devices. It is noted that a great number of seniors resort to assistive walking devices, such as a cane or a walker, in order to compensate for decrements in balance, gain mobility and overcome the fear of falling. In 2011, 8.5 million U.S.~seniors aged 65 and older reported having assistive walking devices, with a cane being most commonly used by two thirds of the elderly \cite{Gel15}. In this regard, a correct use of mobility devices becomes essential to guarantee optimal support and avoid postural deformities injuries or physical impairments with the purpose of re-establishing a normal gait. 

% this work
Using electromagnetic sensing modality, we consider classifying different walking styles, and thus demonstrate the effectiveness of radar in detecting subclasses of gait abnormalities. We show that radar can present a viable, convenient and contact-less supplement or alternative to using wearable sensors (for an overview see e.g.~\cite{Tao12}), which have widely been use to study human gait (for recent works see e.g.~\cite{Pha17,Ren17,Bro15}). As opposed to prior radar-based human gait classification methods, which consider walks with and without arm swinging \cite{Mob09,Tiv10,Tiv15}, or different speeds of walking \cite{Cle15,Ric15}, we focus on detecting differences in the lower limbs kinematics.

For this purpose, we devise a new approach based on predefined features for classifying gaits, where normal, pathological and assisted walks are considered. Here, two types of limping gait are analyzed, where one or both legs are not swinging normally. Further, we consider two different synchronization styles between the cane and the legs, as detection of walking aids has gained increased interest in the latest past \cite{Gur17,Sey18}. 

In addition to physical-based feature extractions, the paper considers automatic learning via subspace analysis. Applied in the cadence domain, we show that features based on principal component analysis (PCA) lead to desirable results that outperform those involving sum-of-harmonics modeling. Considering five different gait classes, a normal walking is correctly detected in 93\% of the cases. Although viewed in the same category as neural networks in unsupervised feature learning, PCA does not demand the same level of computations as deep learning approaches \cite{Jok16,Kim16,Sey18}, neither does it necessitate a very large number of (training) samples.  

In order to proof the applicability of the proposed method for detection of gait asymmetries, we collected radar data of four individuals with different diagnosed gait disorders. The corresponding radar data representations reveal characteristic features that indicate gait disorders. Applying the proposed classification method, the gait asymmetry is correctly detected with an accuracy of 100\% for three of the four test subjects.

% paper organization
The remainder of the paper is organized as follows. For analyzing backscattered radar data from human motions, Section~\ref{sec:signalrepresentations} briefly motivates and outlines different radar data representations, which can be utilized for feature extraction. In Section~\ref{sec:featextraction}, we propose feature extraction techniques for gait classification. Corresponding results are presented and discussed in Section~\ref{sec:results}. Conclusions are given in Section~\ref{sec:conclusion}.