\section{Introduction}

\label{sec_intro}
\textbf{Context:} In recent years Software Defined Networking (SDN) has gained popularity as a means to bring scalability and programmability to network architecture. Particularly in Data Center and Optical Networks, SDN has been shown to offer a high degree of network reconfigurability, reduction in capital expenditure, and a platform for Network Function Virtualization (NFV) \cite{sdn_comprehensive_survey}. These benefits have led to a number of research efforts to apply SDN within the IEEE 802.15.4 low power mesh networking standard, which underpins many Industrial Internet of Things (IIoT) networks. However, these networks are constrained in terms of reliability, throughput, and energy. These limitations mean that the traditional, high-overhead approach of SDN cannot be mapped directly to a Low Power and Lossy Network (LLN). Specifically, the problem is twofold: SDN control messages can be prone to delay and jitter, and the burden of controller overhead can severely affect other network traffic flows.

\textbf{Challenge:} One of the core principles behind SDN is the separation of the control and data planes: centralizing the control plane at an external SDN controller. In a traditional, wired, SDN architecture, network peripherals such as routers, load-balancers, and firewalls, are reduced to dumb switches which can be programmed to perform these tasks through manipulation of their tables using a SDN protocol such as OpenFlow. This programmability is possible due to low latency, reliable links between network devices and the controller, ensuring that network decisions are made rapidly, and that controller decisions are made with an up-to-date global view of the network. 

In IEEE 802.15.4 networks the link between a controller and network device cannot be guaranteed. Controller communication needs to navigate multiple hops across a mesh, survive unreliable links, tackle low throughput, and compete with other traffic. In a centralized SDN architecture, this means that nodes cannot be assured of successful and timely decisions from the controller, with no guarantee on delay and packet loss. 

\textbf{Approach:} However, the recent standardization effort from the IETF 6TiSCH Working Group (WG) \cite{6tisch_ietf_architecture} aims to enable reliable IPv6 communications on top of IEEE 802.15.4-2015 Time Scheduled Channel Hopping (TSCH) networks. This is particularly relevant in industrial process control, automation, and monitoring applications: where failures or loss of communications can jeopardize safety processes, or have knock-on effects on processes down-the-line. The frequency and time diversity achieved from the TSCH schedule gives greater protection against a lossy network environment, and by specifying how that schedule is managed, 6TiSCH provides a platform to deliver Quality of Service (QoS) guarantees to network traffic. Additionally, 6TiSCH introduces the concept of \textit{tracks}: a Layer-2 mesh-under forwarding mechanism \cite{6lowpan_routing_rfc} which allows low-latency paths across the network. Tracks are created by slicing the TSCH schedule, and reserving buffer resources, at each hop along a route between a source and destination node. Each cell represents an atomic unit of bandwidth, and allocated cells are dedicated to that particular track. Since each cell is scheduled at a known time, tracks are essentially a deterministic slice of the network bandwidth, with a guaranteed minimum bounded latency \cite{6tisch_ietf_architecture}. 

\textbf{Contribution:} We aim to exploit the deterministic and low latency properties of 6TiSCH tracks to slice the TSCH schedule for the SDN control path, and provide a reliable link between constrained SDN nodes and a controller in a IEEE 802.15.4 low power wireless network. By using tracks to overcome the control link unreliability, the challenge of implementing a centralized SDN architecture within a LLN becomes more feasible, which in turn allows greater scalability and re-programmability in IoT sensor networks.

\textbf{Outline:} The rest of this paper is organized as follows. In Section \ref{sec_background} we present the problem of introducing a high-overhead SDN architecture given the constraints of a low power mesh network. We then introduce the 6TiSCH architecture, giving an overview of the terminology. Section \ref{sec_motivation} presents the main contribution of this paper, discussing how Layer-2 slicing using 6TiSCH tracks can isolate SDN control traffic in a LLN, and presents an integrated SDN/6TiSCH stack with track allocation for SDN controller links. Finally, we evaluate results of simulations in Section \ref{sec_results}, and draw conclusions in Section \ref{sec_conclusion}.

