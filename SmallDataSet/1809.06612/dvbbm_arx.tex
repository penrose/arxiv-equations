\documentclass[a4paper,11pt]{article}
\pdfoutput=1
\usepackage{amssymb}
\usepackage[utf8]{inputenc}
\usepackage{textcomp}
\usepackage{latexsym}
\usepackage{epstopdf}
\usepackage{epsfig}
\usepackage{graphicx}
\usepackage{amsmath}
\usepackage{multirow}
\usepackage{subfigure}
\usepackage{a4wide}
\usepackage{cite}
\usepackage{rotating}
\usepackage{hyperref}
%---- page parameters ----------------------------------------------------------------------------------------------------------------------------------
\jot = 1.5ex
%\parskip 5pt plus 1pt
%\parindent 0pt
\evensidemargin -0.2in   \oddsidemargin  -0.2in
%\textwidth 6.7in         \textheight 9.5in
\topmargin -2.25cm        \headsep    1.0cm

%---- command (re)definitions ----------------------------------------------------------------------------------------------------------------------------
\newcommand{\capdef}{}
\newcommand{\mycaption}[2][\capdef]{\renewcommand{\capdef}{#2}
\caption[#1]{{\footnotesize #2}}}
%%%%%%%%%%%%%%%%%%%% math environments %%%%%%%%%%%%
\newcommand{\bwt}{\begin{widetext}}
\newcommand{\ewt}{\end{widetext}}
\newcommand{\be}{\begin{equation}}
\newcommand{\ee}{\end{equation}}
 \newcommand{\bdm}{\begin{displaymath}}
\newcommand{\edm}{\end{displaymath}}
\newcommand{\bea}{\begin{eqnarray}}
\newcommand{\eea}{\end{eqnarray}}
\newcommand{\nn}{\nonumber}
\newcommand{\red}{\color{red}}
\newcommand{\blue}{\color{blue}}
\newcommand{\green}{\color{green}}
\newcommand{\mrk}{\vskip 2mm {\color{green}[MARK]}\vskip 2mm }
%%%%%%%%%%%%%%%%%%%%%%%%%%%%%%%%%%%%%%%%%%
%---- symbol short-hands and redefinitions -----------------------------
\def\eq#1{{Eq.~(\ref{#1})}}
\def\eqs#1#2{{Eqs.~(\ref{#1})--(\ref{#2})}}
\def\fig#1{{Fig.~\ref{#1}}}
\def\figs#1#2{{Figs.~\ref{#1}--\ref{#2}}}
\def\Table#1{{Table~\ref{#1}}}
\def\Tables#1#2{{Tables~\ref{#1}--\ref{#2}}}
\def\sect#1{{Sect.~\ref{#1}}}
\def\sects#1#2{{Sects.~\ref{#1}--\ref{#2}}}
\def\app#1{{Appendix~\ref{#1}}}
\def\apps#1#2{{Apps.~\ref{#1}--\ref{#2}}}
\def\citelist#1#2{{[\citenum{#1}--\citenum{#2}]}}
%%%%%%%%%%%%%%%%%%%%%%%%%%%% %%%%%%%%%%%%%%%%%%%%%%%%%%%%%%%%%%%%
\def\vev#1{\left\langle #1 \right\rangle}
\def\abs#1{\left| #1\right|}
\def\mod#1{\abs{#1}}
\def\Im{\mbox{Im}\,}
\def\Re{\mbox{Re}\,}
\def\Tr{\mbox{Tr}\,}
\def\det{\mbox{det}\,}
\def\etal{\hbox{\it et al.}}
\def\ie{\hbox{\it i.e.}{}}
\def\eg{\hbox{\it e.g.}{}}
\def\etc{\hbox{\it etc}{}}
\def\chain#1#2{\mathrel{\mathop{ell\longrightarrow}\limits^{#1}_{#2}}}
%%%%%%%%%%%%%%%%%%%%%%% frequently used %%%%%%%%%%%%%%%%%%%%%%%%%%%%%%%
\def\stheta{\sin^22\theta_{13}}
\def\dma{\Delta m_{31}^2}
\def\dms{\Delta m_{21}^2}
\def\deltacp{\delta_\mathrm{CP}}
\def\GSM{SU(3)_{c}\otimes SU(2)_{L}\otimes U(1)_{Y}}
%%%%%%%%%%%%%%%%%%%%%%%%%%%%%%%%%%%%%%%%%%%%%%%%%%%%%%%%%%%%%%%%%%%%%%%

%%%%%%%%%%%%%%%%%%%%
\newcommand{\rr}[1]{{\color{red}#1}}
%%%%%%%%%%%%%%%%%%%%

%--------------- typesetting ------------------------------------------
\newcommand{\sepA}{\rule[-0.8cm]{0cm}{1.6cm}}
\newcommand{\sepB}{\rule[-1cm]{0cm}{2cm}}
\newcommand{\sepC}{\rule[-1.2cm]{0cm}{2.4cm}}
\newcommand{\sepD}{\rule[-1.4cm]{0cm}{2.8cm}}
\usepackage{slashed}
\newcommand{\mathsym}[1]{{}}
\newcommand{\unicode}{{}}
\newcommand{\D}{\displaystyle}
%\newcommand{\dms}{\mbox{$\Delta m^2_{\odot}$}}
%\newcommand{\dma}{\mbox{$\Delta m^2_{\rm A}$}}
\newcommand{\meff}{\mbox{$\langle m \rangle$}}
\newcommand{\baz}{\begin{array}{cc}}
\newcommand{\bad}{\begin{array}{ccc}}
\newcommand{\ba}{\begin{array}{c}}
\newcommand{\ea}{\end{array}}
%\newcommand{\be}{\begin{equation}}
%\newcommand{\ee}{\end{equation}}
%\newcommand{\bea}{\begin{eqnarray}}
%\newcommand{\eea}{\end{eqnarray}}
%\newcommand{\nn}{\nonumber}
\newcommand{\bi}{\begin{itemize}}
\newcommand{\ei}{\end{itemize}}
\newcommand{\bmt}{\begin{pmatrix}}
\newcommand{\emt}{\end{pmatrix}}
\newcommand{\bt}{\begin{tabular}}
\newcommand{\et}{\end{tabular}}
\newcommand{\hl}{\hline}
\newcommand{\ovl}{\overline}
\newcommand{\ul}{\underline}
\newcommand{\noi}{\noindent}
\newcommand{\benu}{\begin{enumerate}}
\newcommand{\eenu}{\end{enumerate}}


%---- fix a few hyphenation problems ----------------------------------
\hyphenation{pa-ra-meter pa-ra-meters}
%----------------------------------------------------------------------------------

\begin{document}
%\preprint{TTP13-007, SFB/CPP-13-14}

%\begin{center}
\title{\Large\bf {High Scale Type-II Seesaw, Dominant $W_L-W_L$ Channel Double Beta Decay  within 
Cosmological Bound  and Verifiable LFV Decays in SU(5)}}

%\pacs{12.10.-g, 12.10o.Kt, 14.80.-j}
%\vskip 1cm
\author{M. K. Parida$^{\dagger}$, Rajesh Satpathy $^{*}$\\
Centre of Excllence in Theoretical and Mathematical Sciences\\
Siksha 'O' Anusandhan Deemed to be University (SOADU), Khandagiri
Square,\\
 Bhubaneswar 751030, India}
%\email{minaparida@soa.ac.in}
%\email{
\maketitle

\begin{abstract}
 Very recently  a novel implementation of type-II seesaw mechanism for
neutrino mass has been proposed in SU(5) grand unified theory with a
number of desirable new physical phenomena beyond the standard model.  
Introducing heavy right-haded neutrinos and extra fermion singlets, in this
work we show how the type-I seeaw cancellation mechanism works in this
SU(5) framework.  Besides predicting verifiable
LFV decays,  we further  show that the model predicts
dominant double beta decay  with normal hierarchy or
inverted hierarchy of active light neutrino masses
in concordance with cosmological bound.  In addition, a novel
mechanism for heavy right-handed neutrino mass generation independent of type-II seesaw
predicted mass hierarchy is suggested in this work.
\end{abstract}
\noindent{${}^{\dagger}$email:minaparida@soa.ac.in}\\
\noindent{${}^{*}$email:rajesh.rajesh.satpathy@gmail.com} 
%-------------------------------------------------------------------------------
%%%%%%%%%%%%%%%%%%%%%%%%%%%%%%%%%%%%%%%%%%%%%%%%%%%%%%%%%%%%%%%%%%%%%%%%%%%%%%%%%%%%%
\section{Introduction}\label{sec:intr}
Renormalisable  standard model (SM) predicts  neutrinos to be massless 
 whereas  oscillation experiments prove them
to be massive \cite{Salas:2017,schwetz,forero,fogli,gonzalez}. All the generational mixings have been found to be much larger than the corresponding quark mixings.
Theoretically \cite{Alta:2014,AYS:2014,RNM:2015,RNM:2016,Valle:2016,Valle:2017,gs:2011,nurev:mkpbpn}
%Valle:add,Valle:nuDM}
 neutrino masses are predicted through
various seesaw
mechanisms\cite{Minkowski:1977,Yanagida:1979,Gell-Mann:1979,Glashow:1979,RNM-gs:1980,Valle:1980,Valle:t2,Magg:t2,Lazaridis:t2,RNM-gs:t2,RNM-mkp:t2,inv1,inv2,inv3,inv4,inv5,inv6,inv7,LG:2000,type-III,Akhmedov,mkp:hybrid,Linear,Rad,mkprad:2011,mkprad:2012}.
%ma:2006,mkp:2011,LG:2007} such as
 In a minimal left-right
symmetric \cite{JCP:1974,rnmpati:1975,gs-RNM:1975} grand unified theory (GUT)
like $SO(10)$ \cite{georgi:1974} where the
 parity (P) violation in weak interaction is explained along with
 fermion masses
 \cite{babu-rnm:1993,Joshipura,Altarelli-Blankenburg,Bertollini,Goh-Mohapatra}, a number
of these seesaw mechanisms can be naturally embedded while unifying
 the  three frorces of the SM \cite{mkp-PLB:1983}. More recently
 precision gauge coupling unification has been successfully implemented in
 direct symmetry breaking of $SO(18) \to SM$ which may have high
 potential for new physics \cite{Wilczek-Valle}.
%mkp-PRD:1983,cmp1:1984,cmp2:1984,cmp3:1985,cmp-PLB:1984,Pal:1994,rnm-mkp:1993,l%mpr:1995,Shafi:1984,Jaydeep:1918}.
% With natural inclusion
%of right-handed (RH) neutrinos in its spinorial fermionic representation, the
%model can predict the Dirac neutrino masses or Yukawa couplings by
%fitting the charged fermion masses which play a crucial role in
% seesaw mechanisms, lepton flavour violations (LFVs), and lepton number
%violations (LNVs).

The SO(10) model that predicts the  most popular canonical seesaw as
well as the type-II seesaw has also the potential to explain baryon asymmetry of the universe via
leptogenesis through heavy RH neutrino \cite{fukuyana:1986} or Higgs triplet  decays\cite{Hambye-gs}. But
because of underlying quark lepton symmetry \cite{JCP:1974}, the
type-I seesaw scale as well as RH$\nu$ masses are so large that the model
predicts negligible lepton flavor violating (LFV) decays like $\mu\to e\gamma \,$, $\tau\to
\mu\gamma \,$, $\tau\to e\gamma \,$, and $\mu \to e{\bar e}e$. Similarly
direct mediation of large mass of scalar triplet required for type-II
seesaw   gives negligible contribution to lepton number violating
(LNV) and lepton flavor violating (LFV) decays.
%%%%%%%%%%%%%%%%%%%%%%%%%%%%%%%%%%%%%%%%%%%%%%%%%%%%%%%%%%%%%%%%%%
Ever since the proposal of left-right symmetry, extensive
investigations continue in search of experimentally observable double beta decay\cite{bbexpt1-Klapdoor,bbexpt2,bbexpt3} in the
$W_R-W_R$ channel \cite{gs:bb,mkp-bs:2015}. Adding new dimension to such lepton number violatin g (LNV) process, the
like-sign dilepton production  has been suggested as a
possible means of detection of $W_R$-boson at accelerator energies \cite{Keung-gs:1983},
particularly the LHC \cite{Bajc-gs:2007}. However, no such signals of TeV scale $W_R-$
 have been detected so far.
 Even if $W_R$ mass and seesaw scales are
large and  inaccessible for direct verification, neutrinoless
double beta decay ($0\nu 2\beta$) in the $W_L-W_L$ channel \cite{Valle:t2,Valle:bb1,Valle:nuD1,Valle:MajDM,Valle:nuD2,Valle:CPLRS,Valle:KeVDM} is predicted close to
observable limit with $\tau_{\beta \beta} \ge 10^{25}$ yrs provided
light neutrino masses predicted by such high-scale seesaw mechanisms
are of quasidegenerate (QD) type with masses $m_{\nu} \ge {\cal O}(0.2)$ eV \cite{bbexpt1-Klapdoor}.
But as noted by the recent Planck data such QD type masses   violate the
cosmological bound \cite{Planck15}
\be
\Sigma_{\nu}\equiv {\sum}_{i=1}^3 {\hat m}_i\, < 0.23~(\rm  eV). \label{eq:cbound}.
\ee
 The fact that such
QD type $\nu$ masses violate the cosmological bound may be unravelling another
basic frundamental reason  why  detection of double beta decay continues to
elude experimental observation for several decades. On the other hand if
neutrinos have smaller NH or IH type masses there is no hope for
detection of these LNV events in near future with RH$\nu$ extended SM.
In other words predicting observable double beta decay in the
$W_L-W_L$ channel with left-handed helicities of borh the beta
particles has been a formidable problem confronting theoretical and
experimentral physicists. However, it has been shown that in case of dynamical seesaw mechanism generating Dirac neutrinos the seesaw scale is accessible for direct experimental verification \cite{Valle:dynamical}.
%%%%%%%%%%%%%%%%%%%%%%%%%%%%%%%%%%%%%%%%%%%%%%%%%%%%%%%%%%%%%%%%%%%%%%

The path breaking discovery of  inverse seesaw \cite{inv1,inv2,inv3,inv4,inv5,inv6,inv7} with
one extra singlet fermion per generation not only opened up
the neutrino mass generation mechanism for direct experimental tests,
but also  lifted up 
lepton flavor violating (LFV) decays \cite{Valle:LFV1987} from the abysmal depth of
experimental inaccessibility of negligible branching ratios (
$ Br.(l_\alpha\to l_\beta \gamma) \sim 10^{-50}$) to the illuminating
salvation of profound observability ( Br.$\simeq
10^{-8}-10^{-16}$) \cite{lfvexpt} which has been discussed extensively \cite{ilakovac,antbnd,non-unit}.  Despite inverse seesaw,
 observable double beta decay in the $W_L-W_L$ channel and the non-QD type
 neutrino masses remained mutually
exclusive until both the RH neutrinos and singlet fermions ($S_i$) were
brought into the arena of LFV and LNV
conundrum through the much needed extension of the Higgs sector. The King-Kang
\cite{Kim-Kang:2006} mechanism cancelled out the ruling supremacy of canonical  
seesaw which was profoundly explioted in SO(10) models with the
introduction of both the SO(10) Higgs representations ${16}_H$ and ${126}^{\dagger}_H$ 
\cite{nurev:mkpbpn,mkp-bs:2015,app:2013,Majee:2009,mkp-ARC:2010,pas:2014,bpn-mkp:2015} with successful
prediction of  observable double beta decay in the
$W_L-W_L$ channel \cite{Valle:t2,Valle:bb1}. Very interestingly, even though high scale type-II seesaw
 can 
govern light neutrino masses of any hierarchy, possibility of observable LFV and double beta
decay prediction in the $W_L-W_L$ channel irrespective of light
neutrino mass hierarchies has been  realized at least
theoretically \cite{nurev:mkpbpn,bpn-mkp:2015}.
%%%%%%%%%%%%%%%%%%%%%%%%%%%%%%%%%%%%%%%%%%%%%%%%%%%%%%

The purpose of this work is to point out that there are new
interesting physics realisations with suitable
extensions of a non-SUSY SU(5) GUT model proposed recently \cite{scp:2018} where
type-II seesaw, precision coupling unification, verifiable proton
decay, scalar dark matter and vacuum stability have been already
predicted. However with naturally large type-II seesaw scale $> 10^{9.2}$ GeV observable double beta
decay accessible to ongoing experiments
\cite{bbexpt1-Klapdoor,bbexpt2,bbexpt3} is possible in this model
too with QD type neutrinos only of
common mass
with $|m_0| \ge 0.2$ eV like many other high scale seesaw models as noted above.
In this work we 
make additional prediction that dominant double beta decay in the
$W_L-W_L$ channel  can be realised with NH or IH type
hierarchy consistent with much lighter neutrino masses $|m_{i}|<< 0.2$
eV. Thus this realization is  consistent with
cosmological bound of eq.(\ref{eq:cbound}). Although such possibilities were realized earlier
in SO(10) with TeV scale $W_R$ or $Z^{\prime}$ bosons as noted above,
in SU(5) without any expectation of verifiable left-right symmetry, we have shown here for the first time that 
 the dominant double beta decay is mediated by a
sterile neutrino (Majorana fermion singlet) of ${\cal O}(1)$ GeV mass
of first generartion. The model further predicts LFV decay branching
ratios only $4-5$ orders smaller than the current experimental limits.
An dditional interesting part of the present work is the first suggestion of a new
mechanism for heavy RH$\nu$ mass generation that
permits these masses to have hierarchies independent of conventional
type-II seesaw prediction. Some applications of such masses are
briefly noted. Thus the highlights of the present model are
\begin{itemize}
\item{First implementation of the mechanism for cancellation of type-I seesaw leading to the dominance of type-II seesaw in SU(5).}
\item{Prediction of verifiable LFV decays only $4-5$ orders smaller
  than the current experimental limits.}
\item{Prediction of dominant double beta decay in the $W_L-W_L$
  channel close to the current experimental limits for light neutrino
  masses of NH or IH type in concordance  with cosmological bound.}
\item{Suggestion  of a new mechanism for right-handed neutrino mass generation
independent of type-II predicted mass hierarchy.} 
\item{Precision gauge coupling unification with verifiable proton
  decay which is the same as discussed in \cite{scp:2018}.}
\end{itemize}

%%%%%%%%%%%%%%%%%%%%%%%%%%%%%%%%%%%%%%%%%%%%%%%%%%%%%%55
%%%%%%%%%%%%%%%%%%%%%%%%%%%%%%%%%%%%%%%%%%%%%%%%%%%%%%%%%%%
This paper is organised in the following manner.
In Sec.\ref{sec:model} we briefly review the SU(5) model along with gauge coupling
unification and predictions of the intermediate scales. In Sec.\ref{sec:cancel} we
discuss how type-I seesaw formula for active neutrinos cancels out giving rise to dominance of
type-II seesaw and prediction of another type-I seesaw formula for sterile neutrino mass.
Fit to neutrino oscillation data is discussed in Sec.\ref{sec:t2fit}.
 In Sec.\ref{sec:rhnu} we present our suggestion for a new mechanism of RH$\nu$ mass generation. Predictions on
 LFV branching ratios is discussed in Sec.\ref{sec:lfv}.
Lifetime prediction for double beta decay is presented in Sec.\ref{sec:bb}.
In Sec.\ref{sec:sum}
we discuss the results of this work and state our conclusion.
 
%%%%%%%%%%%%%%%%%%%%%%%%%%%%%%%%%%%%%%%%%%%%%%%%%%%%%%%%%%%%%%%%%%
\section{\bf A NON-SUPERSYMMETRIC SU(5) MODEL}\label{sec:model}
\subsection{Extension of SU(5)}
As noted in ref.\cite{kynshi-mkp:1993} the inclusion of
the scalar $\kappa(3,0,8)\subset {75}_H$ with mass
$M_{\kappa}=10^{9.23} {\rm GeV}$  in the extended non-SUSY SU(5)
achieves precision gauge coupling unification. Then 
  type-II seesaw ansatz for neutrino mass is realised by inserting the
  entire Higgs multiplet ${15}_H\subset SU(5)$ containing the LH Higgs triplet
  $\Delta_L(3,-1, 1)$ at the samae scale $ M_{15_H}=10^{12}{\rm GeV}$.
\bea
\kappa(3,0,8)&\subset&{75}_H, M_{\kappa}=10^{9.2} {\rm GeV},\nonumber\\
\Delta_L(3,-1, 1)&\subset& {15}_H, M_{15_H}=10^{12}{\rm GeV},\nonumber\\
\chi_s(1,0,1)&&,~ M_{\chi_S} \sim {\cal O} (1) {\rm TeV}.\label{eq:kappadeltachi}
\eea
The scalar singlet $\chi_S(1,0,1)$ has played the
crucial interesting role of stabilising the SM scalar potential.
The interjaction of ${15}_H$ at any scale $> 10^{9.23}$ GeV in this
model maintains precision coupling unification.
In the present  model we extend the model further by the inclusion of the
following fermions and scalars. Being singlets under the SM gauge group
they do not affect precision mixing unification  of
ref.\cite{scp:2018}.   
\begin{itemize}
\item{Three right handed neutrino singlets $N_i(i=1,2,3)$,  one for
  each generation, with  masses to be fixed by this model phenomenology.}
\item{Three left-handed Majorana fermion singlets $S_i(i=1,2,3)$, 
one for each generation,  similar to those introduced in case of
inverse seesaw mechanism \cite{inv1,inv2,inv3,inv4,inv5,inv6,inv7}.}
\item{A Higgs scalar singlet $\xi_S(1,0,1)$  
  to generate $S-N$ mixings through its VEV and to ensure vacuum
  stability of the scalar potential.}
\end{itemize}

%%%%%%%%%%%%%%%%%%%%%%%%%%%%%%%%%%%%%%%%%%%%%%%%%%%%%%%%%%%%%%%%%%%%%%%
\subsection{Coupling Unification, GUT Scale, and Proton Lifetime}
As already discussed \cite{scp:2018,kynshi-mkp:1993} using renormalisatin group equations for gauge couplings and the set of  Higgs scalars of eq(\ref{eq:kappadeltachi}),
precision unification has been achieved with the PDG values of input parameters
\cite{PDG:2012,PDG:2014,PDG:2016} on $\sin^2\theta_W(M_Z),\alpha_S(M_Z)$ resulting in the following 
mass scales and  the GUT
fine-structure constant $\alpha_G$ 
\bea
M_U&=& 10^{15.23} {\rm GeV}     , \nonumber\\
M_{\kappa}&=& 10^{9.23} {\rm GeV}, \nonumber\\
M_{\Delta_L}&=& M_{{15}_H}=10^{12} {\rm GeV},\nonumber\\
\frac{1}{\alpha_G}&=& 37.765. \label{eq:solsu5}
\eea

Using threshold effects due to superheavy Higgs scalars
\cite{Weinberg:1980,Hall:1981,Ovrut:1982,mkp:1987,mkp-cch:1989,rnm-mkp:1993,Langacker:1993,lmpr:1995},
 proton lifetime prediction for $p\to e^+\pi^0$ turns out to be in
the experimentally accessible range \cite{Abe:2017}
\be
\tau_p(p\to e^+\pi^0)= \left(1.01\times 10^{34\pm 0.44 }- (5.5\times
10^{35\pm 0.44 } \right) {\rm yrs}.\label{eq:taup}
\ee

Extensive investigations with number of SU(5) GUT extensions have been
carried out with
proton lifetime predictions
consistent with experimental limits
\cite{Dorsner,Nath-Perez:2007,Langacker:1981}. But implementation of type-II
seesaw dominance due to type-I seesaw cancellation resulting in
dominant LFV and LNV decays as dicussed below are new in the context
of non-SUSY SU(5). 



\section{CANCELLATION OF TYPE-I AND DOMINANCE OF TYPE-II SEESAW }\label{sec:cancel}

Due to the introdution of heavy RH$\nu$s in the present model which were absent
in \cite{scp:2018}, it may be natural to presume
apriori that besides type-II seesaw, type-I
seesaw may also contribute substantially to light neutrino masses and mixings. But
it has been noted  that there is a natural mechanism to cancel out
type-I seesaw contribution while maitaining dominance of inverse
seesaw \cite{mkp-bs:2015,Kim-Kang:2006,app:2013,Majee:2009,mkp-ARC:2010,pas:2014} or type-II seesaw or even linear seesaw
\cite{nurev:mkpbpn,bpn-mkp:2015} as the case may be. Briefly we
discuss below how this mechanism operates in the present extended
model resulting in
  type-II seesaw dominance even in the presence of heavy RH$\nu$s.

%In addition to the $G_{2213}$ fermions multiplets of LRS model
%$\psi^i_L(2,1,-1,1),\psi^i_R
%(1,2,-1,1),q_L^i(2,1,1/3,3),q_R^i(1,2,1/3,3)\subset {16}^i_F\subset
%  SO(10) (i=1,2,3)$, we have introduced one extra fermion SO(10) singlet per ge%neration $S_i(i=1,2,3)$. 



The SM invariant Yukawa Lagrangian of the   model is 
\bea
{\cal L}_{\rm Yuk} &=& Y^{\ell} {\overline{\psi}}_{L}\psi_{R} \Phi  
 +f \psi^c_{L} \psi_{L} \Delta_L \nonumber\\
&+& y_{\chi}{\overline N }^C S \chi_s 
(1/2)M_{N}{\overline N}^C N + h.c. \label{eq:Yuk-Lag} 
\eea 

  
  Using the VEVs of the Higgs fields and denoting 
 $M=y_{\chi}<\chi_S> =y_{\chi}V_{\chi}$,   
$M_D = Y<{\Phi}> $,, 
a $9 \times 9$ neutral-fermion mass matrix has been obtained which upon
block diagonalisation yields  $3\times 3$ mass
matrices  for each of the light neutrino ($\nu_{\alpha}$) , the right handed
 neutrino ($N_{\alpha}$), and the sterile neutrino
 ($S_{\alpha}$) \cite{app:2013,pas:2014,bpn-mkp:2015}.
The block diagonalisation of $9\times 9$ neutral fermion mass matrix
was earler presented in  useful format in ref.\cite{LG:2000}
which have been effectively utilised to study the type-I seesaw
cancellation mechanism in SO(10) models.

In this model the left-handed triplet $\Delta_L$ and RH neutrinos
$M_N$ being much heavier than the other mass scales with 
$M_{\Delta_L}\gg M_N \gg M \gg M_D$ are at first integrated out from the Lagrangian leading to 
\begin{eqnarray}
- \mathcal{L}_{\rm eff} &=& \left(m_{\nu}^{II}+ M_D \frac{1}{M_N} M^T_D\right)_{\alpha \beta}\, \nu^T_\alpha \nu_\beta +
\left(M_D \frac{1}{M_N} M^T \right)_{\alpha m}\, \left(\overline{\nu_\alpha} S_m + \overline{S_m} \nu_\alpha \right)
\nonumber \\
&&\hspace*{4.0cm} +\left(M \frac{1}{M_N} M^T\right)_{m n}\, S^T_m S_n  \, ,
\end{eqnarray}
which, in the $\left(\nu,~ S\right)$ basis, gives the $6 \times 6$ mass matrix
\begin{eqnarray}
\mathcal{M}_{\rm eff} = \left( \begin{array}{cc}
    M_DM_N^{-1} M^T_D+m_{\nu}^{II}  &  M_D M_N^{-1} M^T    \\
    MM_N^{-1} M_D^T       & MM_N^{-1} M^T  
        \end{array} \right) \, ,
\label{eqn:eff_numatrix}       
\end{eqnarray}
while the $3\times 3$ heavy RH neutrino mass matrix $M_N$ is the other part of the full 
$9 \times 9$ neutrino mass matrix. This $9 \times 9$ mass matrix $\tilde{\mathcal{M}}_{\rm 
\tiny BD}$ which results from the first step of block diagonalization procedure as discussed 
above and in the appendix is
\begin{eqnarray}
\mathcal{W}^\dagger_1 \mathcal{M}_\nu \mathcal{W}^*_1 = 
\tilde{\mathcal{M}}_{\rm \tiny BD}
 = \bmt \mathcal{M}_{\rm eff} & 0\\
0& M_N
\emt \, ,
\end{eqnarray}
Defining
\bea
 X &=& M_D\,M^{-1},\nonumber\\
 Y&=&M\, M^{-1}_N,\nonumber\\
 Z&=&M_D\,M^{-1}_N.\label{eq:defXYZ}
\eea

The transfrmation matrix $\mathcal{W}_1$ has been derived as shown in
eqn.\,(\ref{app:w1})\cite{app:2013,bpn-mkp:2015}
\bea 
\mathcal{W}_1=\bmt
1-\frac{1}{2}ZZ^\dagger & -\frac{1}{2}ZY^\dagger & Z \\
-\frac{1}{2}YZ^\dagger & 1-\frac{1}{2}YY^\dagger & Y \\
-Z^\dagger & -Y^\dagger & 1-\frac{1}{2}(Z^\dagger Z + Y^\dagger Y)
\emt
 \label{app:w1}
\eea

\noindent
After the second step of block 
diagonalization, the type-I seesaw contribution cancels out and gives in the $\left(\nu, S, N \right)$ basis
\begin{eqnarray}
\mathcal{W}^\dagger_2 \tilde{\mathcal{M}}_{\rm \tiny BD} \mathcal{W}^*_2 = \mathcal{M}_{\rm \tiny BD}
= \bmt m_\nu &0&0\\
0&m_{\cal S}&0\\
0&0&m_{\cal N}
\emt\, ,
\label{eqn:block-form}    
\end{eqnarray}
where $\mathcal{W}_2$ has been derived in eqn.\,(\ref{app:w2})
\cite{app:2013,bpn-mkp:2015}. We ahve used the bare mass of $S_i$ and
VEV of $\chi_L(2,-1/2,1)$ to be vanshing i,e $\mu_S=0,<\chi_L>=0$
to  get the form suiatable for this model building
\bea 
\mathcal{W}_2 
=
\bmt 
1-\frac{1}{2}XX^\dagger &X & 0\\
-X^\dagger & 1-\frac{1}{2}X^\dagger X & 0 \\
0 & 0 & 1
\emt
 \label{app:w2}
\eea

In eq.(\ref{eqn:block-form}), the three $3 \times 3$ matrices are 
\bea
m_{\nu} & = & m_{\nu}^{II}=fv_L  \\ 
m_{\cal S}& = &  - M M^{-1}_NM^T  \\ 
m_{\cal N} &= &  M_N \, ,
\label{eq:mass}
\eea
the first of these being the well known type-II seesaw formula and the
second is the emergence of the corresponding type-I seesaw formula for
sterile fermion mass. The third of the above equations represents the
heavy RH$\nu$ mass matrix.

In the third step, $m_{\nu}$, $m_{\cal S}$, and $m_{\cal N}$ are further diagonalized 
by the respective unitary matrices to give their corresponding eigenvalues
\begin{eqnarray}
U^\dagger_\nu\, m_{\nu}\, U^*_{\nu}  &=& \hat{m}_\nu = 
         \text{diag}\left(m_{1}, m_{2}, m_{3}\right)\, , \nonumber \\ 
U^\dagger_S\, m_{\cal S}\, U^*_{S}  &=& \hat{m}_S = 
         \text{diag}\left(m_{S_1}, m_{S_2}, m_{S_3}\right)\, , \nonumber \\
U^\dagger_N\, m_{\cal N}\, U^*_{N}  &=& \hat{m}_N = 
         \text{diag}\left(M_{N_1}, M_{N_2}, M_{N_3}\right)\, .
\label{eq:nudmass}
\end{eqnarray}
\noindent
The complete mixing matrix \cite{LG:2000,app:2013} diagonalizing the above $9 \times 9$ 
neutrino mass matrix given in 
(\ref{eqn:numatrix}) turns out to be
 \begin{eqnarray}
\mathcal{V}&\equiv&
\bmt 
{\cal V}^{\nu\hat{\nu}}_{\alpha i} & {\cal V}^{\nu{\hat{S}}}_{\alpha j} & {\cal V}^{\nu \hat{N}}_{\alpha k} \\
{\cal V}^{S\hat{\nu}}_{\beta i} & {\cal V}^{S\hat{S}}_{\beta j} & {\cal V}^{S\hat{N}}_{\beta k} \\
{\cal V}^{N\hat{\nu}}_{\gamma i} & {\cal V}^{N\hat{S}}_{\gamma j} & {\cal V}^{N\hat{N}}_{\gamma k} 
\emt \\
&=&\bmt 
\left(1-\frac{1}{2}XX^\dagger \right) U_\nu  & 
\left(X-\frac{1}{2}ZY^\dagger \right) U_{S} & 
Z\,U_{N}     \\
-X^\dagger\, U_\nu   &
\left(1-\frac{1}{2} \{X^\dagger X + YY^\dagger \}\right) U_{S} &
\left(Y-\frac{1}{2} X^\dagger Z\right) U_{N}   \\
0 &-Y^\dagger\, U_{S} & \left(1-\frac{1}{2}Y^\dagger Y\right)\, U_{N} 
\emt \, ,
 \label{eqn:Vmix-extended}
\end{eqnarray}
as shown in the appendix. In eqn.\,(\ref{eqn:Vmix-extended}) $X = M_D\,M^{-1}$, $Y=M\, 
M^{-1}_N$ and $Z=M_D\,M^{-1}_N$.
%%%%%%%%%%%%%%%%%%%%%%%%%%%%%%%%%%%%%%%%%%%%%%%%%%%%%%%%%%%%%%%%%%%%%%%

The mass of the singlet fermion is aquired through a type-I seesaw mechanism 

\bea
m_S=-M\frac{1}{M_N}M^T \label{matms}
\eea
where $M$ is the $N-S$ mixing mass term in the Yukawa Lagrangian  eq.(\ref{Yuk-Lag}). 


\section{ TYPE-II SEESAW FIT TO OSCILLATION DATA}\label{sec:t2fit}
\subsection{ Neutrino Mass Matrix from Oscillation Data}
%Using eq.(\ref{dia1}) we have the $3\times 3$ neutrino mass matrix in
% flavor space
Using diagonalisation of neutrino mass matrix $(m_\nu)$ by the PMNS matrix $U_{\rm PMNS}$
\begin{equation}
m_\nu = U_{\rm PMNS}~diag(m_1, m_2, m_3) U_{\rm PMNS}^T ,\label{mnu}
\end{equation}
where $m_i (i=1,2,3)$ denote the mass eigen vlaues. For neutrino mixings we use the abbreviated cyclic notations $t_i=\sin\theta_{jk},c_i=\cos\theta_{jk}$ where $i,j,k$ are cyclic permutations of generational numbers $1,2,3$.
In the standard parametrisation  we \cite{PDG:2012,PDG:2014,PDG:2016}
\begin{equation}
 U_{\rm PMNS}= \left( \begin{array}{ccc} c_{3} c_{2}&
                      t_{3} c_{2}&
                      t_{2} e^{-i\delta_D}\cr
-t_{3} c_{1}-c_{3} t_{1} t_{2} e^{i\delta_D}& c_{3} c_{1}-
t_{3} t_{1} t_{2} e^{i\delta_D}&
t_{1} c_{2}\cr
t_{3} t_{1} -c_{3} c_{1} s_{2} e^{i\delta_D}&
-c_{3} t_{1} -t_{3} c_{1} t_{2} e^{i\delta_D}&
c_{1} c_{2}\cr
\end{array}\right) 
diag(e^{\frac{i \alpha_M}{2}},e^{\frac{i \beta_M}{2}},1)
\end{equation}
where  $\delta_D$ is the Dirac CP phase and $(\alpha_M,\beta_M)$ are Majorana phases. 

Here we present numerical analyses within $3\sigma$ 
limit of the neutrino oscillation data in the  type-II seesaw framework \cite{scp:2018}. As we do not have any experimental information 
about Majorana phases, they are determined by means of random
sampling: i,e  from the set of randomly generated values, each
confined within the maximum allowed limit of $2\pi$  only one set of
values for  $(\alpha_M,\beta_M)$ is chosen.
Very recent analysis of the oscillation data has determined
the  $3\sigma$  and $1\sigma$ 
limits of Dirac CP phase $\delta_D$ \cite{Salas:2017},
 The best fit values 
of $\delta_D$ in the normally ordered (NO) and invertedly ordered (IO) cases are near $1.2\pi$ and $1.5\pi$, respectively, which we utlise for the sake of simplicity.


 Global fit to the oscillation data \cite{Salas:2017}  is summarised below
 including respective parameter uncertainties at  $3\sigma$ level
\begin{eqnarray}
&&\theta_{12}\/^{\circ}=34.5\pm 3.25,\,\, \theta_{23}\/^{\circ}({\rm NO})=41.0\pm 7.25,\nonumber \\
&& \theta_{23}\/^{\circ}({\rm IO})=50.5\pm 7.25,\,\theta_{13}\/^{\circ}({\rm NO})=8.44\pm0.5,\nonumber\\
&&\theta_{13}\/^{\circ}({\rm IO})=8.44\pm 0.5,\,\delta_{D}/\pi({\rm NO})=1.40\pm 1.0,\nonumber\\
 &&\delta_{D}/\pi({\rm IO})=1.44\pm 1.0,\nonumber \\
 &&\Delta m_{21}^2=(7.56\pm0.545)\times 10^{-5}{\rm eV}^2, \nonumber\\
&&|\Delta m_{31}|^2({\rm NO})=(2.55\pm 0.12)\times 10^{-3}{\rm eV}^2,\nonumber\\
&&|\Delta m_{31}|^2({\rm IO})=(2.49\pm 0.12)\times 10^{-3}{\rm eV}^2.
\label{oscdata}
\end{eqnarray}
We denote the cosmologically constrained parameter, the sum of the
three active neutrino masses, 
\be
\Sigma_{\nu}={\sum}_{i=1}^3 {\hat m}_i \label{eq:sumnu}
\ee
whose cosmologically determined upper bound has been given in eq.(\ref{eq:cbound}).
For normally hierarchical (NH), inverted hierarchical (IH), and
quasi-degenerate (QD) patterns, the experimental data on  mass
squared differences have been fitted by the following  light
neutrino mass eigen values with the respective values of the cosmological
parameter $\Sigma_{\nu}$
\bea
{\hat m}_{\nu}&=& (0.00127, 0.008838 ,0.04978) ~{\rm eV}\,\, (\rm
{NH})\nonumber\\   
\Sigma_{\nu}&=&0.059888~{\rm eV},\nonumber\\ 
{\hat m}_{\nu}&=& ( 0.04901,0.04978,0.00127)~{\rm eV}\,\, (\rm {IH})\nonumber\\
\Sigma_{\nu}&=&0.059888~~{\rm eV},\nonumber\\ 
{\hat m}_{\nu}&=& (  0.2056,0.2058,0.2) ~{\rm eV}\,\, (\rm {QD}),\nonumber\\
\Sigma_{\nu}&=&0.6114  ~~{\rm eV}. 
\label{eq:mnusumnu}
\eea
The last line clearly shows violation of cosmological bound in the QD case.
Using oscillation data and best fit values of the mixings we have also determined the PMNS mixing matrix
numerically
\bea
U_{\rm {PMNS}}=\begin{pmatrix} 0.816&0.56&-0.0199-0.0142i\\
-0.354-0.0495i&0.675-0.0346i&0.650\\
0.450-0.0568i&-0.485-0.0395i&0.75\end{pmatrix}.\label{eq:numupmns} 
\eea
\subsection{Determination of Majorana Yukawa Coupling Matrix} 

Now inverting the relation ${\hat m}_\nu=U_{PMNS}^\dagger {\mathcal M}_\nu U_{PMNS}^*$
where ${\hat m}_\nu$ is the diagonalised neutrino mass matrix, we determine ${\mathcal M}_{\nu}$ for three different cases and further determine the corresponding values of the $f$ matrix using $f={\mathcal M}_\nu/v_L$
where we use the predicted value of $v_L=0.1$ eV.\\ 
%Noting that $M_N=fV_R={{\mathcal M}_{\nu}V_R}/ v_L$, we have also derived eigen
%values of the RH neutrino mass matrix  ${\hat M}_{N_i}$ as the
%positive square root of the $i^{th}$ eigen value of the Hermitian matrix
%$M_N^{\dag}M_N$.\\
\par\noindent{\bf NH}\\
\bea
f =
\begin{pmatrix} 0.117+0.022i & -0.124-0.003i  &   0.144+0.025i\\
-0.124-0.003i  & 0.158-0.014i  & -0.141+0.017i\\ 0.144+0.025i &-0.141+0.017i& 0.313-0.00029i \end{pmatrix}\label{fNH}
\eea
%\bea
% |{\hat M}_N| ={\rm diag}( 160, 894, 4870) ~{\rm GeV}.\,\ \label{MNNH}
%\eea
\vspace{0.2cm}
\noindent{\bf IH}\\
\bea
f =  
\begin{pmatrix} 0.390-0.017i & 0.099+0.01i  &  -0.16+0.05i\\
0.099+0.01i  & 0.379+0.02i  & 0.176+0.036i\\-0.16+0.05i &0.176+0.036i& 0.21-0.011i \end{pmatrix}  \label{fIH}
\eea
%\bea
% |{\hat M}_N| ={\rm diag}( 4880, 4910, 131) ~{\rm GeV}.\,\ \label{MNIH}
%\eea
\noindent{\bf QD}\\
\bea
 f =
\begin{pmatrix} 2.02+0.02i & 0.0011+0.02i  &  -0.019+0.3i\\
0.0011+0.02i  & 2.034+0.017i  & 0.021+0.21i\\-0.019+0.3i &0.021+0.21i& 1.99-0.04i \end{pmatrix} \label{fQD}
\eea
%For $v_L=0.1$ eV, we have 
%\bea
% |{\hat M}_N| ={\rm diag}( 21.46, 20.34, 18.87) ~{\rm TeV}\ \label{MNQDp1}
%\eea
%but for  $v_L=0.5$ eV, we obtain
%\bea
% |{\hat M}_N| ={\rm diag}( 4.3, 4.08, 3.77) ~{\rm TeV}.\ \label{MNQDp5}
%\eea



Randomly chosen Majorana phases \cite{scp:2018}
$\alpha_M=74.84^\circ,\beta_M=112.85^\circ$ and the central value of
the Dirac phase $\delta_D=218^\circ$ have been used in this analysis.
Using the well known definition of the Jarlskog invariant 
\be
J_{CP}=-t_{3}c_{3}t_{2}c_{2}^2t_{1}c_{1} \sin \delta_D, \label{eq:jcp}
\ee
and keeping $\delta_D$ at its best fit values
we have estmted the predicted allowed ranges of the CP-Violating parameter in both cases.
\bea
J_{CP}&=&0.0175-0.0212 \,\,(\rm NH)\, \nonumber\\
J_{CP}&=&0.0302-0.0365\,\,(\rm IH)\,\,\label{eq:numjcp}
\eea  
where the variables have been permitted to acquire values within their respective $3\sigma$ ranges of the oscillation data. Besides these there are non-unitarity contributions which have been discussed extensively in the literature.
%%%%%%%%%%%%%%%%%%%%%%%%%%%%%%%%%%%%%%%%%%%%%%%
%%%%%%%%%%%%%%%%%%%%%%%%%%%%%%%%%%%%%%%%%%%%%%%
\subsection{Scaling Transformation of Solutions}
In general there could be type-II seesaw models characterizing different seesaw scales and induced VEVs  matching the given set of neutrino oscillation data represented by the same neutrino mass matrix. For two such models
\bea
&&m_{\nu}=f^{(1)}v_L^{(1)} \nonumber\\
&&= f^{(2)}v_L^{(2)}, \label{eq:sc1}
\eea
Then the $f-$ matrix in one case is determined up to good approximation in terms of the other from the knowledge of the two seesaw scales\\
\be
f^{(1)} \simeq f^{(2)}\frac{M_{\Delta^{(1)}}}{M_{\Delta^{(2)}}}.\label{eq:sc2}
\ee
At $M_{\Delta^{(1)}}=10^{12}$ GeV our solutions are the same as in \cite{scp:2018}. 
In view of this scaling relation,  we can determine the values of the
Majorana Yukawa matrix in the present case from the estimations of
\cite{scp:2018}. For example, if we choose $M_{\Delta^{(1)}}=10^{10}$ GeV in the present case compared to
$M_{\Delta^{(2)}}=10^{12}$ GeV in \cite{scp:2018}, we rescale the solutions of
\cite{scp:2018} by a  factor $10^{-2}$ to derive solutions in the present case. 
Thus  graphical representations of solutions are similar to those of
ref.\cite{scp:2018} for $M_{\Delta^{(2)}}=10^{12}$ GeV which we do not
repeat here. The values of magnitudes of $f_{ij}$ at any new scale are
obtained by rescaling them by the appropriate scaling factor while the
phase angles remain the same as in \cite{scp:2018}.

\subsection{Dirac Neutrino Mass Matrix}
The Dirac neutrino mass matrix $M_D$ plays crucial role in predicting
LFV and LNV decays. In certain SO(10) models
\cite{babu-rnm:1993,Joshipura,bpn-mkp:2015,ap:2012} this is usually determined by
fitting the charged fermion masses at the GUT scale and equating it with the upquark mass matrix. The fact that $M_D^0\simeq M_{u}^0$ at the GUT scale follows from the underlying quark lepton symmetry \cite{JCP:1974} of SO(10). In SU(5) itself, however, there is no
such symmetry to dictate the structure of $M_D$ in terms of quark matrices. Also
in this SU(5) model we do not attempt any charged fermion mass fit at the GUT
scale or above it. Since the Dirac neutrino mass matrix is not predicted by the SU(5) symmetry itself, for the sake of simplicity and to derive maximal effects on LFV and LNV decays, we  assume $M^0_D$ to be equal to the up-quark mas
matrix $M_u^0$ at the GUT
scale. Noting that $N$ is SU(5) singlet fermion, in the context of
relevant Yukawa interaction Lagrangian 
\be
-{\cal L}_{\rm Yuk}= [Y_{N}{\overline 5}_F.{\bf 1}_F.{5}_H + Y_{u} {10}_F.{10}_F.{5}_H+....]+h.c.,\label{eq:su5yuk}
\ee
this assumption is equivalent to alighment of the two Yukawa couplings
\be
Y_{N}\simeq Y_{u} . \label{eq:eqyuk}
\ee 
This alignment is naturally predicted in SO(10) or
SO(18)\cite{Wilczek-Valle}, but in the present SU(5) case it is assumed.

We realise this matrix $M_D$ using renormalisation group equations for fermion masses and gauge couplings and their numerical solutions \cite{dp:2001} starting from the PDG values \cite{PDG:2012,PDG:2014,PDG:2016} of fermion masses at the electroweak scale. 
Following the bottom-up approach and using the down quark diagonal basis, the quark masses and the CKM mixings are 
extrapolated from low energies using renormalisation group (RG) equations
\cite{dp:2001,Stella:2016}. After assuming the approximate equality $M^0_D\simeq M^0_u$ at the GUT scale where $M^0_u$ is the up-quark mass matrix,  the top-down approach is
exploited to run down this mass matrix $M^0_D$ using  RG equations
\cite{dp:2001}. Then
   the value of $M_D$ near $1-10$ TeV scale turns out to be  
\bea
M_D \simeq\footnotesize\begin{pmatrix}
0.014&0.04-0.01i&0.109-0.3i\\
0.04+0.01i&0.35&2.6+0.0007i\\
0.1+0.3i&2.6-0.0007i&79.20
\end{pmatrix}{GeV}.  \label{eq:mdmu}
\eea

As already noted above, although on the basis of SU(5) symmetry alone there may not be any reason for the rigorous validity of eq.(\ref{eq:mdmu}), in what follows  we study the implications of this assumed value of $M_D$ to examine maximum possible impact on LFV and LNV decays discussed in Sec.\ref{sec:lfv} and Sec.\ref{sec:bb}. Another
reason is that the present assumption on $M_D$ may be justified in direct SO(10)
breaking to the SM which we plan to pursue in a future work \cite{mkp-rs:plan}.


\section{RIGHT-HANDED NEUTRINO  MASS IN SU(5) vs. SO(10)}\label{sec:rhnu}
\subsection{RH$\nu$ Mass in SO(10)}
The fermions responsible for type-I and type-II  seesaw are the  LH
leptonic doublets and the RH fermionic singlets of three generations.
 In SO(10) the left handed lepton doublet  $(\nu, l)^T$
and the right-handed neutrino $N$ are in the same representation ${16}_F$.
\bea
(\nu, l)^T & \subset & {16}_F,\nonumber\\
N& \subset & {16}_F.\label{eq:fermiso10}
\eea
In SO(10) GUT the Higgs representation ${126}^{\dagger}_H$ contains  both the left-handed and the right-handed triplets carrying $B-L=-2$,
\be
{126}^{\dagger}_H=\Delta_L(3,1,-2,1)+\Delta_R(1,3,-2,1)+.....\label{eq:126h}\\
\ee
where the quantum numbers are under the left-right symmetry group $SU(2)_L\times SU(2)_R \times U(1)_{B-L}\times SU(3)_C$. 
The common Yukawa coupling $f_{10}$ in the Yukawa term
\be
-{\cal L}_{\rm Yuk10}=  f_{10} {16}_F{16}_F{126}_H^{\dagger},\label{eq:Yuk10}
\ee
 generates the dilepton-Higgs triplet interactios both in the left-handed and right-handed sectors giving rise to type-I and type-II seesaw mechanisms.
The RH neutrino mass is generated through the VEV of the neutral component of the of $\Delta_R$
\be
M_N=f_{10} \langle \Delta_R^0 \rangle .\label{eq:MNso10}.
\ee
The type-II seesaw contribution to light  neutrino mass is 
\be
{\cal M}_{\nu}=f_{10} v_L  \label{eq:t2so10}
\ee
where $v_L$ is the corresponding induced VEV of $\Delta_L$
\be
v_L=\lambda_{10}\frac{\langle \Delta_R^0\rangle v^2_{ew}}{M_{\Delta_L}^2}. \label{eq:vlso10}
\ee
Here $\lambda_{10}$ is the quartic coupling in the part of the scalar
potential
\bea
 V_{10}&=&\lambda_{10}\Delta_L^{\dagger}\Delta_R \Phi \Phi \nonumber\\
&\subset& \lambda_{10}{126}_H^{\dagger}{126}_{H}{10}_H{10}_H.\label{eq:v10}
\eea 
Thus with type-II seesaw dominance, the predicted heavy RH neutrino masses in SO(10)  follow the same hierarchical pattern as the active light neutrino masses
\be
M_{N_1}:M_{N_2}:M_{N_3}::m_1:m_2:m_3 .\label{eq:rhmso10}
\ee
\subsection{RH$\nu$ Mass in SU(5)}

Feynman diagram for type-II seesaw mechanism in the present model is
shown in Fig.\ref{fig:su5mod2}
%%%%%%%%%%%%%%%%%%%%%%%%%%%%%%%%%%%%%%%%%%%%%%%%%%%%%%%%%%%%%%%%%%%%%%%%
\begin{figure}[htbp]
 \includegraphics[width=6cm,height=4.5cm,angle=0]{feyn28.pdf}
 \caption{Feynman diagram representing type-II seesaw mechanim for
   neutrino mass generation in SU(5). Scalar fields $\phi, \sigma_S$ and
   $\Delta_L$ represent SM Higgs doublet, singlet, and LH triplet as
   defined in the text. This diagram defines the trilinear coupling
   mass $\mu_{\Delta}=\lambda \langle \sigma_S \rangle$.}   
\label{fig:su5mod2}
\end{figure}
%%%%%%%%%%%%%%%%%%%%%%%%%%%%%%%%%%%%%%%%%%%%%%%%%%%%%%%%
In contrast to SO(10) where the LH leptonic doublet and the RH$\nu$ are
in the single representation ${16}_F$, in SU(5) they are in different fermionic
representations
\bea
(\nu, l)^T & \subset & {\overline 5}_F,\nonumber\\
N& \subset & {\bf 1}_F.\label{eq:fermiso10}
\eea
In SU(5), while the dilepton Higgs interaction is given by
\be
-{\cal L}_{Yukll}=f {\bar 5}_F{\bar 5}_F{15}_H,\label{eq:Yukll5}
\ee
the RH neutrino mass is  generated through
\be 
-{\cal L}_{YukNN}= (1/2)f_NN N \sigma_S +h.c.\label{eq:YukNN5}
\ee 
The fact that $N$ is a singlet under SU(5) forces $\sigma_S$ to be a singlet too . Further 
this singlet $\sigma_S$ must carry $B-L=-2$ as its VEV generates the
heavy  Majorana mass
\be 
M_N=f_N\langle \sigma_S \rangle .\label{eq:MN}
\ee
In sharp contrast to SO(10) where the LH triplet $\Delta_L$
and the RH triplet $\Delta_R$ scalars contained in the same representation
${126}_H^{\dagger}$ generate the type-II seesaw and $M_N$ the
situation in SU(5) is different. Since  LH
triplet $\Delta_L(3,-1,2)$ mediating type-II seesaw belongs to Higgs
representation ${15}_H\subset $ SU(5) and $\sigma_S$ belongs to a completely
different representation which is a singlet ${\bf 1}\subset$ SU(5), the
two relevant Majorana type couplings in general may not be equal
\be
f_N\neq f.               \label{eq:unequal}
\ee
Also this assertion is further strengthened if we do not assume SU(5)
to be a remnant of SO(10).

Then the RH neutrino mass hierarchy can be decoupled from  the type-II seesaw prediction.
It is interesting to note that in SU(5)
\be
v_L=\frac{\mu_{\Delta}v_{\rm ew}^2}{M_{\Delta}^2}
\ee
where $\mu_{\Delta}$ is the trilinear coupling in the potential term
\be
V_{tri}=\mu_{\Delta}\Delta_L \Phi \Phi +h.c. \label{eq:tlpot}
\ee
 The VEV of this singlet $\sigma_S$ can explain the dynamical origin of this trilinear coupling through its VEV $v_{\sigma}=\langle \sigma_S \rangle$ 
\be
\mu_{\Delta}=\lambda v_{\sigma}, \label{eq:model}
\ee
where $\lambda$ is the quartic coupling in the potential term
\bea
V_{ql}&=&\lambda \sigma_S.\Delta_L\Phi\Phi + h.c.\\
&\subset&\lambda \sigma_S. {15}_H{5}_H{5}_H+h.c \label{eq:qpot}
\eea
where the second line represents the SU(5) origin. For GUT-scale $U(1)_{B-L}$ symmetry breaking driving VEV $v_{\sigma}\simeq M_{GUT}$ in the absence of any intermediate symmetry, it is possible to ensure $\mu_{\Delta}\simeq M_{\Delta_L}$ for 
\be
\lambda \simeq \frac{M_{\Delta_L}}{M_{GUT}}. \label{eq:lamdelgut}
\ee
Thus the SU(5) model gives similar explanation for quartic coupling as
in direct breaking of SO(10).
But the predicted hierarchy of RH$\nu$ masses may not in general
follow the same hierarchical pattern as given by SO(10) as shown in eq.(\ref{eq:rhmso10}).
This is precisely because the eq.(\ref{eq:rhmso10}) follows because the
same diagonalisation matrix $U_{PMNS}$ diagonalises the LH and the RH
neutrino mass matrices which is further rooted in the fact that same
Majorana coupling $f_{10}$ that generates the type-II seesaw mass term
also generates $M_N$. But because of the general possibility $f_N\neq
f$, the RH$\nu$s may acquire a completely different pattern depending
upon the value of $f_N$. Unlike
SO(10), these masses are also allowed to be quite different from the
type-II seesaw scale.

Even if the value of $v_{\sigma}$
may be needed to be near $M_{\Delta_L}$, the value of $M_N$ is allowed
to be considerably lower by finetuning the value of $f_N$.
Our LFV and LNV decay phenomenology as discussed below may need $M_N=1-10$ TeV which is realizable using this new technique in SU(5). In contrast SO(10) needs
$U(1)_R\times U(1)_{B-L}$ or $SU(2)_R\times U(1)_{B-L}$ gauge symmetry
and hence new gauge bosons  near the TeV scale to generate such RH$\nu$
masses which should be detected at LHC \cite{mkp-bs:2015,bpn-mkp:2015}.   
Thus a new mechanism for RH$\nu$ mass emerges here by noting the
coupling $f_N\neq f$ which has the potential to generate RH$\nu$
masses in the range $100-10^{15}$ GeV. Thus the RH$\nu$ mass
predictions in the two GUTs in the presence of type-II seesaw
dominance are
\par\noindent{\bf{\underline {Type-II Seesaw Dominated SO(10)}:-}}\\
\be
M_{N_i}\simeq \frac{m_i M^2_{\Delta_L}}{ v_{ew}^2}. \label{eq:mnso10}
\ee
\par\noindent{\bf{\underline{Type-II Seesaw Dominated SU(5)}:-}}\\
\be
M_{N_i}=\left[{\cal O}(10){\rm GeV} - {\cal
    O}(M_{\Delta_L})\right].\label{eq:mnsu5}
\ee 
Here $m_i,i=1,2,3$ represents the three mass eigen values of light
neutrinos. It is to be noted that $m_i$ is absent in the RHS of
eq.(\ref{eq:mnsu5}) in the SU(5) case.
\section{\bf LEPTON FLAVOR VIOLATIONS}\label{sec:lfv}
In the SM extensions there has been extensive investigation of lepton
flavor violating phenemena $l_{\alpha} \to l_{\beta}+\gamma$ and other
processes like $\mu \to e{\bar e}e$  including
unitarity violations \cite{ilakovac,antbnd,non-unit}. In the flavor
basis we use the
standard charged current Lagrangian  
\begin{eqnarray}
\mathcal{L}_{\rm CC} &=& -\frac{1}{\sqrt{2}}\, \sum_{\alpha=e, \mu, \tau}
[g_{2L} \overline{\ell}_{\alpha \,L}\, \gamma_\mu {\nu}_{\alpha \,L}\, W^{\mu}_L] 
       + \text{h.c.} 
\label{eqn:ccint-flavor}
\end{eqnarray}

In  predicting the LFV
branching ratios  we have used the relevant formulas of
\cite{ilakovac}  while assuming a simplifying diagonal structure for
M,
\be
M = {\rm diag.}~(M_1, M_2, M_3),\label{eq:M}
\ee
which, in comdination with eq.{\ref{eq:mdmu}), gives the elements of the
  $\nu-S$ mixing matrix 
\bea
{\cal V}^{(lS)}=
\begin{pmatrix}
{M_D}_{e1}/M_1& {M_D}_{e2}/M_2&{M_D}_{e3}/M_3\\ 
{M_D}_{\mu 1}/M_1& {M_D}_{\mu 2}/M_2&{M_D}_{\mu 3}/M_3\\ 
{M_D}_{\tau 1}/M_1& {M_D}_{\tau 2}/M_2&{M_D}_{\tau 3}/M_3
\end{pmatrix}.
\label{HLMIX}
\eea
The $S-N$ mixing matrix 
\be
V^{(SN)}=\frac{M}{M_N}. \label{eq:mixsn} 
\ee

Noting that the physical neutrino flavor state $\nu_{\alpha}$ is a mixture of ${\hat \nu}$, ${\hat S}$ and ${\hat N}$
\be
 \nu_{\alpha}=U_{\alpha i}{\hat \nu}_i+V^{lS}_{\alpha i}{\hat S}
+V^{(SN)}_{\alpha i}{\hat N_i}.\label{eq:nusnmix}
\ee
where $U\sim U_{PMNS}$ and the other two mixings violate unitarity. For large $M_N\gg M$ the third term in the RHS of eq.(\ref{eq:nusnmix}} can be dropped leading to the unitarity violation parameter $\eta$
\be
U^{\prime}\simeq(1-\eta)U_{PMNS}.\label{eq:nonuni}
\ee
where
\bea
\eta_{\alpha\beta}&=&(1/2){(X.X^{\dagger})}_{\alpha\beta},\nonumber\\
X&=&\frac{M_D}{M}. \label{eq:X}
\eea
There has been extensive discussion on the constraint imposed on this
parameter
\cite{antbnd,non-unit}. The largest out of these is $\eta_{\tau \tau}\le 0.0027$. Theoretically
\be
\frac{1}{2}\left[\sum_i \frac{{M_D}_{\tau i}.{M_D}_{\tau i}^*}{M_i^2}\right]\le 0.0027.\label{eq:etanum}
\ee
In the completely degenerate case of $S-N$ mixing, $M_1=M_2=M_3=M$ we get
\be  
M\ge 1280 {\rm GeV} \label{eq:Mdeg}
\ee
%%%%%%%%%%%%%%%%%%%%%%%%%%%%%%%%%%%%%%%%%%%%%%%%%%%%%%%%%%%%%%%%%%


 The  RH neutrinos in the present  model being degenerate of masses
 $M_{N_i}\gg m_{S_i}$ have much less significant contributtions than the singlet fermions. The predicted branching ratios 
being only few to four orders less than the current experimental
limits \cite{lfvexpt} are verifiable by ongoing searches,

\bea
BR(\mu \to e\gamma)&=&5.90\times 10^{-17},\nonumber\\
BR(\tau \to e\gamma)&=&7.80\times 10^{-16},\nonumber\\
BR(\tau \to \mu\gamma)&=&2.39\times 10^{-12}.\label{lfvbr}
\eea 
For the sake of comoleteness we present the variation of LFV decay branching ratios as a function of the lightest neutrino mass in Fig. \ref{lfvbrs}. \\     
%%%%%%%%%%%%%%%%%%%%%%%%%%%%%%%%%%%%%%%%%%%%%%%%%%%%%%%%%%%%%%%%%%%%
  \begin{figure}[htbp]
 \includegraphics[width=10cm,height=8cm,angle=0]{lfvt2.pdf}
 \caption{Variation of LFV decay branching ratios as a funtion of the
   lightest neutrino mass. Colored horizontal lines represent 
${\rm I}:BR(\mu \to e\gamma)$, ${\rm II}:BR(\tau \to e\gamma)$, and ${\rm III}:BR(\tau \to \mu\gamma)$. }   
\label{lfvbrs}
\end{figure}
%%%%%%%%%%%%%%%%%%%%%%%%%%%%%%%%%%%%%%%%%%%%%%%%%%%%%%%%%%%%%%%%%%%%%

 In this approach the LFV decay rate
mediated by the $W_L$ boson in the loop depends predominantly upon $N-S$ mixing matrix $M$
and the Dirac neutrino mass matrix $M_D$, although subdominantly upon
the RH$\nu$ mass matrix $M_N$. However in the high scale type-II seesaw
 ansatz in this case
 LFV decay rate is independent of  light neutrino masses. This behavior
 of LFV decay rates are clearly exhibited in Fig.\ref{lfvbrs} where
 the three branching ratios have maintained constancy with the
 variation of $m_{\nu}$.


%%%%%%%%%%%%%%%%%%%%%%%%%%%%%%%%%%%%%%%%%%%%%%%%%%%%%%%%%%%%%%

\section{\bf DOMINANT $W_L-W_L$-CHANNEL DOUBLE BETA DECAY WITHIN COSMOLOGICAL BOUND}\label{sec:bb}
\subsection{Double Beta Decay Mediation by Sterle Neutrinos}  

As the $W_R$ boson has mass $> 10^{15}$ GeV   and the doubly charged Higgs bosons
have masses $> 10^{9.2}$ GeV, they have negligible contributions for direct mediations of $0\nu\beta\beta$ process.
 Feynman diagrams for
$0\nu\beta\beta$ decay amplitude due to the exchanges of Majorana
fermions $\nu$, $S$, and $N$ are shown in Fig. \ref{fig:feynall}.
%%%%%%%%%%%%%%%%%%%%%%%%%%%%%%%%%%%%%%%%%%%%%%%%%%%%%%%%%%%%%%%%%%%%%%%%%%
\begin{figure}[htbp]
 \includegraphics[width=15cm,height=5cm,angle=0]{feyn_allLL.pdf}
%{feyn_allLL.pdf}
 \caption{Feynman diagrams representing neutrino-less double beta
   decay due to
exchanges of all three types of  Majorana frmions $\nu$,$S$ and $N$.}   
\label{fig:feynall}
\end{figure}
%%%%%%%%%%%%%%%%%%%%%%%%%%%%%%%%%%%%%%%%%%%%%%%%%%%%%%%%%%%%%%%%%%%%%%%%%%
In Fig.\ref{fig:feyns} we present  Feynman diagram for
$0\nu\beta\beta$ decay amplitude due to the sterile neutrino  exchange 
where its mass insertion has been explicitly indicated.
%%%%%%%%%%%%%%%%%%%%%%%%%%%%%%%%%%%%%%%%%%%%%%%%%%%%%%%%%%%%%%%%%%%%%%%%
\begin{figure}[htbp]
 \includegraphics[width=6cm,height=4.5cm,angle=0]{sterile1c.pdf}
 \caption{Feynman diagram representing neutrino-less double beta
   decay amplitude due to
exchanges of sterile neutrino $S$ with explicit mass isertion $m_s=-M\frac{1}{M_N}M^T$.}   
\label{fig:feyns}
\end{figure}
%%%%%%%%%%%%%%%%%%%%%%%%%%%%%%%%%%%%%%%%%%%%%%%%%%%%%%%%
Mass eigen values of different sterile neutrinos for different sets of 
 $(M_1,M_2,M_3)$ consistent  with constraints on unitarity violating parameters 
$\eta_{\alpha \beta}$ are presented in Table \ref{tab:msMi}. We have
used the singlet fermion  mass seesaw formula of eq.(\ref{eq:mass})
and $M_{N_1}=M_{N_2}=M_{N_3}= 4000$ GeV. 
%%%%%%%%%%%%%%%%%%%%%%%%%%%%%%%%%%%%%%%%%%%%%%%%%%%%%%%%%%%
\begin{table}[!h]
\caption{Prediction of singlet fermion masses for different values of
  $(M_1,M_2,M_3)$ where we have used  $M_{N_1}=M_{N_2}=M_{N_3}= 4000$ GeV. }
\begin{center}
\begin{tabular}{|c|c|}
\hline
{ $M$ (GeV)} & {$\hat{m}_s$ (GeV)} \\
\hline
$(60,1200,1200)$ & $(0.9,360,360)$ \\
\hline
$(70,1200,1200)$ & $(1.22,360,360)$ \\
\hline
$(80,1200,1200)$ & $(1.60,360,360)$ \\
\hline
$(90,1200,1200)$ & $(2.00,360,360)$ \\
\hline
$(100,1200,1200)$ & $(2.50,360,360)$ \\
\hline
$(110,1200,1200)$ & $(3.00,360,360)$ \\
\hline
$(120,1200,1200)$ & $(3.60,360,360)$ \\
\hline
$(130,1200,1200)$ & $(4.22,360,360)$ \\
\hline
$(140,1200,1200)$ & $(4.90,360,360)$ \\
\hline
$(150,1200,1200)$ & $(5.62,360,360)$ \\
\hline
\end{tabular}
\end{center}
\label{tab:msMi}
\end{table}
%%%%%%%%%%%%%%%%%%%%%%%%%%%%%%%%%%%%%%%%%%%%%%%%%%%%%%%%%%%%%
These solutions are displayed in Fig.\ref{fig:msM}. 
%%%%%%%%%%%%%%%%%%%%%%%%%%%%%%%%%%%%%%%%%%%%%%%%%%%%%%%%%%%%
\begin{figure}[htbp]
 \includegraphics[width=9cm,height=12cm,angle=-90]{ms123.pdf}
 \caption{Prediction of singlet fermion mass eigen values as a function of $N-S$ mixing mass parameters $M_i(i=1,2,3)$ for $M_{N_i}= 4\, {\rm TeV} (i=1,2,3)$. The horizontal red coloured line represents solutions for two other eigen values for $M_2=M_3=1200$ GeV.} 
 \label{fig:msM}
\end{figure}
%%%%%%%%%%%%%%%%%%%%%%%%%%%%%%%%%%%%%%%%%%%%%%%%%%%%%%%%%%%%

We use normalisations necassary for different contributions \cite{Vergados,Doi},  due to exchanges of
 light-neutrinos, sterile neutrinos, and the heavy
RH neutrinos  in the $W_L-W_L$ channel. They 
lead to the inverse half life \cite{app:2013,pas:2014,bpn-mkp:2015},   
\begin{eqnarray}
  \left[T_{1/2}^{0\nu}\right]^{-1} &\simeq &
  G_{01}|\frac{{\cal M}^{0\nu}_\nu}{m_e}|^2|({\large\bf
    M}^{ee}_{\nu} +{\large\bf M}^{ee}_{S}+{\large\bf M}^{ee}_{N})|^2,\nonumber\\
&=& K_{0\nu}|({\large\bf
    M}^{ee}_{\nu} +{\large\bf M}^{ee}_{S}+{\large\bf M}^{ee}_{N})|^2,\nonumber\\
&=& K_{0\nu}|{\large\bf
    M}_{\rm eff} |^2.
  \label{invhalf}
\end{eqnarray}
Here  $G_{01}= 0.686\times 10^{-14} {\rm yrs}^{-1}$, ${\cal
  M}^{0\nu}_{\nu} = 2.58-6.64$, and $K_{0\nu}= 1.57\times 10^{-25} {\rm yrs}^{-1}
{\rm eV}^{-2}$. In eq.(\ref{invhalf}) the three effective mass parametes  have been defined as 
\begin{eqnarray}
{\large \bf  M}^{\rm ee}_{\nu} =\sum^{}_{i} \left(\mathcal{V}^{\nu \nu}_{e\,i}\right)^2\, {m_{\nu_i}}
\label{effmassparanus} 
\end{eqnarray}
\bea
{\large \bf  M}^{\rm ee}_{S} = \sum^{}_{i} \left(\mathcal{V}^{\nu
  S}_{e\,i}\right)^2\, \frac{|p|^2}{{\hat m}_{S_i}} 
  \label{effmassparanus2} 
\eea
\begin{eqnarray}
{\large \bf  M}^{\rm ee}_{N} = \sum^{}_{i} \left(\mathcal{V}^{\nu
  N}_{e\,i}\right)^2\, \frac{|p|^2}{M_{N_i}},
\label{effmassparanus3} 
\end{eqnarray}
with 
\begin{eqnarray}
&&{\large\bf M}_{\rm eff}={\large\bf M}^{ee}_{\nu} +{\large\bf
    M}^{ee}_{S}+{\large\bf M}^{ee}_{N}.\label{sumeff}.
\end{eqnarray}

The quantity ${\hat m}_{S_i}$ is the i-th eigen value of the  $S-$ fermion mass matrix
$m_S$. The magnitude of neutrino virtuality momentum $|p|$ has been estimated to be in the allowed range 
$|p|= 120$ MeV$-200$ MeV \cite{Vergados,Doi}.
The RH$\nu$s being much heavier than the singlet fermions, their
contributions have been neglected. 

\subsection{Singlet Fermion Assisted Enhanced Double Beta Decay Rate} 
We use neutrino oscillation date to estimate $M_{\nu}^{\rm ee}$ for NH and IH cases
with the values of Dirac phase and Majorana phases as discussed above. We further use the values of $M_i$ from Table \ref{tab:msMi} and Fig. \ref{fig:msM} and the Dirac neutrino mass matrix from eq.(\ref{eq:mdmu} to estimate $M_S^{\rm ee}$ 
while treating the RH$\nu$ mass at its assumed degenerate value of $M_{N_i}=4 {\rm TeV} (i=1,2,3)$. 
The variation of effective parameter $m_{\rm ee}$ as a function of
lightest neutrino mass is shown in Fig.\ref{fig:eff2} when $m_{s_1}=2$ GeV.
%%%%%%%%%%%%%%%%%%%%%%%%%%%%%%%%%%%%%%%%%%%%%%%%%%%%%%%%%%%%
\begin{figure}[htbp]
 \includegraphics[width=9cm,height=9cm,angle=-90]{bbeff2.pdf}
 \caption{Variation of effective mass parameter as a function of
   lightest active neutrino mass $m_1$ for $m_{s_1}=2$ GeV. For
   comparison, predictions in the standard model suplementd by light neutrino masses
   of NH type is shown by green dot-dashed curve. For IH pattern of
   mass hierarchy the standard prediction is shown by red dot-dashed
   curve.}   
\label{fig:eff2}
\end{figure}
%%%%%%%%%%%%%%%%%%%%%%%%%%%%%%%%%%%%%%%%%%%%%%%%%%%%%%%%%%%

As noted from the analytic formulas the effective mass parameter in
the singlet femion dominated case being inversely proportional to
$m_{s_1}$, it will proportionately decrease for the larger value of
the mediating particle mass. This feature has been shown in Fig.\ref{fig:eff4}.
%%%%%%%%%%%%%%%%%%%%%%%%%%%%%%%%%%%%%%%%%%%%%%%%%%%%%%%%%%%%
\begin{figure}[htbp]
 \includegraphics[width=7cm,height=8cm,angle=-90]{bbeff4.pdf}
 \caption{Same as Fig.\ref{fig:eff2} but for for $m_{s_1}=4.0$ GeV. }   
\label{fig:eff4}
\end{figure}
%%%%%%%%%%%%%%%%%%%%%%%%%%%%%%%%%%%%%%%%%%%%%%%%%%%%%%%
We present predictions of double-beta decay half life as a function of
the singlet fermion mass in Fig.\ref{fig:bbls1}. It is clear that while for
$m_{s_1}=2$ GeV the half life saturates the current expeerimental
limit, for larger masses the halftife increases.
%%%%%%%%%%%%%%%%%%%%%%%%%%%%%%%%%%%%%%%%%%%%%%%%%%%%%%%
\begin{figure}[htbp]
 \includegraphics[width=9cm,height=12cm,angle=-90]{bbhnh.pdf}
 \caption{Prediction of double beta decay half-life 
 as a function of
   sterile neutrino mass  $m_{s_1}$ GeV (blue shaded curve) where the NH type  light
   neutrino and the sterile neutrino exchange contributions
   have been included. Effects of much larger masses $(m_{S_2}, m_{S_3})\gg
   m_{S_1}$ have been neglected. The spread in the curve reflects uncertainty in the virtuality momentum $p=120-190$ MeV. For comparison the standard
   prediction with NH and IH pattern of light neutrino mass
   hierarchies are shown by the two respective horizontal lines.The
   bottom most thick red horizontal line closest to the X-axis represents 
    overlapping experimental bounds from different groups
   \cite{bbexpt1-Klapdoor, bbexpt2,bbexpt3}.  }   
\label{fig:bbls1}
\end{figure}
%%%%%%%%%%%%%%%%%%%%%%%%%%%%%%%%%%%%%%%%%%%%%%%%%%%%%%% 
Neglecting heavy RH$\nu$ contribitions but including those due to the lightest
sterile neutrino and the IH type light neutrinos our predictions of
half life as a function of the lightest sterile neutrino mass is shown
in Fig. \ref{fig:bbIH}  

%%%%%%%%%%%%%%%%%%%%%%%%%%%%%%%%%%%%%%%%%%%%%%%%%%%%%%
\begin{figure}[htbp]
 \includegraphics[width=9cm,height=12cm,angle=-90]{bbhin.pdf}
 \caption{Same as Fig.\ref{fig:bbls1} but with contributions of IH type light neutrinos cambined with lightest sterile neutrinos.}
\label{fig:bbIH}
\end{figure}
%%%%%%%%%%%%%%%%%%%%%%%%%%%%%%%%%%%%%%%%%%%%%%%%%%




Predicted lifetimes are seen to decrease with increasing sterile
neutrino mass. The sterile neutrino exchange contribution completely
dominates over light neutrino
exchange contributions for 
$m_{S_1}=1.3-7$ GeV in case of IH but for $m_{S_1}=1.5-20$ GeV in case of NH.  
At $m_{S_1}\simeq 1.5 GeV$ both types of solutions saturate the
current laboratory limits reached by deifferent experimental
groups . For double beta decay half-life expectations in standard and
non-standard scenarios see  ref.\cite{Rodejohann}. 

  




%%%%%%%%%%%%%%%%%%%%%%%%%%%%%%%%%%%%%%%%%%%%%%%%%%%%%%%%%%%%%%%%%%%%%%%
\section{SUMMARY, DISCUSSION AND CONCLUSION}\label{sec:sum}
A recently proposed scalar extension of minimal non-SUSY SU(5) GUT
has been found to realize  precision gauge coupling unification, high
scale type-II seesaw ansatz for neutrino masses, and prediction of  a
WIMP scalar DM candidate that also completes vacuum stability of the
scalar potential. But the LFV decays are predicted to have negligible
rates inaccessible to ongoing searches in foreseeable future. Like
wise experimentally verifiable double beta decay rates measurable by
different search experiments are possible only for quasi-degenerate
neutrino mass spectrum with large common mass scale $|m_0| \ge 0.2$
that violates the recently measured cosmological bound $\sum_i m_i \le
0.1$ eV. In order to remove these deficits we have extended this model
by the addiion of three RH$\nu$s, three extra Majorana fermion
singlets $S_i(i=1,2,3)$ and a scalar singlet $\xi_S(1,0,1)$ that
generates $N-S$ mixing mass term through its vacuum expectation
value. In the original thory  of type-I seesaw cancellation mechanism,
although the choice of particles is same as $N_i$, $S_i$ and
$\xi_S(1,0,1)$,  the neutrino mass is given by double seesaw
\cite{Kim-Kang:2006}. Further there is no grand unification of gauge couplings or prediction of proton decay in this model \cite{Kim-Kang:2006}, and the scalar potential of the model has vacuum instability . In addition the  
 $N_i$ are not gauged. The model does not  predict dominant contributions to double beta decay through this mechanism with NH or IH type 
neutrin masses. In non-SUSY  SO(10) models of unification of three forces,
 implementing  the cancellation of type-I seesaw \cite{mkp-bs:2015,app:2013,pas:2014}, the TeV scale RH neutrinos are gauged but the neutrino masses are controlled by inverse seesaw. But in \cite{nurev:mkpbpn,bpn-mkp:2015} the RH$\nu$s are gauged and the neutrino mass formula is linear seesaw or type-II seesaw \cite{nurev:mkpbpn}.  In all type-II seesaw
dominated  SO(10) models, the  RH$\nu$ masses have the same hierarchy
as the left-handed neutrino
masses:$M_{N_1}:M_{N_2}:M_{N_3}::m_1:m_2:m_3$. This happens precisely
because the left-handed and the right-handed dilepton Yukawa
interactions originate from the same SO(10) invariant term:
$f{16}_F{16}_F {126}^{\dagger}$. In SU(5), however, as the LH triplet $\Delta_L(3,-1,1)$
generating type-II seesaw and the  singlet $\sigma_S(1,0,1)$
generating RH$\nu$s  belong to different scalar representations
${15}_H\subset SU(5)$ and ${\bf 1}_H\subset SU(5)$, respectively, they
can  possess different Majorana couplings in their respecive Yukawa interactions:$fll{\Delta_L}^C$ and
$f_N\sigma_S N N$. Because of this reason the the generated RH$\nu$
masses through $M_N=f_N\langle \sigma_s \rangle$ no longer follows the
predicted type-II hierarchical pattern. Then the allowed fine tuning
$|f_N|<< |f|$ permits $M_N\sim {\cal O}(1-10)$ TeV  RH neutrino mass scale even though,
unlike SO(10) models, there are no low mss $W_R$ or $Z^{\prime}$ bosons at this scale in this SU(5) model. The apprehension of unacceptably large active neutrino mass generation through type-I seesaw mechanism is rendered inoperative
through  the well established procedure of cancellation mechanism that is
also shown to operate profoundly in this SU(5) model. Such RH$\nu$s  generating $N-S$  mixing mass $M\simeq {\cal O} (100-1000)$ GeV now reproduce the well known results on
 LFV decay branching ratios only $4-5$ orders lower than the current
 experimental limit as well as the extensively investigated
 non-unitarity effects. Through the sterile neutrino canonical seesaw
 formula emerging from this cancellation mechanism (in the presence of
 $N_i$),  $m_S=-M\frac{1}{M_N}M^T$, this mechanism predicts their masses over a wide range of values, $m_{s_1}={\cal O} (1-100)$ GeV and $m_{s_2},m_{s_3} \sim {\cal O}(10-1000)$ GeV. The lightest sterile neutrino mass $m_{s_1}$ now predicts dominant double beta decay in the $W_L-W_L$ channel through the $\nu-S$ mixing close to the current experimental limits even though the light neutrino masses are of NH or IH type ($m_i << |0.2|$ eV) which satisfy the cosmological bound. For larger values of $m_{S_1}$ the predicted decay rate decreases  
and the sterile neutrino contribution becomes negligible for $m_{s_i}>> 50$ GeV.
In the limiting case when all the singlet fermion masses have such large values, the double beta decay rates asymptotically  approach the respective standard NH or IH type  contributions.   
The new mechanism of RH$\nu$ mass generation also allows the second
and the third generation sterile neutrino masses to be
quasi-degenerate (QD) near $1-10$ TeV scale while keeping $m_{S_1}\sim
1-10$ GeV suitable for dominant double beta decay mediation. There is a possibility that such TeV
scale QD masses while maintaining observable predictions on LFV decays  can effectively generate baryon asymetry of the universe
via resonant leptogenesis \cite{bpn-mkp:2015}.
Vacuum stability of the scalar potential can be implemented though the scalar singlet
$\xi_S(1,0,1)$ following the method of
\cite{falkowski:2014,spc:2017}. A scalar singlet DM can be easily
accommodated as discussed in \cite{scp:2018}. Irrespective of scalar DM,
the model can also  accommodate a Majorana fermion singlet dark matter \cite{strumia:2006} which can emerge from the additional fermionic representation ${24}_F\subset SU(5)$ .         

The predictions of new fermions has an additional advantage over scalars as 
these masses are protected by leptonic global symmetries \cite{tHooft}.
The predictions of such Majorana type sterile neutrinos  can be tested
by high enegy and high luminousity accelerators through their
like-sign dilepton production processes  \cite{bpn-mkp:arx}. For example at LHC they can mediate the process $pp\to W_L X \to l^{\pm}l^{\pm}jjX$ where the jets could manifest as mesons. It would be quite interesting to examine emergence of such SU(5) theory as a remnant of SO(10) or $E_6$ GUTs. 

We conclude that even in the presence of SM as effective gauge theory
  descending from a suitable SU(5) extension, it is possible to predict
  experimentally accessible double beta decay rates in the $W_L-W_L$ channel satisfying the
  cosmological bound on active neutrino masses as well as verfiable
  LFV decays. The RH$\nu$ masses can be considerably
  different from those constrained by conventional type-I or type-II
  seesaw frameworks which are instrumental in predicting interesting physical phenomena even
  if there are no non-standard heavy gauge bosons anywhere below the 
GUT scale. 
 

            
\section {\bf ACKNOWLEDGMENT }
M. K. P. thanks the Science and Engineering Research Board, Department
of Science and Technology, Government of India for grant of research
project SB/S2/HEP-011/2013. R.S. thanks  Siksha 'O' Anusandhan 
University for research fellowship.\\


\begin{thebibliography}{99}
\bibitem{Salas:2017}
P.F. de Salas (Valencia U., IFIC), D.V. Forero, C.A. Ternes,
M. Tortola, J.W.F. Valle, ``{\em Status of Neutrino Oscillations 2018: First Hint for Normal Ordering and Improved CP Sensitivity }'',e-Print: arXiv:1708.01186v2[hep-ph][INSPIRE]. 


\bibitem{schwetz} T. Schwetz, M. Tartola, J. W. F. Valle, ``{\em Global neutrino data and recent reactor fluxes:status of three-flavour oscillation parameters}'',New J. Phys. 13 (2011) 063004[arXiv:1103.0734] [INSPIRE].

\bibitem{forero} D. V. Forero, M. Tartola, J. W. F. Valle, ``{\em Neutrino oscillations refitted }'',Phys. Rev. D 90(2014) 093006 [arXiv:1405.7540] [INSPIRE].

\bibitem{fogli}  G. L. Fogli, E. Lisi, A. Marrone, A. Palazzo, A. M. Rotunno, ``{\em Global analysis of neutrino masses, mixings and phases: entering the era of leptonic CP-violation searches }'', Phys. Rev. {\bf D 86} (2012) 013012[arXiv:1205.5254] [INSPIRE].

\bibitem{gonzalez} M.Gonzalez-Garcia, M. Maltoni, T. Schwetz, ``{\em Global Analyses of Neutrino Oscillation Experiments }'', Nucl. Phys. B 908 (2016) 199[arXiv:1512.06856][INSPIRE].
%\bibitem{nurev} 
%\bibitem{gs:rev:2011}
\bibitem{Alta:2014}
  G. Altarelli, ``{\em Neutrinos Today: An
   Introduction}'', in Proceedings: {\em 49th Recontres de Moriond on
 Electroweak Interactons and Unified Theories}, Thuile, Italy, March
 15-22, (2014);
\bibitem{AYS:2014} A. Yu. Smirnov, ``{\em Theories of Neutrino Masses and Mixings}'',
Nuovo Cim.\ C {\bf 037}, no. 03, 29 (2014)
%  doi:10.1393/ncc/i2014-11761-y
  [arXiv:1402.6580 [hep-ph]];
%Nuovo Cim. {\bf C 037}
%  (2014) no.3, 29-37;
\bibitem{RNM:2015}
R. N. Mohapatra, ``{\em From Old Symmetries to New Symmet%ries:
    Quarks, Leptons, and B-L}'', in ``50 Years of Quarks'' pp 245-263
  (World Scientific, 2015);
\bibitem{RNM:2016} 
R. N. Mohapatra, ``{\em Neutrino Mass as a
   Signal for TeV Scale Physics}'',  Nucl. Phys. {\bf B 908} (2016)
  423-435;
%\bibitem{AYS:rev:2014}
\bibitem{Valle:2016} 
O. G. Miranda, J. W. F. Valle, ``{\em Neutrino Oscillation and
   Seesaw Origin of Neutrino Masses}'',
Nucl.\ Phys.\ B {\bf 908}, 436 (2016)
%  doi:10.1016/j.nuclphysb.2016.03.027
  [arXiv:1602.00864 [hep-ph]];  
%   Nucl. Phys. {\bf B 908} {2016}
%436-455;
\bibitem{Valle:2017} J. W. F. Valle, ``{\em Neutrino Physics from A to Z : Two Lectures at Corfu  }'', PoS CORFU2016 (2017) 007 arXiiv:1705.00872.    
\bibitem{gs:2011} G. Senjanovic, ``{\em Neutrino mass: From LHC to
  grand unification}'', Riv. Nuovo Cim. {\bf 34}
  (2011) 1-68. 
\bibitem{nurev:mkpbpn} M. K. Parida, B. P. Nayak, ``{\em Singlet
  fermion assisted dominant seesaw with lepton flavor and number
  violations and leptogenesis }'', Adv. High Energy Phys. {\bf 2017}
  (2017) {\bf 4023493} arXiv:1607.07236[hep-ph].
%%%%%%%%%%%%%%%%%%%%%%%%%%%%%%%%%%%%%%%%%%%%%%%%%%%%%%%%%%%%%%%%%%  
%%%%%%%%%%%%%%%%%%%%%%%%%%%%%%%%%%%%%%%%%%%%%%%%%%%%%%%%%%%%%%%%%%
\bibitem{Minkowski:1977} P.~Minkowski,
 ``{\em {mu $\to$ e gamma at a Rate of One Out of 1-Billion Muon  Decays?}}'',
 Phys. Lett. {\bf B 67}, (1977)  421.
%%CITATION = PHLTA,B67,421;%%
\bibitem{Yanagida:1979}
T.~Yanagida, in {\em Workshop on Unified Theories, KEK Report
79-18}, p.~95, 1979.
\bibitem{Gell-Mann:1979}
M.~Gell-Mann, P.~Ramond and R.~Slansky, {\em Supergravity},
p.~315,
\newblock Amsterdam: North Holland, 1979.
\bibitem{Glashow:1979}
S.~L. Glashow, {\em 1979 Cargese Summer Institute on Quarks and
Leptons},
 p.~687,
\newblock New York: Plenum, (1980).
\bibitem{RNM-gs:1980} R.~N.~Mohapatra and G.~Senjanovic,
  ``{\em Neutrino Mass and Spontaneous Parity Violation},''
  Phys.\ Rev.\ Lett.\  {\bf 44}, (1980)  912.
\bibitem{Valle:1980} J. Schechter, J. W. F. Valle, ``{\em Neutrino Masses in $SU(2)\times U(1)$ Theories }'', Phys. Rev. {\bf D\, 22},  (1980) 2227.
\bibitem{Valle:t2}  J. Schechter, J.W. F.  Valle, ``{\em Neutrinoless double beta decay in
  $SU(2)\times U(1)_Y$ theories}'', Phys. Rev. {\bf D 25} (1982) 2951;
\bibitem{Magg:t2} M. Magg, C. Wetterich, Phys. Lett. {\bf B 94},  (1980) 61 .
\bibitem{Lazaridis:t2} G. Lazaridis, Q. Shafi, C. Wetterich,
 Nucl. Phys. {\bf B 181}, (1981) 287.
\bibitem{RNM-gs:t2} R. N. Mohapatra and G. Senjanovic,
  Phys. Rev. {\bf D 23}, (1981)  165.
\bibitem{Ma-Us:t2}  E. Ma, U. Sarkar, ``{\em Neutrino Masses and
  Leptogenesis with Heavy Triplets}'', Phys. Rev. Lett. {\bf 80}
  (1998) 5716, arXiv:hep-ph/9802445. 
\bibitem{RNM-mkp:t2} R. N. Mohapatra, M. K. Parida,``{\em Type-II seesaw dominance in non-supersymmetric SO(10) grand unification}'',Phys. Rev. {\bf D  84} (2011) 095021,[arXiv:1109.2188].    

\bibitem{inv1} R. N. Mohapatra, Phys. Rev. Lett.{\bf 56}
  (1986) 561.
\bibitem{inv2}R.N. Mohapatra,J. W. F. Valle, Phys. Rev. {\bf D 34}
  (1986) 1642.
\bibitem{inv3}  D. Wyler, L. Wolfenstein, `` {\em Massless neutrinos
  in left-right symmetric models}'', Nucl. Phys. {\bf B 218} (1983)
  205.

\bibitem{inv4}C. H. Albright, `` {\em Search for solutions of superstring
    neutrino mass problem }'',  Phys. Lett. {\bf B 178} (1986) 219;  

\bibitem{inv5}  S. Nandi, U. Sarkar, `` {\em A solution to neutrino mass problem in
  superstring $E_6$ theory }'', Phys. Rev. Lett. {\bf 56} (1986) 564.

\bibitem{inv6} E. Witten, `` {\em New issues in manifolds with SU(3)
  holonomy }'', Nucl. Phys. {\bf B 268} (1986) 79;
\bibitem{inv7}E. Ma, `` {\em Lepton number non-conservation in $E_6$ inspired
  superstring models }'', Phys. Lett. {\bf B 191} (1987) 287.
\bibitem{LG:2000}  W.~Grimus, L.~Lavoura, JHEP {\bf 0011}, (2000)   042 ; arXiv:
  0008179 [hep-ph];
 P. M. Ferreira, W. Grimus, D. Jurciuconis,
  L. Lavoura, ``{\em Flavor symmetries in a renormalizable
    SO(10)}'',Nucl. Phys. {\bf B 906} (2016) 289-320,
  arXiv:1510.02641[hep-ph].
%\bibitem{mitra-gs-vissani:2012} M. Mitra, G. Senjanovic, F. Vissani,
%  Nucl. Phys. {\bf B 856}, (2012)  26.

\bibitem{type-III} R. Foot, H. Lew, X. G. He, G. C. Joshi,
  Z. Phys. {\bf C 44} (1989) 441; B. Bajc,  G. Senjanovic, ``{\em
    Seesaw at LHC}'', JHEP {\bf 0708} (2007) 014 ~arXiv:hep-ph/0612092. 
\bibitem{Akhmedov} E.~K.~Akhmedov and M.~Frigerio,
  %``Duality in Left-Right Symmetric Seesaw Mechanism,''
  Phys.\ Rev.\ Lett.\  {\bf 96} (2006) 061802
  [hep-ph/0509299]; 
  %%CITATION = HEP-PH/0509299;%%
  %35 citations counted in INSPIRE as of 11 Mar 2015
  E.~K.~Akhmedov and M.~Frigerio,
  %``Interplay of type I and type II seesaw contributions to neutrino mass,''
  JHEP {\bf 0701} (2007) 043
  [hep-ph/0609046].
  %%CITATION = HEP-PH/0609046;%%
  %24 citations counted in INSPIRE as of 11 Mar 2015

\bibitem{mkp:hybrid}M. K. Parida, B. P. Nayak, R. Satpathy, R. L. Awasthi, ``{\em
  Standard coupling unification in SO(10), hybrid seesaw neutrino mass,
  leptogenesis, dark matter, and proton lifetime predictions}'', 
  JHEP {\bf 1704} (2017) 075, [arXiv:1608.03956[hep-ph]][{\small IN}SPIRE].

\bibitem{Linear} E. Akhmedov, M. Lindner, E. Schnapka, J. W. F. Valle,
  `` {\em Left-right symmetry breaking in NJL approach}'',
  Phys. Lett. {\bf B 368} (1996) 270-280, hep-ph/9507275; M. Malinsky,
  J. C. Romao, J. W. F. Valle, `` {\em  Novel supersymmetric SO(10) seesaw
mechanism}'', Phys. Rev. Lett. {\bf 95} (2005) 161801, hep-ph/0506296;
  S. M. Barr, Phys. Rev. Lett. {\bf 92} (2004) 101601, hep-ph/0309152. 
 
\bibitem{Rad} E. Ma, Phys. Rev. {\bf D 73} (2006) 077031; 
\bibitem{mkprad:2011} M. K. Parida, Phys. Lett. {\bf B 704} (2011) 206;
\bibitem{mkprad:2012}  M. K. Parida, Pramana, {\bf 79} (2012) 1271.  
\bibitem{JCP:1974} J. C. Pati, A. Salam, Phys. Rev. {\bf D 10} (1974) 275. 
\bibitem{rnmpati:1975}. R. N. Mohapatra, J. C. Pati, Phys. Rev. {\bf D 11},
 566, (1975) 2558.
\bibitem{gs-RNM:1975}
 G. Senjanovi\'c, R. N. Mohapatra, Phys. Rev.
{\bf D 12},  (1975)  1502; G.~Senjanovic,
  %``Spontaneous Breakdown of Parity in a Class of Gauge Theories,''
  Nucl.\ Phys.\ B {\bf 153}, 334 (1979).
  %%CITATION = NUPHA,B153,334;%%
  %3
\bibitem{georgi:1974} H. Georgi, Particles and Fields, {\it Proceedings of APS
Division of Particles and Fields,} ed C. Carlson, (AIP, New York, 1975), p.575;
 H. Fritzsch, P. Minkowski, Ann. Phys. (Berlin) {\bf 93}, 193 (1975).
\bibitem{babu-rnm:1993} K. S. Babu and R. N. Mohapatra,
  Phys. Rev. Lett. {\bf 70},  (1993) 2845.
\bibitem{Joshipura} A. S. Joshipura, K. M. Patel , ``{\em Fermion
  masses in SO(10) model}'', Phys. Rev. {\bf D 83 } (2011) 095002,
  arXiv:1102.5148[hep-ph].
\bibitem{Altarelli-Blankenburg} G. Altarelli,
  G. Blankenburg,''{\em Different SO(10) paths to fermion masses and mixings}'',
  JHEP {\bf 1103} (2011) 133, arXiv:1012.2697[hep-ph].

\bibitem{Bertollini}  S. Bertolini, T. Schwetz, M. Malinsky, ``{\em Fermion masses and mixings in SO(10) models and the neutrino challenge to supersymmetric grand unified theories}'', Phys. Rev.{\bf D 73} 115012, [hep-ph/0605006v4].

\bibitem{Goh-Mohapatra} H. S. Goh, R. N. Mohapatra, and S. Nasri, Phys. Rev. {\bf D\,70}, 075022 (2004).

\bibitem{mkp-PLB:1983} 
D. Chang, R. N. Mohapatra, and M. K. Parida, Phys. Rev. Lett.
 {\bf 52}, 1072 (1984);
%\bibitem{cmp2:1984}
  D. Chang, R. N. Mohapatra, and M. K. Parida,
 Phys. Rev. {\bf D 30}, 1052 (1984);
%\bibitem{cmp3:1985}
 D. Chang, R. N. Mohapatra, J. Gipson, R. E. Marshak,
  and M. K. Parida, Phys. Rev. {\bf D 31}, 1718 (1985);
%\bibitem{cmp4:PLB:1984}
 D. Chang, R. N. Mohapatra, M. K. Parida,  Phys. Lett. {\bf 142B} (1984) 55-58;
M. K. Parida, Phys. Lett. {\bf B 126 }, 220
  (1983); M. K. Parida, Phys. Rev. {\bf D 17} (1983) R2383 ({\bf Rapid Communication}),
%\bibitem{mkp-pkp:1991} 
M. K. Parida, P.K. Patra, Phys. Rev. Lett. {\bf 66} (1991) 858 ;
%\bibitem{mkp-pkp:1992} 
M. K. Parida, P.K. Patra, Phys. Rev. Lett. {\bf 68} (1992)754;
%\bibitem{rnm-PLB:1992} 
M. K. Parida, Phys. Rev. {\bf D 57} (1998) 2762; 
R. N. Mohapatra, Phys. Lett. {\bf B 285} (1992) 235;
%\bibitem{Pal:1994}
 N. Deshpande, E. Keith, P. Pal, Phys. Rev. {\bf D 47}(1993) 2892;
% \bibitem{Shafi:1982}  
T. W. B. Kibble, G. Lazaridis, Q. Shafi, Phys. Rev. {\bf D 26 } (1982)
435;
%\bibitem{Shafi:1984}
 Q. Shafi, C. Wetterich, Phys. Rev. Lett. {\bf 52} (1984) 875;
%bibitem{Berto:2009}
 S. Bertolini, Luca Di Luzio, M. Malinsky, ``{\em Intermediate Mass Scales in the Non-Supersymmetric SO(10) Grand Unification: A Reappraisal }'',Phys. Rev. {\bf D 80} (2009) 015013, arXiv:0903.4049[hep-ph];
%\bibitem{mprs:2007} 
S.K. Majee, M. K. Parida,
  A. Raychaudhuri, U. Sarkar, Phys. Rev. {\bf D 75 } (2007) 075003, arXiv:hep-ph/0701109;

%\bibitem{Jaydeep:2018} 
J. Chakraborty, S. K. Patra, S. Mohanty, Phys. Rev.{\bf D 97} (2018)
{\bf no.9, 095010}, arXiv:1711.11391[hep-ph].
\bibitem{Wilczek-Valle} M. Reig, J. W. F. Valle, C. A. Vaquera-Arazuo,
  F. Wilczek, ``{\em A Model for Comprehensive Unification}'',Phys. Lett. {\bf B774 } (2017) 667-670, arXiv:1706.03116[hep-ph].

\bibitem{fukuyana:1986} S. Fukuyama and T. Yanagida, Phys. Lett. {\bf
  B 174} (1986) 45.
\bibitem{Hambye-gs} T. Hambye, E. Ma, U. Sarkar, ``{\em Supersymmetric
  Triplet Higgs Model of Neutrino Mass and Leptogenesis}'',
  Nucl. Phys. {\bf B602} (2001) 23,hep-ph/0011192; 
T. Hambye, G. Senjanovic, ``{ \em Consequences of
  Triplet Seesaw for Leptogenesis }'', Phys. Lett. {\bf B582} (2004)
73-81; G. D'Ambrosio, T. Hambye, A. Hector, A. Rossi, {\em
  Leptogenesis in the Minimal Supersymmetric Tri[let Seesaw Model}'',
  Phys. Lett. {\bf B604} (2004) 199-206.     
%\bibitem{soni:1979} G. Beall, M. Bander, and A. Soni,
 % Phys. Rev. Lett. {\bf 48}, (1982)  848. 

\bibitem{bbexpt1-Klapdoor} H.V. Klapdor-Kleingrothaus, A. Dietz, L. Baudis, G. Heusser, I.V. Krivosheina, S. Kolb, B. Majorovits, H. Pas, H. Strecker, V. Alexeev, A. Balysh, A. Bakalyarov, S.T. Belyaev, V.I. Lebedev, S. Zhukov (Kurchatov Institute, Moscow, Russia)
Eur.Phys.J. {\bf A\,12}, (2001)  147.

\bibitem{bbexpt2} C. Arnaboldi et al. [CUORICINO Collaboration], Phys. Rev. {\bf C\,78}, (2008)  035502 ;
         C. E. Aalseth et al. [ IGEX Collaboration ], Phys. Rev. {\bf D\,65},  (2002) 092007.    

\bibitem{bbexpt3} J. Argyriades et al. [NEMO Collaboration], Phys. Rev. {\bf C\,80}, (2009) 032501 ;
         I. Abt, M. F. Altmann, A. Bakalyarov, I. Barabanov, C. Bauer, E. Bellotti, S. T. Belyaev, 
         L. B. Bezrukov et al., [hep-ex/0404039];
\bibitem{gs:bb}M .Nemevsk, G. Senjanovic, V. Tello, Phys. Rev. Lett. {\bf B 110},  (2013) 151802.
\bibitem{mkp-bs:2015} M. K. Parida, B. Sahoo, ``{\em Planck scale
  induced left-right gauge theory at LHC and experimental tests}'',
  Nucl.Phys. {\bf B 906} (2016) 77-104 arXiv:1411.6748[hep-ph]. 


\bibitem{Keung-gs:1983} W. Y. Keung, G. Senjanovic, ``{\em Majorana
  neutrinos and the production of right-handed charged gauge
  bosons}'', Phys. Rev. Lett. {\bf 50} (1983) 1427.  
\bibitem{Bajc-gs:2007} B. Bajc, G. Senjanovic, ``{\em Seesaw at LHC}'', JHEP, 
{\bf 0708} (2007) 014 ~arXiv:hep-ph/0612029. 

\bibitem{Valle:bb1}J. W. F. Valle,''{\em Neutrinoless double beta
  decay with quasidirac neutrinos}'', Phys. Rev. {\bf D 27} (1983)
  1672-1674.

\bibitem{Valle:nuD1}J. Schechter, J.W. F.  Valle, ``{\em Neutrino Decay and Spontaneous
  Violation of  Lepton Number}'', Phys. Rev. {\bf D 25} (1982) 774.
\bibitem{Valle:MajDM} G. Gelmini,, David N. Schramm,
J.W.F. Valle, `` {\em Majorons: A Simultaneous Solution to the Large and Small Scale Dark Matter Problems}'',
Phys.Lett. 146B (1984) 311-317.
\bibitem{Valle:nuD2} G.B. Gelmini, J.W.F. Valle, ``{ \em Fast Invisible Neutrino Decays}'', Phys.Lett. 142B (1984) 181-187
    (1984).
\bibitem{Valle:CPLRS} J.W.F. Valle, ``{\em 
 Leptonic {CP} Violation and Left-right Symmetry}'', Phys.Lett. 138B (1984) 155-158.
\bibitem{Valle:KeVDM}V. Berezinsky, J.W.F. Valle, ``{\em The KeV majoron as a dark matter particle}'' Phys.Lett. B318 (1993) 360-366, hep-ph/9309214.
 
\bibitem{Planck15}{\bf Planck Collaboration}, P.A. Ade {\em et al.}, ``{ \em
  Planck 2015 results. XIII. Cosmological parameter}'',
  Astron. Astrophys. {\bf 594} (2016) A13, arXiv:1502.01589[astro-phy].
\bibitem{Valle:dynamical} J. W. F. Valle, C. A. Vaquera Arajuo, ``{\em Dynamical Seesaw Mechanism for Dirac Neutrinos}'', Phys. Lett.{\bf B755} (2016) 363, arXiv:1601.05237[hep-ph]. 
\bibitem{Valle:LFV1987} J. Bernabeu, A. Santamaria, J. Vidal, A. Mendez, J. W. F. Valle, ``{\em Lepton Flavor Non-COnservation in High Energies in a Super String Inspired Standard Model}'',Phys. Lett. {\bf B187} (1987) 303-308.
\bibitem{lfvexpt} J. Adam {\em et al.} (MEG Collaboration),
Phys. Rev. Lett.{\bf 107}, 171801 (2011); arxiv: 1107.5547[hep-ex];
K. Hayasaka {\em et al,} (Belle Collaboration), Phys. Lett. {\bf B
  666}, 16 (2008), arxiv:0705.0650[hep-ex]; M. L. Brooks et al. [MEGA Collaboration], Phys. Rev. Lett. {\bf 83}, 1521 (1999); 
 B. Aubert [The BABAR Collaboration], arXiv:0908.2381 [hep-ex]; 
   Y. Kuno (PRIME Working Group), Nucl. Phys. B. Proc. Suppl. {\bf
     149}, 376 (2005);  F. R. Joaquim, A. Rossi, Nucl. Phys. {\bf B\,765}, 71 (2007).

%\bibitem{hussein} N. Bakhel, T. Hussein, M. Y. Khlopov, arXiv: 1406.3254 [hep-p%h].

%\bibitem{cms} S. Chatrchyan et. al. [CMS Collaboration], Phys. Rev. Lett. {\bf %B 109},  (2012) 261802.

%\bibitem{snovic} G. Senjanovic, Int. J. Mod. Phys. A 20,  (2011), 1469, aXiv:10%12.4104 [hep-ph].
\bibitem{ilakovac}
  A.~Ilakovac, A.~Pilaftsis,
  %``Flavor violating charged lepton decays in seesaw-type models,''
  Nucl.\ Phys.\  {\bf B\,437}, (1995)  491 
  [hep-ph/9403398];
 F.~Deppisch, J.~W.~F.~Valle,
  %``Enhanced lepton flavor violation in the supersymmetric 
%inverse seesaw model,''
 Phys.\ Rev.\  {\bf D\,72},  (2005) 036001  [hep-ph/0406040];  
C.~Arina, F.~Bazzocchi, N.~Fornengo, J.~C.~Romao, J.~W.~F.~Valle,
  %``Minimal supergravity sn
%eutrino dark matter and inverse seesaw neutrino masses,''
  Phys.\ Rev.\ Lett.\  {\bf 101},  (2008)  161802 ; arXiv:0806.3225 [hep-ph]; 
 M.~Malinsky, T.~Ohlsson, Z.~Xing, H.~Zhang,
  %``Non-unitary neutrino mixing and CP violation 
%in the minimal inverse seesaw model,''
  Phys.\ Lett.\  {\bf B\,679},  (2009) 242-248 ; arXiv:0905.2889 [hep-ph];
 M.~Hirsch, T.~Kernreiter, J.~C.~Romao, A.~Villanova del Moral,
  %``Minimal Supersymmetric Inverse Seesaw: Neutrino masses, lepton flavour violation and LHC phenomenology,''
  JHEP {\bf 1001} (2010) 103; arXiv:0910.2435 [hep-ph];
F.~Deppisch, T.~S.~Kosmas, J.~W.~F.~Valle,
  %``Enhanced mu- - e- conversion in nuclei in the inverse seesaw model,''
  Nucl.\ Phys.\  {\bf B\,752},  (2006) 80 ; arXiv:0910.3924 [hep-ph]; 
  S. P. Das, F. F. Deppisch, O. Kittel, J. W. F. Valle, Phys. Rev. {\bf D\,86},  (2012) 055006.
\bibitem{antbnd} S. Antusch, J. P. Baumann, and E. Fernandez-Martinez,
  Nucl. Phys. {\bf B 180}, (2009) 369 ; S. Antusch, J. P. Baumann, and
  E. Fernandez-Martinez,and J. Lopez-Pavon, Nucl. Phys. {\bf B 810}  (2009) 369 ;
S. Antush,~C. Biggio, E. Fernandez-Martinez, M. Belen Gavela, 
J. Lopez-Pavon, J. High Energy Phys. {\bf 10} (2006) 084; D.V. Forero,
S. Morisi, M. Tartola and J. W. F. Valle, J. High Energy Phys. {\bf 09} (2011) 142.
\bibitem{non-unit} E. Fernandez-Martinez, M. B. Gavela, J. Lopez-Pavon and O. Yasuda, Phys. Lett. {\bf B\,649}, (2007)  427 ;  
                   K. Kanaya, Prog. Theor. Phys.,{\bf 64}, (1980)  2278; 
                   J. Kersten and A. Y. Smirnov, Phys. Rev. {\bf D\,76},  (2007) 073005 ; 
                   M. Malinsky, T. Ohlsson, H. Zhang, Phys. Rev. {\bf D\, 79},  (2009) 073009; 
                   G. Altarelli and D. Meloni, Nucl. Phys. {\bf B\, 809},  (2009) 158 ; 
                   F. del Aguila and J. A. Aguilar-Saavedra, Phys. Lett. {\bf B\, 672},  (2009) 158 ; 
                   F. del Aguila and J. A. Aguilar-Saavedra and J. de Blas, Acta Phys. Polon. {\bf B\,40}, 
                    (2009) 2901 ; arXiv:0910.2720 [hep-ph]; 
                   A. van der Schaaf, J. Phys. {\bf G\, 29},  (2003) 2755; Y. Kuno, Nucl. Phys. {\bf B}, Proc. Suppl. {\bf 149},  (2005) 376.

\bibitem{Kim-Kang:2006}S. K. Kang and C. S. Kim, Phys. Lett. {\bf B\,646}, 248 (2007)       

\bibitem{app:2013}R. L. Awasthi, M. K. Parida, and  S. Patra, J. High
Energy Phys. {\bf 1308} (2013) 122, arXiv:1302.0672[hep-ph];
M. K. Parida, S. Patra, Phys. Lett {\bf B 718} (2013)
1407,arXiv:1211.5000 [hep-ph]. 
%\bibitem{hussein} N. Bakhel, T. Hussein, M. Y. Khlopov, arXiv: 1406.3254 [hep-p%h].
\bibitem{Majee:2009} S. K. Majee, M. K. Parida, A. Raychaudhuri,
  Phys. Lett. {\bf B 668 }(4) (2008) 299-302, arXiv:0807.3959[hep-ph].

\bibitem{mkp-ARC:2010} M. K. Parida and A. Raychaudhuri, Phys. Rev. {\bf D\,82}, 093017 (2010);  arXiv:1007.5085[hep-ph]. 
                 
\bibitem{pas:2014} M.~K.~Parida, R.~L.~Awasthi, P.~K.~Sahu, JHEP {\bf 1501} (2015) 045, arXiv:1401.1412 [hep-ph].
\bibitem{bpn-mkp:2015}
B.P.Nayak and M.K.Parida, Eur. Phys. J. {\bf C 75},  (2015) 5,183,
arxiv:1312.3185 [hep-ph]
\bibitem{scp:2018} Biswonath Sahoo, Mainak Chakraborty, M. K. Parida, ``{\em
  Neutrino mass, Coupling Unification, Verifiable Proton Decay, Vacuum
  Stability and WIMP Dark Matter in SU(5)}'', Adv. High Energy
  Phys.{\bf 2018} (2018) 4078657 [arXiv:1404.01803[hep-ph]].
\bibitem{kynshi-mkp:1993} M.L. Kynshi, M.K. Parida, ``{\em Higgs
  Scalar in the Grand Desert with Observable Proton Lifetime in SU(5)
  and Small Neutrino Masses in SO(10)  }'',Phys.Rev. {\bf D 47} (1993)
  4830 ({\bf Rapid Communication}).
\bibitem{PDG:2012} Particle Data Group,
 J. Beringer et al., ``{\em Review of Particle Physics (RPP) }'',
 Phys. Rev. {\bf D 86} (2012) 010001. 
\bibitem{PDG:2014} K. A. Olive {\em et al.} (Particle Data Group), ``{\em Review of Particle Physics (RPP) }'', Chin.
  Phys. {\bf C 38 } (2014) 090001. 
\bibitem{PDG:2016} C. Patrignani {\em et al} (Particle
  Data Group), ``{\em Review of Particle Physics (RPP) }'',
  Chin. Phys. {\bf C 40} (2016) no.10, 100001.

\bibitem{Weinberg:1980} S. Weinberg, ``{\em Effective Gauge Theories  }'' Phys. Lett. {\bf B  91 } (1980) 51.
\bibitem{Hall:1981} L. Hall, ``{\em Grand Unification of Effective Gauge Theories  }'', Nucl. Phys. {\bf B  178} (1981) 75.
\bibitem{Ovrut:1982} B. Ovrut, H. Schnitzer, ``{\em Effective Field Theory in Background Field Gauge }'',Phys. Lett. {\bf B 110} (1982) 139.
\bibitem{mkp:1987} M. K. Parida, ``{\em Heavy Particle Effects in Grand
  Unified Theories with Fine-Structure Constant
  Matching}'',Phys. Lett. {\bf B 196 } (1987) 163.
\bibitem{mkp-cch:1989} M. K. Parida, C. C. Hazra,
M. K. Parida, C. C. Hazra, `` {\em Superheavy
  Higgs Scalar Effects in Effective Gauge Theories From SO(10) Grand
  Unification With Low Mass Right-handed Gauge Bosons}'',
  Phys.Rev. D40 (1989) 3074-3085.

\bibitem{rnm-mkp:1993} R. N. Mohapatra, M. K. Parida, Phys. Rev. {\bf
  D 47 } (1993) 264, arXiv:hep-ph/9204234; 

\bibitem{Langacker:1993} P. Langacker, N. Polonsky, Phys. Rev. {\bf D 49} (1994) 1454, arXiv:hep-ph/9306235.
\bibitem{lmpr:1995}  Dae-Gyu Lee,
  R. N. Mohapatra, M. K. Parida, M. Rani, Phys. Rev. {\bf D 51} (1995)
  229, arXiv:hep-ph/9404238.
\bibitem{Abe:2017}{\bf Superkamiokande Collaboration}, K. Abe, Y. Haga,
  Y. Hayato, M. Ikeda, K. Iyogi {\it et al.}, `` {\em Search for Proton Decay via
  $p\to e^+\pi^0$ and $ p\to \mu^+\pi^0$ in 0.31 Megaton Years
  Exposoure of Water Cherenkov Detector}'', Phys. Rev. {\bf D 95} 
  (2017) {\bf no.1} 012004 arXiv:1610.03597[hep-ex][INSPIRE].
  
\bibitem{Dorsner} 
  I.~Doršner, S.~Fajfer, A.~Greljo, J.~F.~Kamenik and N.~Košnik,
  ``{\em Physics of leptoquarks in precision experiments and at particle colliders}'',
  Phys.\ Rept.\  {\bf 641}, 1 (2016)
%  doi:10.1016/j.physrep.2016.06.001
  [arXiv:1603.04993 [hep-ph]];
%  I. Dorsner, S. Fajfer, A. Greljo, Phs. Rept. {\bf
%   64} (2016) 1-68;
 I.~Dorsner,
  ``{\em A scalar leptoquark in SU(5)}'',
  Phys.\ Rev.\ D {\bf 86}, 055009 (2012)
%  doi:10.1103/PhysRevD.86.055009
  [arXiv:1206.5998 [hep-ph]];
%   I. Dorsner, Phys. Rev. {\bf D 86} (2012) 055009;
I.~Dorsner, S.~Fajfer and I.~Mustac,
  ``{\em Light vector-like fermions in a minimal SU(5) setup}'',
  Phys.\ Rev.\ D {\bf 89}, no. 11, 115004 (2014)
%  doi:10.1103/PhysRevD.89.115004
  [arXiv:1401.6870 [hep-ph]];
 M. K. Parida, P. K. Patra, A. K. Mohanty, ``{\em
  Gravity induced large grand unification mass in SU(5) with higher
  dimensional operator}'', Phys. Rev. {\bf D 39} (1989) 316-322; 
 
I.~Doršner, S.~Fajfer and N.~Košnik,
  ``{\em Leptoquark mechanism of neutrino masses within the grand unification framework}'',
  Eur.\ Phys.\ J.\ C {\bf 77}, no. 6, 417 (2017)
%  doi:10.1140/epjc/s10052-017-4987-2
  [arXiv:1701.08322 [hep-ph]];  
%  I. Dorsner, S. Fajfer, N. Kosnik, EPJC {\bf 77}
%   (2017) 417; 
  I.~Dorsner, S.~Fajfer and N.~Kosnik,
  ``{\em Heavy and light scalar leptoquarks in proton decay}'',
  Phys.\ Rev.\ D {\bf 86}, 015013 (2012)
%  doi:10.1103/PhysRevD.86.015013
  [arXiv:1204.0674 [hep-ph]]; 
%   I. Dorsner, S. Fajfer, N. Kosnik, Phys. Rev. {\bf D 86}
%   (2012) 015013;
I.~Dorsner and P.~Fileviez Perez,
  ``{\em Upper Bound on the Mass of the Type III Seesaw Triplet in an SU(5) Model}'',
  JHEP {\bf 0706}, 029 (2007)
%  doi:10.1088/1126-6708/2007/06/029
  [hep-ph/0612216];
%   I. Dorsner, P. Filviez Perez, JHEP {\bf 0706} (2007)
%   029;
P.~Fileviez Perez,
  ``{\em Supersymmetric Adjoint SU(5)}'',
  Phys.\ Rev.\ D {\bf 76}, 071701 (2007)
%  doi:10.1103/PhysRevD.76.071701
  [arXiv:0705.3589 [hep-ph]]   .
%   P. Filviez Perez, Phys. Rev. {\bf D 76} (2007) 071701. 


 \bibitem{Nath-Perez:2007} P. Nath, P. Filviez Perez, Phys. Rep. {\bf 441} (2007) 191, arXiv:hep-ph/0601023.
 
\bibitem{Langacker:1981}  
 P. Langacker, Phys. Rept.{\bf 72 }, 185 (1981).


\bibitem{dp:2001}C. R. Das and M. K. Parida, ``{\em New Formulas and
  Predictions for Running Fermion Masses in SM, 2HDM, and MSSM}'' Eur. Phy. J. {\bf C 20}
  (2001) 121, arXiv:0010004[hep-ph]; M. K. Parida, B. Purkayastha, Eur. Phy. J. {\bf C 14}
  (2000) 159, arXiv:9902374[hep-ph]; M. K. Parida, N. N. Singh,
  Phys. Rev. {\bf D 59} (1999) 032002, arXiv:9710328[hep-ph].
\bibitem{Stella:2016} D. Meloni, T. Ohlsson, S. Riad, ``{\em
  Renormalization Group Running of Fermion Observables in an Extended
  Non-Supersymmetric SO(10) Model }'', JHEP {\bf 03} (2017)
  045,arXiv:1612.07973[hep-ph] [{\small IN}SPIRE].
  
\bibitem{ap:2012} R. L. Awasthi  and M. K. Parida ,Phys.Rev.{\bf D 86}
  , (2012)  093004, e-Print:arXiv:1112.1286[hep-ph].

\bibitem{mkp-rs:plan} M. K. Parida, Rajesh Satpathy, Under Preparation.

\bibitem{Vergados} G. Pantis, F. Simkovic, J. Vergados, and A. Faessler, Phys. Rev. {\bf C\,53},  (1996) 695 ; 
arXiv:nucl-th/9612036 [nucl-th]; 
J. Suhonen and O. Civitarese, Phys. Rept.{\bf C 300} (1998) 123; 
J. Kotila and F. Iachello, Phys. Rev. {\bf C\,85},  (2012)  034316 ;
arXiv:1209.5722 [nucl-th].
\bibitem{Doi}R. N. Mohapatra, Phys. Rev. {\bf D 34 } (1986) 3457; M. Doi, T. Kotani, and E. Takasugi, Prog. Theor. Phys. Suppl. {\bf 83} (1985) 1;
F. Simkovic, G. Pantis, J. Vergados, and A. Faessler, Phys. Rev. {\bf C\,60},  (1999)  055502 ;
arXiv:hep-ph/9905509 [hep-ph]; A. Faessler, A. Meroni, S. T. Petcov, F. SimkovicNucl. Phys. {\bf B 692} (2004) 303; arXiv:hep-ph/0309342.  
\bibitem{falkowski:2014} A. Falkowski, C. Gross, O. Lebedev, ``{\em A
  Second Higgs from Higgs Portal}'', JHEP {\bf 05} (2015) 057,
  arXiv:1502.01361 [hep-ph].
\bibitem{spc:2017} B. Sahoo, M. K. Parida, M. Chakraborty, ``{\em
  Matter Parity Violating Dark Matter Decay in Minimal SO(10),
  Unification, Vacuum Stability and Verifiable Proton Decay }'',
  e-Print: arXiv:1707.01286[hep-ph].  
\bibitem{strumia:2006} M.~Cirelli, A.~Strumia and M.~Tamburini,
  ``{\em Cosmology and Astrophysics of Minimal Dark Matter}'',
  Nucl.\ Phys.\ B {\bf 787}, 152 (2007)
  [arXiv:0706.4071 [hep-ph]].

\bibitem{tHooft}G. 't Hooft, in Proceedings of the 1979 Cargese Summer Institute on Recent Developments in Gauge Theories, edited by
 G. t Hooft et al. (Plenum Press, New York, 1980).
\bibitem{Rodejohann} Shao-Feng Ge, W. Rodejohann, K. Zuber, ``{\em
  Half-life Expectations of Neutrinoless Double Beta Decay in Standard
  and Non-Standard Secnarios}'', Phys. Rev. {\bf D 96} (2017) {\bf
  no.5}, 055019 [arXiv:1707.07904'[hep-ph]]. 
\bibitem{bpn-mkp:arx} B. P. Nayak, M. K. Parida, ``{\em Dilepton
  Events with Displaced Vertices, Double Beta Decay, and Resonant
  Leptogenesis with Type-II Seesaw Dominance, TeV Scale $Z^{\prime}$
  and Heavy Neutrinos }'', e-Print:arXiv:1509.06192 [hep-ph].  

\end{thebibliography}


\end{document}


