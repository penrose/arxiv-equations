\documentclass[10pt]{wlscirep}

\title{Variation of the spin textures of 2-species spin-1 condensates
studied beyond the single spatial mode approximation and the experimental
identification of these textures}

\author[1]{Y. Z. He}
\author[2]{Y. M. Liu}
\author[1,*]{C. G. Bao}
\affil[1]{School of Physics, Sun Yat-Sen University, Guangzhou, 510275, P. R. China}
\affil[2]{Department of physics, Shaoguan University, shaoguan, 510205, P. R. China}
\affil[*]{Corresponding author: C.G. Bao, stsbcg@mail.sysu.edu.cn}

\begin{abstract}
Based on the numerical solutions of the coupled Gross-Pitaevskii equations, the spin-texture of a Bose-Einstein condensate with two kinds of spin-1 atoms has been studied. Besides, the probabilities of an atom in spin component $\mu$ and of two correlated atoms one in $\mu$ and one in $\nu$ have been calculated. By an analysis of the probabilities, three (and only three) types of texture have been found. (i) Both species are in polar phase. (ii) Both species are in ferromagnetic (f) phase. The two kinds of aligned spins can be parallel or anti-parallel to each other. (iii) One in f-phase, one in quasi-f phase (when the total spin $S$ is given along the $Z$-axis, the spin-state in the quasi-f phase will have a part of atoms having spin-component $\mu=1$ while all the rest having $\mu =-1$). This finding simplifies the previous classification. Moreover, we found that the variation of the spin-textures can be sensitively reflected by these probabilities. Therefore, the theoretical and experimental studies on these probabilities provide a way to determine the parameters involved.
\end{abstract}

\begin{document}

\flushbottom
\maketitle



\section*{Introduction}

It is well known that the Bose-Einstein condensates (BEC) as man-made and
controllable systems are promising in application. In particular, due to the
optical traps, the spin-degrees of freedom can be liberated and therefore
the inherent physics becomes much richer\cite{ref5,ref6,ref7,ref8,ref9,ref10}.
The so called spinor BEC (S-BEC) is a very exquisite system. Although the
spin-dependent forces are very weak, the spin-textures are decisively
determined by them. When the temperature is sufficiently low (say, lower
than $10^{-10}$K), the spatial degrees of freedom
are nearly frozen and the spin degrees of freedom play essential roles. In
this case, the S-BEC might be an ideal system for realizing exquisite
control.

The ground state (g.s.) of 1-species S-BEC with spin-1 atoms has two well
known phases, namely, the polar and ferromagnetic phases. Due to the
influence of the spin-dependent interspecies interactions, the spin-textures
of 2-species S-BEC become more complicated. These textures have been studied
by a number of authors \cite{ml,ref11,ref12,ref13,ref14,ref15,ref16,ref17,ref18}.
The present paper is
one along this line. The condensate is a mixture of A-atoms and B-atoms. In
the previous study the single mode approximation has been adopted. The
details of the spatial wave functions have not yet been taken into account.
Instead, the coupled Gross-Pitaevskii equations (CGP) have been numerically
solved in this paper. Thereby we can obtain the spatial wave functions and
evaluate their effect.

Furthermore, the probability of an atom (A- or B-atom) having its
spin-component in $\mu$ denoted as $P_{\mu}$, and the probability of two
atoms (of the same kind or different kinds) having their spin-components in
$\mu$ and $\nu$ denoted as $P_{\mu\nu}$ have been calculated. How $P_{\mu}$
and $P_{\mu\nu}$ match the variation of the spin-texture has been
studied. Based on the probabilities, the emphasis of this paper is placed on
clarifying the physical picture of the spin-textures. Since $P_{\mu}$ and
$P_{\mu\nu}$ are observables (say, via the time-of-flight images), the
spin-texture can be thereby identified via experimental measurements of
these probabilities. In particular, the critical points where a
spin-texture-transition (STT) occurs can be identified. It was found that
the locations of the critical points depend on the spatial wave functions
seriously (as shown below). Therefore, the measurement of $P_{\mu}$ and 
$P_{\mu\nu}$ can provide useful information on the spin-textures. Together
with the exact solution of the CGP, this information can be used to
determine the values of the parameters involved (say, the strengths of
interactions).

In the following derivation, the fractional
parentage coefficients of spin-1 many-body systems have been introduced
\cite{bao05,bao06}. With these coefficients, all the
matrix elements of 1-body and 2-body spin-operators can be straightly
obtained without knowing the detailed expressions of the related
spin-states. Thus these coefficients facilitate greatly the following
derivation.



\section*{The coupled Gross-Pitaevskii equations and the spin-textures}

Let the condensate contain $N_{A}$ A-atoms and $N_{B}$ B-atoms. Let $X$
denotes A or B. The intra-species interaction for spin-1 atoms is $%
V_{X}=\sum_{1\leq i<j\leq N_{X}}\delta (\mathbf{r}_{i}-\mathbf{r}%
_{j})(c_{X0}+c_{X2}\hat{\mathbf{F}}_{i}^{X}\cdot \hat{\mathbf{F}}_{j}^{X})$,
where $\hat{\mathbf{F}}_{i}^{X}$ is the spin operator for the $i$-th $X$%
-atom. When the singlet-pairing force has been ignored, the inter-species
interaction is $V_{AB}=\sum_{1\leq i\leq N_{A}}\sum_{1\leq i^{\prime }\leq
N_{B}}\delta (\mathbf{r}_{i}-\mathbf{r}_{i^{\prime }})(c_{AB0}+c_{AB2}\hat{%
\mathbf{F}}_{i}^{A}\cdot \hat{\mathbf{F}}_{i^{\prime }}^{B})$, In the g.s.
all the particles of the same kind will condense to a spatial state (denoted
as $\varphi _{X}$) which is most favorable for binding. Let $\Xi $ denotes a
general normalized total spin-state for both species. Then, the g.s. can be
written as
\begin{equation}
\Psi _{o}=\prod_{i=1}^{N_{A}}\varphi _{A}(\mathbf{r}_{i})\prod_{j=1}^{N_{B}}%
\varphi _{B}(\mathbf{r}_{j})\Xi .  \label{twf}
\end{equation}

Let $\vartheta_{S_XM_X}^{N_X}$ denotes a normalized and all-symmetric
spin-state for the $X$-atoms, where all the spins are coupled to $S_X$ and
its $Z$-component $M_X$. According to the theory given in Ref. \citen{katr},
$N_X-S_X$ must be even, $\vartheta_{S_X,M_X}^{N_X}$ is unique when $S_X$ and
$M_X$ are given, and the set $\{\vartheta_{S_X,M_X}^{N_X}\}$ (where $S_X=N_X$%
, $N_X-2$, to 0 or 1) is complete for all-symmetric spin-states with $N_X$
spin-1 atoms. We further introduce $(\vartheta_{S_A}^{N_A}%
\vartheta_{S_B}^{N_B})_{SM}$ where $S_A$ and $S_B$ are coupled to $S$ and $M$
. The set $\{(\vartheta_{S_A}^{N_A}\vartheta_{S_B}^{N_B})_{SM}\}$ is
complete for the total spin-states which is invariant against the
permutation of the same kind of spins. Thus, in general, we have the
expansion $\Xi=\sum_{S_AS_BS}D_{S_AS_BS}(\vartheta_{S_A}^{N_A}%
\vartheta_{S_B}^{N_B})_{SM}$, where $D_{S_AS_BS}$ are the coefficients for
expansion.

Let $\frac{1}{2}m_X\omega_X^2 r^2$ be the isotropic trap for the $X$-atoms.
Let us introduce $m$ and $\omega$, and use $\hbar\omega$ and $\lambda\equiv%
\sqrt{\hbar/(m\omega)}$ as the units for energy and length throughout this
paper. Then, the Hamiltonian is
\begin{equation}
\hat{H} = \hat{K}_A+\hat{K}_B+V_A+V_B+V_{AB},  \label{h}
\end{equation}
where $\hat{K}_X=\sum_{i=1}^{N_X}\hat{h}_X(i)$, $\hat{h}_X(i)=\frac{1}{2}(-%
\frac{m}{m_X}\nabla_i^2+\gamma_X r_i^2)$ and $\gamma_X=\frac{m_X\omega_X^2}{%
m\omega^2}$.

Since the spin-space is expanded by $\vartheta _{S_{X}}^{N_{X}}$ which is
all-symmetric, making use of the symmetry, the Hamiltonian is equivalent to
an effective Hamiltonian $H_{\mathrm{eff}}$, in which $V_{X}$ is replaced by
\begin{equation}
V_{X}^{\mathrm{eff}}=\sum_{i<j}\delta (\mathbf{r}_{i}-\mathbf{r}%
_{j})[c_{X0}+c_{X2}\frac{1}{N_{X}(N_{X}-1)}(\hat{\mathbf{S}}_{X}\cdot \hat{%
\mathbf{S}}_{X}-2N_{X})],  \label{vefx}
\end{equation}%
and $V_{AB}$ is replaced by
\begin{equation}
V_{AB}^{\mathrm{eff}}=\sum_{i,j}\delta (\mathbf{r}_{i}-\mathbf{r}%
_{j})[c_{AB0}+c_{AB2}\frac{1}{2N_{A}N_{B}}(\hat{\mathbf{S}}\cdot \hat{%
\mathbf{S}}-\hat{\mathbf{S}}_{A}\cdot \hat{\mathbf{S}}_{A}-\hat{\mathbf{S}}%
_{B}\cdot \hat{\mathbf{S}}_{B})],  \label{vefab}
\end{equation}%
where $\hat{\mathbf{S}}_{X}=\sum_{i}\hat{\mathbf{F}}_{i}^{X}$ is the total
spin-operator for the $X$-atoms and $\hat{\mathbf{S}}=\hat{\mathbf{S}}_{A}+%
\hat{\mathbf{S}}_{B}$ is the total spin-operator for both kinds of atoms. In
the derivation of Eqs.(\ref{vefx}) and (\ref{vefab}),
the following two equalities $\sum_{i<j}\hat{\mathbf{F}}_{i}^{X}\cdot \hat{%
\mathbf{F}}_{j}^{X}=\frac{1}{2}(\hat{\mathbf{S}}_{X}\cdot \hat{\mathbf{S}}%
_{X}-2N_{X})$ and $\sum_{i}\sum_{j}\hat{\mathbf{F}}_{i}^{A}\cdot \hat{%
\mathbf{F}}_{j}^{B}=\frac{1}{2}(\hat{\mathbf{S}}\cdot \hat{\mathbf{S}}-\hat{%
\mathbf{S}}_{A}\cdot \hat{\mathbf{S}}_{A}-\hat{\mathbf{S}}_{B}\cdot \hat{%
\mathbf{S}}_{B})$ have been introduced. It is clear that $H_{\mathrm{eff}}$
keeps $S_{A}$, $S_{B}$, and $S$ to be conserved.

With the above good quantum numbers, the g.s. can be rewritten as
\begin{equation}
\Psi_o = \prod_{i=1}^{N_A} \varphi_A(\mathbf{r}_i) \prod_{j=1}^{N_B}
\varphi_B(\mathbf{r}_j) (\vartheta_{S_A}^{N_A}\vartheta_{S_B}^{N_B})_{SM}
\equiv \Psi_{S_AS_BSM},  \label{gs}
\end{equation}
where $S_A$, $S_B$, and $S$ together with $\varphi_A$ and $\varphi_B$ will
be determined via a standard variational procedure.

Since $\vartheta_{S_X,M_X}^{N_X}$ is unique as being proved in Ref. %
\citen{katr}, the spin-texture described by $(\vartheta_{S_A}^{N_A}%
\vartheta_{S_B}^{N_B})_{SM}$ is unique. Therefore,
the label $(S_AS_BS)$ can be used to classify the spin-textures (note that $%
M $ is not necessary in the label because it manifests only the geometry)%
\cite{ref14}.

From $\delta(\langle\Psi_{S_AS_BSM}|H_{\mathrm{eff}}|\Psi_{S_AS_BSM}\rangle/%
\langle\Psi_{S_AS_BSM}|\Psi_{S_AS_BSM}\rangle)=0$, we obtain the CGP for $%
\varphi_A$ and $\varphi_B$ as\cite{ml}
\begin{eqnarray}
&&(\hat{h}_A+\alpha_{11}\varphi_A^2+\alpha_{12}\varphi_B^2-\varepsilon_A)%
\varphi_A = 0,  \label{cgp1} \\
&&(\hat{h}_B+\alpha_{21}\varphi_A^2+\alpha_{22}\varphi_B^2-\varepsilon_B)%
\varphi_B = 0,  \label{cgp2}
\end{eqnarray}
where
\begin{eqnarray*}
 &&
 \alpha_{11}=D_A N_A,\ \ \
 \alpha_{12}=D_{AB}N_B,\ \ \
 \alpha_{21}=D_{AB}N_A,\ \ \
 \alpha_{22}=D_B N_B, \\
 &&
 D_X=c_{X0}+\frac{S_X(S_X+1)-2N_X}{N_X(N_X-1)}c_{X2}, \ \ \
 D_{AB}=c_{AB0}+\frac{S(S+1)-S_A(S_A+1)-S_B(S_B+1)}{2N_A N_B}c_{AB2}.
\end{eqnarray*}
Besides, $\varphi_A$ and $\varphi_B$ are required to be
normalized.

Once the set $(S_A,S_B,S)$ has been presumed, we can solve the CGP to obtain
$\varphi_A$ and $\varphi_B$, together with the total energy
\begin{equation}
E_{S_AS_BS} = N_A\langle\varphi_A|\hat{h}_A|\varphi_A\rangle
+N_B\langle\varphi_B|\hat{h}_B|\varphi_B\rangle +\frac{N_A(N_A-1)}{2}D_AI_A +%
\frac{N_B(N_B-1)}{2}D_BI_B +N_AN_BD_{AB}I_{AB},  \label{th}
\end{equation}
where $I_X=\int\mathrm{d}\mathbf{r}|\varphi_X(\mathbf{r})|^4$ and $%
I_{AB}=\int\mathrm{d}\mathbf{r}|\varphi_A(\mathbf{r})\varphi_B(\mathbf{r}%
)|^2 $. From a series of presumed $(S_A,S_B,S)$, we can find out the optimal
set $(S_A,S_B,S)$ which leads to the minimum of the total energy. This set
is for the g.s..

Based on the numerical solutions of the CGP, we concentrate on studying the
effect of the spin-dependent inter-species interaction on spin-textures. In
the follows, $c_{AB2}$ is considered to be variable.
The other parameters are listed in the captions of Fig.\ref{fig1} to Fig.\ref%
{fig3}. They are so chosen to assure that the g.s. is a miscible state, in
which the A- and B-atoms have a better overlap (otherwise, the effect of $%
c_{AB2}$ is weak). $N_A\leq N_B$ is assumed. The variation of the
spin-texture of the g.s. against $c_{AB2}/c_{AB0}$ is demonstrated by the
variation of the good quantum numbers $(S_A,S_B,S)$. Three cases are
considered. (i) Both $c_{A2}$ and $c_{B2}$ are
positive. (ii) Both $c_{A2}$ and $c_{B2}$ are
negative. (iii) $c_{A2}>0$ and $c_{B2}<0$. The
variations are plotted in Fig.\ref{fig1}, \ref{fig2} and \ref{fig3},
respectively.

\begin{figure}[htbp]
\centering \resizebox{0.6\columnwidth}{!}{\includegraphics{STT1.eps} }
\caption{(color online) Variation of the total energy $E_{S_A S_B S}$ and
the spin-texture (specified by $(S_A,S_B,S)$) of the g.s. against $%
c_{AB2}/c_{AB0}$ from $-0.1$ to $0.1$. The parameters are given as $N_A=1000$%
, $N_B=1200$, $m_A/m_B=23/87$, $\gamma_A=\gamma _B=1$; $%
c_{A0}=10^{-3}$, $c_{B0}=2c_{A0}$, $c_{AB0}=0.9c_{A0}$; $c_{A2}=c_{A0}/50$, $%
c_{B2}=c_{B0}/50$. The units for energy and length are $\hbar\omega$
and $\sqrt{\hbar/(m\omega)}$. The solid line is for 
$E_{S_AS_BS}-E_o$, where $E_o$ is the total energy at the
left-end of the curve. The dashed line is for $S_A/N_A$, the dash-dot line
is for $S_B/N_A$, the dash-dot-dot line is for $S/N_A$.}
\label{fig1}
\end{figure}

\begin{figure}[htbp]
\centering \resizebox{0.6\columnwidth}{!}{\includegraphics{STT2.eps}}
\caption{(color online) The parameters are the same as in Fig.\ref%
{fig1} but with $c_{A2}=-c_{A0}/50$, $c_{B2}=-c_{B0}/50$.}
\label{fig2}
\end{figure}

\begin{figure}[htbp]
\centering \resizebox{0.6\columnwidth}{!}{\includegraphics{STT3.eps}}
\caption{(color online) The parameters are the same as in Fig.\ref%
{fig1} but with $c_{B2}=-c_{B0}/50$.}
\label{fig3}
\end{figure}

\begin{figure}[htbp]
\centering \resizebox{0.6\columnwidth}{!}{\includegraphics{STT4.eps}}
\caption{(color online) $P_{\mu}^A$ (a) and $P_{\mu}^B$ (b)
against $c_{AB2}/c_{AB0}$. The parameters are the same as in Fig.\ref%
{fig1}. Solid line is for $\mu=1$, dotted line is for $\mu=0$%
, and dash line is for $\mu=-1$. The points $P_{1}$ to $P_{5}$ mark
the boundary of the zones, refer to Fig.1.}
\label{fig4}
\end{figure}

\begin{figure}[htbp]
\centering \resizebox{0.6\columnwidth}{!}{\includegraphics{STT5.eps}}
\caption{(color online)Selected $P_{\mu\nu}^{AB}$ against $%
c_{AB2}/c_{AB0}$. The parameters are the same as in Fig.\ref{fig1}. $%
\mu$ ($\nu$) is the component of an A-atom (a B-atom). $%
\mu$ and $\nu$ are marked in the figure.}
\label{fig5}
\end{figure}

\begin{figure}[htbp]
\centering \resizebox{0.6\columnwidth}{!}{\includegraphics{STT6.eps}}
\caption{(color online) $P_{\mu}^A$ (a) and $P_{\mu}^B$ (b)
against $c_{AB2}/c_{AB0}$. The parameters are the same as in Fig.\ref%
{fig2}.}
\label{fig6}
\end{figure}

\begin{figure}[htbp]
\centering \resizebox{0.6\columnwidth}{!}{\includegraphics{STT7.eps}}
\caption{(color online) Selected $P_{\mu\nu}^{AB}$ against $%
c_{AB2}/c_{AB0}$. The parameters are the same as in Fig.\ref{fig2}.}
\label{fig7}
\end{figure}

\begin{figure}[htbp]
\centering \resizebox{0.6\columnwidth}{!}{\includegraphics{STT8.eps}}
\caption{(color online) $P_{\mu}^A$ (a) and $P_{\mu}^B$ (b)
against $c_{AB2}/c_{AB0}$. The parameters are the same as in Fig.\ref%
{fig3}.}
\label{fig8}
\end{figure}

{In} Fig.\ref{fig1} to 3, the total energy $E_{S_{A}S_{B}S}$ is also given.
Ahead of carrying on an analysis of the figures, we are going to another
topic which is related to the experimental identification of the
spin-textures.



\section*{Probability of an atom with spin-component $\mu$ and the
probability of a pair of atoms with $\mu$ and $\nu$}

It is believed that, in different spin-textures, the spins are correlated in
different ways. Accordingly, the probability of an A-atom or a B-atom lying
in a specific spin-component $\mu$ (denoted as $P_{\mu}$) is prescribed by
the spin-texture. Moreover, the probability of two atoms that one in $\mu$
and one in $\nu$ (denoted as $P_{\mu\nu}$) is also prescribed by the
spin-texture. Therefore, information on the spin-textures can be extracted
from these probabilities. To obtain the analytical expressions for these
probabilities, a crucial point is to extract the
spin-state of an atom or of two atoms from the total spin-state. By using
the fractional parentage coefficients developed for spin-1 many-body systems%
\cite{bao05,bao06}, the extraction can be realized even when the details of
the total spin-state have not yet been known. The analytical expressions of
the fractional parentage coefficients are given in the Appendix.

With the coefficients Eq.(\ref{fpc}) given in the Appendix, the total
spin-state $(\vartheta_{S_A}^{N_A}\vartheta_{S_B}^{N_B})_{SM}$ can be
rewritten as
\begin{equation}
(\vartheta_{S_A}^{N_A} \vartheta_{S_B}^{N_B})_{SM} = \sum_{\mu}
\chi_{\mu}(i) \sum_{S^{\prime }} [ T_{\mu}^{S^{\prime }}
(\vartheta_{S_A+1}^{N_A-1} \vartheta_{S_B}^{N_B})_{S^{\prime }M-\mu }
+Q_{\mu}^{S^{\prime }} (\vartheta_{S_A-1}^{N_A-1}
\vartheta_{S_B}^{N_B})_{S^{\prime }M-\mu }],  \label{tss}
\end{equation}
where the spin-state of an A-atom in $\mu$ component has been extracted.
\begin{eqnarray}
& T_{\mu}^{S^{\prime }} = C_{1\mu;S^{\prime },M-\mu}^{SM} a_{S_A}^{N_A}
\sqrt{(2S_A+1)(2S^{\prime }+1)} W(1,S_A+1,S,S_B;S_AS^{\prime }),&  \label{p}
\\
& Q_{\mu}^{S^{\prime }} = C_{1\mu;S^{\prime },M-\mu}^{SM} b_{S_A}^{N_A}
\sqrt{(2S_A+1)(2S^{\prime }+1)} W(1,S_A-1,S,S_B;S_AS^{\prime }),&  \label{q}
\end{eqnarray}
where the W-coefficient of Racah has been introduced. Then, the
normalization of the total wave function $\Psi_{S_AS_BSM}$ can be expanded
as
\begin{equation}
\langle\Psi_{S_AS_BSM}|\Psi_{S_AS_BSM}\rangle = \sum_{\mu}P_{\mu}^A,
\end{equation}
where
\begin{equation}
P_{\mu }^A = \sum_{S^{\prime }} [(T_{\mu}^{S^{\prime
}})^2+(Q_{\mu}^{S^{\prime }})^2],
\end{equation}
is just the probability of an A-atom with its spin lying at $\mu$. The
probability of a B-atom at $\mu$, i.e. $P_{\mu}^B$, can be obtained simply
by interchanging $N_A\leftrightarrow N_B$ and $S_A\leftrightarrow S_B$ in
the above formulae.

When the coupled spin-state of two A-atoms has been extracted by using Eq.(%
\ref{tswf}), we have the expansion
\begin{equation}
\langle\Psi_{S_AS_BSM}|\Psi_{S_AS_BSM}\rangle =
\sum_{\mu,\nu}P_{\mu\nu}^{AA},
\end{equation}
where
\begin{eqnarray}
& P_{\mu\nu}^{AA} = \sum_{S^{\prime }_AS^{\prime }} (X_{S^{\prime
}_AS^{\prime }}^{\mu\nu})^2,& \\
& X_{S_A^{\prime }S^{\prime }}^{\mu\nu} = \sum_{\lambda}
C_{1,\mu;1,\nu}^{\lambda,\mu+\nu} C_{\lambda,\mu+\nu;S^{\prime
},M-\mu-\nu}^{SM} h_{\lambda S_A^{\prime }S_A}^{N_A} \sqrt{%
(2S_A+1)(2S^{\prime }+1)} W(\lambda S_A^{\prime }SS_B;S_AS^{\prime }),&
\end{eqnarray}
where $\lambda$ runs over $0$ and $2$. $P_{\mu\nu}^{AA}$ is the probability
of two A-atoms with spin-components $\mu$ and $\nu$, respectively. The
probability $P_{\mu\nu}^{BB}$ can be similarly obtained by interchanging the
indexes $A$ and $B$.

Let $i$ ($i^{\prime }$) denotes an A-atom (a B-atom). When $\chi_{\mu }(i)$
and $\chi_{\nu}(i^{\prime })$ have been simultaneously extracted, by using
Eq.(\ref{fpc}) we have the expansion
\begin{equation}
\langle\Psi_{S_AS_BSM}|\Psi_{S_AS_BSM}\rangle = \sum_{\mu\nu} P_{\mu \nu
}^{AB},
\end{equation}
where
\begin{eqnarray}
& P_{\mu\nu}^{AB} = \sum_{S_{AB}} [ (Z_{S_{AB}}^{I,\mu\nu})^2
+(Z_{S_{AB}}^{II,\mu\nu})^2 +(Z_{S_{AB}}^{III,\mu\nu})^2
+(Z_{S_{AB}}^{IV,\mu\nu})^2 ],& \\
& Z_{S_{AB}}^{I,\mu\nu} = \sum_{\lambda} C_{1,\mu;1,\nu}^{\lambda,\mu+\nu}
C_{\lambda,\mu+\nu;S_{AB},M-\mu-\nu}^{SM} a_{S_A}^{N_A} a_{S_B}^{N_B} \sqrt{%
(2\lambda+1)(2S_{AB}+1)(2S_A+1)(2S_B+1)} \left(
\begin{array}{ccc}
1 & 1 & \lambda \\
S_A+1, & S_B+1, & S_{AB} \\
S_A & S_B & S%
\end{array}
\right), &
\end{eqnarray}
where the 9-$j$ symbol has been introduced, $\lambda$ runs over $0$, $1$,
and $2$. $Z_{S_{AB}}^{II,\mu\nu}$ is similar to $Z_{S_{AB}}^{I,\mu\nu}$ but
with $a_{S_B}^{N_B}$ being changed to $b_{S_B}^{N_B}$ and the index $S_B+1$
in the 9-$j$ symbol being changed to $S_B-1$. $Z_{S_{AB}}^{III,\mu\nu}$ is
similar to $Z_{S_{AB}}^{I,\mu\nu}$ but with $a_{S_A}^{N_A}\rightarrow
b_{S_A}^{N_A}$ and $S_A+1\rightarrow S_A-1$. $Z_{S_{AB}}^{IV,\mu\nu}$ is
similar to $Z_{S_{AB}}^{I,\mu\nu}$ but with $a_{S_A}^{N_A}a_{S_B}^{N_B}%
\rightarrow b_{S_A}^{N_A}b_{S_B}^{N_B}$, $S_A+1\rightarrow S_A-1$, and $%
S_B+1\rightarrow S_B-1$.

Numerical examples of these probabilities are shown in Fig.\ref{fig4} to Fig.%
\ref{fig8}. Where $M=S$ is given, namely, the $Z$-axis is chosen lying along
the orientation of $S$.



\section*{Variation of the spin-texture against the inter-species interaction%
}

Note that, when the spin-spin force is repulsive (attractive), two
anti-parallel (parallel) spins will be lower in energy. This point is
important to the following analysis.

(1) The case $c_{A2}>0$ and $c_{B2}>0$

Recall that, if $c_{AB2}=0$, both species are in polar phase due to both $%
c_{A2}$ and $c_{B2}$ are positive. Thus, the spins are two-by-two coupled to
zero to form the singlet-pairs ($s$-pairs) in which the two spins are
anti-parallel so as to minimize the repulsion caused by $c_{A2}$ and $c_{B2}$%
. When $c_{AB2}>0$, Fig.\ref{fig1} demonstrates that, in the domain marked
by zone III, $(S_A,S_B,S)=(0,0,0)$. Thus the polar-polar phase remains
unchanged when $|c_{AB2}|$ is not so large. On the other hand, Fig.\ref{fig4}
demonstrates that all the $P_{\mu}^A=P_{\mu}^B=1/3$ in zone III. In other
words, the spin-states are both isotropic. This coincides with the texture $%
(0,0,0)$.

When $c_{AB2}$ increases and exceeds $p_{3}$ and enters into
zone IV but not very close to $p_{4}$, from Fig.\ref{fig1} we
know that $S_{A}=N_{A}$, $S_{A}>S_{B}$, and $S=S_{A}-S_{B}$. Since $S$ is
given lying along the $Z$-axis, $S_{A}$ ($S_{B}$) must be lying along $+Z$%
-axis ($-Z$-axis) to assure $S=S_{A}-S_{B}$. The value of $S_{B}$ at
$p_{3}$ (denoted as $S_{B}|_{p_{3}}$) is
determined by the input parameters. The crossing over $p_{3}$
causes a spin-texture-transition (STT) from $(0,0,0)$ to $(N_{A},S_{B}|_{p_{3}},N_{A}-S_{B}|_{p_{3}})$.
Thus $p_{3}$ is the critical point of the STT. During the STT, all
the spins of the A-atoms are suddenly free from the $s$-pairs and lying
along the $Z$-axis. This is shown in Fig.\ref{fig4}a where $P_{1}^{A}=1$ , $%
P_{0}^{A}=0$, and $P_{-1}^{A}=0$ in zone IV. The fact that $P_{0}^{A}=0$
implies a complete breakdown of all the $s$-pairs of the A-atoms. Meanwhile
all the spins of the B-atoms are also suddenly free from the $s$-pairs and
re-align along the $\pm Z$-axis as shown in Fig.\ref{fig4}b where $%
P_{0}^{B}=0$ and $P_{-1}^{B}>P_{1}^{B}>0$ in the zone IV. Accordingly, $%
S_{B}=N_{B}(P_{-1}^{B}-P_{1}^{B})$ lying along the $-Z$-axis.

When $c_{AB2}$ increases from $p_{3}$, $P_{-1}^{B}$ increases
linearly and $P_{1}^{B}$ decreases as shown in Fig.\ref{fig4}b. Accordingly,
$S_{B}$ becomes larger and larger while $S=S_{A}-S_{B}$ becomes smaller.
When $c_{AB2}\rightarrow p_{4}$, $S_{B}\rightarrow S_{A}$ and
$S\rightarrow 0$. In a narrow domain neighboring to $p_{4}$,
the numbers of A- and B-atoms with $\mu =0$ increase greatly. This is shown
in Fig.\ref{fig4}a and \ref{fig4}b by the very narrow and sharp peak of $%
P_{0}^{X}$ peaked at $p_{4}$, where all the six $P_{\mu
}^{X}=1/3$.

When $c_{AB2}$ increases further from $p_{4}$ and enters into
zone V, in a narrow domain at the right side of $p_{4}$, $%
P_{\mu }^{A}$ is changed from $1/3$ (at $p_{4}$) to $\delta
_{\mu ,-1}$. The magnitude $S_{A}=N_{A}$ remains unchanged. However, recall
that $P_{\mu }^{A}=\delta _{\mu ,1}$ in zone IV, the orientation of $S_{A}$
has reversed. Meanwhile, the spins of the B-atoms change also swiftly from
being isotropic (at $p_{4}$) to being partially polarized, a
larger part points to the $Z$-axis while a smaller part points to the $-Z$%
-axis. This is shown in Fig.\ref{fig4}b where the three $P_{\mu }^{B}$ are
changed from $1/3$ to $P_{1}^{B}>P_{-1}^{B}>0$ and $P_{0}^{B}=0$. When $%
c_{AB2}$ increases further in zone V, $P_{1}^{B}$ ($P_{-1}^{B}$) keeps
monotonically increasing (decreasing). Accordingly, $%
S_{B}=N_{B}(P_{1}^{B}-P_{-1}^{B})$ keeps increasing against $c_{AB2}$ but
its orientation is changed from being arbitrary (at $p_{4}$)
to lying along the $+Z$-axis. Thus, from zone IV to zone V, the crossing
over $p_{4}$ leads also to a great change in the spin-texture
(including the emergence and vanish of atoms with $\mu =0$ and the reverse
of the orientations of $S_{A}$ and $S_{B}$). Therefore, $p%
_{4} $ is also a critical point of the STT.

It is mentioned that, in zone V, $P_{-1}^{B}$ is decreasing. Thus it will
eventually become zero as shown in Fig.\ref{fig4}b marked by $p_{5}$.
Crossing over $p_{5}$, the next domain is zone VI.
Where all the A-atoms (B-atoms) are fully polarized along the $-Z$-axis ($+Z$%
-axis) when $N_{B}>N_{A}$ is assumed. Accordingly, $S=N_{B}-N_{A}$. In this
case, the spin of any A-atom and the spin of any B-atom are anti-parallel.
In this way, the repulsion raised by the positive $c_{AB2}$ is minimized.
The associated texture in zone VI is denoted as $(N_{A},N_{B},N_{B}-N_{A})$,
and is named anti-parallel ferro-ferro (f-f) phase. Since the effect of $%
c_{AB2}$ has been optimized in this texture, it is stable against the
further increase of $c_{AB2}$. The turning point $p_{5}$
marks the boundary of this phase

When $c_{AB2}$ decreases from zero, the zone III extends to the left until $%
c_{AB2}=p_{2}$. It implies that the texture $(0,0,0)$ holds
at both sides of $c_{AB2}=0$ from $p_{2}$ to $p%
_{3}$, When $|c_{AB2}|$ exceeds $p_{2}$ and enters into zone
II, all the A-atoms align along the $Z$-axis. Meanwhile, a larger part of
the B-atoms align along the $Z$-axis while a smaller part along $-Z$-axis,
none of them have $\mu =0$. Accordingly, $S_{A}=N_{A}$, $%
S_{B}=N_{B}(P_{1}^{B}-P_{-1}^{B})>0$, and $S=S_{A}+S_{B}$. In this zone, $%
P_{-1}^{B}$ ($P_{1}^{B}$) decreases (increases) with the decrease of $c_{AB2}
$. When $c_{AB2}=p_{1}$, $P_{-1}^{B}$ arrives at zero and all
the B-atoms are also lying along the $Z$-axis. Accordingly, $S_{A}=N_{A}$, $%
S_{B}=N_{B}$, and $S=N_{A}+N_{B}$. In this case the two spins in any A-B
pairs (each is composed of an A-atom and a B-atom) are parallel so as to
maximize the attraction raised by the negative $c_{AB2}$. This texture is
denoted as $(N_{A},N_{B},N_{A}+N_{B})$ and is named parallel f-f phase which
is stable against the further decrease of $c_{AB2}$. The turning point
$p_{1}$ marks the boundary of this phase.

Comparing Fig.\ref{fig1} and \ref{fig4}, we see that the variation and
transition of the spin-texture are sensitively reflected in $P_{\mu }^{X}$.
In particular, the five critical and turning points $p_{1}$
to $p_{5}$ are clearly shown in $P_{\mu }^{X}$. Besides,
selected $P_{\mu \nu }^{AB}$ ($\mu $ is for an A-atom and $\nu $ is for a
B-atom) are shown in Fig.\ref{fig5}. Information on the spin-textures can
also be extracted from this figure. Say, all the $P_{\mu \nu }^{AB}=1/9$ in
zone III and at the boundary separating Zone IV and V, this demonstrates the
isotropism when $S=0$. $P_{-1,1}^{AB}=1$ in zone VI, this demonstrates
directly the anti-parallel f-f phase. $P_{1,1}^{AB}=1$ in zone I, this
demonstrates directly the parallel f-f phase. The case that $%
P_{1,1}^{AB}+P_{1,-1}^{AB}=1$ and $P_{1,1}^{AB}>P_{1,-1}^{AB}>0$ appears
only in zone II, this demonstrates that all the A-atoms are lying along the $%
Z$-axis, while a larger (smaller) part of the B-atoms are lying along
(opposite to) the $Z$-axis. The case that $P_{1,1}^{AB}+P_{1,-1}^{AB}=1$ and
$P_{1,-1}^{AB}>P_{1,1}^{AB}>0$ appears only in zone IV, and so on.

(2) The case $c_{A2}<0$ and $c_{B2}<0$

It is shown in Fig.\ref{fig2} that, when the two intra-species
spin-dependent interactions are both attractive, both kinds of species will
keep the ferro-phase disregarding how $c_{AB2}$ is. Nonetheless, the two
groups of aligned atoms will point to the same direction if $c_{AB2}<0$
(i.e., the g.s. has $S=N_{A}+N_{B}$ and is in parallel f-f phase), or point
to opposite directions if $c_{AB2}>0$ (i.e., the g.s. has $S=|N_{A}-N_{B}|$
and is in anti-parallel f-f phase). This is confirmed by $P_{\mu }^{X}$ and $%
P_{\mu \nu }^{AB}$ as shown in Fig.\ref{fig6} and \ref{fig7}. Obviously, $%
c_{AB2}=0$ is a critical point where the transition $%
(N_{A},N_{B},N_{A}+N_{B})\rightarrow (N_{A},N_{B},|N_{A}-N_{B}|)$ occurs.

(3) The case $c_{A2}>0$ and $c_{B2}<0$

It is shown in Fig.\ref{fig3} that $S_{B}=N_{B}$ holds in the whole range of
$c_{AB2}$. Thus the B-atoms are in the ferro-phase and very stable against $%
c_{AB2}$. When $c_{AB2}=0$ the A-atoms are in the polar phase and,
accordingly, $P_{\mu }^{A}=1/3$ as shown in Fig.\ref{fig8}. However, the
polar phase is extremely fragile against $c_{AB2}$. Once $|c_{AB2}|$
deviates from zero, $P_{0}^{A}$ falls into zero in an extremely narrow
domain surrounding $c_{AB2}=0$ (refer to the sub-figure inside Fig.\ref{fig8}%
). Remind that $P_{0}^{A}=0$ implies the emergence of all the $s$-pairs.
Thus all the $s$-pairs will be completely broken
outside the extremely narrow domain where the free
A-atoms are either have $\mu =1$ or $\mu =-1$. When $c_{AB2}$ increases
further and arrives at $p_{2}$ (marked in Fig.\ref{fig8}), $%
P_{-1}^{A}$ increases from $1/3$ to 1 while $P_{1}^{A}$ decreases from $1/3$
to 0. Accordingly, $S_{A}$ increases from 0 to $N_{A}$ but lying along the $%
-Z$-axis. When $c_{AB2}\geq P_{2}$, all the A-atoms are aligned and lying
along the opposite direction of the B-atoms. Thus, $p_{2}$ is
a turning point, and it marks the boundary of the anti-parallel f-f phase,
which is stable against the further increase of $c_{AB2}$. Similarly,
$p_{1}$ is also a turning point, and it marks the boundary of
the parallel f-f phase.

Comparing Fig.\ref{fig1} (where the zone III for the $(0,0,0)$ texture is
broad) and Fig.\ref{fig3} (where the domain for $(0,0,0)$ is only a point),
we know that the stability of the $s$-pairs of the A-atoms depends on the
status of the B-atoms. If both are in $s$-pairs, then both are stable until $%
|c_{AB2}|$ exceeds a critical value. Alternatively, if the B-atoms are
aligned, the A-atom will be attracted to align along the same direction (if $%
c_{AB2}<0$) or the opposite direction (if $c_{AB2}>0$) with them. This leads
to the extremely fragile $(0,0,0)$ texture.



\section*{Summary}

We have solved the CGP numerically to obtain the solutions for 2-species
spin-1 BEC. The knowledge of the spin-textures is extracted from these
solutions. The results qualitatively support the previous results based on
the single-spatial mode approximation. Thus, this approximation is confirmed
to be applicable in the qualitative sense. The variation of the
spin-textures is further analyzed via the 1-body and 2-body probability
densities $P_{\mu }^{X}$ and $P_{\mu \nu }^{AB}$. The analysis leads to the
following understanding on the physical picture of the spin-texture. Recall
that the $Z$-axis is chosen to be lying along $S$.

Remind that the g.s. of the 1-species spin-1 BEC has two well known phases,
the polar and ferromagnetic phases. Thus, for 2-species, it is naturally to
guess that there would be three phases: polar-polar, polar-ferro, and f-f
phases. However, by solving the CGP, we know that the polar-polar phase
exists only if both species are polar in nature (i.e., $c_{A2}>0$ and $%
c_{B2}>0$) and $|c_{AB2}|$ is smaller than a critical value.

The presumed polar-ferro phase does not exist (unless $c_{AB2}=0$). This is
because the polar phase will become extremely fragile when it is influenced
by a nearby ferro phase even when $|c_{AB2}|$ is very weak. The aligned
atoms in ferro phase (say, the B-atoms) lure the atoms in polar phase (say,
the A-atoms) to align with them along the same or opposite direction (refer
to Fig.\ref{fig3} and Fig.\ref{fig8}). Consequently, all the $s$-pairs of
the A-atoms will be broken when $|c_{AB2}|$ deviates from zero. This is
confirmed by the exact solution of the CGP, which leads to $P_{0}^{A}=0$
(whereas if the $s$-pairs survive, $P_{0}^{A}$ cannot be zero). After being
free from the $s$-pairs, the A-atoms will fall into a quasi-ferro (qf) phase
in which the atoms are divided into two parts lying parallel and
anti-parallel to the direction of $S$, respectively. When $c_{AB2}<0$, more
A-atoms are in the former part. When $c_{AB2}$ decreases further, this part
will become bigger and bigger until all A-atoms are parallel to the $Z$%
-axis. Whereas when $c_{AB2}>0$, more A-atoms are in the latter part. Thus,
instead of the polar-ferro phase, the 2-BEC has the qf-ferro phase. The main
feature of the qf phase is the vanish of the $\mu =0$ component (as shown
in Fig.\ref{fig8}).

Finally, the existence of the presumed f-f phase is confirmed. It can be
further divided into parallel f-f phase and antiparallel f-f phase depending
on the relative orientation of the spins of the two species.

The above results from 2-species BEC give a hint to multi-species BEC. Say,
for 3-species BEC, it is guessed that the polar-polar-polar could emerge
only if all intra-species spin-dependent interactions are repulsive and the
strengths of the inter-species spin-dependent interactions fall inside a
limited scope, the polar-f-f phase does not exist but replaced by qf-f-f
phase, and there would be f-f-f phase with two groups of spins lying along a
direction while the third group lying along the opposite direction, and so
on. These suggestions remain to be checked.

Due to the existence of different phases, phase transitions are inevitable.
It turns out that all the critical points can be sensitively reflected in
the probability densities $P_{\mu}^X$ and $P_{\mu\nu}^{XX^{\prime }}$. The
theoretical and experimental studies on these quantities will deepen the
understanding of the spin-textures, and provide a way for determining the
parameters involved. In particular, the spin-textures are found to be
sensitive to the spin-dependent forces, the parameters of these very weak
forces could be thereby determined.



\section*{Appendix}

Let $\vartheta_{S,M}^N$ be the normalized all-symmetric total spin-state of $%
N$ spin-1 atoms of the same kind, all the spins are coupled to $S$ and $M$.
Then, $\chi(i)$ (the spin-state of the $i$-th particle) can be extracted as%
\cite{bao05,bao06}
\begin{equation}
\vartheta_{SM}^N = a_S^N [\chi(i)\vartheta_{S+1}^{N-1}]_{SM} +b_S^N [\chi
(i)\vartheta_{S-1}^{N-1}]_{SM},  \label{fpc}
\end{equation}
where $a_S^N$ and $b_S^N$ are the fractional parentage coefficients for
extracting one particle. They appear as
\begin{eqnarray}
&&a_S^N=\sqrt{\frac{[1+(-1)^{N-S}](N-S)(S+1)}{2N(2S+1)}},  \nonumber \\
&&b_S^N=\sqrt{\frac{[1+(-1)^{N-S}]S(N+S+1)}{2N(2S+1)}},  \label{ab}
\end{eqnarray}
where $\vartheta_{S\pm 1}^{N-1}$ are also normalized and all-symmetric but
for $N-1$ spins coupled to $S\pm 1$.

Let the $i$-th and $j$-th spins be coupled to $\lambda$, and the coupled
spin-state is denoted by $[\chi(i)\chi(j)]_{\lambda}$. Then this pair can be
extracted as
\begin{equation}
\vartheta_{SM}^N = \sum_{\lambda S^{\prime }} h_{\lambda S^{\prime }S}^N \{
[\chi(i)\chi(j)]_{\lambda }\vartheta_{S^{\prime }}^{N-2} \}_{SM},
\label{tswf}
\end{equation}
where the coefficients $h_{\lambda S^{\prime }S}^N$ are the fractional
parentage coefficients for extracting two particles. They are
\begin{eqnarray}
h_{0,SS}^N &=&[\frac{(N+S+1)(N-S)}{3N(N-1)}]^{1/2},  \nonumber \\
h_{2,S+2,S}^N &=&[\frac{(S+1)(S+2)(N-S)(N-S-2)}{(2S+1)(2S+3)N(N-1)}]^{1/2},
\nonumber \\
h_{2,S,S}^N &=&[\frac{S(2S+2)(N-S)(N+S+1)}{3(2S-1)(2S+3)N(N-1)}]^{1/2},
\nonumber \\
h_{2,S-2,S}^N &=&[\frac{S(S-1)(N+S+1)(N+S-1)}{(2S-1)(2S+1)N(N-1)}]^{1/2}.
\label{h2}
\end{eqnarray}
All the other $h_{\lambda S^{\prime }S}^N$ are zero. Furthermore, when $S=0$%
, $h_{2,S,S}^N$ and $h_{2,S-2,S}^N$ should be zero.



\begin{thebibliography}{99}
\bibitem{ref5} Stamper-Kurn, D. M., Andrews, M. R., Chikkatur, A. P.,
Inouye, S., Miesner, H. J., Stenger, J., and Ketterle, W., Optical
Confinement of a Bose-Einstein Condensate, \textit{Phys. Rev. Lett.} \textbf{%
80}, 2027 (1998).

\bibitem{ref6} Ho, T. L., Spinor Bose Condensates in Optical Traps, \textit{%
Phys. Rev. Lett.} \textbf{81}, 742 (1998).

\bibitem{ref7} Law, C. K., Pu, H., and Bigelow, N. P., Quantum Spins Mixing
in Spinor Bose-Einstein Condensates, \textit{Phys. Rev. Lett.} \textbf{81},
5257 (1998).

\bibitem{ref8} Goldstein, Elena V., and Meystre, Pierre, Quantum theory of
atomic four-wave mixing in Bose-Einstein condensates, \textit{Phys. Rev. A}
\textbf{59}, 3896 (1999).

\bibitem{ref9} Ho, T. L., and Yip, S. K., Fragmented and Single Condensate
Ground States of Spin-1 Bose Gas, \textit{Phys. Rev. Lett.} \textbf{84},
4031 (2000).

\bibitem{ref10} Koashi, M., and Ueda, M., Exact Eigenstates and Magnetic
Response of Spin-1 and Spin-2 Bose-Einstein Condensates, \textit{Phys. Rev.
Lett.} \textbf{84}, 1066 (2000).

\bibitem{ml} Luo, M., Li, Z. B., and Bao, C. G., Bose-Einstein condensate of
a mixture of two species of spin-1 atoms, \textit{Phys. Rev. A} \textbf{75},
043609 (2007).

\bibitem{ref11} Xu, Z. F., Zhang, Yunbo, and You, L., Binary mixture of
spinor atomic Bose-Einstein condensates, \textit{Phys. Rev. A} \textbf{79},
023613 (2009).

\bibitem{ref12} Shi,Yu, Ground states of a mixture of two species of spinor
Bose gases with interspecies spin exchange, \textit{Phys. Rev. A} \textbf{82}%
, 023603 (2010).

\bibitem{ref13} Xu, Z. F., L\"{u}, R., and You, L., Quantum entangled ground
states of two spinor Bose-Einstein condensates, \textit{Phys. Rev. A}
\textbf{84}, 063634 (2011).

\bibitem{ref14} Shi, Yu, and Ge, Li, Three-dimensional quantum phase diagram
of the exact ground states of a mixture of two species of spin-1 Bose gases
with interspecies spin exchange, \textit{Phys. Rev. A} \textbf{83}, 013616
(2011).

\bibitem{ref15} Shi, Yu, and Ge, Li, Ground states of a mixture of two
species of spin-1 Bose gases with interspecies spin exchange in a magnetic
field, \textit{Int. J. Mod. Phys. B} \textbf{26}, 1250002 (2012).

\bibitem{ref16} Xu, Z. F., Mei, J. W., L\"{u}, R., and You, L.,
Spontaneously axisymmetry-breaking phase in a binary mixture of spinor
Bose-Einstein condensates, \textit{Phys. Rev. A} \textbf{82}, 053626 (2010).

\bibitem{ref17} Zhang, J., Li, T. T., and Zhang, Yunbo, Interspecies singlet
pairing in a mixture of two spin-1 Bose condensates, \textit{Phys. Rev. A}
\textbf{83}, 023614 (2011).

\bibitem{ref18} Irikura, Naoki, Eto, Yujiro, Hirano, Takuya, and Saito,
Hiroki, Ground-state phases of a mixture of spin-1 and spin-2 Bose-Einstein
condensates, \textit{Phys. Rev. A} \textbf{97}, 023622 (2018).

\bibitem{bao05} Bao, C. G., and Li, Z. B., First excited band of a spinor
Bose-Einstein condensate, \textit{Phys. Rev. A} \textbf{72}, 043614 (2005).

\bibitem{bao06} Bao., C. G., ne-body and Two-body Fractional Parentage
Coefficients for Spinor Bose-Einstein Condensation, \textit{Front. Phys.
China} \textbf{1}, 92-96 (2006)

\bibitem{katr} Katriel, J., Weights of the total spins for systems of
permutational symmetry adapted spin-1 particles, \textit{Journal of
Molecular Structure: THEOCHEM} \textbf{547}, 1-11 (2001).

\bibitem{zbli} Li, Z. B., Liu, Y. M., Yao, D. X., and Bao, C. G., Two types
of phase diagrams for two-species Bose-Einstein condensates and the combined
effect of the parameters, \textit{J. Phys. B: At. Mol. Opt. Phys.} \textbf{50%
}, 135301 (2017).

\bibitem{liu} Liu, Y. M., He, Y. Z., and Bao, C. G., Singularity in the
matrix of the coupled Gross-Pitaevskii equations and the related
state-transitions in three-species condensates, \textit{Scientific reports}
\textbf{7}, 6585 (2017).
\end{thebibliography}



\section*{Acknowledgements}

Supported by the National Natural Science Foundation of China under Grants
No.11372122, 11274393, 11574404, and 11275279; the Open Project Program of
State Key Laboratory of Theoretical Physics, Institute of Theoretical
Physics, Chinese Academy of Sciences, China(No.Y4KF201CJ1); the National
Basic Research Program of China (2013CB933601); and the
Natural Science Foundation of Guangdong of China (2016A030313313).



\section*{Author contributions}

Y. Z. He is responsible to the numerical calculation. Y. M. Liu is
responsible to the theoretical derivation. C. G. Bao provides the idea,
write the paper, and responsible to the whole paper. All authors reviewed
the manuscript.



\section*{Additional information}

\textbf{Competing Interests:} The authors declare that they have no
competing interests.

\end{document}
