\documentclass[pre,
twocolumn,
%reprint
%longbibliography,
%groupedaddress
]{revtex4-1}
\usepackage{graphicx}
\usepackage{amsmath, amssymb, amsthm}
\usepackage{epsfig, color}
\usepackage{systeme}
\usepackage{multirow, comment, enumerate}
\usepackage{hyperref}
\usepackage{enumitem}
\usepackage{esint}
\usepackage{lipsum}

\definecolor{myblue}{RGB}{0,50,200}
\hypersetup{
	colorlinks,
	citecolor=myblue,
	linkcolor=myblue,
	urlcolor=myblue
}

\newcommand{\overbar}[1]{\mkern 1.5mu\overline{\mkern-1.5mu#1\mkern-1.5mu}\mkern 1.5mu}

\allowdisplaybreaks

\theoremstyle{definition}
\newtheorem{theorem}{\textit{Theorem}}
\theoremstyle{definition}
\newtheorem{definition}{\textit{Definition}}

\newcommand{\bol}{\boldsymbol}
\newcommand{\mca}{\mathcal}
\newcommand{\mbb}{\mathbb}
\newcommand{\mrm}{\mathrm}
\newcommand{\eq}[1]{\begin{equation}#1\end{equation}}
\newcommand{\eqs}[1]{\begin{equation*}#1\end{equation*}}
\newcommand{\sube}[2]{\begin{subequations}#1\label{#2}\end{subequations}}
\newcommand{\al}[1]{\begin{align}#1\end{align}}
\newcommand{\als}[1]{\begin{align*}#1\end{align*}}
\newcommand{\aln}[1]{\begin{aligned}#1\end{aligned}}
\newcommand{\avg}[1]{\langle #1\rangle}
\newcommand{\avgb}[1]{\big\langle #1\big\rangle}
\newcommand{\avgB}[1]{\Big\langle #1\Big\rangle}
\newcommand{\avgbb}[1]{\bigg\langle #1\bigg\rangle}
\newcommand{\avgl}[1]{\left\langle #1\right\rangle}
\newcommand{\set}[1]{\lbrace #1\rbrace}
\newcommand{\inl}[1]{$ #1 $}
\newcommand{\bra}[1]{\left( #1 \right)}
\newcommand{\bras}[1]{\left[ #1 \right]}
\newcommand{\brab}[1]{\left\{ #1 \right\}}
\newcommand{\arcsinh}{\text{arcsinh}}
\newcommand{\norm}[1]{\Vert #1 \Vert}
% japanese characters
\usepackage[whole]{bxcjkjatype}
% convenience commands for comments
\definecolor{mygreen}{RGB}{0, 100, 0}
\newcommand{\com}[1]{\textcolor{red}{#1}}
\newcommand{\rep}[1]{\textcolor{mygreen}{#1}}
\newcommand{\sol}[1]{\textcolor{blue}{#1}}
\usepackage{soul}

\begin{document}
\title{Thermodynamic uncertainty relation for time-delayed Langevin systems}
\author{Tan Van Vu}
\email{tan@biom.t.u-tokyo.ac.jp}
\author{Yoshihiko Hasegawa}
\email{hasegawa@biom.t.u-tokyo.ac.jp}
\affiliation{Department of Information and Communication Engineering, Graduate School of Information Science and Technology, The University of Tokyo, Tokyo 113-8656, Japan}

%\date{\today}

\begin{abstract}
Thermodynamic uncertainty relation which states a universal trade-off between the fluctuation of a non-vanishing current and the dissipation has been found for various Markovian systems.
However, such relation remains unveiled in the cases of non-Markovian systems.
We study here thermodynamic uncertainty relation for time-delayed Langevin systems.
We find that for a finite observation time the fluctuation of a non-vanishing current at the steady state is constrained by a generalized dissipation, which is equal to the total entropy production in the absence of time delay.
We show that the rate of this dissipation is always positive and bounded from below by the square of arbitrary current in the system.
We numerically verify the uncertainty relation and show its validity in three systems: periodic Brownian motion, simple two-dimensional model, and Brusselator system.
\end{abstract}

\pacs{}
\maketitle

\section{Introduction}\label{sect.1}
During the last two decades, stochastic thermodynamics (ST) has been developed into a consistent framework for describing small systems, which are fluctuating and far from thermal equilibrium \cite{Sekimoto.1998.PTPS,Denis.2002.AP,Seifert.2005.PRL,Seifert.2012.RPP,Yannick.2015.PA,Jaco.2016.EPL,Gong.2016.PRL,Goldt.2017.PRL}.
The traditional thermodynamic quantities, e.g., work, heat, and entropy production, have been generalized to the level of stochastic trajectories \cite{Sekimoto.1998.PTPS,Seifert.2005.PRL}.
The first and second laws were extended to be applicable for both ensemble and individual trajectory levels.
ST has been successfully applied to study many systems such as optical system \cite{Tietz.2006.PRL}, colloidal particle \cite{Blickle.2006.PRL}, and biochemical reaction networks \cite{Schmiedl.2007.JCP,Schmiedl.2007.JSP}.

In recent years, thermodynamic uncertainty relation (TUR) which unifies two features of non-equilibrium systems: non-vanishing currents and non-zero dissipation, has been discovered in various Markovian systems.
Specifically, TUR provides that for a system in a non-equilibrium steady-state with observation time \inl{\mca{T}}, the ratio of the variance \inl{\mrm{Var}[\Theta]} and square of the mean \inl{\avg{\Theta}^2} of an arbitrary non-vanishing current \inl{\Theta} integrated up to time \inl{\mca{T}} is bounded from below by the reciprocal of the total entropy production \inl{\avg{\Sigma}} accumulated by \inl{\mca{T}} as
\eq{
\frac{\mrm{Var}[\Theta]}{\avg{\Theta}^2}\ge\frac{2k_{\mrm{B}}}{\avg{\Sigma}}.\label{eq.unc.rel}
}
In other words, this relation shows that there exists a trade-off between the precision and the thermodynamic cost.
Such relation has analogously appeared in many other contexts \cite{Mehta.2012.PNAS,Hasegawa.2018.PRE}.
TUR was first studied for biomolecular processes in \cite{Barato.2015.PRL} and proved for steady-state multicyclic networks within linear response theory.
However, numerical pieces of evidence \cite{Barato.2015.PRL} have suggested that this relation holds even beyond linear response.
With the aid of the large deviation theory, TUR for long observation time has been derived for steady-state currents in Markov jump processes \cite{Gingrich.2016.PRL}, which are arbitrarily far from equilibrium.
Since then, several variations of TUR have been proposed \cite{Pietzonka.2016.PRE}.
Reference \cite{Polettini.2016.PRE} provided a tighter bound for thermodynamically consistent currents of overdamped diffusion processes, in which \inl{\avg{\Sigma}} in Eq.~\eqref{eq.unc.rel} is replaced by the minimum entropy production that can be attained by a system that sustains the current \inl{\Theta}.
Various forms of TUR for long observation time have been proposed in other contexts such as enzyme kinetics \cite{Barato.2015.JPCB}, self-propelled colloidal particles \cite{Falasco.2016.PRE}, molecular motors \cite{Patrick.2016.JSM}, stochastic pumps \cite{Rotskoff.2017.PRE}, first passage time problems \cite{Garrahan.2017.PRE,Gingrich.2017.PRL}, Brownian motion \cite{Hyeon.2017.PRE}, discrete time stochastic processes \cite{Karel.2017.EPL,Chiuchiu.2018.PRE}, ballistic transport \cite{Brandner.2018.PRL}, and biological motors \cite{Hwang.2018.JPCL}.
Numerical experiments \cite{Pietzonka.2017.PRE} suggested that TUR still holds in the case of finite observation time.
After that, again by using the tool of large deviation theory, Ref.~\cite{Horowitz.2017.PRE} provided proof of finite-time TUR for steady-state current fluctuations in Markov jump processes.
By bounding the generating function, Ref.~\cite{Andreas.2018.JSM} showed that TUR is also valid for the continuous state-space systems described by Langevin equations.
For the special case of the fluctuating entropy production, i.e., \inl{\Theta=\Sigma}, TUR reduces to \inl{\mrm{Var}[\Sigma]/\avg{\Sigma}\ge 2k_{\mrm{B}}}, which has been derived in \cite{Pigolotti.2017.PRL}.
TUR also can be re-derived from an information inequality, which shows a promising connection between stochastic thermodynamics and statistical inference theory \cite{Hasegawa.2018.arxiv}.

Till now, TUR has been studied only for Markovian systems.
However, delays which cause non-Markovian behavior are inevitable in many real-world stochastic processes such as biological systems \cite{Bratsun.2005.PNAS,Gupta.2013.PRL} and time-delayed feedback \cite{Kim.1999.PRL,Masoller.2003.PRL,Lichtner.2012.PRE}.
Therefore, it is pertinent to ask: Does TUR still hold for non-Markovian systems?
In the present paper, we address this fundamental question for time-delayed Langevin systems.
Intriguingly, we find that for finite observation time at a steady state the fluctuation of an arbitrary current is bounded from below by the reciprocal of a generalized dissipation \inl{\avg{\Sigma_{\mrm{g}}}}, which is same with entropy production in the absence of time delay.
This generalized quantity satisfies the fluctuation theorem and hence the second law, i.e., \inl{\avg{\Sigma_{\mrm{g}}}\ge 0} \cite{Jiang.2011.PRE}.
We also show a stronger entropic bound for \inl{\avg{\Sigma_{\mrm{g}}}} that it is constrained by the square of the magnitude of arbitrary current in the system.
In the presence of time delay, \inl{\avg{\Sigma_{\mrm{g}}}} is different from the total entropy production, which can be negative due to entropy flux of delayed feedback.
We numerically validate our result in three models: periodic Brownian motion, simple two-dimensional model, and chemical oscillating reaction called Brusselator system.


\section{Model}\label{sect.2}
We consider a time-delayed system with dynamical variables \inl{\bol{x}(t)=[x_1(t),\dots,x_N(t)]^\top}, described by a set of coupled Langevin equations as follows:
\eq{
\dot{\bol{x}} = \bol{F}(\bol{x},\bol{x}_\tau)+\sqrt{2D}\bol{\xi},\label{eq.del.Lan}
}
where \inl{\bol{x}_\tau\equiv\bol{x}(t-\tau)}, \inl{\bol{F}(\bol{x},\bol{x}_\tau)\in\mbb{R}^N} is a drift force, \inl{\bol{\xi}(t)=[\xi_1(t),\dots,\xi_N(t)]^\top} is white Gaussian noise with correlation \inl{\avg{\xi_i(t)\xi_j(t')}=\delta_{ij}\delta(t-t')}, and \inl{\tau\ge 0} denotes the delay time of the system.
The noise intensity \inl{D} is related to temperature \inl{T} as \inl{D=k_{\rm B}T}.
To make the entropy dimensionless, we set hereinafter \inl{k_{\rm B}=1}.
Eq.~\eqref{eq.del.Lan} is interpreted as Ito stochastic integration.
Let \inl{P(\bol{x},t)} be the probability density function (PDF) for the system to be in state \inl{\bol{x}} at time \inl{t}.
Then the corresponding Fokker-Planck equation (FPE) is \cite{Frank.2003.PRE,Frank.2005.PRE}
\al{
\partial_t P(\bol{x},t)=&-\sum_{i=1}^N\partial_{x_i}\bras{\int F_i(\bol{x},\bol{y}) P(\bol{y},t-\tau;\bol{x},t)d\bol{y}}\nonumber\\
&+D\sum_{i=1}^N\partial_{x_i}^2P(\bol{x},t)\equiv -\sum_{i=1}^N\partial_{x_i}J_i(\bol{x},t),
}
where \inl{P(\bol{y},t-\tau;\bol{x},t)} is a joint probability density that the system takes value \inl{\bol{x}} at time \inl{t} and \inl{\bol{y}} at time \inl{t-\tau}, and
\als{
J_i(\bol{x},t) &=\int F_i(\bol{x},\bol{y})P(\bol{y}, t-\tau;\bol{x},t)d\bol{y}-D\partial_{x_i}P(\bol{x},t)\\
&\equiv \overbar{F}_i(\bol{x}) P(\bol{x},t) - D\partial_{x_i}P(\bol{x},t),
}
is the probability current. Here,
\eq{
\overbar{F}_i(\bol{x}) = \int F_i(\bol{x},\bol{y})P(\bol{y},t-\tau |\bol{x},t)d\bol{y}
}
is a delay-averaged force.
The FPE can be rewritten as
\eq{
\partial_tP(\bol{x},t) =-\sum_{i=1}^N\partial_{x_i}[\overbar{F}_i(\bol{x})P(\bol{x},t)] + D\sum_{i=1}^N\partial_{x_i}^2P(\bol{x},t).\label{eq.FokPla.equ}
}
Now, we consider a stochastic trajectory \inl{\Gamma=\{\bol{x}(t)\}} that starts from \inl{t=t_s} and ends at \inl{t=t_e}.
Analogous to the case of non-delayed system, the system entropy can be defined as \inl{S(t)=-\ln P(\bol{x}(t),t)}.
Thus, the change of system entropy is \inl{\Delta S=-\ln{P(\bol{x}(t_e),t_e)}+\ln{P(\bol{x}(t_s),t_s)}}.
The change of medium entropy can be expressed via heat dissipation \inl{\mca{Q}[\Gamma]} as \inl{\Delta S_{\rm m}=\mca{Q}[\Gamma]/T}, where 
\eq{
\mca{Q}[\Gamma]=\int_{t_s}^{t_e} \bol{F}(\bol{x},\bol{x}_\tau)^\top\dot{\bol{x}}dt.
}
Then one obtains the change of total entropy as
\eq{
\Delta S^{\rm tot}\equiv\Delta S+\Delta S^{\rm m}.
}
However, due to the delay-induced entropy flux, the ensemble average of \inl{\Delta S^{\rm tot}} can be negative, i.e., \inl{\avg{\Delta S^{\rm tot}}<0} \cite{Jiang.2011.PRE}.
The agent that operates the time-delayed feedback can be considered as a kind of Maxwell demon.
We define a delay-averaged heat disspation
\eq{
\overbar{\mca{Q}}[\Gamma]=\int_{t_s}^{t_e} \overbar{\bol{F}}(\bol{x})^\top\dot{\bol{x}}dt,
}
and a generalized dissipation \inl{\Delta S_{\rm g}^{\rm tot}} as
\eq{
\Delta S_{\rm g}^{\rm tot}=\Delta S+\overbar{\mca{Q}}[\Gamma]/T.
}
Since
\eq{
\avg{\dot{S}_{\rm g}^{\rm tot}}=\int\frac{\norm{\bol{J}(\bol{x},t)}^2}{DP(\bol{x},t)}d\bol{x}\geq 0,
}
it is guaranteed that \inl{\avg{\Delta S_{\rm g}^{\rm tot}}} is always non-negative.
Here \inl{\norm{.}} denotes the Euclidean norm, i.e., \inl{\norm{\bol{x}}=\sqrt{x_1^2+\dots+x_N^2}}.
We note that in the absence of time delay, \inl{\Delta S_{\rm g}^{\rm tot}=\Delta S^{\rm tot}} and thus, the inequality \inl{\avg{\Delta S_{\rm g}^{\rm tot}}\ge 0} can be viewed as the generalized second law.
Moreover, \inl{\Delta S_{\rm g}^{\rm tot}} also satisfies the integral fluctuation theorem \cite{Jiang.2011.PRE}, i.e., \inl{\avg{e^{-\Delta S_{\rm g}^{\rm tot}}}=1}.

Now, for an arbitrary trajectory \inl{\Gamma}, we consider a generalized current \inl{\Theta[\Gamma]} induced by the dynamics of Eq.~\eqref{eq.del.Lan} as follows:
\eq{
\Theta[\Gamma]=\int_{t_s}^{t_e}\bol{\Lambda}(\bol{x})^\top\circ\dot{\bol{x}}dt,
}
where \inl{\bol{\Lambda}(\bol{x})\in\mbb{R}^N} is an arbitrary projection function and ``\inl{\circ}''-symbol indicates the Stratonovich product.
Depending on the form of \inl{\bol{\Lambda}(\bol{x})}, such a current can be the travel distance of a particle or the total entropy production.
From the definition, the ensemble average of \inl{\dot{\Theta}} is 
\eq{
\avg{\dot{\Theta}}=\int\bol{\Lambda}(\bol{x})^\top\bol{J}(\bol{x},t)d\bol{x}.
}
By applying Cauchy-Schwartz inequality twice, we obtain
\als{
\avg{\dot{\Theta}}^2&=\bra{\int\bol{\Lambda}(\bol{x})^\top\bol{J}(\bol{x},t)d\bol{x}}^2\\
&\le\bra{\sum_{i=1}^N\sqrt{\int DP(\bol{x},t)\Lambda_i(\bol{x})^2d\bol{x}\int \frac{J_i(\bol{x},t)^2}{DP(\bol{x},t)}d\bol{x}}}^2\\
&\le\int DP(\bol{x},t)\norm{\bol{\Lambda}(\bol{x})}^2d\bol{x}\int \frac{\norm{\bol{J}(\bol{x},t)}^2}{DP(\bol{x},t)}d\bol{x}.
}
This is equivalent to \inl{\avg{\dot{S}_{\rm g}^{\rm tot}}\avg{D\norm{\bol{\Lambda}}^2}\ge\avg{\dot{\Theta}}^2} and implies that the average of generalized dissipation rate is bounded by the magnitude of any arbitrary current.
When delay time \inl{\tau=0}, this inequality reduces to an entropic bound for non-delayed Langevin systems \cite{Andreas.2018.PRE}.

\section{Thermodynamic uncertainty relation}
From now on, we assume that the system is in a steady state with distribution \inl{P^{\rm ss}(\bol{x})} and current \inl{\bol{J}^{\rm ss}(\bol{x})}.
Suppose that we observe the current \inl{\Theta[\Gamma]} with observation time \inl{\mca{T}}, namely \inl{t_e-t_s=\mca{T}}.
Our goal is to derive a relation between the fluctuation of this current and the average of generalized dissipation \inl{\avg{\Sigma_{\mrm{g}}}} given by
\eq{
\avg{\Sigma_{\mrm{g}}}\equiv\avg{\Delta S_{\rm g}^{\rm tot}}=\mca{T}\avg{\dot{S}_{\rm g}^{\rm tot}}=\mca{T}\int\frac{\norm{\bol{J}^{\rm ss}(\bol{x})}^2}{DP^{\rm ss}(\bol{x})}d\bol{x}.\label{eq.gen.dis.def}
}
At the steady state, the average of \inl{\Theta} is 
\eq{
\avg{\Theta}=\int_{t_s}^{t_e}\avg{\bol{\Lambda}(\bol{x})^\top\dot{\bol{x}}}dt=\mca{T}\int\bol{\Lambda}(\bol{x})^\top\bol{J}^{\rm ss}(\bol{x})d\bol{x}.
}
Our derivation is based on Ref.~\cite{Andreas.2018.JSM}.
To derive a bound for the fluctuation of the current \inl{\Theta[\Gamma]}, we consider its scaled cumulant generating function
\eq{
\chi(k,\Theta)=\ln{\avgl{e^{k\Theta[\Gamma]}}}=\ln\bra{\int e^{k\Theta[\Gamma]}\mca{P}(\Gamma)\mca{D}\Gamma},\label{eq.gen.fun.def}
}
where \inl{\mca{P}(\Gamma)} is the probability of observing the trajectory \inl{\Gamma}.
Given the validity of the FPE in Eq.~\eqref{eq.FokPla.equ}, \inl{\mca{P}(\Gamma)} can be expresses as \cite{Risken.1989}
\al{
\mca{P}(\Gamma)&=\mca{N}\exp\bra{-\int_{t_s}^{t_e}\mca{A}(\bol{x}(t),\dot{\bol{x}}(t))dt},\label{eq.pat.int}\\
\mca{A}(\bol{x}(t),\dot{\bol{x}}(t))&\equiv\frac{1}{4D}\norm{\dot{\bol{x}}-\overbar{\bol{F}}}^2,
}
where \inl{\mca{A}(\bol{x},\dot{\bol{x}})} is a stochastic action and \inl{\mca{N}} is a normalization constant.
The continuous-time integral in Eq.~\eqref{eq.pat.int} is interpreted as the limit of a discrete sum, using pre-point (Ito) discretization.
The cross terms such as \inl{\int\overbar{\bol{F}}(\bol{x})^\top\dot{\bol{x}}dt} should be interpreted as \inl{\int\overbar{\bol{F}}(\bol{x})^\top\bullet\dot{\bol{x}}dt}, where ``\inl{\bullet}''-symbol denotes the Ito product.
Although the action \inl{\mca{A}(\bol{x},\dot{\bol{x}})} in pre- and mid-point discretization appears to be different, we can show that both of the discretizations reduces to the same result (see Appendix \ref{app.pat.int}).
Now we consider a modified dynamics
\eq{
\dot{\bol{x}}=\overbar{\bol{F}}(\bol{x})+\bol{A}(\bol{x})+\sqrt{2D}\bol{\xi},
}
where
\eq{
\bol{A}(\bol{x})=ck\frac{\bol{J}^{\rm ss}(\bol{x})}{P^{\rm ss}(\bol{x})},
}
and \inl{c} is an arbitrary constant.
We note that the modification of the drift terms of Eq.~\eqref{eq.del.Lan} is analogous to the selection of optimal current in the tool of large deviation theory.
The FPE corresponding to the modified dynamics is
\al{
\partial_t P_{\bol{A}}(\bol{x},t)=&-\sum_{i=1}^N\partial_{x_i}\bras{(\overbar{F}_i(\bol{x})+A_i(\bol{x}))P_{\bol{A}}(\bol{x},t)}\nonumber\\
&+D\sum_{i=1}^N\partial_{x_i}^2P_{\bol{A}}(\bol{x},t)\label{eq.FokPla.mod.dyn}.
}
It can be proved that the steady-state distribution \inl{P_{\bol{A}}^{\rm ss}(\bol{x})} of the modified dynamics is same with the original one, i.e., \inl{P_{\bol{A}}^{\rm ss}(\bol{x})=P^{\rm ss}(\bol{x})} (see Appendix \ref{app.ste.sta.mod.dyn}).
The probability current of modified dynamics \inl{\bol{J}_{\bol{A}}^{\rm ss}} has the relation \inl{\bol{J}_{\bol{A}}^{\rm ss}(\bol{x})=(1+ck)\bol{J}^{\rm ss}(\bol{x})}.
Plugging the equality \inl{\Theta[\Gamma]=\int_{t_s}^{t_e}\dot{\Theta}dt} into Eq.~\eqref{eq.gen.fun.def}, we can prove that (see Appendix \ref{app.pro.bou.gen.fun})
\eq{
\chi(k,\Theta)\ge k\avgl{\int_{t_s}^{t_e}\dot{\Theta}dt}_{\bol{A}}-\frac{1}{4D}\avgl{\int_{t_s}^{t_e}\norm{A}^2dt}_{\bol{A}},
}
where \inl{\avg{..}_{\bol{A}}} denotes the average over all the realizations of modified dynamics.
Since
\begin{widetext}
\al{
\avgl{\int_{t_s}^{t_e}\dot{\Theta}dt}_{\bol{A}}&=\avgl{\int_{t_s}^{t_e}\bol{\Lambda}(\bol{x})^\top\dot{\bol{x}}dt}_{\bol{A}}=(1+ck)\mca{T}\int\bol{\Lambda}(\bol{x})\bol{J}^{\rm ss}(\bol{x})d\bol{x}=(1+ck)\avg{\Theta},\\
\frac{1}{4D}\avgl{\int_{t_s}^{t_e}\norm{\bol{A}}^2dt}_{\bol{A}}&=\frac{\mca{T}}{4D}\int \norm{\bol{A}(\bol{x})}^2P_{\bol{A}}^{\rm ss}(\bol{x})d\bol{x}=\frac{(ck)^2\mca{T}}{4}\int \frac{\norm{\bol{J}^{\rm ss}(\bol{x})}^2}{DP^{\rm ss}(\bol{x})}d\bol{x}=\frac{(ck)^2\avg{\Sigma_{\mrm{g}}}}{4},
}
\end{widetext}
we obtain the following inequality
\eq{
\chi(k,\Theta)\geq (1+ck)k\avg{\Theta}-\frac{(ck)^2\avg{\Sigma_{\mrm{g}}}}{4}.\label{eq.ine1}
}
To maximize the right-hand side of Eq.~\eqref{eq.ine1}, we choose \inl{c=2\avg{\Theta}/\avg{\Sigma_{\rm g}}}, and thus obtain a lower bound of generating function as
\eq{
\chi(k,\Theta)\ge k\avg{\Theta}+\frac{k^2\avg{\Theta}^2}{\avg{\Sigma_{\mrm{g}}}}.\label{eq.bou.gen.fun}
}
The relation of the variance of \inl{\Theta} and the generating function is given by
\eq{
{\rm Var}[\Theta]=\left.\frac{d^2\chi(k,\Theta)}{dk^2}\right|_{k=0}.
}
Since \inl{\chi(0,\Theta)=0,~\partial_k\chi(k,\Theta)|_{k=0}=\avg{\Theta}}, we only need to compare the coefficients of the second derivative of both sides in Eq.~\eqref{eq.bou.gen.fun} at \inl{k=0}.
Consequently, TUR for delayed Langevin system is obtained as
\eq{
\frac{{\rm Var}[\Theta]}{\avg{\Theta}^2}\ge\frac{2}{\avg{\Sigma_{\mrm{g}}}}.\label{eq.TUR}
}
This inequality is our main result.
In the absence of time delay, this relation reduces to the conventional TUR which was derived in the literature. Thus, our result can be considered as a generalized TUR for Langevin systems.

\section{Examples}
We now numerically validate TUR in three models: periodic Brownian motion, simple two-dimensional model, and Brusselator system.
At the steady state, the generalized dissipation \inl{\avg{\Sigma_{\rm g}}} can be expressed as \inl{\avg{\Sigma_{\mrm{g}}}=\avg{\int_{t_s}^{t_e}\overbar{\bol{F}}(\bol{x})^\top\dot{\bol{x}}dt}}.
Since calculating \inl{\avg{\Sigma_{\mrm{g}}}} as Eq.~\eqref{eq.gen.dis.def} is difficult, we use this equation to numerically evaluate \inl{\avg{\Sigma_{\mrm{g}}}} in all three models.

\subsection{Periodic Brownian motion}
We consider a Brownian motion of a particle in one-dimensional periodic potential as follows \cite{Juniper.2016.PRE}:
\eq{
\dot{x}=F(x,x_{\tau})+\sqrt{2D}\xi,\label{eq.Lan.exa1}
}
where \inl{F(x,x_\tau)=a\sin\bra{2\pi x/L}+b\sin\bra{2\pi x_\tau/L}+c}, and \inl{\xi} is white Gaussian noise.
Here \inl{a,b,c,L>0} are constants.
The FPE corresponding to Eq.~\eqref{eq.Lan.exa1} is
\eq{
\partial_tP(x,t)=-\partial_x\bras{\overbar{F}(x)P(x,t)}+D\partial_x^2P(x,t),\label{eq.FokPla.exa1}
}
where the delay-averaged force \inl{\overbar{F}(x)} is
\eq{
\overbar{F}(x)=\int F(x,y)P(y,t-\tau|x,t)dy.\label{eq.Fba.def.exa1}
}
Equation~\eqref{eq.FokPla.exa1} has a periodic steady-state solution \inl{P^{\rm ss}(x)}, i.e., \inl{P^{\rm ss}(x+L)=P^{\rm ss}(x)}.
To evaluate the generalized dissipation \inl{\avg{\Sigma_{\mrm{g}}}}, one first must determine \inl{\overbar{F}(x)}.
However, due to unknown distribution \inl{P(y,t-\tau|x,t)}, it is a nontrivial task to obtain an explicit form of \inl{\overbar{F}(x)}.
Fortunately, for sufficient small \inl{\tau}, we can use short time propagator to approximate the transition probability \inl{P(y,t-\tau|x,t)} as \cite{Risken.1989,Frank.2005.PRE}
\eq{
P(y,t-\tau|x,t)=\frac{1}{\sqrt{4\pi D\tau}}\exp\bra{-\frac{(y-x-F(x,x)\tau)^2}{4D\tau}}.\label{eq.pro.tra.app.exa1}
}
Plugging Eq.~\eqref{eq.pro.tra.app.exa1} to Eq.~\eqref{eq.Fba.def.exa1}, we obtain the explicit form of \inl{\overbar{F}(x)} as
\al{
\overbar{F}(x)&=a\sin(2\pi x/L)+be^{-(2\pi/L)^2D\tau}\nonumber\\
&\times\sin\bra{\frac{2\pi}{L}\bras{x+((a+b)\sin(2\pi x/L)+c)\tau}}+c.\label{eq.Fba.ana.exa1}
}
As a confirmation, in the limit \inl{\tau\to 0} we have \inl{\overbar{F}(x)\to F(x,x)}, which is consistent with Eq.~\eqref{eq.Fba.def.exa1}.
At the steady-state, the probability current is a constant and satisfies the following relation:
\eq{
J^{\rm ss}=\overbar{F}(x)P^{\rm ss}(x)-D\partial_xP^{\rm ss}(x).
}
\begin{figure}[t]
	\centering
	\includegraphics[width=0.23\textwidth]{FIG1a.pdf}
	\includegraphics[width=0.23\textwidth]{FIG1b.pdf}
	\includegraphics[width=0.23\textwidth]{FIG1c.pdf}
	\includegraphics[width=0.23\textwidth]{FIG1d.pdf}
	\caption{(a)(b) Steady-state PDFs \inl{P^{\rm ss}(x)} of Brownian motion. The parameters are \inl{a=1,L=2\pi}, (a) \inl{b=c=0.1,D=0.2,\tau=0.1}, and (b) \inl{b=c=0.5,D=0.5,\tau=0.5}. The (blue) solid line represents the theoretical result defined in Eq.~\eqref{eq.PDF.exa1} and the (orange) dots express the numerical result of that. (c) Numerical verification of TUR. For each random trajectory generated by Eq.~\eqref{eq.Lan.exa1}, we plot the fluctuation of current \inl{\mrm{Var}[\Theta]/\avg{\Theta}^2} as a function of \inl{\avg{\Sigma_{\rm g}}}. The dashed line is \inl{2/\avg{\Sigma_{\rm g}}}, which is the saturating case of TUR. The parameters are \inl{a\in[0.1,1],b,c\in[0.01,0.5],D\in[0.05,0.5],\tau\in[0.05,0.5],\mca{T}\in[0.5,5]}, and \inl{L=2\pi}. (d) Relation between generalized dissipation \inl{\avg{\Sigma_{\rm g}}} and medium entropy production \inl{\avg{\Delta S^{\rm m}}}.}\label{fig.exa1}
\end{figure}
Noting that \inl{P^{\rm ss}(x)} is bounded when \inl{x\to\pm\infty}, we can explicitly obtain the stationary distribution \inl{P^{\rm ss}(x)} as \cite{Hyeon.2017.PRE}:
\eq{
P^{\rm ss}(x)=\frac{J^{\rm ss}}{D}e^{-\phi(x)/D}\bra{\frac{\varphi_+(L)}{1-e^{-cL/D}}-\varphi_{-}(x)},\label{eq.PDF.exa1}
}
where 
\eqs{
\phi(x)=-\int_0^x\overbar{F}(x') dx',~\varphi_{\pm}(x)=\int_{0}^xe^{\pm \phi(x')/D}dx'.
}
The current \inl{J^{\rm ss}} is determined by using the normalization condition \inl{\int_0^LP^{\rm ss}(x)dx=1} and is given by
\eq{
J^{\rm ss}=\frac{D(1-e^{-cL/D})}{\varphi_{+}(L)\varphi_{-}(L)-(1-e^{-cL/D})\int_0^Le^{-\phi(x)/D}\varphi_{+}(x)dx}.
}
First, we verify the validity of the approximation based on short time propagator.
We numerically solve Eq.~\eqref{eq.Lan.exa1} with time step \inl{\Delta t=1.0\times 10^{-3}} and collect \inl{1.6\times 10^7} realizations to obtain the steady-state PDF.
We compare the analytical and numerical results for two cases: \inl{\tau=0.1} and \inl{\tau=0.5}.
The results are plotted in Figs.~\ref{fig.exa1}(a)(b) and other parameters are described in the caption of that.
The results show that the analytical and numerical PDFs are identical for both cases of delay time.
This implies that short time propagator yields a good approximation when delay time \inl{\tau} is small.
Next, we validate TUR for the projection function \inl{\Lambda(x)\equiv 1}.
In this case, the current \inl{\Theta=\int_{t_s}^{t_e}\dot{x}dt} corresponds to the postision of the particle.
Using Monte Carlo simulations, we numerically evaluate the thermodynamic cost \inl{\avg{\Sigma_{\rm g}}} and the moments of the current \inl{\Theta}, i.e., variance \inl{\mrm{Var}[\Theta]} and mean \inl{\avg{\Theta}}.
We randomly select \inl{a,b,c,D,\tau,\mca{T}}, and repeat simulations for \inl{2.0\times 10^6} times at each of the selected parameter settings (ranges of the random parameters are shown in the caption of Fig.~\ref{fig.exa1}(c)).
Figure~\ref{fig.exa1}(c) plots \inl{\mrm{Var}[\Theta]/\avg{\Theta}^2} as a function of \inl{\avg{\Sigma_{\rm g}}}, where points denote random calculations of the ratio and the dashed line denotes \inl{2/\avg{\Sigma_{\rm g}}} corresponding to the saturating case of Eq.~\eqref{eq.TUR}.
It can be seen that all the points are located above the line, which empirically verifies TUR.
We also compare \inl{\avg{\Sigma_{\rm g}}} with medium entropy production \inl{\avg{\Delta S^{\rm m}}} in Fig.~\ref{fig.exa1}(d).
Here, \inl{\avg{\Delta S^{\rm m}}} is evaluated as \inl{\avg{\Delta S^{\rm m}}=\avg{\int_{t_s}^{t_e}F(x,x_\tau)\dot{x}/Tdt}}.
For all parameter settings, we plot the value of \inl{\avg{\Sigma_{\rm g}}} as a function of \inl{\avg{\Delta S^{\rm m}}}, where points represent random calculations and the dashed line corresponds to the case that \inl{\avg{\Sigma_{\rm g}}=\avg{\Delta S^{\rm m}}}.
We can see that for selected parameters, all points are located below the line, which implies that the generalized dissipation is smaller than heat dissipation.

\subsection{A simple two-dimensional system}
Next, we consider a simple two-dimensional system described by the following coupled equations:
\eq{
\dot{\bol{x}}=\bol{F}(\bol{x},\bol{x}_\tau)+\sqrt{2D}\bol{\xi},\label{eq.Lan.exa2}
}
where
\eq{
\bol{F}(\bol{x},\bol{x}_\tau)=\begin{bmatrix}
-ax_1+bx_{2,\tau}\\
-ax_2-bx_{1,\tau}
\end{bmatrix}
,
}
and \inl{\bol{\xi}} is white Gaussian noise with correlation \inl{\avg{\xi_i(t)\xi_j(t')}=\delta_{ij}\delta(t-t')}.
Here, \inl{x_{i,\tau}\equiv x_i(t-\tau)} and \inl{a,b>0} are given constants.
The system is manipulated under a parabolic potential with linear delay-feedback.
The corresponding FPE is
\eq{
\partial_tP(\bol{x},t)=-\sum_{i=1}^2\partial_{x_i}[\overbar{F}_i(\bol{x})P(\bol{x},t)]+D\sum_{i=1}^2\partial_{x_i}^2P(\bol{x},t),\label{eq.FokPla.exa2}
}
where \inl{\overbar{F}_i(\bol{x})=\int F_i(\bol{x},\bol{y})P(\bol{y},t-\tau|\bol{x},t)d\bol{y}}.
Analogously, for small delay time \inl{\tau}, the transition probability can be approximated by the short time propagator
\als{
P(\bol{y},t-\tau|\bol{x},t)&=\frac{1}{\sqrt{(4\pi D\tau)^2}}\times\\
&\exp\bra{-\frac{\norm{\bol{y}-\bol{x}-\bol{F}(\bol{x},\bol{x})\tau}^2}{4D\tau}}.
}
Then, the delay-averaged force \inl{\overbar{F}_1(x),\overbar{F}_2(x)} is approximately calculated as follows:
\als{
\overbar{F}_1(\bol{x})&=-(a+b^2\tau)x_1+(b-ab\tau)x_2,\\
\overbar{F}_2(\bol{x})&=-(a+b^2\tau)x_2-(b-ab\tau)x_1.
}
For convenience, we set \inl{A\equiv a+b^2\tau,B\equiv b-ab\tau}.
By assuming that the steady-state PDF \inl{P^{\rm ss}(\bol{x})} is a function of \inl{\norm{\bol{x}}}, we obtain the steady-state solution of Eq.~\eqref{eq.FokPla.exa2} as
\eq{
P^{\rm ss}(\bol{x})=\frac{A}{2\pi D}e^{-A\norm{\bol{x}}^2/2D}.\label{eq.PDF.exa2}
}
\begin{figure}[t]
	\centering
	\includegraphics[width=0.23\textwidth]{FIG2a.pdf}
	\includegraphics[width=0.23\textwidth]{FIG2b.pdf}
	\includegraphics[width=0.23\textwidth]{FIG2c.pdf}
	\includegraphics[width=0.23\textwidth]{FIG2d.pdf}
	\caption{(a)(b) Steady-state marginal PDFs \inl{P^{\rm ss}(x_i)}, where (a) \inl{i=1} and (b) \inl{i=2}. The parameters are \inl{a=0.5,b=0.1,D=0.5}, and \inl{\tau=0.5}. The (blue) solid line represents the analytical result defined in Eq.~\eqref{eq.PDF.exa2} and the (orange) dots express the numerical result. (c) Numerical verification of TUR. For each random trajectory generated by Eq.~\eqref{eq.Lan.exa2}, the fluctuation of current \inl{\mrm{Var}[\Theta]/\avg{\Theta}^2} is plotted as a function of \inl{\avg{\Sigma_{\rm g}}}. The dashed line is \inl{2/\avg{\Sigma_{\rm g}}}, which is the saturating case of Eq.~\eqref{eq.TUR}. The parameters are \inl{a\in[0.1,0.5],b\in[0.01,0.1],D\in[0.05,0.5],\tau\in[0.05,0.5]}, and \inl{\mca{T}\in[0.5,5]}. (d) Comparision between generalized dissipation \inl{\avg{\Sigma_{\rm g}}} and medium entropy production \inl{\avg{\Delta S^{\rm m}}}.}\label{fig.exa2}
\end{figure}
To check the goodness of approximation based on short time propagator for small delay time, we run simulation for Eq.~\eqref{eq.Lan.exa2} to obtain the PDF \inl{P^{\rm ss}(\bol{x})} and compare with the analytical result defined in Eq.~\eqref{eq.PDF.exa2}.
The number of realizations is \inl{1.6\times 10^7} and the time resolution is \inl{\Delta t=1.0\times 10^{-3}}.
The marginal PDFs \inl{P^{\rm ss}(x_1)} and \inl{P^{\rm ss}(x_2)} are plotted in Figs.~\ref{fig.exa2}(a)(b) for the case of \inl{\tau=0.5}.
The other parameters are shown in the corresponding caption.
We can see that analytical and numerical results are consistent, which assures the validity of approximation of \inl{\overbar{\bol{F}}(\bol{x})}.
Next, we validate TUR for the current \inl{\Theta=\int_{t_s}^{t_e} \left\{(-ax_1+bx_2)\dot{x}_1+(-ax_2-bx_1)\dot{x}_2\right\}dt}.
When the noise and time delay are small, this current is positive. 
Using Monte Carlo simulations, we numerically evaluate the fluctuation of current \inl{\mrm{Var}[\Theta]/\avg{\Theta}^2} and the thermodynamic cost \inl{\avg{\Sigma_{\rm g}}}.
We randomly select \inl{a,b,D,\tau,\mca{T}}, and repeat simulations for \inl{2.0\times 10^6} times at each of the selected parameter setting (ranges of the random parameters are shown in the caption
of Figure~\ref{fig.exa2}(c)).
In Fig.~\ref{fig.exa2}(c), we plot \inl{\mrm{Var}[\Theta]/\avg{\Theta}^2} as a function of \inl{\avg{\Sigma_{\rm g}}}.
The meaning of dots and dashed line are the same as in Fig.~\ref{fig.exa1}.
As can be seen, all the points are above the line, which numerically validates Eq.~\eqref{eq.TUR}.
Again, generalized dissipation \inl{\avg{\Sigma_{\rm g}}} and medium entropy production \inl{\Delta S^{\rm m}} are compared in Fig.~\ref{fig.exa2}(d).
We can see that for the selected range of parameters, although the generalized dissipation is smaller than the medium entropy production, they are very close to each other.
This shows a strong correlation between \inl{\avg{\Sigma_{\rm g}}} and \inl{\avg{\Delta S^{\rm m}}} in this system.

\subsection{Brusselator system}
We here study a non-Markovian Brusselator system, which is a paradigmatic model for oscillations in chemical reaction systems. The system is described as follows:
\eq{
\dot{\bol{x}}=\bol{F}(\bol{x},\bol{x}_\tau)+\sqrt{2D}\bol{\xi},\label{eq.Lan.exa3}
}
where
\eq{
\bol{F}(\bol{x},\bol{x}_\tau)=\begin{bmatrix}
1-(b+1)x_1+cx_{1,\tau}^2x_{2,\tau}\\
bx_1-cx_1^2x_2
\end{bmatrix}
,
}
and \inl{b,c\in\mbb{R}_{>0}} are positive constants.
The definition of \inl{x_{i,\tau}} is analogous as in Eq.~\eqref{eq.Lan.exa2}.
We note that this model is not associated with a real-world chemical system, but instead in order to illustrate the validity of TUR.
The corresponding FPE is
\eq{
\partial_tP(\bol{x},t)=-\sum_{i=1}^2\partial_{x_i}[\overbar{F}_i(\bol{x})P(\bol{x},t)]+D\sum_{i=1}^2\partial_{x_i}^2P(\bol{x},t),\label{eq.FokPla.exa3}
}
where \inl{\overbar{F}_i(\bol{x})=\int F_i(\bol{x},\bol{y})P(\bol{y},t-\tau|\bol{x},t)d\bol{y}}.
Analogously, we can approximate \inl{\overbar{F}_i(\bol{x})} for small \inl{\tau} and obtain
\eq{
\aln{
\overbar{F}_1(\bol{x})&=1-(b+1)x_1+c(2D\tau+(x_1+F_1(\bol{x},\bol{x})\tau)^2)\\
&\times(x_2+F_2(\bol{x},\bol{x})\tau),\\
\overbar{F}_2(\bol{x})&=bx_1-cx_1^2x_2.
}\label{eq.Fbar.exam3}
}
\begin{figure}[t]
	\centering
	\includegraphics[width=0.23\textwidth]{FIG3a.pdf}
	\includegraphics[width=0.23\textwidth]{FIG3b.pdf}
	\includegraphics[width=0.23\textwidth]{FIG3c.pdf}
	\includegraphics[width=0.23\textwidth]{FIG3d.pdf}
	\caption{(a)(b) Steady-state marginal PDFs \inl{P^{\rm ss}(x_i)}, where (a) \inl{i=1} and (b) \inl{i=2}. The parameters are \inl{b=c=0.1,D=0.5}, and \inl{\tau=0.1}. The (blue) solid line and (orange) dots represents the numerical results of the dynamics generated by Eq.~\eqref{eq.Lan.exa3} and Eq.~\eqref{eq.FokPla.exa3} with approximated \inl{\overbar{\bol{F}}(\bol{x})}, respectively. (c) Numerical verification of TUR. For each random trajectory generated by Eq.~\eqref{eq.Lan.exa3}, the fluctuation of current \inl{\mrm{Var}[\Theta]/\avg{\Theta}^2} is plotted as a function of \inl{\avg{\Sigma_{\rm g}}}. The dashed line is \inl{2/\avg{\Sigma_{\rm g}}}, which is the saturating case of Eq.~\eqref{eq.TUR}. The parameters are \inl{b=c\in[0.01,0.1],D\in[0.05,0.5],\tau\in[0.05,0.1]}, and \inl{\mca{T}\in[0.5,5]}. (d) Relation between generalized dissipation \inl{\avg{\Sigma_{\rm g}}} and medium entropy production \inl{\avg{\Delta S^{\rm m}}}.}\label{fig.exa3}
\end{figure}
We note that in Eq.~\eqref{eq.Fbar.exam3}, \inl{\overbar{F}_1(\bol{x})} is approximative while \inl{\overbar{F}_2(\bol{x})} is exact one since \inl{F_2(\bol{x},\bol{x}_\tau)} does not involve any delay terms.
Even though explicit expression of \inl{\overbar{\bol{F}}(\bol{x})} is available, it is still difficult to analytically calculate the steady-state solution of Eq.~\eqref{eq.FokPla.exa3}.
Therefore, to evaluate the accuracy of approximation based on short time propagator for small delay time, we numerically solve both Eqs.~\eqref{eq.Lan.exa3} and \eqref{eq.FokPla.exa3} to obtain the PDF \inl{P^{\rm ss}(\bol{x})} and compare these results.
Again, the number of realizations and the time resolution used in the simulation are \inl{1.6\times 10^7} and \inl{1.0\times 10^{-3}}, respectively.
Figures \ref{fig.exa3}(a)(b) plot the marginal PDFs \inl{P^{\rm ss}(x_1)} and \inl{P^{\rm ss}(x_2)} for the case of \inl{\tau=0.1}.
The other parameters are \inl{b=c=0.1}, and \inl{D=0.5}.
As can be seen, short time propagator yields a good approximation for the PDFs of original dynamics.
The PDFs generated by Eqs.~\eqref{eq.Lan.exa3} and \eqref{eq.FokPla.exa3} with approximated \inl{\overbar{\bol{F}}(\bol{x})} are consistent.
Next, we validate TUR for the current 
\eq{
\Theta=\int_{t_s}^{t_e}\begin{bmatrix}
1-(b+1)x_1+cx_{1}^2x_{2}\\
bx_1-cx_1^2x_2
\end{bmatrix}^\top\dot{\bol{x}}dt.
}
The projection function \inl{\bol{\Lambda}(\bol{x})} is equal to \inl{\bol{F}(\bol{x},\bol{x})}, thus the current is guaranteed to be positive when the noise and time delay are small. 
%Using Monte Carlo simulations, we numerically evaluate the fluctuation of current \inl{\mrm{Var}[\Theta]/\avg{\Theta}^2} and the thermodynamic cost \inl{\avg{\Sigma_{\rm g}}}.
We randomly select \inl{b,c,D,\tau,\mca{T}}, and repeat simulations for \inl{2.0\times 10^6} times at each of the selected parameter setting (ranges of the random parameters are shown in the caption
of Figure~\ref{fig.exa3}(c)).
We plot \inl{\mrm{Var}[\Theta]/\avg{\Theta}^2} as a function of \inl{\avg{\Sigma_{\rm g}}} in Fig.~\ref{fig.exa3}(c), where the points and the dashed line have same meaning as in Fig.~\ref{fig.exa1}.
As can be seen, all the points lie above the line and Eq.~\eqref{eq.TUR} is empirically verified.
In Fig.~\ref{fig.exa3}(d), we compare \inl{\avg{\Sigma_{\rm g}}} with \inl{\Delta S^{\rm m}}.
%The value of \inl{\avg{\Sigma_{\rm g}}} is plotted as a function of \inl{\avg{\Delta S^{\rm m}}}, where the dots are random calculations and the dashed line expresses the case that \inl{\avg{\Sigma_{\rm g}}=\avg{\Delta S^{\rm m}}}.
We can see that while the the generalized dissipation is always non-negative, the medium entropy production can be negative for some parameter settings, which implies the violation of the second law.
It also can be observed that \inl{\avg{\Sigma_{\rm g}}} is always larger than \inl{\avg{\Delta S^{\rm m}}} in this system.

\section{Conclusions}\label{sect.Conclusion}
In summary, we have presented TUR for time-delayed Langevin systems.
We showed that the fluctuation of an arbitrary non-vanishing current is bounded from below by the reciprocal of the generalized dissipation \inl{\avg{\Sigma_{\rm g}}}.
This generalized dissipation is always non-negative and satisfies the integral fluctuation.
We also proved that the rate of \inl{\avg{\Sigma_{\rm g}}} is not only positive but bounded from below by the square of the magnitude of any current in the delayed system.
In the absence of time delay, \inl{\avg{\Sigma_{\rm g}}} becomes the total entropy production, and our TUR recovers the conventional TUR of non-delayed Langevin systems.
The derived TUR holds on arbitrary time scales in a non-equilibrium steady state.
Our result complements the universal relations that characterize the non-equilibrium systems.
This generalization of TUR for delayed Langevin systems can be used as a tool for estimating a hidden thermodynamic quantity of real-world systems involving time delay from finite-time experimental data.


\section*{Acknowledgment}
This work was supported by MEXT KAKENHI Grant No. JP16K00325.

\appendix
\section{Equivalent between pre- and mid-point discretization}\label{app.pat.int}
There are two typical schemes of discretization in path integral: pre-point which is used in this paper, and mid-point discretizations.
Following the Ref.~\cite{Adib.2008.JPCB}, we show that both discretizations reduce to the same path
integral representation for additive noise systems.

For simplicity, we explain here for a univariate case.
The generalization for a multivariate case is entirely analogous.
Both of the representations are given by
\al{
\mathcal{P}_{\rm p}(\Gamma)&=\mca{N}\exp\left(-\int_{t_s}^{t_e}\frac{\left(\dot{x}-F(x)\right)^{2}}{4D}dt\right),\label{eq.pre.poi.pat.int}\\
\mathcal{P}_{\rm m}(\Gamma)&=\mca{N}\exp\left(-\int_{t_s}^{t_e}\left[\frac{\left(\dot{x}-F(x)\right)^{2}}{4D}+\frac{1}{2}\partial_xF(x)\right]dt\right),\label{eq.mid.poi.pat.int}
}
where \inl{\mathcal{P}_{\rm p},\mathcal{P}_{\rm m}} correspond to pre- and mid-point scheme, respectively.
Equation~\eqref{eq.mid.poi.pat.int} contains an additional term \inl{\partial_xF(x)}, which makes the difference with Eq.~\eqref{eq.pre.poi.pat.int}.
However, in Eqs.~\eqref{eq.mid.poi.pat.int} and \eqref{eq.pre.poi.pat.int}, the cross term \inl{\int F(x)\dot{x}dt} should be interpreted differently for pre- and mid-point discretizations: \inl{\int F(x)\dot{x}dt} is equal to \inl{\int F(x)\bullet\dot{x}dt} for pre-point and to \inl{\int F(x)\circ\dot{x}dt} for mid-point.
The mid- and pre-point schemes can be converted via the following relation:
\eq{
\int F(x)\circ\dot{x}dt=\int F(x)\bullet\dot{x}dt+\int D\partial_xF(x)dt.\label{eq.mid.pre.con}
}
By substituting Eq.~\eqref{eq.mid.pre.con} into Eq.~\eqref{eq.mid.poi.pat.int}, we easily obtain that Eq.~\eqref{eq.mid.poi.pat.int} is the same as Eq.~\eqref{eq.pre.poi.pat.int}.
Therefore, it can be concluded that both of the discretizations reduce to the same path-integral representation.

\section{Steady-state distribution of modified dynamics}\label{app.ste.sta.mod.dyn}
Substituting \inl{P_{\bol{A}}(\bol{x},t)=P^{\rm ss}(\bol{x})} into Eq.~\eqref{eq.FokPla.mod.dyn}, we have
\als{
&\sum_{i=1}^N\partial_{x_i}\bras{(\overbar{F}_i(\bol{x},t)+A_i(\bol{x},t))P_{\bol{A}}(\bol{x},t)}+D\sum_{i=1}^N\partial_{x_i}^2P_{\bol{A}}(\bol{x},t)\\
&=-\sum_{i=1}^N\partial_{x_i}J_i^{\rm ss}(\bol{x})-\sum_{i=1}^N\partial_{x_i}\bras{A_i(\bol{x})P^{\rm ss}(\bol{x})}\\
&=-\sum_{i=1}^N\partial_{x_i}J_i^{\rm ss}(\bol{x})-\sum_{i=1}^Nck\partial_{x_i}J_i^{\rm ss}(\bol{x})\\
&=-(1+ck)\sum_{i=1}^N\partial_{x_i}J_i^{\rm ss}(\bol{x})=0.
}
This implies that the steady-state distribution \inl{P_{\bol{A}}^{\rm ss}(\bol{x})} remains unchanged, i.e., \inl{P_{\bol{A}}^{\rm ss}(\bol{x})=P^{\rm ss}(\bol{x})}.

\section{Proof of the bound on generating function}\label{app.pro.bou.gen.fun}
The probability \inl{\mca{P}_{\bol{A}}(\Gamma)} of observing a trajectory \inl{\Gamma} in modified dynamics can be derived analogously from the FPE defined in Eq.~\eqref{eq.FokPla.mod.dyn} and has the following form:
\al{
&\mca{P}_{\bol{A}}(\Gamma)=\mca{N}_{\bol{A}}\exp\bra{-\int_{t_s}^{t_e}\mca{A}_{\bol{A}}(\bol{x}(t),\dot{\bol{x}}(t))dt},\\
&\mca{A}_{\bol{A}}(\bol{x}(t),\dot{\bol{x}}(t))\equiv\frac{1}{4D}\norm{\dot{\bol{x}}-\overbar{\bol{F}}-\bol{A}}^2,
}
where \inl{\mca{A}_{\bol{A}}} is a stochastic action associated with modified dynamics and \inl{\mca{N}_{\bol{A}}} is a normalizing constant.
Since the normalizing constant does not depend on the drift term, we have \inl{\mca{N}_{\bol{A}}=\mca{N}}.
The action \inl{\mca{A}} can be rewritten in term of \inl{\mca{A}_{\bol{A}}} as
\eq{
\mca{A}(\bol{x},\dot{\bol{x}})=\mca{A}_{\bol{A}}(\bol{x},\dot{\bol{x}})+\frac{1}{4D}\bra{2\bol{A}^\top(\dot{\bol{x}}-\overbar{\bol{F}}-\bol{A})+\norm{\bol{A}}^2}.
}
Now, we transform the generating function \inl{\chi(k,\Theta)} and obtain a lower bound as follows:
\begin{widetext}
\als{
\chi(k,\Theta)&=\ln\bra{\int\mca{N}\exp\bras{\int_{t_s}^{t_e}\brab{k\dot{\Theta}-\mca{A}}dt}\mca{D}\Gamma}\\
%&=\ln\bra{\int\mca{N}\exp\bra{\int_{t_s}^{t_e}\bra{k\dot{\Theta}-\mca{A}_{\bol{A}}-\frac{1}{4D}\bra{2\bol{A}^\top(\dot{\bol{x}}-\overbar{\bol{F}}-\bol{A})+\norm{\bol{A}}^2}}dt}\mca{D}\Gamma}\\
&=\ln\bra{\int\exp\bras{\int_{t_s}^{t_e}\brab{k\dot{\Theta}-\frac{1}{4D}\bra{2\bol{A}^\top(\dot{\bol{x}}-\overbar{\bol{F}}-\bol{A})+\norm{\bol{A}}^2}}dt}\mca{P}_{\bol{A}}(\Gamma)\mca{D}\Gamma}\\
&\ge\int\bra{\int_{t_s}^{t_e}\bras{k\dot{\Theta}-\frac{1}{4D}\brab{2\bol{A}^\top(\dot{\bol{x}}-\overbar{\bol{F}}-\bol{A})+\norm{\bol{A}}^2}}dt}\mca{P}_{\bol{A}}(\Gamma)\mca{D}\Gamma.
}
\end{widetext}
The lower bound in last line is obtained by applying Jensen's inequality to the logarithm.
The term \inl{\bol{A}^\top\bra{\dot{\bol{x}}-\overbar{\bol{F}}-\bol{A}}} becomes \inl{\sqrt{2D}\bol{A}^\top\bol{\xi}} and vanishes due to pre-point discretization as follows:
\eq{
\avgB{\int\sqrt{2D}\bol{A}^\top\bol{\xi}dt}=\avgB{\lim_{\Delta t\to 0}\sum_{i}\sqrt{2D}\bol{A}(\bol{x}_{i-1})^\top\bol{\xi}_{i}\Delta t}=0.
}
Thus, we obtain the inequality
\eq{
\chi(k,\Theta)\ge k\avgl{\int_{t_s}^{t_e}\dot{\Theta}dt}_{\bol{A}}-\frac{1}{4D}\avgl{\int_{t_s}^{t_e}\norm{A}^2dt}_{\bol{A}}.
}

\begin{thebibliography}{46}%
	\makeatletter
	\providecommand \@ifxundefined [1]{%
		\@ifx{#1\undefined}
	}%
	\providecommand \@ifnum [1]{%
		\ifnum #1\expandafter \@firstoftwo
		\else \expandafter \@secondoftwo
		\fi
	}%
	\providecommand \@ifx [1]{%
		\ifx #1\expandafter \@firstoftwo
		\else \expandafter \@secondoftwo
		\fi
	}%
	\providecommand \natexlab [1]{#1}%
	\providecommand \enquote  [1]{``#1''}%
	\providecommand \bibnamefont  [1]{#1}%
	\providecommand \bibfnamefont [1]{#1}%
	\providecommand \citenamefont [1]{#1}%
	\providecommand \href@noop [0]{\@secondoftwo}%
	\providecommand \href [0]{\begingroup \@sanitize@url \@href}%
	\providecommand \@href[1]{\@@startlink{#1}\@@href}%
	\providecommand \@@href[1]{\endgroup#1\@@endlink}%
	\providecommand \@sanitize@url [0]{\catcode `\\12\catcode `\$12\catcode
		`\&12\catcode `\#12\catcode `\^12\catcode `\_12\catcode `\%12\relax}%
	\providecommand \@@startlink[1]{}%
	\providecommand \@@endlink[0]{}%
	\providecommand \url  [0]{\begingroup\@sanitize@url \@url }%
	\providecommand \@url [1]{\endgroup\@href {#1}{\urlprefix }}%
	\providecommand \urlprefix  [0]{URL }%
	\providecommand \Eprint [0]{\href }%
	\providecommand \doibase [0]{http://dx.doi.org/}%
	\providecommand \selectlanguage [0]{\@gobble}%
	\providecommand \bibinfo  [0]{\@secondoftwo}%
	\providecommand \bibfield  [0]{\@secondoftwo}%
	\providecommand \translation [1]{[#1]}%
	\providecommand \BibitemOpen [0]{}%
	\providecommand \bibitemStop [0]{}%
	\providecommand \bibitemNoStop [0]{.\EOS\space}%
	\providecommand \EOS [0]{\spacefactor3000\relax}%
	\providecommand \BibitemShut  [1]{\csname bibitem#1\endcsname}%
	\let\auto@bib@innerbib\@empty
	%</preamble>
	\bibitem [{\citenamefont {Sekimoto}(1998)}]{Sekimoto.1998.PTPS}%
	\BibitemOpen
	\bibfield  {author} {\bibinfo {author} {\bibfnamefont {K.}~\bibnamefont
			{Sekimoto}},\ }\href {\doibase 10.1143/PTPS.130.17} {\bibfield  {journal}
		{\bibinfo  {journal} {Prog. Theor. Phys. Supp.}\ }\textbf {\bibinfo {volume}
			{130}},\ \bibinfo {pages} {17} (\bibinfo {year} {1998})}\BibitemShut
	{NoStop}%
	\bibitem [{\citenamefont {Evans}\ and\ \citenamefont
		{Searles}(2002)}]{Denis.2002.AP}%
	\BibitemOpen
	\bibfield  {author} {\bibinfo {author} {\bibfnamefont {D.~J.}\ \bibnamefont
			{Evans}}\ and\ \bibinfo {author} {\bibfnamefont {D.~J.}\ \bibnamefont
			{Searles}},\ }\href {\doibase 10.1080/00018730210155133} {\bibfield
		{journal} {\bibinfo  {journal} {Adv. Phys.}\ }\textbf {\bibinfo {volume}
			{51}},\ \bibinfo {pages} {1529} (\bibinfo {year} {2002})}\BibitemShut
	{NoStop}%
	\bibitem [{\citenamefont {Seifert}(2005)}]{Seifert.2005.PRL}%
	\BibitemOpen
	\bibfield  {author} {\bibinfo {author} {\bibfnamefont {U.}~\bibnamefont
			{Seifert}},\ }\href {\doibase 10.1103/PhysRevLett.95.040602} {\bibfield
		{journal} {\bibinfo  {journal} {Phys. Rev. Lett.}\ }\textbf {\bibinfo
			{volume} {95}},\ \bibinfo {pages} {040602} (\bibinfo {year}
		{2005})}\BibitemShut {NoStop}%
	\bibitem [{\citenamefont {Seifert}(2012)}]{Seifert.2012.RPP}%
	\BibitemOpen
	\bibfield  {author} {\bibinfo {author} {\bibfnamefont {U.}~\bibnamefont
			{Seifert}},\ }\href {http://stacks.iop.org/0034-4885/75/i=12/a=126001}
	{\bibfield  {journal} {\bibinfo  {journal} {Rep. Prog. Phys.}\ }\textbf
		{\bibinfo {volume} {75}},\ \bibinfo {pages} {126001} (\bibinfo {year}
		{2012})}\BibitemShut {NoStop}%
	\bibitem [{\citenamefont {Decker}(2015)}]{Yannick.2015.PA}%
	\BibitemOpen
	\bibfield  {author} {\bibinfo {author} {\bibfnamefont {Y.~D.}\ \bibnamefont
			{Decker}},\ }\href {\doibase 10.1016/j.physa.2015.01.073} {\bibfield
		{journal} {\bibinfo  {journal} {Physica A}\ }\textbf {\bibinfo {volume}
			{428}},\ \bibinfo {pages} {178 } (\bibinfo {year} {2015})}\BibitemShut
	{NoStop}%
	\bibitem [{\citenamefont {Fuchs}\ \emph {et~al.}(2016)\citenamefont {Fuchs},
		\citenamefont {Goldt},\ and\ \citenamefont {Seifert}}]{Jaco.2016.EPL}%
	\BibitemOpen
	\bibfield  {author} {\bibinfo {author} {\bibfnamefont {J.}~\bibnamefont
			{Fuchs}}, \bibinfo {author} {\bibfnamefont {S.}~\bibnamefont {Goldt}}, \ and\
		\bibinfo {author} {\bibfnamefont {U.}~\bibnamefont {Seifert}},\ }\href
	{http://stacks.iop.org/0295-5075/113/i=6/a=60009} {\bibfield  {journal}
		{\bibinfo  {journal} {EPL}\ }\textbf {\bibinfo {volume} {113}},\ \bibinfo
		{pages} {60009} (\bibinfo {year} {2016})}\BibitemShut {NoStop}%
	\bibitem [{\citenamefont {Gong}\ \emph {et~al.}(2016)\citenamefont {Gong},
		\citenamefont {Lan},\ and\ \citenamefont {Quan}}]{Gong.2016.PRL}%
	\BibitemOpen
	\bibfield  {author} {\bibinfo {author} {\bibfnamefont {Z.}~\bibnamefont
			{Gong}}, \bibinfo {author} {\bibfnamefont {Y.}~\bibnamefont {Lan}}, \ and\
		\bibinfo {author} {\bibfnamefont {H.~T.}\ \bibnamefont {Quan}},\ }\href
	{\doibase 10.1103/PhysRevLett.117.180603} {\bibfield  {journal} {\bibinfo
			{journal} {Phys. Rev. Lett.}\ }\textbf {\bibinfo {volume} {117}},\ \bibinfo
		{pages} {180603} (\bibinfo {year} {2016})}\BibitemShut {NoStop}%
	\bibitem [{\citenamefont {Goldt}\ and\ \citenamefont
		{Seifert}(2017)}]{Goldt.2017.PRL}%
	\BibitemOpen
	\bibfield  {author} {\bibinfo {author} {\bibfnamefont {S.}~\bibnamefont
			{Goldt}}\ and\ \bibinfo {author} {\bibfnamefont {U.}~\bibnamefont
			{Seifert}},\ }\href {\doibase 10.1103/PhysRevLett.118.010601} {\bibfield
		{journal} {\bibinfo  {journal} {Phys. Rev. Lett.}\ }\textbf {\bibinfo
			{volume} {118}},\ \bibinfo {pages} {010601} (\bibinfo {year}
		{2017})}\BibitemShut {NoStop}%
	\bibitem [{\citenamefont {Tietz}\ \emph {et~al.}(2006)\citenamefont {Tietz},
		\citenamefont {Schuler}, \citenamefont {Speck}, \citenamefont {Seifert},\
		and\ \citenamefont {Wrachtrup}}]{Tietz.2006.PRL}%
	\BibitemOpen
	\bibfield  {author} {\bibinfo {author} {\bibfnamefont {C.}~\bibnamefont
			{Tietz}}, \bibinfo {author} {\bibfnamefont {S.}~\bibnamefont {Schuler}},
		\bibinfo {author} {\bibfnamefont {T.}~\bibnamefont {Speck}}, \bibinfo
		{author} {\bibfnamefont {U.}~\bibnamefont {Seifert}}, \ and\ \bibinfo
		{author} {\bibfnamefont {J.}~\bibnamefont {Wrachtrup}},\ }\href {\doibase
		10.1103/PhysRevLett.97.050602} {\bibfield  {journal} {\bibinfo  {journal}
			{Phys. Rev. Lett.}\ }\textbf {\bibinfo {volume} {97}},\ \bibinfo {pages}
		{050602} (\bibinfo {year} {2006})}\BibitemShut {NoStop}%
	\bibitem [{\citenamefont {Blickle}\ \emph {et~al.}(2006)\citenamefont
		{Blickle}, \citenamefont {Speck}, \citenamefont {Helden}, \citenamefont
		{Seifert},\ and\ \citenamefont {Bechinger}}]{Blickle.2006.PRL}%
	\BibitemOpen
	\bibfield  {author} {\bibinfo {author} {\bibfnamefont {V.}~\bibnamefont
			{Blickle}}, \bibinfo {author} {\bibfnamefont {T.}~\bibnamefont {Speck}},
		\bibinfo {author} {\bibfnamefont {L.}~\bibnamefont {Helden}}, \bibinfo
		{author} {\bibfnamefont {U.}~\bibnamefont {Seifert}}, \ and\ \bibinfo
		{author} {\bibfnamefont {C.}~\bibnamefont {Bechinger}},\ }\href {\doibase
		10.1103/PhysRevLett.96.070603} {\bibfield  {journal} {\bibinfo  {journal}
			{Phys. Rev. Lett.}\ }\textbf {\bibinfo {volume} {96}},\ \bibinfo {pages}
		{070603} (\bibinfo {year} {2006})}\BibitemShut {NoStop}%
	\bibitem [{\citenamefont {Schmiedl}\ and\ \citenamefont
		{Seifert}(2007)}]{Schmiedl.2007.JCP}%
	\BibitemOpen
	\bibfield  {author} {\bibinfo {author} {\bibfnamefont {T.}~\bibnamefont
			{Schmiedl}}\ and\ \bibinfo {author} {\bibfnamefont {U.}~\bibnamefont
			{Seifert}},\ }\href {\doibase 10.1063/1.2428297} {\bibfield  {journal}
		{\bibinfo  {journal} {J. Chem. Phys.}\ }\textbf {\bibinfo {volume} {126}},\
		\bibinfo {pages} {044101} (\bibinfo {year} {2007})}\BibitemShut {NoStop}%
	\bibitem [{\citenamefont {Schmiedl}\ \emph {et~al.}(2007)\citenamefont
		{Schmiedl}, \citenamefont {Speck},\ and\ \citenamefont
		{Seifert}}]{Schmiedl.2007.JSP}%
	\BibitemOpen
	\bibfield  {author} {\bibinfo {author} {\bibfnamefont {T.}~\bibnamefont
			{Schmiedl}}, \bibinfo {author} {\bibfnamefont {T.}~\bibnamefont {Speck}}, \
		and\ \bibinfo {author} {\bibfnamefont {U.}~\bibnamefont {Seifert}},\ }\href
	{\doibase 10.1007/s10955-006-9148-1} {\bibfield  {journal} {\bibinfo
			{journal} {J. Stat. Phys.}\ }\textbf {\bibinfo {volume} {128}},\ \bibinfo
		{pages} {77} (\bibinfo {year} {2007})}\BibitemShut {NoStop}%
	\bibitem [{\citenamefont {Mehta}\ and\ \citenamefont
		{Schwab}(2012)}]{Mehta.2012.PNAS}%
	\BibitemOpen
	\bibfield  {author} {\bibinfo {author} {\bibfnamefont {P.}~\bibnamefont
			{Mehta}}\ and\ \bibinfo {author} {\bibfnamefont {D.~J.}\ \bibnamefont
			{Schwab}},\ }\href {\doibase 10.1073/pnas.1207814109} {\bibfield  {journal}
		{\bibinfo  {journal} {Proc. Natl. Acad. Sci. U.S.A.}\ }\textbf {\bibinfo
			{volume} {109}},\ \bibinfo {pages} {17978} (\bibinfo {year}
		{2012})}\BibitemShut {NoStop}%
	\bibitem [{\citenamefont {Hasegawa}(2018)}]{Hasegawa.2018.PRE}%
	\BibitemOpen
	\bibfield  {author} {\bibinfo {author} {\bibfnamefont {Y.}~\bibnamefont
			{Hasegawa}},\ }\href {\doibase 10.1103/PhysRevE.98.032405} {\bibfield
		{journal} {\bibinfo  {journal} {Phys. Rev. E}\ }\textbf {\bibinfo {volume}
			{98}},\ \bibinfo {pages} {032405} (\bibinfo {year} {2018})}\BibitemShut
	{NoStop}%
	\bibitem [{\citenamefont {Barato}\ and\ \citenamefont
		{Seifert}(2015{\natexlab{a}})}]{Barato.2015.PRL}%
	\BibitemOpen
	\bibfield  {author} {\bibinfo {author} {\bibfnamefont {A.~C.}\ \bibnamefont
			{Barato}}\ and\ \bibinfo {author} {\bibfnamefont {U.}~\bibnamefont
			{Seifert}},\ }\href {\doibase 10.1103/PhysRevLett.114.158101} {\bibfield
		{journal} {\bibinfo  {journal} {Phys. Rev. Lett.}\ }\textbf {\bibinfo
			{volume} {114}},\ \bibinfo {pages} {158101} (\bibinfo {year}
		{2015}{\natexlab{a}})}\BibitemShut {NoStop}%
	\bibitem [{\citenamefont {Gingrich}\ \emph {et~al.}(2016)\citenamefont
		{Gingrich}, \citenamefont {Horowitz}, \citenamefont {Perunov},\ and\
		\citenamefont {England}}]{Gingrich.2016.PRL}%
	\BibitemOpen
	\bibfield  {author} {\bibinfo {author} {\bibfnamefont {T.~R.}\ \bibnamefont
			{Gingrich}}, \bibinfo {author} {\bibfnamefont {J.~M.}\ \bibnamefont
			{Horowitz}}, \bibinfo {author} {\bibfnamefont {N.}~\bibnamefont {Perunov}}, \
		and\ \bibinfo {author} {\bibfnamefont {J.~L.}\ \bibnamefont {England}},\
	}\href {\doibase 10.1103/PhysRevLett.116.120601} {\bibfield  {journal}
		{\bibinfo  {journal} {Phys. Rev. Lett.}\ }\textbf {\bibinfo {volume} {116}},\
		\bibinfo {pages} {120601} (\bibinfo {year} {2016})}\BibitemShut {NoStop}%
	\bibitem [{\citenamefont {Pietzonka}\ \emph
		{et~al.}(2016{\natexlab{a}})\citenamefont {Pietzonka}, \citenamefont
		{Barato},\ and\ \citenamefont {Seifert}}]{Pietzonka.2016.PRE}%
	\BibitemOpen
	\bibfield  {author} {\bibinfo {author} {\bibfnamefont {P.}~\bibnamefont
			{Pietzonka}}, \bibinfo {author} {\bibfnamefont {A.~C.}\ \bibnamefont
			{Barato}}, \ and\ \bibinfo {author} {\bibfnamefont {U.}~\bibnamefont
			{Seifert}},\ }\href {\doibase 10.1103/PhysRevE.93.052145} {\bibfield
		{journal} {\bibinfo  {journal} {Phys. Rev. E}\ }\textbf {\bibinfo {volume}
			{93}},\ \bibinfo {pages} {052145} (\bibinfo {year}
		{2016}{\natexlab{a}})}\BibitemShut {NoStop}%
	\bibitem [{\citenamefont {Polettini}\ \emph {et~al.}(2016)\citenamefont
		{Polettini}, \citenamefont {Lazarescu},\ and\ \citenamefont
		{Esposito}}]{Polettini.2016.PRE}%
	\BibitemOpen
	\bibfield  {author} {\bibinfo {author} {\bibfnamefont {M.}~\bibnamefont
			{Polettini}}, \bibinfo {author} {\bibfnamefont {A.}~\bibnamefont
			{Lazarescu}}, \ and\ \bibinfo {author} {\bibfnamefont {M.}~\bibnamefont
			{Esposito}},\ }\href {\doibase 10.1103/PhysRevE.94.052104} {\bibfield
		{journal} {\bibinfo  {journal} {Phys. Rev. E}\ }\textbf {\bibinfo {volume}
			{94}},\ \bibinfo {pages} {052104} (\bibinfo {year} {2016})}\BibitemShut
	{NoStop}%
	\bibitem [{\citenamefont {Barato}\ and\ \citenamefont
		{Seifert}(2015{\natexlab{b}})}]{Barato.2015.JPCB}%
	\BibitemOpen
	\bibfield  {author} {\bibinfo {author} {\bibfnamefont {A.~C.}\ \bibnamefont
			{Barato}}\ and\ \bibinfo {author} {\bibfnamefont {U.}~\bibnamefont
			{Seifert}},\ }\href {\doibase 10.1021/acs.jpcb.5b01918} {\bibfield  {journal}
		{\bibinfo  {journal} {J. Phys. Chem. B}\ }\textbf {\bibinfo {volume} {119}},\
		\bibinfo {pages} {6555} (\bibinfo {year} {2015}{\natexlab{b}})}\BibitemShut
	{NoStop}%
	\bibitem [{\citenamefont {Falasco}\ \emph {et~al.}(2016)\citenamefont
		{Falasco}, \citenamefont {Pfaller}, \citenamefont {Bregulla}, \citenamefont
		{Cichos},\ and\ \citenamefont {Kroy}}]{Falasco.2016.PRE}%
	\BibitemOpen
	\bibfield  {author} {\bibinfo {author} {\bibfnamefont {G.}~\bibnamefont
			{Falasco}}, \bibinfo {author} {\bibfnamefont {R.}~\bibnamefont {Pfaller}},
		\bibinfo {author} {\bibfnamefont {A.~P.}\ \bibnamefont {Bregulla}}, \bibinfo
		{author} {\bibfnamefont {F.}~\bibnamefont {Cichos}}, \ and\ \bibinfo {author}
		{\bibfnamefont {K.}~\bibnamefont {Kroy}},\ }\href {\doibase
		10.1103/PhysRevE.94.030602} {\bibfield  {journal} {\bibinfo  {journal} {Phys.
				Rev. E}\ }\textbf {\bibinfo {volume} {94}},\ \bibinfo {pages} {030602}
		(\bibinfo {year} {2016})}\BibitemShut {NoStop}%
	\bibitem [{\citenamefont {Pietzonka}\ \emph
		{et~al.}(2016{\natexlab{b}})\citenamefont {Pietzonka}, \citenamefont
		{Barato},\ and\ \citenamefont {Seifert}}]{Patrick.2016.JSM}%
	\BibitemOpen
	\bibfield  {author} {\bibinfo {author} {\bibfnamefont {P.}~\bibnamefont
			{Pietzonka}}, \bibinfo {author} {\bibfnamefont {A.~C.}\ \bibnamefont
			{Barato}}, \ and\ \bibinfo {author} {\bibfnamefont {U.}~\bibnamefont
			{Seifert}},\ }\href {http://stacks.iop.org/1742-5468/2016/i=12/a=124004}
	{\bibfield  {journal} {\bibinfo  {journal} {J. Stat. Mech.}\ }\textbf
		{\bibinfo {volume} {2016}},\ \bibinfo {pages} {124004} (\bibinfo {year}
		{2016}{\natexlab{b}})}\BibitemShut {NoStop}%
	\bibitem [{\citenamefont {Rotskoff}(2017)}]{Rotskoff.2017.PRE}%
	\BibitemOpen
	\bibfield  {author} {\bibinfo {author} {\bibfnamefont {G.~M.}\ \bibnamefont
			{Rotskoff}},\ }\href {\doibase 10.1103/PhysRevE.95.030101} {\bibfield
		{journal} {\bibinfo  {journal} {Phys. Rev. E}\ }\textbf {\bibinfo {volume}
			{95}},\ \bibinfo {pages} {030101} (\bibinfo {year} {2017})}\BibitemShut
	{NoStop}%
	\bibitem [{\citenamefont {Garrahan}(2017)}]{Garrahan.2017.PRE}%
	\BibitemOpen
	\bibfield  {author} {\bibinfo {author} {\bibfnamefont {J.~P.}\ \bibnamefont
			{Garrahan}},\ }\href {\doibase 10.1103/PhysRevE.95.032134} {\bibfield
		{journal} {\bibinfo  {journal} {Phys. Rev. E}\ }\textbf {\bibinfo {volume}
			{95}},\ \bibinfo {pages} {032134} (\bibinfo {year} {2017})}\BibitemShut
	{NoStop}%
	\bibitem [{\citenamefont {Gingrich}\ and\ \citenamefont
		{Horowitz}(2017)}]{Gingrich.2017.PRL}%
	\BibitemOpen
	\bibfield  {author} {\bibinfo {author} {\bibfnamefont {T.~R.}\ \bibnamefont
			{Gingrich}}\ and\ \bibinfo {author} {\bibfnamefont {J.~M.}\ \bibnamefont
			{Horowitz}},\ }\href {\doibase 10.1103/PhysRevLett.119.170601} {\bibfield
		{journal} {\bibinfo  {journal} {Phys. Rev. Lett.}\ }\textbf {\bibinfo
			{volume} {119}},\ \bibinfo {pages} {170601} (\bibinfo {year}
		{2017})}\BibitemShut {NoStop}%
	\bibitem [{\citenamefont {Hyeon}\ and\ \citenamefont
		{Hwang}(2017)}]{Hyeon.2017.PRE}%
	\BibitemOpen
	\bibfield  {author} {\bibinfo {author} {\bibfnamefont {C.}~\bibnamefont
			{Hyeon}}\ and\ \bibinfo {author} {\bibfnamefont {W.}~\bibnamefont {Hwang}},\
	}\href {\doibase 10.1103/PhysRevE.96.012156} {\bibfield  {journal} {\bibinfo
			{journal} {Phys. Rev. E}\ }\textbf {\bibinfo {volume} {96}},\ \bibinfo
		{pages} {012156} (\bibinfo {year} {2017})}\BibitemShut {NoStop}%
	\bibitem [{\citenamefont {Proesmans}\ and\ \citenamefont {den
			Broeck}(2017)}]{Karel.2017.EPL}%
	\BibitemOpen
	\bibfield  {author} {\bibinfo {author} {\bibfnamefont {K.}~\bibnamefont
			{Proesmans}}\ and\ \bibinfo {author} {\bibfnamefont {C.~V.}\ \bibnamefont
			{den Broeck}},\ }\href {http://stacks.iop.org/0295-5075/119/i=2/a=20001}
	{\bibfield  {journal} {\bibinfo  {journal} {EPL}\ }\textbf {\bibinfo {volume}
			{119}},\ \bibinfo {pages} {20001} (\bibinfo {year} {2017})}\BibitemShut
	{NoStop}%
	\bibitem [{\citenamefont {Chiuchi\`u}\ and\ \citenamefont
		{Pigolotti}(2018)}]{Chiuchiu.2018.PRE}%
	\BibitemOpen
	\bibfield  {author} {\bibinfo {author} {\bibfnamefont {D.}~\bibnamefont
			{Chiuchi\`u}}\ and\ \bibinfo {author} {\bibfnamefont {S.}~\bibnamefont
			{Pigolotti}},\ }\href {\doibase 10.1103/PhysRevE.97.032109} {\bibfield
		{journal} {\bibinfo  {journal} {Phys. Rev. E}\ }\textbf {\bibinfo {volume}
			{97}},\ \bibinfo {pages} {032109} (\bibinfo {year} {2018})}\BibitemShut
	{NoStop}%
	\bibitem [{\citenamefont {Brandner}\ \emph {et~al.}(2018)\citenamefont
		{Brandner}, \citenamefont {Hanazato},\ and\ \citenamefont
		{Saito}}]{Brandner.2018.PRL}%
	\BibitemOpen
	\bibfield  {author} {\bibinfo {author} {\bibfnamefont {K.}~\bibnamefont
			{Brandner}}, \bibinfo {author} {\bibfnamefont {T.}~\bibnamefont {Hanazato}},
		\ and\ \bibinfo {author} {\bibfnamefont {K.}~\bibnamefont {Saito}},\ }\href
	{\doibase 10.1103/PhysRevLett.120.090601} {\bibfield  {journal} {\bibinfo
			{journal} {Phys. Rev. Lett.}\ }\textbf {\bibinfo {volume} {120}},\ \bibinfo
		{pages} {090601} (\bibinfo {year} {2018})}\BibitemShut {NoStop}%
	\bibitem [{\citenamefont {Hwang}\ and\ \citenamefont
		{Hyeon}(2018)}]{Hwang.2018.JPCL}%
	\BibitemOpen
	\bibfield  {author} {\bibinfo {author} {\bibfnamefont {W.}~\bibnamefont
			{Hwang}}\ and\ \bibinfo {author} {\bibfnamefont {C.}~\bibnamefont {Hyeon}},\
	}\href {\doibase 10.1021/acs.jpclett.7b03197} {\bibfield  {journal} {\bibinfo
			{journal} {J. Phys. Chem. Lett.}\ }\textbf {\bibinfo {volume} {9}},\
		\bibinfo {pages} {513} (\bibinfo {year} {2018})}\BibitemShut {NoStop}%
	\bibitem [{\citenamefont {Pietzonka}\ \emph {et~al.}(2017)\citenamefont
		{Pietzonka}, \citenamefont {Ritort},\ and\ \citenamefont
		{Seifert}}]{Pietzonka.2017.PRE}%
	\BibitemOpen
	\bibfield  {author} {\bibinfo {author} {\bibfnamefont {P.}~\bibnamefont
			{Pietzonka}}, \bibinfo {author} {\bibfnamefont {F.}~\bibnamefont {Ritort}}, \
		and\ \bibinfo {author} {\bibfnamefont {U.}~\bibnamefont {Seifert}},\ }\href
	{\doibase 10.1103/PhysRevE.96.012101} {\bibfield  {journal} {\bibinfo
			{journal} {Phys. Rev. E}\ }\textbf {\bibinfo {volume} {96}},\ \bibinfo
		{pages} {012101} (\bibinfo {year} {2017})}\BibitemShut {NoStop}%
	\bibitem [{\citenamefont {Horowitz}\ and\ \citenamefont
		{Gingrich}(2017)}]{Horowitz.2017.PRE}%
	\BibitemOpen
	\bibfield  {author} {\bibinfo {author} {\bibfnamefont {J.~M.}\ \bibnamefont
			{Horowitz}}\ and\ \bibinfo {author} {\bibfnamefont {T.~R.}\ \bibnamefont
			{Gingrich}},\ }\href {\doibase 10.1103/PhysRevE.96.020103} {\bibfield
		{journal} {\bibinfo  {journal} {Phys. Rev. E}\ }\textbf {\bibinfo {volume}
			{96}},\ \bibinfo {pages} {020103} (\bibinfo {year} {2017})}\BibitemShut
	{NoStop}%
	\bibitem [{\citenamefont {Dechant}\ and\ \citenamefont {ichi
			Sasa}(2018)}]{Andreas.2018.JSM}%
	\BibitemOpen
	\bibfield  {author} {\bibinfo {author} {\bibfnamefont {A.}~\bibnamefont
			{Dechant}}\ and\ \bibinfo {author} {\bibfnamefont {S.}~\bibnamefont {ichi
				Sasa}},\ }\href {http://stacks.iop.org/1742-5468/2018/i=6/a=063209}
	{\bibfield  {journal} {\bibinfo  {journal} {J. Stat. Mech.}\ }\textbf
		{\bibinfo {volume} {2018}},\ \bibinfo {pages} {063209} (\bibinfo {year}
		{2018})}\BibitemShut {NoStop}%
	\bibitem [{\citenamefont {Pigolotti}\ \emph {et~al.}(2017)\citenamefont
		{Pigolotti}, \citenamefont {Neri}, \citenamefont {Rold\'an},\ and\
		\citenamefont {J\"ulicher}}]{Pigolotti.2017.PRL}%
	\BibitemOpen
	\bibfield  {author} {\bibinfo {author} {\bibfnamefont {S.}~\bibnamefont
			{Pigolotti}}, \bibinfo {author} {\bibfnamefont {I.}~\bibnamefont {Neri}},
		\bibinfo {author} {\bibfnamefont {E.}~\bibnamefont {Rold\'an}}, \ and\
		\bibinfo {author} {\bibfnamefont {F.}~\bibnamefont {J\"ulicher}},\ }\href
	{\doibase 10.1103/PhysRevLett.119.140604} {\bibfield  {journal} {\bibinfo
			{journal} {Phys. Rev. Lett.}\ }\textbf {\bibinfo {volume} {119}},\ \bibinfo
		{pages} {140604} (\bibinfo {year} {2017})}\BibitemShut {NoStop}%
	\bibitem [{\citenamefont {Hasegawa}\ and\ \citenamefont
		{Vu}(2018)}]{Hasegawa.2018.arxiv}%
	\BibitemOpen
	\bibfield  {author} {\bibinfo {author} {\bibfnamefont {Y.}~\bibnamefont
			{Hasegawa}}\ and\ \bibinfo {author} {\bibfnamefont {T.~V.}\ \bibnamefont
			{Vu}},\ }\href {https://arxiv.org/abs/1809.03292} {\bibfield  {journal}
		{\bibinfo  {journal} {arXiv:1809.03292}\ } (\bibinfo {year}
		{2018})}\BibitemShut {NoStop}%
	\bibitem [{\citenamefont {Bratsun}\ \emph {et~al.}(2005)\citenamefont
		{Bratsun}, \citenamefont {Volfson}, \citenamefont {Tsimring},\ and\
		\citenamefont {Hasty}}]{Bratsun.2005.PNAS}%
	\BibitemOpen
	\bibfield  {author} {\bibinfo {author} {\bibfnamefont {D.}~\bibnamefont
			{Bratsun}}, \bibinfo {author} {\bibfnamefont {D.}~\bibnamefont {Volfson}},
		\bibinfo {author} {\bibfnamefont {L.~S.}\ \bibnamefont {Tsimring}}, \ and\
		\bibinfo {author} {\bibfnamefont {J.}~\bibnamefont {Hasty}},\ }\href
	{\doibase 10.1073/pnas.0503858102} {\bibfield  {journal} {\bibinfo  {journal}
			{Proc. Natl. Acad. Sci. U.S.A.}\ }\textbf {\bibinfo {volume} {102}},\
		\bibinfo {pages} {14593} (\bibinfo {year} {2005})}\BibitemShut {NoStop}%
	\bibitem [{\citenamefont {Gupta}\ \emph {et~al.}(2013)\citenamefont {Gupta},
		\citenamefont {L\'opez}, \citenamefont {Ott}, \citenamefont
		{Josi\ifmmode~\acute{c}\else \'{c}\fi{}},\ and\ \citenamefont
		{Bennett}}]{Gupta.2013.PRL}%
	\BibitemOpen
	\bibfield  {author} {\bibinfo {author} {\bibfnamefont {C.}~\bibnamefont
			{Gupta}}, \bibinfo {author} {\bibfnamefont {J.~M.}\ \bibnamefont {L\'opez}},
		\bibinfo {author} {\bibfnamefont {W.}~\bibnamefont {Ott}}, \bibinfo {author}
		{\bibfnamefont {K.~c.~v.}\ \bibnamefont {Josi\ifmmode~\acute{c}\else
				\'{c}\fi{}}}, \ and\ \bibinfo {author} {\bibfnamefont {M.~R.}\ \bibnamefont
			{Bennett}},\ }\href {\doibase 10.1103/PhysRevLett.111.058104} {\bibfield
		{journal} {\bibinfo  {journal} {Phys. Rev. Lett.}\ }\textbf {\bibinfo
			{volume} {111}},\ \bibinfo {pages} {058104} (\bibinfo {year}
		{2013})}\BibitemShut {NoStop}%
	\bibitem [{\citenamefont {Kim}\ \emph {et~al.}(1999)\citenamefont {Kim},
		\citenamefont {Park},\ and\ \citenamefont {Pyo}}]{Kim.1999.PRL}%
	\BibitemOpen
	\bibfield  {author} {\bibinfo {author} {\bibfnamefont {S.}~\bibnamefont
			{Kim}}, \bibinfo {author} {\bibfnamefont {S.~H.}\ \bibnamefont {Park}}, \
		and\ \bibinfo {author} {\bibfnamefont {H.-B.}\ \bibnamefont {Pyo}},\ }\href
	{\doibase 10.1103/PhysRevLett.82.1620} {\bibfield  {journal} {\bibinfo
			{journal} {Phys. Rev. Lett.}\ }\textbf {\bibinfo {volume} {82}},\ \bibinfo
		{pages} {1620} (\bibinfo {year} {1999})}\BibitemShut {NoStop}%
	\bibitem [{\citenamefont {Masoller}(2003)}]{Masoller.2003.PRL}%
	\BibitemOpen
	\bibfield  {author} {\bibinfo {author} {\bibfnamefont {C.}~\bibnamefont
			{Masoller}},\ }\href {\doibase 10.1103/PhysRevLett.90.020601} {\bibfield
		{journal} {\bibinfo  {journal} {Phys. Rev. Lett.}\ }\textbf {\bibinfo
			{volume} {90}},\ \bibinfo {pages} {020601} (\bibinfo {year}
		{2003})}\BibitemShut {NoStop}%
	\bibitem [{\citenamefont {Lichtner}\ \emph {et~al.}(2012)\citenamefont
		{Lichtner}, \citenamefont {Pototsky},\ and\ \citenamefont
		{Klapp}}]{Lichtner.2012.PRE}%
	\BibitemOpen
	\bibfield  {author} {\bibinfo {author} {\bibfnamefont {K.}~\bibnamefont
			{Lichtner}}, \bibinfo {author} {\bibfnamefont {A.}~\bibnamefont {Pototsky}},
		\ and\ \bibinfo {author} {\bibfnamefont {S.~H.~L.}\ \bibnamefont {Klapp}},\
	}\href {\doibase 10.1103/PhysRevE.86.051405} {\bibfield  {journal} {\bibinfo
			{journal} {Phys. Rev. E}\ }\textbf {\bibinfo {volume} {86}},\ \bibinfo
		{pages} {051405} (\bibinfo {year} {2012})}\BibitemShut {NoStop}%
	\bibitem [{\citenamefont {Jiang}\ \emph {et~al.}(2011)\citenamefont {Jiang},
		\citenamefont {Xiao},\ and\ \citenamefont {Hou}}]{Jiang.2011.PRE}%
	\BibitemOpen
	\bibfield  {author} {\bibinfo {author} {\bibfnamefont {H.}~\bibnamefont
			{Jiang}}, \bibinfo {author} {\bibfnamefont {T.}~\bibnamefont {Xiao}}, \ and\
		\bibinfo {author} {\bibfnamefont {Z.}~\bibnamefont {Hou}},\ }\href {\doibase
		10.1103/PhysRevE.83.061144} {\bibfield  {journal} {\bibinfo  {journal} {Phys.
				Rev. E}\ }\textbf {\bibinfo {volume} {83}},\ \bibinfo {pages} {061144}
		(\bibinfo {year} {2011})}\BibitemShut {NoStop}%
	\bibitem [{\citenamefont {Frank}\ \emph {et~al.}(2003)\citenamefont {Frank},
		\citenamefont {Beek},\ and\ \citenamefont {Friedrich}}]{Frank.2003.PRE}%
	\BibitemOpen
	\bibfield  {author} {\bibinfo {author} {\bibfnamefont {T.~D.}\ \bibnamefont
			{Frank}}, \bibinfo {author} {\bibfnamefont {P.~J.}\ \bibnamefont {Beek}}, \
		and\ \bibinfo {author} {\bibfnamefont {R.}~\bibnamefont {Friedrich}},\ }\href
	{\doibase 10.1103/PhysRevE.68.021912} {\bibfield  {journal} {\bibinfo
			{journal} {Phys. Rev. E}\ }\textbf {\bibinfo {volume} {68}},\ \bibinfo
		{pages} {021912} (\bibinfo {year} {2003})}\BibitemShut {NoStop}%
	\bibitem [{\citenamefont {Frank}(2005)}]{Frank.2005.PRE}%
	\BibitemOpen
	\bibfield  {author} {\bibinfo {author} {\bibfnamefont {T.~D.}\ \bibnamefont
			{Frank}},\ }\href {\doibase 10.1103/PhysRevE.72.011112} {\bibfield  {journal}
		{\bibinfo  {journal} {Phys. Rev. E}\ }\textbf {\bibinfo {volume} {72}},\
		\bibinfo {pages} {011112} (\bibinfo {year} {2005})}\BibitemShut {NoStop}%
	\bibitem [{\citenamefont {Dechant}\ and\ \citenamefont
		{Sasa}(2018)}]{Andreas.2018.PRE}%
	\BibitemOpen
	\bibfield  {author} {\bibinfo {author} {\bibfnamefont {A.}~\bibnamefont
			{Dechant}}\ and\ \bibinfo {author} {\bibfnamefont {S.-i.}\ \bibnamefont
			{Sasa}},\ }\href {\doibase 10.1103/PhysRevE.97.062101} {\bibfield  {journal}
		{\bibinfo  {journal} {Phys. Rev. E}\ }\textbf {\bibinfo {volume} {97}},\
		\bibinfo {pages} {062101} (\bibinfo {year} {2018})}\BibitemShut {NoStop}%
	\bibitem [{\citenamefont {Risken}(1989)}]{Risken.1989}%
	\BibitemOpen
	\bibfield  {author} {\bibinfo {author} {\bibfnamefont {H.}~\bibnamefont
			{Risken}},\ }\href@noop {} {\emph {\bibinfo {title} {The Fokker-Planck
				Equation: Methods of Solution and Applications}}},\ \bibinfo {edition} {2nd}\
	ed.\ (\bibinfo  {publisher} {Springer},\ \bibinfo {year} {1989})\BibitemShut
	{NoStop}%
	\bibitem [{\citenamefont {Juniper}\ \emph {et~al.}(2016)\citenamefont
		{Juniper}, \citenamefont {Straube}, \citenamefont {Aarts},\ and\
		\citenamefont {Dullens}}]{Juniper.2016.PRE}%
	\BibitemOpen
	\bibfield  {author} {\bibinfo {author} {\bibfnamefont {M.~P.~N.}\
			\bibnamefont {Juniper}}, \bibinfo {author} {\bibfnamefont {A.~V.}\
			\bibnamefont {Straube}}, \bibinfo {author} {\bibfnamefont {D.~G. A.~L.}\
			\bibnamefont {Aarts}}, \ and\ \bibinfo {author} {\bibfnamefont {R.~P.~A.}\
			\bibnamefont {Dullens}},\ }\href {\doibase 10.1103/PhysRevE.93.012608}
	{\bibfield  {journal} {\bibinfo  {journal} {Phys. Rev. E}\ }\textbf {\bibinfo
			{volume} {93}},\ \bibinfo {pages} {012608} (\bibinfo {year}
		{2016})}\BibitemShut {NoStop}%
	\bibitem [{\citenamefont {Adib}(2008)}]{Adib.2008.JPCB}%
	\BibitemOpen
	\bibfield  {author} {\bibinfo {author} {\bibfnamefont {A.~B.}\ \bibnamefont
			{Adib}},\ }\href {\doibase 10.1021/jp0751458} {\bibfield  {journal} {\bibinfo
			{journal} {J. Phys. Chem. B}\ }\textbf {\bibinfo {volume} {112}},\ \bibinfo
		{pages} {5910} (\bibinfo {year} {2008})}\BibitemShut {NoStop}%
\end{thebibliography}%


\end{document}